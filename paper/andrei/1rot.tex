\documentclass[a4paper]{article}

\usepackage[margin=1.1in]{geometry}
\usepackage[utf8]{inputenc}
%\usepackage[T1]{fontenc}
%\usepackage{textcomp}
\usepackage[spanish]{babel}
\usepackage{amsmath, amssymb}
\usepackage{amsthm}
\usepackage{braket}
\decimalpoint

\DeclareMathOperator{\R}{\mathbb{R}}
\DeclareMathOperator{\C}{\mathbb{C}}
\DeclareMathOperator{\N}{\mathbb{N}}
\DeclareMathOperator{\Z}{\mathbb{Z}}
\DeclareMathOperator{\Tr}{Tr}
\DeclareMathOperator{\tr}{tr}

\title{Wigner function for one rotation}
\author{Ernesto Camacho Ramírez}
\begin{document}
  \maketitle

  I was under the understanding that this was trivial. But
  the numerical evidence just doesn't match up, so I will
  try to prove it and see if there is a faulty step
  anywhere.  Let's consider the standard Wigner function
  kernel:
  \begin{equation}
    \label{eqn:ray_kernel}
    w(\alpha,\beta)
    = \ket{\tilde\alpha}\bra{\tilde\alpha}
    + \sum_{\xi,\nu}^{} \delta_{\beta,\xi\alpha+\nu} 
    \ket{\psi_{\xi,\nu}}\bra{\psi_{\xi,\nu}}
    - I,
  \end{equation}
  where $\ket{\kappa}$ is the computational basis,
  $\ket{\tilde\kappa}$ is the Fourier basis and
  $\ket{\psi_{\xi,\nu}} = V_\xi \ket{\nu}$ where $V_\xi$ is
  the ray rotation operator for a given slope $\xi$. These
  rotation operators are defined by their action on the
  horizontal $Z_\alpha$ operators:
  \begin{equation}
    V_\xi Z_\alpha V_\xi^{*}
    = c_{\alpha,\xi}^* \chi(\xi \alpha^2) Z_\alpha
    X_{\xi\alpha},
  \end{equation}
  where $V_\xi$ is unitary and can be expressed diagonally
  in the Fourier basis:
  \begin{equation}
    V_\xi
    = \sum_{\kappa}^{} c_{\kappa,\xi} \ket{\tilde\kappa}
    \bra{\tilde\kappa},
  \end{equation}
  with the coefficients satisfying the recurrence relation
  \begin{equation}
    c_{\alpha,\xi} c_{\kappa,\xi}
    = c_{\alpha+\kappa,\xi} \chi(\xi\kappa\alpha).
  \end{equation}
  This condition must be satisfied in order for the Wigner
  kernel to have the tomographic property. In order to
  explícitly calculate these operators we must solve the
  recurrence relation. In order to match up with the results
  of Klimov, Muñoz, Sanchez-Soto we can follow their
  methodology. This consists in first noting that for
  $\alpha = \kappa$ we obtain:
  \begin{equation}
    c_{\alpha,\xi}^2 = \chi(\xi\alpha^2).
  \end{equation}
  By choosing a basis $\{\sigma_j\}$ of the Galois field
  $GF(2^N)$, we can solve this equation (non-uniquely) for
  each basis element $\sigma_j$. Of course we have multiple
  sets of solutions considering the square root, so the
  authors \textit{fix} a choice of signs. In the 2017 paper
  they decided to choose the all positive solutions. Having
  fixed a solution for the basis, we can decompose any
  element of the field as a linear combination of said basis
  elements and use the recurrence relation to obtain the
  general formula:
  \begin{equation}
    c_{\alpha,\xi}
    = c_{\sum_{j=1}^{} a_j \sigma_j,\xi}
    = \chi\left[
    \mu \sum_{k=1}^{N-1} \left(
      a_k \sigma_k 
      \sum_{j=k+1}^{N} a_j \sigma_j
    \right) \right]
    \prod_{l=1}^N c_{a_l \sigma_l, \mu}.
  \end{equation}

  Given this solution for the recurrence equation we can now
  obtain the explicit expression of the ray rotation
  operators.

  In the 2017 paper, the authors use the same solution
  method for curves more general than those of the rays.  A
  curve of the form $\beta = f(\alpha)$ for suitable $f$ can
  always be transformed into the horizontal ray. In the
  operator viewpoint this can be accomplished by a
  symplectic operator $P_f$ such that
  \begin{equation}
    P_f Z_\alpha P_f^* \sim Z_\alpha X_{f(\alpha)}.
  \end{equation}
  This operator can once again be expressed diagonally in
  the Fourier basis:
  \begin{equation}
    P_f = \sum_{\kappa}^{} c_{\kappa,f}
    \ket{\tilde\kappa}\bra{\tilde\kappa},
  \end{equation}
  where the coefficients $c_{\kappa,f}$ now satisfy the
  recurrence relation:
  \begin{equation}
    c_{\kappa,f} c_{\alpha,f}
    = \chi(\alpha f(\kappa)) c_{\kappa+\alpha,f},
    \quad c_{0,f} = 1.
  \end{equation}
  Again, assuming that $\alpha = \kappa$ we obtain the
  non-uniquely solvable equation:
  \begin{equation}
    c_{\kappa,f}^2 = \chi(\kappa f(\kappa)).
  \end{equation}
  We employ the same solution method as before, i.e., we
  solve this equation for the basis elements and fix a sign
  convention, which in our case is the all positive
  solutions. Then we use the recurrence relation to obtain
  the following general formula:
  \begin{equation}
    c_{\alpha,f}
    = c_{\sum_{j=1} a_j \sigma_j, f}
    = \prod_{l=1}^N c_{a_l \sigma_l, f} \chi\left[
    \sum_{k=1}^{N-1} \left(\sum_{j=k+1}^N a_j
    \sigma_j\right) f(a_k \sigma_k) \right],
    \quad c_{0,f} = 1.
  \end{equation}
  Which is, of course, very similar to the ray formula. We
  can use this rotation operator $P_f$ to obtain new sets of
  orthonormal basis from the standard set
  $\ket{\psi_{\xi,\nu}}$. The nature of the $P_f$ implies
  that these new sets preserve the mutual unbiasedness of
  the standard sets. We label these elements as:
  \begin{equation}
    \ket{\psi_{\xi,\nu}^f}
    = P_f \ket{\psi_{\xi,\nu}},
    \quad
    \ket{\tilde\nu^f}
    = P_f \ket{\tilde\nu}.
  \end{equation}

  Now onto our main question. Given a Wigner kernel
  $w(\alpha,\beta)$ made up of the standard states, under
  what conditions is Wigner function of a state from a
  \textit{rotated set} equal to a delta function?
  \textit{(Just as a thought, isn't this kinda of like
    asking if the Wigner kernel is covariant under rotation
    operations? We know that for states from the standard
    set, the standard kernel will produce a delta function.
    So if the kernel is covariant under rotations then this
    would imply that the Wigner function of a rotated state
    would just be translated in phase space, and therefore
    it would still be a delta function? But we know that for
    the qubit case, there is no
  covariant-under-all-symplectic-transformations
  Wigner kernel.)}

  From the definition of the ray kernel
  (\ref{eqn:ray_kernel}) we have for an arbitrary state
  $\rho = \ket{\psi_{\xi',\nu'}^f}
  \bra{\psi_{\xi',\nu'}^f}$:
  \begin{align}
    W_\rho(\alpha,\beta)
    &= \Tr(\rho w(\alpha,\beta)) \\
    &=
    \braket{
      \psi_{\xi',\nu'}^f|w(\alpha,\beta)|\psi_{\xi',\nu'}^f
    } \\
    &= \braket{\psi_{\xi',\nu'}^f|\tilde\alpha}
    \braket{\tilde\alpha|\psi_{\xi',\nu'}^f} 
    + \sum_{\xi,\nu}^{} \delta_{\beta,\xi\alpha+\nu}
    \braket{\psi_{\xi',\nu'}^f|\psi_{\xi,\nu}}
    \braket{\psi_{\xi,\nu}|\psi_{\xi',\nu'}^f}
    - \braket{\psi_{\xi',\nu'}^f|\psi_{\xi',\nu'}^f} \\
    &= \left|\braket{\psi_{\xi',\nu'}^f|\tilde\alpha}\right|^2
    + \sum_{\xi,\nu}^{} \delta_{\beta,\xi\alpha+\nu}
    \left|\braket{\psi_{\xi',\nu'}^f|\psi_{\xi,\nu}}\right|^2
    - 1.
  \end{align}

  Let's first analyze the sum term in the middle of the last
  equation. The properties of the inner products of states
  from different sets of MUBs are being studied right now,
  and for the moment the only thing that seems to be
  interesting is that they all seem to be equal to
  $\frac{1}{2^k}$ to some $k \le N$. For now let's
  calculate an arbitrary inner product. First recall that
  \begin{equation}
    \braket{\lambda|\tilde\kappa}
    = \bra{\lambda} F\ket{\kappa} \\
    = \bra{\lambda} \frac{1}{\sqrt{d}}\sum_{m,n}^{}
    \chi(mn) \ket{m}\braket{n|\kappa} \\
    %&= \bra{\lambda}\frac{1}{\sqrt{d}} \sum_{m}^{}
    %\chi(m\kappa) \ket{m} \\
    %&= \frac{1}{\sqrt{d}}\sum_{m}^{} \chi(m\kappa) 
    %\braket{\lambda|m} \\
    = \frac{1}{\sqrt{d}} \chi(\lambda \kappa).
  \end{equation}
  Therefore
  \begin{align}
    \braket{\psi_{\xi,\nu}|\tilde\kappa}
    &= \bra{\nu} V_\xi^* \ket{\tilde\kappa} \\
    &= \bra{\nu} 
    \left( 
      \sum_{m}^{} c_{m,\xi}^* \ket{\tilde m}\bra{\tilde m}
    \right) 
    \ket{\tilde\kappa} \\
    &= \sum_{m}^{} c_{m,\xi}^* \braket{\nu|\tilde m} 
    \braket{\tilde m|\tilde \kappa} \\
    &= c_{\kappa,\xi}^* \braket{\nu|\tilde\kappa} \\
    &= \frac{1}{\sqrt{d}} c_{\kappa,\xi}^* \chi(\nu\kappa).
  \end{align}
  Using this we obtain the following expression for the
  inner product of two states:
  \begin{align}
    \braket{\psi_{\xi,\nu}|\psi_{\xi',\nu'}^f}
    &= \braket{\psi_{\xi,\nu}| P_f |\psi_{\xi',\nu'}} \\
    &= \bra{\psi_{\xi,\nu}} 
    \left( \sum_{\kappa}^{} c_{\kappa,f} 
      \ket{\tilde\kappa}\bra{\tilde\kappa}
    \right) 
    \ket{\psi_{\xi',\nu'}} \\
    &= \sum_{\kappa}^{} c_{\kappa,f} 
    \braket{\psi_{\xi,\nu}|\tilde\kappa}
    \braket{\tilde\kappa|\psi_{\xi',\nu'}} \\
    &= \sum_{\kappa}^{} c_{\kappa,f}
    \left( 
      \frac{1}{\sqrt{d}} c_{\kappa,\xi}^* \chi(\nu\kappa)
    \right) 
    \left( 
      \frac{1}{\sqrt{d}} c_{\kappa,\xi'}^* \chi(\nu'\kappa)
    \right)^* \\
    &= \frac{1}{d} \sum_{\kappa}^{} c_{\kappa,f} 
    c_{\kappa,\xi}^* c_{\kappa,\xi'} \chi(\kappa(\nu+\nu')).
  \end{align}

  Doesn't seem like we can do much more than this for now,
  so let's proceed by finding its absolute square:

  \begin{align}
    \braket{\psi_{\xi',\nu'}^f|\psi_{\xi,\nu}}
    \braket{\psi_{\xi,\nu}|\psi_{\xi',\nu'}^f}
    &= \left( 
      \frac{1}{d} \sum_{\kappa}^{} c_{\kappa,f} 
      c_{\kappa,\xi}^* c_{\kappa,\xi'} \chi(\kappa(\nu+\nu'))
    \right)^*
    \left( 
      \frac{1}{d} \sum_{\kappa}^{} c_{\kappa,f} 
      c_{\kappa,\xi}^* c_{\kappa,\xi'} \chi(\kappa(\nu+\nu'))
    \right) \\
    &= \frac{1}{d^2} 
    \left( 
      \sum_{\kappa}^{} c_{\kappa,f}^* 
      c_{\kappa,\xi} c_{\kappa,\xi'}^* \chi(\kappa(\nu+\nu'))
    \right)
    \left( 
      \sum_{\kappa}^{} c_{\kappa,f} 
      c_{\kappa,\xi}^* c_{\kappa,\xi'} \chi(\kappa(\nu+\nu'))
    \right) \\
    &= \frac{1}{d^2} 
    \sum_{\kappa,\kappa'}^{}  
    (c_{\kappa,f}^* c_{\kappa',f})
    (c_{\kappa,\xi'}^* c_{\kappa',\xi'})
    (c_{\kappa',\xi}^* c_{\kappa,\xi})
    \chi((\kappa+\kappa')(\nu+\nu')).
    \label{eqn:squared_modulus}
  \end{align}

  Before continuing we will express the recurrence relation
  in a different manner to accomadate the previous
  equations. For the curve rotation coefficients we recall
  that 
  \begin{equation}
    c_{\kappa,f}^2
    = \chi(\kappa f(\kappa))
    = \chi(\kappa f(\kappa))^*
    = \left( c_{\kappa,f}^2 \right)^*
    = \left( c_{\kappa,f}^* \right)^2.
  \end{equation}
  Therefore
  \begin{align}
    c_{\kappa,f}^* c_{\kappa',f}
    &= \left(c_{\kappa,f}^*\right)^2 (c_{\kappa,f}
    c_{\kappa',f}) \\
    &= \chi(\kappa f(\kappa)) \chi(\kappa' f(\kappa))
    c_{\kappa+\kappa',f}  \\
    &= \chi\left( \kappa(f(\kappa) +f(\kappa')) \right) 
    c_{\kappa+\kappa',f} \\
    &= \chi\left( \kappa f(\kappa + \kappa') \right) 
    c_{\kappa+\kappa',f},
  \end{align}
  where we have used the property $\chi(\kappa' f(\kappa)) =
  \chi(\kappa f(\kappa'))$ for all $\kappa,\kappa' \in
  GF(2^N)$ and the additivity of the curve $f$. An analogous
  property holds for the ray rotation coefficients:
  \begin{equation}
    c_{\kappa,\xi}^* c_{\kappa',\xi}
    = \chi\left( \kappa \xi(\kappa+\kappa') \right) 
    c_{\kappa+\kappa',\xi}.
  \end{equation}
  Using this last expression we can see that:
  \begin{equation}
    c_{\kappa,\xi} c_{\kappa',\xi}^*
    = \chi(\kappa \xi(\kappa + \kappa')) c_{\kappa +
    \kappa',\xi}^*.
  \end{equation}

  Substituting these expressions in equation
  (\ref{eqn:squared_modulus}) and using the change of
  variables $\eta = \kappa + \kappa'$ we obtain:
  \begin{align}
    \left|\braket{\psi_{\xi,\nu}|\psi_{\xi',\nu'}^f}\right|^2
    &= \frac{1}{d^2} 
    \sum_{\kappa,\kappa'}^{}  
    (c_{\kappa,f}^* c_{\kappa',f})
    (c_{\kappa,\xi'}^* c_{\kappa',\xi'})
    (c_{\kappa',\xi}^* c_{\kappa,\xi})
    \chi((\kappa+\kappa')(\nu+\nu')) \\
    %&= \frac{1}{d^2} \sum_{\kappa,\kappa'}^{} 
    %\chi\left( \kappa f(\kappa+\kappa') \right) 
    %c_{\kappa+\kappa',f}
    %\chi\left( \kappa \xi'(\kappa+\kappa') \right) 
    %c_{\kappa+\kappa',\xi'}
    %\chi\left( \kappa \xi(\kappa+\kappa') \right) 
    %c_{\kappa+\kappa',\xi} \\
    %&\qquad\qquad\qquad
    %\chi\left( (\kappa+\kappa')(\nu+\nu') \right) \\
    &= \frac{1}{d^2} \sum_{\kappa,\kappa'}^{} 
    \chi\left( \kappa f(\kappa+\kappa') + 
      \kappa(\xi + \xi')(\kappa+\kappa')
    +(\kappa+\kappa') (\nu+\nu') \right)
    c_{\kappa+\kappa',f} c_{\kappa+\kappa',\xi'}
    c_{\kappa+\kappa',\xi}^* \\
    &= \frac{1}{d} \sum_{\kappa,\eta}^{} 
    \chi\left[
      \kappa(f(\eta) + (\xi+\xi')\eta)
    \right]
    \chi(\eta(\nu+\nu'))
    c_{\eta,f} c_{\eta,\xi'} c_{\eta,\xi}^* \\
    &= \frac{1}{d} \sum_{\eta}^{} 
    \chi\left( \eta(\nu+\nu') \right) 
    c_{\eta,f}c_{\eta,\xi'}c_{\eta,\xi}^* 
    \delta_{f(\eta), (\xi+\xi')\eta}.
  \end{align}

  I have computationally verified the change of variables,
  so now we will proceed by summing over the slopes of the
  lines that pass through the point in question. So we have:

  \begin{align}
    \sum_{\xi,\nu}^{} 
    \delta_{\beta,\xi\alpha+\nu}
    \left|
    \braket{\psi_{\xi,\nu}|\psi_{\xi',\nu'}^f}
    \right|^2
    &= \sum_{\xi,\nu}^{} 
    \delta_{\beta,\xi\alpha+\nu}
    \left(
      \frac{1}{d} \sum_{\eta}^{} 
      \chi\left( \eta(\nu+\nu') \right) 
      c_{\eta,f}c_{\eta,\xi'}c_{\eta,\xi}^* 
      \delta_{f(\eta), (\xi+\xi')\eta}
    \right) \\
    &= \frac{1}{d}
    \left( 
      \sum_{\xi,\eta}^{} 
      \chi\left( \eta(\beta + \xi\alpha+\nu') \right) 
      c_{\eta,f}c_{\eta,\xi'}c_{\eta,\xi}^* 
      \delta_{f(\eta), (\xi+\xi')\eta}
    \right). 
  \end{align}

  Notice that $\delta_{f(\eta), (\xi+\xi')\eta} = 1$ only
  when $\eta = 0$ or $\xi = \eta^{-1}f(\eta) + \xi'$. We
  split the sum in these to cases and sum over $\xi$ to
  obtain:
  \begin{align}
    d \sum_{\xi,\nu}^{} 
    \delta_{\beta,\xi\alpha+\nu}
    \left|
    \braket{\psi_{\xi,\nu}|\psi_{\xi',\nu'}^f}
    \right|^2
    &= \sum_{\xi}^{} 1
    + \sum_{\xi, \eta \in F^*}^{} 
    \chi\left( \eta(\beta + \xi\alpha+\nu') \right) 
    c_{\eta,f}c_{\eta,\xi'}c_{\eta,\xi}^* 
    \delta_{f(\eta), (\xi+\xi')\eta} \\
    &= d + \sum_{\eta \in F^*}^{} 
    \chi\left\{
      \eta\left[
        \beta + \left( \eta^{-1}f(\eta) + \xi' \right)
        \alpha + \nu'
      \right]
    \right\}
    c_{\eta,f} c_{\eta,\xi'}
    c_{\eta,\eta^{-1}f(\eta) + \xi'}^* \\
    &= d + \sum_{\eta \in F^*}^{} 
    \chi\left[
      \eta(\beta + \xi'\alpha + \nu') + f(\eta)\alpha
    \right] 
    c_{\eta,f} c_{\eta,\xi'}
    c_{\eta,\eta^{-1}f(\eta) + \xi'}^* \\
    &= d + \sum_{\eta \in F^*}^{} 
    \chi\left[
      \eta(\beta + \xi'\alpha + \nu' + f(\alpha))
    \right] 
    c_{\eta,f} c_{\eta,\xi'}
    c_{\eta,\eta^{-1}f(\eta) + \xi'}^*. 
  \end{align}

  Notice that if the rotation coefficients somehow manage to
  vanish, then we will be left with a delta function. So the
  question is now, \textit{if and how} can we remove these
  terms from the sum. Basically my claim is that the
  following is \textit{not} true:
  \begin{equation}
    c_{\eta,f} c_{\eta,\xi'}
    c_{\eta,\eta^{-1}f(\eta) + \xi'}^*
    = 1,
  \end{equation}
  for all $\eta \in F^*$ and $\xi' \in F$. Equivalently we
  are asking when the following recurrence relation is true:
  \begin{equation}
    c_{\eta,f} c_{\eta,\xi'}
    = 
    c_{\eta,\eta^{-1}f(\eta) + \xi'}.
  \end{equation}
  The first problem I see is that the rotation coefficients
  are calculated in a slightly different manner for the ray
  and curve operators. When substituting $\xi =
  \eta^{-1}f(\eta)+\xi'$ in the coefficient $c_{\eta,\xi}^*$ 
  we must remember that such coefficient is calculated using
  the function $g(\kappa) = \xi \kappa$, and so the new $g$
  is given by $g(\kappa) = \left( \eta^{-1}f(\eta)+\xi
  \right) \kappa$. This is just a thought but, recall that in
  the general formula of the coefficients we use the curve
  (ray or real curve) to calculate the coefficients for the
  basis. This information is always used in the calculation
  for arbitrary elements, and so simply replacing the
  coefficient indices doesn't change this information.
  
  Whatever the reason may be, the calculation does not
  verify computationally for the case of the curve
  $f(\alpha) = \alpha + \alpha^2 + \alpha^4$, which by one
  rotation $P_f$ produces the $(1,6,2)$ factorization MUBs,
  and so at least for this case we cannot reduce the
  expression to a delta function.

\end{document}

\documentclass[a4paper]{article}

\usepackage[margin=1.1in]{geometry}
\usepackage[utf8]{inputenc}
%\usepackage[T1]{fontenc}
%\usepackage{textcomp}
\usepackage[spanish]{babel}
\usepackage{amsmath, amssymb}
\usepackage{amsthm}
\usepackage{braket}
\decimalpoint

\DeclareMathOperator{\R}{\mathbb{R}}
\DeclareMathOperator{\C}{\mathbb{C}}
\DeclareMathOperator{\N}{\mathbb{N}}
\DeclareMathOperator{\Z}{\mathbb{Z}}
\DeclareMathOperator{\Tr}{Tr}
\DeclareMathOperator{\tr}{tr}

\title{Fixed phase space example}
\author{Ernesto Camacho Ramírez}
\begin{document}
  \maketitle

  To get us up to speed, we recall the method for
  calculating the rotation operator coefficients. We can
  solve (non-uniquely) the following recurrence relation:
  \begin{equation}
    c_{\alpha,\mu} c_{\kappa,\mu}
    = c_{\alpha+\kappa,\mu} \chi(\mu\kappa\alpha),
  \end{equation}
  for the basis elements. Having fixed a solution for the
  basis, we can decompose any element of the field as a
  linear combination of said basis elements and use the
  recurrence relation to obtain the general formula:
  \begin{equation}
    c_{\alpha,\mu}
    = c_{\sum_{j=1}^{} a_j \sigma_j,\mu}
    = \chi\left[
    \sum_{k=1}^{N-1} \left(
      a_k \sigma_k
      \sum_{j=k+1}^{N} \mu a_j \sigma_j
    \right) \right]
    \prod_{l=1}^N c_{a_l \sigma_l, \mu}.
  \end{equation}
  Using the rotation coefficients we can calculate the
  corresponding phase $\phi(\alpha,\beta)$ for any point in
  phase space. Finally we can assign a displacement operator
  $D(\alpha,\beta)$ at every point in the following manner:
  \begin{equation}
    D(\alpha,\beta)
    = \phi(\alpha,\beta) Z_\alpha X_\beta.
  \end{equation}

  We are trying to see is if for some fixed choice of discrete
  phase space given by the phases of the rays, there exists
  a curve (not included in the set of rays) such that the
  operators given by these points satisfy the equation
  \begin{equation}
    D(\alpha_1,f(\alpha_1)) D(\alpha_2,f(\alpha_2))
    = D(\alpha_1+\alpha_2, f(\alpha_1+\alpha_2)),
  \end{equation}
  or more generaly for a curve given by a set of points
  $\{(\alpha_i,\beta_i)\}$:
  \begin{equation}
    \label{eqn:prop}
    D(\alpha_1,\beta_1) D(\alpha_2,\beta_2)
    = D(\alpha_1+\alpha_2, \beta_1+\beta_2).
  \end{equation}
  
  By choosing the all positive basis solutions for each ray
  we obtain the following phase space (we fix the order the
  of the field by the powers of the generating element
  $\sigma$, where the rows are labeled by $\alpha$ and the
  columns by $\beta$):
  \begin{align}
    \Gamma
    &=
    \displaystyle \left(\begin{array}{rrrrrrrr}
    1 & 1 & 1 & 1 & 1 & 1 & 1 & 1 \\
    1 & -1 & -i & 1 & -i & i & i & -1 \\
    1 & -i & -1 & i & -i & i & 1 & -1 \\
    1 & 1 & i & i & i & 1 & 1 & i \\
    1 & -i & -i & i & -1 & 1 & i & -1 \\
    1 & i & i & 1 & 1 & i & 1 & i \\
    1 & i & 1 & 1 & i & 1 & i & i \\
    1 & -1 & -1 & i & -1 & i & i & -i
    \end{array}\right) \\
  \end{align}

  Now we consider equation (5.19) from the Annals article
  which shows the exceptional curve given the by the set of
  points
  \begin{equation}
    (0,0), (\sigma^4,0), (\sigma^4, \sigma^5), (\sigma^3,
    \sigma^7), (\sigma^3, \sigma^4), (\sigma^6, \sigma^4),
    (\sigma^6, \sigma^7), (0, \sigma^5).
  \end{equation}
  This curve can be obtained by the equation:
  \begin{equation}
    \beta^2 + \sigma^5 \beta
    = \sigma^6 \alpha + \sigma^2 \alpha^2.
  \end{equation}
  The amount of points that actually satisfy this equation
  is 16, I'm not sure exactly how we extract the points
  above, but I have verified that this curve (the 8 points
  given above) is closed under addition. \textit{For fixed
  phases, the operators indexed by this curve do satisfy the
property (\ref{eqn:prop})}.

  As an illustrative example (for sanity check) consider the
  points $(\sigma^6,\sigma^4)$ and $(\sigma^3,\sigma^7)$.
  Their phases according to $\Gamma$ are
  \begin{equation}
    \phi(\sigma^6,\sigma^4) = i
    \quad
    \text{and}
    \quad
    \phi(\sigma^3,\sigma^7) = i.
  \end{equation}
  The displacement operators are:
  \begin{equation}
    D(\sigma^6,\sigma^4) = 
    \displaystyle \left(\begin{array}{rrrrrrrr}
    0 & 0 & 0 & 0 & i & 0 & 0 & 0 \\
    0 & 0 & -i & 0 & 0 & 0 & 0 & 0 \\
    0 & i & 0 & 0 & 0 & 0 & 0 & 0 \\
    0 & 0 & 0 & 0 & 0 & 0 & i & 0 \\
    -i & 0 & 0 & 0 & 0 & 0 & 0 & 0 \\
    0 & 0 & 0 & 0 & 0 & 0 & 0 & i \\
    0 & 0 & 0 & -i & 0 & 0 & 0 & 0 \\
    0 & 0 & 0 & 0 & 0 & -i & 0 & 0
    \end{array}\right)
  \end{equation}
  And
  \begin{equation}
    D(\sigma^3,\sigma^7) = 
    \displaystyle \left(\begin{array}{rrrrrrrr}
    0 & 0 & 0 & 0 & 0 & 0 & 0 & i \\
    0 & 0 & 0 & i & 0 & 0 & 0 & 0 \\
    0 & 0 & 0 & 0 & 0 & 0 & -i & 0 \\
    0 & -i & 0 & 0 & 0 & 0 & 0 & 0 \\
    0 & 0 & 0 & 0 & 0 & -i & 0 & 0 \\
    0 & 0 & 0 & 0 & i & 0 & 0 & 0 \\
    0 & 0 & i & 0 & 0 & 0 & 0 & 0 \\
    -i & 0 & 0 & 0 & 0 & 0 & 0 & 0
    \end{array}\right)
  \end{equation}
  Their product is
  \begin{equation}
    D(\sigma^6,\sigma^4) D(\sigma^3,\sigma^7) =
    \displaystyle \left(\begin{array}{rrrrrrrr}
    0 & 0 & 0 & 0 & 0 & 1 & 0 & 0 \\
    0 & 0 & 0 & 0 & 0 & 0 & -1 & 0 \\
    0 & 0 & 0 & -1 & 0 & 0 & 0 & 0 \\
    0 & 0 & -1 & 0 & 0 & 0 & 0 & 0 \\
    0 & 0 & 0 & 0 & 0 & 0 & 0 & 1 \\
    1 & 0 & 0 & 0 & 0 & 0 & 0 & 0 \\
    0 & -1 & 0 & 0 & 0 & 0 & 0 & 0 \\
    0 & 0 & 0 & 0 & 1 & 0 & 0 & 0
    \end{array}\right)
  \end{equation}
  Which is equal to the operator:
  \begin{equation}
    D(\sigma^6+\sigma^3,\sigma^4 +\sigma^7) = 
    \displaystyle \left(\begin{array}{rrrrrrrr}
    0 & 0 & 0 & 0 & 0 & 1 & 0 & 0 \\
    0 & 0 & 0 & 0 & 0 & 0 & -1 & 0 \\
    0 & 0 & 0 & -1 & 0 & 0 & 0 & 0 \\
    0 & 0 & -1 & 0 & 0 & 0 & 0 & 0 \\
    0 & 0 & 0 & 0 & 0 & 0 & 0 & 1 \\
    1 & 0 & 0 & 0 & 0 & 0 & 0 & 0 \\
    0 & -1 & 0 & 0 & 0 & 0 & 0 & 0 \\
    0 & 0 & 0 & 0 & 1 & 0 & 0 & 0
    \end{array}\right)
  \end{equation}

  As an example that doesn't work we can consider the curve
  \begin{equation}
    \beta
    = \sigma^6 \alpha + \sigma^3 \alpha^2 + \sigma^5 \alpha^4.
  \end{equation}
  In particular the property (\ref{eqn:prop}) does not hold
  for the points $(\sigma^2,\sigma^4)$ and
  $(\sigma^3,\sigma^3)$ of this curve.


\end{document}

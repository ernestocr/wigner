\documentclass[a4paper]{article}

\usepackage[utf8]{inputenc}
\usepackage[T1]{fontenc}
\usepackage{textcomp}
\usepackage{amsmath, amssymb}
\usepackage{amsthm}
\usepackage{physics}

\DeclareMathOperator{\R}{\mathbb{R}}
\DeclareMathOperator{\C}{\mathbb{C}}
\DeclareMathOperator{\N}{\mathbb{N}}
\DeclareMathOperator{\Z}{\mathbb{Z}}

\title{Matriz de probabilidad}
\author{Ernesto Camacho Ramírez}
\begin{document}
  \maketitle

  Using the complete set of MUBs generated by the ray
  partition, we calculate the probabilities
  \begin{equation}
    p_\kappa^{(\lambda)}
    = \bra{\psi_{\kappa}^{\lambda}}
    \hat\rho
    \ket{\psi_{\kappa}^{\lambda}},
  \end{equation}
  corresponding to some state $\hat\rho$ and some eigenstate
  assigned to the ray parametrized by $\lambda$. In
  particular we calculate the matrix of probabilities (of
  size $9 \times 8$) for a state $\hat\rho_1$ corresponding
  to the abelian curve given by Ec.  (\ref{ac2}), $\beta =
  \alpha + \sigma^3 \alpha^2 + \sigma^{5} \alpha^{4}$ and
  for a state $\hat\rho_2$ of the non-abelian curve given by
  Ec.  (\ref{nac}) $\beta = \sigma^2 \alpha + \sigma^3
  \alpha^2 + \sigma^{5} \alpha^{4}$:

  \begin{equation}
    \left( p_\kappa^\lambda \right) =
    \displaystyle \left(\begin{array}{rrrrrrrr}
    \frac{1}{8} & \frac{1}{8} & \frac{1}{8} & \frac{1}{8} &
    \frac{1}{8} & \frac{1}{8} & \frac{1}{8} & \frac{1}{8} \\
    [6pt]
    \frac{1}{4} & 0 & \frac{1}{4} & 0 & 0 & 0 & \frac{1}{4}
                & \frac{1}{4} \\ [6pt]
    \frac{1}{8} & \frac{1}{8} & \frac{1}{8} & \frac{1}{8} &
    \frac{1}{8} & \frac{1}{8} & \frac{1}{8} & \frac{1}{8} \\
    [6pt]
    \frac{1}{8} & \frac{1}{8} & \frac{1}{8} & \frac{1}{8} &
    \frac{1}{8} & \frac{1}{8} & \frac{1}{8} & \frac{1}{8} \\
    [6pt]
    \frac{1}{4} & 0 & 0 & 0 & \frac{1}{4} & \frac{1}{4} & 0
                & \frac{1}{4} \\ [6pt]
    \frac{1}{4} & 0 & 0 & \frac{1}{4} & \frac{1}{4} & 0 &
    \frac{1}{4} & 0 \\ [6pt]
    \frac{1}{8} & \frac{1}{8} & \frac{1}{8} & \frac{1}{8} &
    \frac{1}{8} & \frac{1}{8} & \frac{1}{8} & \frac{1}{8} \\
    [6pt]
    \frac{1}{2} & \frac{1}{2} & 0 & 0 & 0 & 0 & 0 & 0 \\
    [6pt]
    \frac{1}{4} & 0 & \frac{1}{4} & \frac{1}{4} & 0 &
    \frac{1}{4} & 0 & 0
    \end{array}\right),
    \quad
    \text{for state } \hat\rho_1.
  \end{equation}

  \begin{equation}
    \left( p_{\kappa}^\lambda \right) =
    \displaystyle \left(\begin{array}{rrrrrrrr}
    \frac{1}{8} & \frac{1}{8} & \frac{1}{8} & \frac{1}{8} &
    \frac{1}{8} & \frac{1}{8} & \frac{1}{8} & \frac{1}{8} \\
    [6pt]
    \frac{1}{2} & \frac{1}{2} & 0 & 0 & 0 & 0 & 0 & 0 \\
    [6pt]
    \frac{1}{8} & \frac{1}{8} & \frac{1}{8} & \frac{1}{8} &
    \frac{1}{8} & \frac{1}{8} & \frac{1}{8} & \frac{1}{8} \\
    [6pt]
    \frac{1}{4} & 0 & \frac{1}{4} & \frac{1}{4} & 0 &
    \frac{1}{4} & 0 & 0 \\ [6pt]
    \frac{1}{4} & 0 & 0 & \frac{1}{4} & \frac{1}{4} & 0 &
    \frac{1}{4} & 0 \\ [6pt]
    0 & \frac{1}{4} & \frac{1}{4} & \frac{1}{4} & 0 & 0 &
    \frac{1}{4} & 0 \\ [6pt]
    \frac{1}{8} & \frac{1}{8} & \frac{1}{8} & \frac{1}{8} &
    \frac{1}{8} & \frac{1}{8} & \frac{1}{8} & \frac{1}{8} \\
    [6pt]
    0 & \frac{1}{4} & 0 & \frac{1}{4} & \frac{1}{4} &
    \frac{1}{4} & 0 & 0 \\ [6pt]
    \frac{1}{8} & \frac{1}{8} & \frac{1}{8} & \frac{1}{8} &
    \frac{1}{8} & \frac{1}{8} & \frac{1}{8} & \frac{1}{8}
    \end{array}\right),
    \quad
    \text{for state } \hat\rho_2,
  \end{equation}
  where $\lambda \in \mathbb F \cup \{\infty\}$ and $\kappa
  \in \mathbb F$ (the rows of the matrix are indexed by
  $\lambda$ and the columns by $\kappa$). Additionally, the
  reconstruction formula can be expressed as
  \begin{equation}
    \hat\rho
    = \sum_{\lambda}^{} \sum_{\kappa}^{} 
    p_\kappa^{(\lambda)} 
    \ket{\psi_\kappa^\lambda}\bra{\psi_\kappa^\lambda} -
    \hat I
    = \sum_{\lambda}^{} \sum_{\kappa}^{} 
    \left( p_\kappa^{(\lambda)} - \frac{1}{p^{n} + 1} \right) 
    \ket{\psi_\kappa^\lambda}\bra{\psi_\kappa^\lambda}.
  \end{equation}
  The matrices of the coefficients $p_\kappa^{(\lambda)} -
  (p^{n} + 1)^{-1}$ for both states are given below:

  \begin{equation}
    \displaystyle \left(\begin{array}{rrrrrrrr}
    \frac{1}{72} & \frac{1}{72} & \frac{1}{72} &
    \frac{1}{72} & \frac{1}{72} & \frac{1}{72} &
    \frac{1}{72} & \frac{1}{72} \\ [6pt]
    \frac{5}{36} & -\frac{1}{9} & \frac{5}{36} &
    -\frac{1}{9} & -\frac{1}{9} & -\frac{1}{9} &
    \frac{5}{36} & \frac{5}{36} \\ [6pt]
    \frac{1}{72} & \frac{1}{72} & \frac{1}{72} &
    \frac{1}{72} & \frac{1}{72} & \frac{1}{72} &
    \frac{1}{72} & \frac{1}{72} \\ [6pt]
    \frac{1}{72} & \frac{1}{72} & \frac{1}{72} &
    \frac{1}{72} & \frac{1}{72} & \frac{1}{72} &
    \frac{1}{72} & \frac{1}{72} \\ [6pt]
    \frac{5}{36} & -\frac{1}{9} & -\frac{1}{9} &
    -\frac{1}{9} & \frac{5}{36} & \frac{5}{36} &
    -\frac{1}{9} & \frac{5}{36} \\ [6pt]
    \frac{5}{36} & -\frac{1}{9} & -\frac{1}{9} &
    \frac{5}{36} & \frac{5}{36} & -\frac{1}{9} &
    \frac{5}{36} & -\frac{1}{9} \\ [6pt]
    \frac{1}{72} & \frac{1}{72} & \frac{1}{72} &
    \frac{1}{72} & \frac{1}{72} & \frac{1}{72} &
    \frac{1}{72} & \frac{1}{72} \\ [6pt]
    \frac{7}{18} & \frac{7}{18} & -\frac{1}{9} &
    -\frac{1}{9} & -\frac{1}{9} & -\frac{1}{9} &
    -\frac{1}{9} & -\frac{1}{9} \\ [6pt]
    \frac{5}{36} & -\frac{1}{9} & \frac{5}{36} &
    \frac{5}{36} & -\frac{1}{9} & \frac{5}{36} &
    -\frac{1}{9} & -\frac{1}{9}
    \end{array}\right),
    \quad
    \text{for state } \hat\rho_1,
  \end{equation}
  \begin{equation}
    \displaystyle \left(\begin{array}{rrrrrrrr}
    \frac{1}{72} & \frac{1}{72} & \frac{1}{72} &
    \frac{1}{72} & \frac{1}{72} & \frac{1}{72} &
    \frac{1}{72} & \frac{1}{72} \\ [6pt]
    \frac{7}{18} & \frac{7}{18} & -\frac{1}{9} &
    -\frac{1}{9} & -\frac{1}{9} & -\frac{1}{9} &
    -\frac{1}{9} & -\frac{1}{9} \\ [6pt]
    \frac{1}{72} & \frac{1}{72} & \frac{1}{72} &
    \frac{1}{72} & \frac{1}{72} & \frac{1}{72} &
    \frac{1}{72} & \frac{1}{72} \\ [6pt]
    \frac{5}{36} & -\frac{1}{9} & \frac{5}{36} &
    \frac{5}{36} & -\frac{1}{9} & \frac{5}{36} &
    -\frac{1}{9} & -\frac{1}{9} \\ [6pt]
    \frac{5}{36} & -\frac{1}{9} & -\frac{1}{9} &
    \frac{5}{36} & \frac{5}{36} & -\frac{1}{9} &
    \frac{5}{36} & -\frac{1}{9} \\ [6pt]
    -\frac{1}{9} & \frac{5}{36} & \frac{5}{36} &
    \frac{5}{36} & -\frac{1}{9} & -\frac{1}{9} &
    \frac{5}{36} & -\frac{1}{9} \\ [6pt]
    \frac{1}{72} & \frac{1}{72} & \frac{1}{72} &
    \frac{1}{72} & \frac{1}{72} & \frac{1}{72} &
    \frac{1}{72} & \frac{1}{72} \\ [6pt]
    -\frac{1}{9} & \frac{5}{36} & -\frac{1}{9} &
    \frac{5}{36} & \frac{5}{36} & \frac{5}{36} &
    -\frac{1}{9} & -\frac{1}{9} \\ [6pt]
    \frac{1}{72} & \frac{1}{72} & \frac{1}{72} &
    \frac{1}{72} & \frac{1}{72} & \frac{1}{72} &
    \frac{1}{72} & \frac{1}{72}
    \end{array}\right),
    \quad
    \text{for state } \hat\rho_2.
  \end{equation}
  The states $\hat\rho_1$ and $\hat\rho_2$ can't be
  expressed as a convex sum of other states as they are both
  pure states, and in particular they can't be expressed as
  a convex sum of the eigenstates assigned to the ray
  partition.
  
\end{document}

\documentclass[a4paper,11pt]{report}
\usepackage[margin=1.3in]{geometry}

\usepackage[utf8]{inputenc}
\usepackage[T1]{fontenc}
\usepackage{textcomp}
%\usepackage[spanish]{babel}
\usepackage[spanish,es-noshorthands]{babel}
\decimalpoint

%\usepackage{csquotes}
%\usepackage[sorting=none]{biblatex}
%\addbibresource{refs.biblatex}

\usepackage{amsmath, amssymb}
\usepackage{amsthm}
\usepackage{bm}
\usepackage{braket}

\usepackage{graphicx}
\usepackage{subfig}

\DeclareMathOperator{\R}{\mathbb{R}}
\DeclareMathOperator{\Q}{\mathbb{Q}}
\DeclareMathOperator{\C}{\mathbb{C}}
\DeclareMathOperator{\N}{\mathbb{N}}
\DeclareMathOperator{\Z}{\mathbb{Z}}
\DeclareMathOperator{\F}{\mathbb{F}}
\DeclareMathOperator{\E}{\mathbb{E}}

\let\H\relax
\DeclareMathOperator{\H}{\mathcal H}
\DeclareMathOperator{\Sz}{\mathcal S}

\DeclareMathOperator{\dom}{Dom}
\DeclareMathOperator{\prob}{Prob}
\DeclareMathOperator{\id}{id}
\DeclareMathOperator{\Tr}{Tr}
\DeclareMathOperator{\tr}{tr}
\DeclareMathOperator{\Op}{Op}
\DeclareMathOperator{\W}{W}
\DeclareMathOperator{\Fr}{\mathcal{F}\!}
\DeclareMathOperator{\GF}{GF}
\DeclareMathOperator{\GR}{GR}
\DeclareMathOperator{\Sl}{\mathfrak{sl}}
\DeclareMathOperator{\Gal}{Gal}
\DeclareMathOperator{\img}{img}

\newtheorem{definition}{Definición}
\newtheorem{theorem}{Teorema}
\newtheorem{proposition}{Proposición}
\newtheorem{lemma}{Lema}
\newtheorem{corollary}{Corolario}
\newtheorem{example}{Ejemplo}
\newtheorem{axiom}{Postulado}
\newtheorem{remark}{Observación}

\title{Garcia, Klimov 2010 - Qubit MUBs generation}
\author{Ernesto}

\begin{document}
    \maketitle

    Summary of Garcia's and Klimov's \textit{Generation of
    bases with definite factorization for an $n$-qubit
    system and mutually unbiased sets construction} written
    in 2010.

    The intention of the paper is to classify the bases for
    $n$-qubit systems for its importance to optimal state
    and process tomography and for the quantification of
    entanglement in a multipartite state.

    They focus on a subset of all possible bases which can
    be constructed from a $2^{n}$-dimensional Hilbert space,
    namely, those obtained from the standard logical basis
    by applying only local Clifford transformations and CNOT
    gates.

    \begin{definition}
        A Clifford gate is a quantum gate that belongs the
        Clifford Group, which is a set of transformations
        which normalize the $n$-qubit Pauli group. Quantum
        circuits that consist of only Clifford gates can be
        efficiently simulated with a classical computer
        (Gottesman-Knill theorem).
    \end{definition}

    Such bases have a definite factorization structure. Each
    basis vetor is divided into the same number of $K$ 
    blocks, $1 \leq K \leq n$, and each block contains $m_K$ 
    completely non-separable states of the same qubits
    (i.e., $\sum_{K}^{} m_K = n$). For $K = n$ we have a
    completely factorizable basis (each element is a tensor
    product of $n$ qubits) and for $K = 1$ the basis is
    non-separable.

    I don't like the following. They state that a
    \textit{general} method for the construction of such
    bases can be formulated in the framework of the
    so-called stabilizer formalism. This is obvious by
    construction, but they say that the bases with
    \textit{definite factorization} (meaning what?) in the
    $2^{n}$-dimensional Hilbert space can be constructed as
    eigenstates of $n$ commuting monomials formed by
    products of $n$ Pauli operators and generators of an
    Abelian group of order $2^{n}$. Where is the direct
    relationship with the factorization?

    They then state that it is clear that the non-separable
    blocks can be considered as common eigenstates of a set
    of commuting monomials each containing a product of
    $m_k$ Pauli operators. This means that the non-separable
    blocks are locally isomorphic to some $m_K$-qubit graph
    state. Graph states are multiparticle entangle states
    that correspond to mathematical graphs, where the
    vertices take the role of quantum spin systems and the
    edges represent Ising interactions.

    A drawback is that the stabilizer formalism does not
    provided a systematic method for \textit{defining} sets
    of such commuting monomials. The authors confer to an
    elegant way for small dimensions using a graph-state
    approach. This method is limited for higher dimensions.

    Then out of the blue, the authors state that the problem
    of bases generation is related to the classification of
    elements of the symplectic group $Sp_{2n}(\Z_2)$. The
    main difficult of this approach consists of factoring
    out the elements of $Sp_{2n}(\Z_2)$ which lead to the
    same bases. The order of this group grows rapidly with
    $n$,

    \begin{equation}
        \left|
        Sp_{2n}(\Z_2)
        = 2^{n^2} \prod_{k=1}^{n} \left( 2^{2k} - 1 \right) 
        \right|.
    \end{equation}

    As an example, the symplectic group corresponding to
    three qubits has order $|Sp_{6}(\Z_2)| = 1,451,520$, but
    there are only 135 \textit{adequate} elements, which are
    the different bases for three qubits.

    Some of the just mentioned bases are mutually unbiased.
    For a $d$-dimensional system it has been found that the
    maximum number of MUBs cannot be greater than $d+1$ and
    this limit is reached if $d = p$ is prime or a power of
    a prime $d = p^{n}$.

    An interesting phenomena related to MUBs in prime power
    dimension is the existence of locally non-isomorphic
    sets of such bases, which are bases with different
    separability structures. The authors plan to show that
    among all possible bases, there are
    $\prod_{k=1}^{n}\left( 2^{k}+1 \right)$ unbiased sets of
    $2^{n}+1$. Moreover, these bases cannot be transformed
    into each other by local transformations. This is
    combinatorial problem.

    The approach of the authors consists of a systematic
    procedure to determine symplectic matrices which
    generate all possible different sets of commuting
    monomials in a non-redundant way, and thus all possible
    bases for an $n$-qubit system.

    \section{Commutativity condition and symplectic group}

    We label the states in the Hilbert space of an $n$-qubit
    system by elements of the finite field $\GF(2^{n})$.
    Denote $\ket \alpha$, $\alpha \in \GF(2^{n})$, an
    orthnormal basis in the Hilbert space, i.e.,
    $\braket{\beta|\alpha} = \delta_{\beta\alpha}$.

    The analog of an orthogonal basis is the so-called
    self-dual basis, satisfying the condition $\tr(\theta_i
    \theta_j) = \delta_{ij}$, so that $a_i = \tr(\alpha
    \theta_i)$ where $\tr : \GF(2^{n}) \to \Z_2$ is the
    field trace. Using this decomposition we can establish a
    correspondence between the Hilbert space $\C^{2^{n}}$ 
    and the tensor product space $\C^2 \otimes \cdots
    \otimes \C^2$:
    \begin{equation}
        \ket \alpha
        \mapsto \ket{a_1} \otimes \cdots \otimes \ket{a_n}
        = \ket{a_1,\ldots,a_n}.
    \end{equation}

    Now we consider the generators of the generalized Pauli
    groups:
    \begin{equation}
        Z_{\beta}
        = \sum_{\alpha \in \GF(2^{n})}^{}
        (-1)^{\tr(\alpha\beta)} \ket \alpha \bra \alpha,
        \quad
        X_{\beta}
        = \sum_{\alpha \in \GF(2^{n})}^{} 
        \ket{\alpha + \beta} \bra{\alpha}.
    \end{equation}
    
    The basis vectors $\ket \alpha$ are eigenstates of the
    operators $Z_\beta$. These operators satisfy the
    important commutation relation:
    
    \begin{equation}
        Z_\alpha X_\beta
        = (-1)^{\tr(\alpha\beta)} X_\beta Z_\alpha.
    \end{equation}
    
    Using the same correspondence between the Hilbert space
    and the tensor product space, we can factorize the
    operators into tensor products of the single-particle
    Pauli operators $\sigma_z$ and $\sigma_x$:

    \begin{equation}
        Z_\alpha
        = \sigma_{z}^{a_1} \otimes \cdots \otimes
        \sigma_z^{a_n},
        \quad
        X_\beta
        = \sigma_x^{b_1} \otimes \cdots \otimes
        \sigma_x^{b_n},
    \end{equation}
    
    where of course $a_j$ and $b_j$ are the expansion
    coefficients of $\alpha$ and $\beta$ using the self-dual
    basis. The simplest set of commuting operators is
    $\{Z_\alpha, \alpha \in \GF(2^{n})\}$ that determines
    the standard logical (completely factorizable) basis
    $\ket \alpha$. Now consider a set of commuting
    monomials

    \begin{equation}
        Y
        = \{Y_i = Z_{\alpha_i}X_{\beta_i} : i =
        1,\ldots,2^{n}\}.
    \end{equation}

    Since any monomial can be obtained from the corrsponding
    $Z_{\alpha_i}$ by local Clifford transformations and
    CNOT operations, $Y_i = U Z_{\alpha_i} U^{*}$, the basis
    $\{U\ket\alpha : \alpha \in \GF(2^{n})\}$ formed by the
    eigenstates of $Y_i$ has a definite factorization
    structure, this means that all the elements of this
    basis are factorized in the same manner. But how trivial
    is it show that in fact any monomial can be obtained
    from a $Z_{\alpha_i}$? And what is meant by
    \textit{local} Clifford operations? I assume it means
    that we take tensor products of Clifford operations and
    act upon the $Z_{\alpha_i}$ operator.

    \begin{proof}
        Let $Y_i = Z_{\alpha_i}X_{\beta_i}$ be an arbitrary
        monomial. Suppose there exists $U$ such that $Y_i =
        U Z_{\alpha_i} U^{*}$, then
        \begin{align*}
            U &= Y_i U Z_{\alpha_i}^{-1} \\
              &= Z_{\alpha_i} X_{\beta_i} U
              Z_{\alpha_i}^{-1} \\
              &= ?
        \end{align*}

        A more direct argument. Since the Clifford gates
        normalize the Pauli group, then $U Z_{\alpha_i}
        U^{*}$ is element of the Pauli group. How to show
        that it is of the form $Z_{\alpha_i} X_{\beta_i}$?
    \end{proof}

    It is not difficult to see that each set $Y$ is
    isomorphic to an Abelian group of order $2^{n}$ with the
    structure $\Z_2 \otimes \Z_2 \otimes \cdots \otimes
    \Z_2$. This means that only $n$ different (non-trivial)
    operators are required to generate the wholse set. First
    of all what is this structure? It is abelian with
    respect to what operation? 

    Ignore these questions for now. The commutator of two
    monomials $Z_{\alpha_1}X_{\beta_1}$ and
    $Z_{\alpha_2}X_{\beta_2}$ is given by

    \begin{equation}
        [
        Z_{\alpha_1}X_{\beta_1}, Z_{\alpha_2}X_{\beta_2}
        ]
        = \left\{
            (-1)^{\tr(\alpha_2\beta_1)}
            - (-1)^{\tr(\alpha_1\beta_2)}
        \right\} Z_{\alpha_2+\alpha_1} X_{\beta_2+\beta_1}.
    \end{equation}

    This takes us to the commutativity condition between
    them to be $\tr(\alpha_1\beta_2 + \beta_1\alpha_2) = 0$.
    In terms of the self-dual basis, this condition takes
    and explicit symplectic form:

    \begin{equation}
        \sum_{i=1}^{n} 
        \left( a_i^{1}b_i^{2} + b_i^{1}a_i^{2} \right) = 0.
    \end{equation}

    Such a condition can be re-written in vector notation,
    like it is done in the stabilizer formalism:

    \begin{equation}
        \left(C_{b^2}^{a^2}\right)^{T}
        J
        C^{a^{1}}_{b^{1}} = 0,
    \end{equation}
    
    where $C_{b}$ is a $2n$-dimensional vector

    \begin{equation}
        \left(C_{b^2}^{a^2}\right)^{T}
        = \begin{bmatrix}
            a_n^k & \cdots & a_1^k & b_1^k & \cdots & b_n^k
        \end{bmatrix},
        \quad k = 1, 2.
    \end{equation}

    This in turn determines the corresponding monomial

    \begin{equation}
        C_b^a \mapsto Z_\alpha X_\beta,
        \quad
        \alpha = \sum_{i=1}^{n} a_i \theta_i,
        \quad
        \beta = \sum_{i=1}^{n} b_i \theta_i,
        \quad
        a_i,b_i \in \Z_2.
    \end{equation}

    Of course $J$ is the standard $2n \times 2n$ symplectic
    matrix

    \begin{equation}
        \begin{bmatrix}
            0_n & \Xi_n \\
            \Xi_n & 0_n,
        \end{bmatrix},
        \quad
        \Xi_n = \begin{bmatrix}
            0 & \cdots & 1 \\
            \vdots & \ddots & \vdots \\
            1 & \cdots & 0
        \end{bmatrix}. 
    \end{equation}

    We can focus in particular on the vectors $C_0^{a}$,
    corresponding to the set $Z_\alpha$. These operators
    trivially satisfy nullity condition. And then comes the
    unbacked statement (is it trivial?): any other
    commutative set $Y' = \{Z_{\alpha'}X_{\beta'}\}$ 
    corresponds to vectors

    \begin{equation}
        C_{b'}^{a'} = P C_0^{a},
    \end{equation}

    where $P$ is a symplectic matrix over $\Z_2$ satisfying
    $P^{T} J P = J$ and $a$ is a binary $n$-tuple. Of course
    it forms a commutative set since $P$ preserves the
    symplectic form, but how do you conclude that
    \textit{all other} sets are obtained in this manner?

    Therefore each $P \in Sp_{2n}(\Z_2)$ generates a set of
    commuting operators starting with the standard set
    $\{Z_\alpha\}$. Not all different symplectic matrices
    produce different commutative sets, and the number of
    commutative sets is essentially less then the order of
    the symplectic group. Thus the main problem one is faced
    with in this method is to find only those matrices that
    generate \textit{different} sets of commuting operators.

    This is where things get interesting. Since the
    symplectic operators are applied to the very particular
    $C_0^a$, whose $n$ last components are zeros, a
    convenient form to disregard partially the symplectic
    matrices that lead to the same commutative sets consists
    of using the Bruhat decomposition of $P \in
    Sp_{2n}(\Z_2)$ into right cosets:

    \begin{equation}
        P = a_\pi b_C^{+},
    \end{equation}

    where $b_C^{+} \in \mathbb B_C^{+}$, $\mathbb B_C^{+}$ 
    is the Borel subgroup (a group of upper triangular
    symplectic matrices on $\Z_2$), $a_\pi \in A_\pi$, and
    each coset $A_\pi$ is constructed inserting some free
    parameters $x_k \in \Z_2$ into the matrix realization
    $[\pi]$ of the Weyl group $W_n$ of permutations $\pi$: a
    group of permutation of $2n$ objects, such that if these
    objects are labeled by natural numbers $1,2,\ldots,2n$,
    it has that $\pi(i) + \pi(2n+1-i) = 2n+1$, $i =
    1,\ldots,n$. 


    
\end{document}

En lo que sigue nos enfocamos a una partícula en una sola
  dimensión, pero se puede extender a varias dimensiones sin
  mucho problema. Sea $A$ un observable y $\rho$ un estado
  de un sistema cuántico, entonces siguiendo la idea
  anterior, nos gustaría poder hacer el siguiente cálculo
  \begin{equation}
    \Tr\left(\rho \hat{A}\right)
    = \iint A(x,p)F(x,p) \, dx \, dp,
  \end{equation}
  donde $F$ es una densidad probabilística en el espacio de
  fase relacionada al estado $\rho$. Evidentemente lo
  primero que necesitamos es una correspondencia entre
  operadores en el espacio de Hilbert y funciones
  integrables en el espacio de fase. Pero inmediatamente nos
  confrontamos con un problema de asignación, debido a la no
  conmutatividad de los operadores de posición y de
  momentum. En partícular consideremos el caso en que
  nuestro operador es la exponencial $e^{i(\xi \hat{x} +
  \eta \hat{p})}$ donde $\xi, \eta \in \R$.  Podríamos
  reemplazar los operadores $\hat{x}$ y $\hat{p}$ de manera
  formal para obtener
  \begin{equation}
    \Tr\left( \rho e^{i\left( \xi \hat{x} + \eta
    \hat{p} \right)} \right)
    = \iint e^{i\left( \xi x + \eta p \right)} F(x,p) \,
    dx \, dp
  \end{equation}
  con una densidad particular $F(x,p)$. Ahora consideremos
  el operador $e^{i\xi \hat{x}} e^{i\eta\hat{p}}$, siguiendo
  el mismo razonamiento obtenemos
  \begin{equation}
    \Tr\left(\rho e^{i\xi \hat{x}} e^{i\eta
    \hat{p}}\right)
    = \iint e^{i\xi x} e^{\eta p} \tilde F(x,p) \, dx \,
    dp
    = \iint e^{i\left( \xi x + \eta p \right)} \tilde
    F(x,p) \, dx \, dp.
  \end{equation}
  Notemos que la función en el espacio de fase que aparece
  en el integrando es la misma para ambos operadores. Pero
  utilizando la identidad Baker-Campbell-Hausdorf sabemos
  que
  \begin{equation}
    e^{i\left( \xi \hat{x} + \eta \hat{p} \right)}
    = e^{i\xi \hat{x}}e^{i\eta \hat{p}}e^{-\frac{1}{2}i\hbar
    \xi \eta}
    \neq e^{i\xi \hat{x}}e^{i\eta \hat{p}}.
  \end{equation}
  Así que en general las distribuciones $F$ y $\tilde F$
  deben ser distintas si no obtendremos los mismos valores
  esperados para distintos operadores en un mismo estado.
  Distintas reglas asociando funciones de operadores no
  conmutativos con sus correspondientes funciones escalares
  nos darán distintas distribuciones
  \cite{leeTheoryApplicationQuantum1995}. En el caso del
  ordenamiento de Weyl, que introduciremos más adelante, se
  puede probar que 
  \[
    e^{\xi x + \eta p} \leftrightarrow e^{\xi \hat{x} + \eta
    \hat{p}},
  \] 
  por lo tanto eligiendo ésta asignación podemos
  \textit{definir} la densidad $F$ mediante la ecuación
  \begin{equation}
    \Tr\left(
      \rho e^{i(\xi \hat{x} + \eta \hat{p})}
    \right)
    = \iint e^{i(\xi x + \eta p)}F(x,p) \, dx \, dp.
  \end{equation}
  Para obtener una expresión de $F$, utilizamos la
  transformación de Fourier y escribimos de manera formal
  \begin{equation}
    \label{eqn:density_from_trace}
    F(x,p)
    = \frac{1}{(2\pi)^2} \iint \Tr\left( \rho 
      e^{i\left( \xi \hat{x} + \eta \hat{p} \right)} \right)
      e^{-i\left( \xi x + \eta p\right)} \, d\xi \, d\eta.
  \end{equation}

  Enseguida buscamos expresar el integrando sin la traza y
  sobre todo sin los operadores $\hat{x}$ y $\hat{p}$. Para
  ésto hacemos uso de la notación de Dirac y expresamos la
  traza mediante una ``expansión'' de los operadores
  $\hat{A}$ y $\rho$ en términos de la posición y luego
  aplicamos el operador exponencial sobre los vectores $\ket
  x$. Primero
  \begin{equation}
    \Tr\left( \rho \hat{A} \right) 
    = \iint \braket{x|\rho|x'} \braket{x'|\hat{A}|x} \, dx
    \, dx'.
  \end{equation}
  Luego, el operador $e^{i\eta \hat{p}}$ actúa sobre $\ket
  x$ como una traslación en el espacio de posición, es decir
  \[
    e^{i\eta \hat{p}} \ket x
    = \ket{x - \eta \hbar},
  \] 
  y el operador $e^{i\xi \hat{x}}$ actúa sobre un $\ket x$ 
  simplemente como una multiplicación por un factor de fase,
  \[
    e^{i\xi \hat{x}} \ket x
    = e^{i\xi x} \ket x.
  \] 
  Finalmente recordando que $e^{i(\xi \hat{x} + \eta
  \hat{p})} = e^{i \xi \hat{x}} e^{i \eta \hat{p}}
  e^{-\frac{1}{2}i\hbar \xi \eta}$, obtenemos
  \begin{align*}
    \Tr\left( \rho e^{i(\xi \hat{x} + \eta \hat{p})} \right) 
    &= \iint \braket{x|\rho|x'} \braket{x'| e^{i(\xi \hat{x}
    + \eta \hat{p})}|x} \, dx \, dx' \\
    &= \iint \braket{x|\rho|x'} \braket{x'|e^{i\eta
    \hat{p}}|x} e^{i\xi x}e^{-\frac{1}{2} i \hbar \xi \eta}
    \, dx \, dx' \\
    &=  \iint \braket{x|\rho|x'} \braket{x'|x - \eta \hbar}
    e^{i\xi x}e^{-\frac{1}{2} i \hbar \xi \eta} \, dx \, dx'
    \\
    &= \int \braket{x|\rho|x - \eta \hbar} e^{i\xi (x
    -\frac{1}{2} \hbar \eta)} \, dx.
  \end{align*}

  Sustituyendo ésta expresión de la traza en el integrando
  de (\ref{eqn:density_from_trace}), y haciendo el cambio de
  variable $x'' = x' - \frac{1}{2}\eta\hbar$, obtenemos una
  expresión que ya no incluye a los operadores $\hat{x}$ y
  $\hat{p}$:
  \begin{align*}
    F(x,p)
    &= \frac{1}{(2\pi)^2} \iiint \braket{x'|\rho|x' -
    \eta\hbar} e^{i\xi(x' - \frac{1}{2}\hbar \eta)}
    e^{-i(\xi x + \eta p)} \, d\xi \, d\eta \, dx'\\
    &= \frac{1}{(2\pi)^2}\iiint \braket{x'|\rho|x'-\eta
    \hbar} e^{i\xi(x' - \frac{1}{2}\hbar\eta - x)} e^{-i\eta
    p} \, d\xi \, d\eta \, dx' \\
    &= \frac{1}{(2\pi)^2}\iiint \braket{x'' + \tfrac{1}{2}
    \eta \hbar|\rho|x'' - \tfrac{1}{2} \eta \hbar}
    e^{i\xi(x'' - x)} e^{-i\eta p} \, d\xi \, d\eta \, dx''.
  \end{align*}

  Utilizando la representación integral de la función delta
  de Dirac (respecto a $\xi$) y luego integrando sobre $x'$,
  las cosas se simplifican aún más:
  \begin{align}
    F(x,p)
    &= \frac{1}{2\pi} \iint \braket{
    x'+\tfrac{1}{2}\eta\hbar | \rho|
    x'-\tfrac{1}{2}\eta\hbar} \delta(x'-x) e^{-i\eta p} \,
    d\eta \, dx' \\
    &= \frac{1}{2 \pi} \int \braket{x +
    \tfrac{1}{2}\eta\hbar | \rho | x - \tfrac{1}{2}
    \eta\hbar}e^{-i \eta p} \, d\eta.
  \end{align}
  Haciendo un cambio de variable más, $y = -\eta \hbar$, por
  fin llegamos a lo que definimos como la transformación de
  Wigner de un estado arbitrario:
  \begin{equation}
    W(x,p)
    := \frac{1}{2\pi\hbar} \int \braket{x -
      \tfrac{1}{2}y|\rho|x+\tfrac{1}{2}y} e^{\frac{i}{\hbar}
    y p} \, dy.
  \end{equation}
  La función $W$ es integrable sobre el espacio de fase
  $\R^{2n}$ y contiene la misma información que el estado
  $\rho$. En el caso en que el estado $\rho = \ket \psi \bra
  \psi$ es puro, inmediatamente llegamos a la expresión de
  Wigner (\ref{eqn:wigners_original}) utilizando la
  representación de la función de onda. Formalmente se puede
  extender ésta idea para transformar operadores que no son
  operadores de densidad, aunque justificar ésto de manera
  rigurosa no es tan trivial. En éste caso la transformación
  de Wigner de un operador lineal $\hat{A}$ es
  \begin{equation}
    A(x,p)
    = \frac{1}{2 \pi \hbar} \int
    \braket{x-\tfrac{1}{2}y|\hat{A}|x+\tfrac{1}{2}y}
    e^{\frac{i}{\hbar}p y} \, dy.
  \end{equation}

  Integrando el producto de las funciones de Wigner de un
  estado $\rho$ y de un observable $\hat{A}$ sobre el
  espacio de fase se puede demostrar fácilmente que
  efectivamente hemos calculado el valor esperado del
  observable, cumpliendo con nuestro objetivo inicial, es
  decir,
  \[
    \Tr\left( \rho\hat{A} \right) 
    = \iint A(x,p)W(x,p) \, dx \, dp.
  \] 

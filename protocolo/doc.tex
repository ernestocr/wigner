\documentclass[a4paper]{article}

\usepackage[margin=1.4in]{geometry}
\usepackage[utf8]{inputenc}
\usepackage[T1]{fontenc}
\usepackage{csquotes}
\usepackage{biblatex}
\addbibresource{refs.biblatex}
\usepackage[spanish]{babel}
\usepackage{amsmath, amssymb}
\usepackage{amsthm}
\decimalpoint
\usepackage{graphicx}

\usepackage{booktabs}

\DeclareMathOperator{\R}{\mathbb{R}}
\DeclareMathOperator{\C}{\mathbb{C}}
\DeclareMathOperator{\N}{\mathbb{N}}

\title{
  \Large Protocolo de tésis\\[1in]
  \huge Funciones de Wigner en el espacio de
  fase discreto\\[1in]
  \large \textbf{Autor}: Ernesto Camacho Ramírez\\
  \textbf{Director de tésis}: Dr. Andrés García Sandoval
}
\date{12 de abril de 2023}
\author{}
\begin{document}

  \begin{titlepage}
    \begin{center}
      \vspace{1.0in}

      \huge
      Universidad de Guadalajara\\
      \vspace{0.2cm}
      \large
      Centro Universitario de Ciencias Exactas e Ingenerías\\
      Licenciatura en Matemáticas\\

      \vspace{0.5in}

      \includegraphics[width=0.3\textwidth]{../thesis/imgs/udg}
      
      \vspace{0.5in}

      \huge
      \textbf{
        Funciones de Wigner en el Espacio de Fase Discreto
      }

      \large
      \vspace{1.0in}
      
      Autor: Ernesto Camacho Ramírez \\
      \vspace{0.2cm}
      Director de tésis: Dr. Andrés García Sandoval \\

      \vfill
           
      \Large Protocolo de Tésis
     \end{center}
  \end{titlepage}


  \newpage
  \section{Introducción}

  El número de elementos de un conjunto general de
  operadores cuánticos unitarios sobre estados de $n$-qudit
  generalmente crece exponencialmente con $n$. Una excepción
  importante a ésta regla involucra el conjunto de
  operadores de Clifford, que actúan sobre estados
  estabilizadores. Los operadores de Clifford son los
  operadores que normalizan el grupo de Pauli, y los
  estados estabilizadores son los eigenestados simultáneos
  de un subconjunto maximal de operadores de Pauli
  conmutativos. Éstos estados juegan un papel importante
  en la corrección de errores cuánticos \cite{gottesman1998}
  y son cerrados bajo la acción de las compuertas de
  Clifford. La simulación eficiente de dichos sistemas con
  una computadora clásica se demostró con el algoritmo
  tableau de Aaronson y Gottesman \cite{aaronson2004,
  gottesman1998} para qubits ($d=2$). La busqueda de un
  explicación de por qué un algoritmo tan eficiente es
  posible para la simulación de circuitos de Clifford ha
  sido un objeto de mucho estudio \cite{gottesman1999,
  howard2014, mari2012}. El progreso reciente ha sido
  resultado del trabajo de Wooters \cite{wootters1987},
  Eisert \cite{mari2012}, Gross \cite{gross2006} y Emerson
  \cite{howard2014}, quienes han formulado una nueva
  perspectiva basada en los espacios de fase discretos de
  estados y operadores en espacios de Hilbert finitos,
  utilizando funciones discretas de Wigner.  La función de
  Wigner en el caso continuo surge de vincular la mecánica
  cuántica con la mecánica clásica estadística, mediante el
  espacio de fase. Las funciones de Wigner nos dan una
  representación alternativa de estados de sistemas
  cuánticos, por medio de \textit{cuasi-distribuciones}
  sobre el espacio de fase. Éstas funciones no son
  distribuciones probabílisticas verdaderas ya que pueden
  tomar valores negativos, y ésta negatividad ha sido
  vinculada con la no-localidad. La generalización a
  sistemas discretos a tomado distintas formas a traves de
  los años, cada una con ventajas y desventajas. Para una
  clase particular de funciones de Wigner discretas, se ha
  demostrado que los estados estabilizadores son análogos
  discretos a los estados Gaussianos en sistemas continuos
  \cite{gross2006}, en el sentido de que tienen funciones de
  Wigner no-negativas. Por otro lado se ha demostrado
  que los operadores de Clifford son mapeos definidos
  positivos, ésto implica que los circuitos de Clifford son
  simulables eficientemente en computadores clásicas.

  En este trabajo, buscamos construir de una manera
  explícita, a los operadores \textit{puntuales de fase}
  discretos (núcleos de la transformación de Wigner) para
  qubits y qutrits. En particular, desarrollamos la
  construcción \textit{estándar} siguiendo la metodología de
  Gibbons y Wootters \cite{wootters1987}, y como
  contribución del trabajo, buscamos una construcción
  \textit{no estándar} de los núcleos por medio de bases no
  equivalentes (bajo transformaciones unitarias) a las bases
  de la construcción estándar. La motivación para realizar
  éste trabajo es que los núcleos de la construcción no
  estándar preservan las propiedades de la construcción
  estándar por lo tanto pueden ser utilizados para definir
  una función de Wigner. En particular, preservan la
  propiedad tomográfica, la cual permite expresar la función
  de Wigner de cualquier estado como una combinación lineal
  de probabilidades observadas. La inequivalencia de las
  construcciones conduce a la posibilidad de encontrar
  estados no estabilizadores con funciones de Wigner
  no-negativas, lo que contrasta con resultados previos para
  el caso discreto \cite{gross2006, galvao2005,
  cormick2006a}.

  \section{Objetivos}

  Los objetivos de éste trabajo de tesis son:
  \begin{itemize}
    \item \textbf{Objetivo general:} Comparar las funciones
      de Wigner discretas que se obtienen mediante la
      construcción estándar y la construcción no estándar.
    \item \textbf{Objetivos particulares:}
      \begin{itemize}
        \item Revisar la bibliografía básica y especializada
          de las funciones de Wigner.
        \item Comprender y construir la función de Wigner
          discreta estándar para qubits y qutrits.
        \item Emular la construcción estándar de la función
          discreta de Wigner para lograr una construcción
          alternativa.
        \item Revisar las propiedades matemáticas y físicas
          de la nueva construcción para compararla con la
          estándar.
      \end{itemize}
  \end{itemize}

  \section{Metodología}

  La construcción de la función de Wigner discreta
  (estándar) para qubits y qutrits, se realizará en la forma
  descrita en Wooters \cite{gibbons2004}, es decir
  utilizando \textit{lineas} en el espacio de fase discreto
  para etiquetar y determinar los núcleos de Wigner
  necesarios.  Por otro lado, la construcción de la función
  de Wigner discreta no estándar para qubits y qutrits, se
  realizará utilizando conjuntos de bases unitariamente no
  equivalentes a los que se utilizan en la construcción de
  Wooters \cite{kantor2012}. Ambas construcciones requerirán
  el uso de un software especializado para su construcción
  explicita, pues los ejemplos más simples requieren
  espacios discretos de dimensión $32 \times 32$ y $27
  \times 27$, para qubits y qutrits, respectivamente.

  \subsubsection{Plan de trabajo}

  Habrá reuniones semanales con el director en donde se
  presentan los avances, dudas y resultados que surgen del
  trabajo. La bibliografía consiste de varios artículos y
  libros sobre la función de Wigner y la mecánica cuántica
  en el espacio de fase, entre otros temas, pero
  especificamente se trabajará siguiendo la metodología de
  Wootters \cite{gibbons2004} para la
  construcción estándar de la función de Wigner discreta.
  Por el momento no se contempla exponer el trabajo en algún
  coloquio, taller o evento. Pero sí existe la posibilidad
  de exponerlo en el seminario de álgebra, la cual se
  agendaría despues de alcanzar el objetivo de la
  construcción no estándar.

  \section{Cronograma de actividades}

  \begin{table}[ht]
    \centering
    \begin{tabular}{p{0.4\linewidth}ccccc}\toprule
       & Enero & Febrero & Marzo & Abril & Mayo \\
      Investigación bibliográfica & . & . & . & & \\
      \addlinespace
      Construcción del espacio de fase discreto & & . & . &
                                                & \\
      \addlinespace
      Construcción de la función de Wigner estándar para
      qubits y qutrits & & . & . & & \\
      \addlinespace
      Construcción de la función de Wigner no estándar para
      qubits y qutrits & & & . & . & \\
      \addlinespace
      Escritura de la tesis & & & . & . & . \\
      \addlinespace
      Revisión por el director & & & & & .
    \end{tabular}
  \end{table}

  \printbibliography
  
\end{document}

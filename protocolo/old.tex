\documentclass[a4paper]{article}

\usepackage[utf8]{inputenc}
\usepackage[T1]{fontenc}
\usepackage{textcomp}
\usepackage[spanish]{babel}
\usepackage{amsmath, amssymb}
\usepackage{amsthm}
\decimalpoint

\usepackage{import}
\usepackage{xifthen}
\pdfminorversion=7
\usepackage{pdfpages}
\usepackage{transparent}
\newcommand{\incfig}[1]{%
  \def\svgwidth{\columnwidth}
  \import{./figures/}{#1.pdf_tex}
}

\DeclareMathOperator{\R}{\mathbb{R}}
\DeclareMathOperator{\C}{\mathbb{C}}
\DeclareMathOperator{\N}{\mathbb{N}}

%\newtheorem{definition}{Definición}
%\newtheorem{theorem}{Teorema}
%\newtheorem{proposition}{Proposición}
%\newtheorem{lemma}{Lema}
%\newtheorem{corollary}{Corolario}

\title{
  \large Protocolo de tésis\\[1in]
  \huge Toolbox para la simulación de la medición de sistemas
cuánticos\\[1in]
  \large Autor: Ernesto Camacho Ramírez\\
  Director de tésis: Dr. Andrés Sandoval García
}
\author{}
\begin{document}
  \maketitle

  \newpage
  \section{Introducción}

  Resumen de la introducción de Nielsen.

  \begin{itemize}
    \item La teoría de la información cuántica implica un
      cambio de perspectiva a la hora de analizar sistemas
      cuánticos, mediante la introducción de la teoría de la
      información y de computación. En lugar de usar la
      teoría cuántica para estudiar sistemas o fenomenos
      naturales, podemos usar la teoría para diseñar
      sistemas.
    \item Pertinente a éste trabajo, deseamos estudiar la
      pregunta, ¿por qué los sistemas cuánticos son
      dificiles de simular de manera clásica?, y ¿qué
      podemos hacer para eficientar la simulación y estudio
      de dichos sistemas?, ¿qué recursos físicos son
      necesarios para la computación cuántica?, etc\ldots
    \item Habido mucho progreso en el área de la
      implementación de computadoras cúanticas. Aún no
      existen computadoras de gran escala pero los avances
      han sido notables (especialemente de \textit{iones
      atrapados}).

    \item ¿La computación cuántica se puede realizar sin el
      uso de la dinámica unitaria, basta con solo la
      medición?

    \item ¿Qué hay de la simulación \textit{clásica} de
      sistemas cuánticos?

    \item La motivación principal para el trabajo en al área
      de la información cuántica es la esperanza de obtener
      algorítmos cuánticos para resolver problemas
      computacionalmente difíciles. Pero, aún no se sabe
      sobre que clase de problemas podemos esperar que las
      computadoras cuánticas superen a las clásicas.

    \item Podemos utilizar una computadora clásica para
      simular una computadora cuántica, pero parecer ser
      imposible hacer la simulación de manera eficiente. Por
      lo tanto las computadoras cuánticas ofrecen un
      incremento en la velocidad sobre las compus cláscias -
      Nielsen. 

    \item Es probable que una de las aplicaciones más
      importantes de las computadoras cuánticas será la
      simulación de sistemas cuánticos que son dificilmente
      simulables en una computadora clásica.

    \item ¿Qué otros problemas podemos resolver más rápido
      que una compu clásica? No sabemos. El diseño de
      algoritmos cuánticos es dificil - Nielsen.

    %To design good quantum algorithms one must \textit{turn
    %off} one's classical intuition for at least part of the
    %design process, using truly quantum effects to achieve
    %the desired algorithmic end.

    %Simulating naturally occurring quantum mechanical
    %systems is a candidate for which quantum computers may
    %excel, but which is believed to be very difficult on a
    %classical computer. The reason for this difficulty is
    %that the number of complex numbers needed to describe a
    %quantum system generally grows exponentially with size
    %of the system.

    %Although a quantum computer can simulate many quantum
    %systems far more efficiently than a classical computer,
    %this does not mean that the fast simulation will allow
    %the desired information about the quantum system to be
    %otained.

    %The simulation of quantum systems is an important
    %problem in many fields, ntoably quantum chemistry, where
    %the computational constraints imposed by classical
    %computers make it difficult to accurately simulate the
    %behavior of even moderately sized molecules, much less
    %the very large molecules that occur in many biological
    %systems.
  \end{itemize}
  
  

  \subsection{Más preguntas}

  ¿Qué tan \textit{poderosas} son las computadoras
  cuánticas? ¿Qué les da su poder? No sabemos. La
  factorización de enteros sugiere un potencial un potencial
  superior a las computadoras clásicas, pero ésto no es
  definitivo. Aún es posible que las computadoras cuánticas
  no sean mas potentes que las clásicas, en el sentido de
  que todo lo que puede ser eficientemente resuelto por una
  computadora cuántica, también sea eficientmente resuelto
  por una computadora clásica.

  \subsection{Teoría de complejidad computacional}

  És un rama de la ciencia computacional que clasifica la
  dificultad de varios problemas computacionales, ya sean
  clásicos o cuánticos. Una clase de complejidad es una
  colección de problemas computacionales que comparten
  caracterísiticas respecto a los recursos requeridos para
  resolverlos. Las dos clases más importantes son $P$ y
  $NP$.

  %It is clear that $P \subset NP$, but it is still not
  %clear that there are any problems in $NP$ which are not in
  %$P$.

  %There is an important subclass of $NP$ problems, the
  %$NP$-complete problems, because there are many important
  %problmes, and because any $NP$-complete problem is at
  %least as hard as all other problems in $NP$. In particular
  %if $P \neq NP$, then it will follow that no $NP$-complete
  %problem can be efficiently solved on a classical computer.

  %It is not known whether quantum computers can be used to
  %quickly solve all the problems in $NP$, despite the fact
  %that they can be used to solve \textit{some} problems,
  %that are believed to be in $NP$ but no in $P$.

  %It would be very exciting if we could solve all $NP$ 
  %problems efficiently on a quantum computer, but quantum
  %parallelism is already ruled out for this.

  %Exactly where $BQP$ fits with respet to $P, NP$ and
  %$PSPACE$ is yet unknown. Although we do know that $BQP$
  %lies somewhere between $P$ and $PSPACE$.

  %An interesting implication is, that if quantum computers
  %are strictly more powerful than classical computers, then
  %it will follow that $P$ is not equal to $PSPACE$. The
  %difficulty of this last question suggests that it may be
  %non-trivial to prove that quantum computers are more
  %powerful than classical computers.
  
  \newpage
  \section{Objetivos}

  Aplicar toolboxes adecuadas para la solución de problemas
  de sistemas cuánticos que involucran muchas partículas.

  Otención del conocmiento mínimo de la mecánica cuántica
  requerido para la comprensión y objetivos de éste trabajo.

  Actividades:
  \begin{itemize}
    \item Investigación bibliográfica de toolboxes para
      sistemas cuánticos.
    \item instalación y manejo adecuado de al menos una
      toolbox seleccionada.
    \item elaboración de "manual" de uso para la toolbox
      instalada.
    \item Uso de la toolbox en la solución de problemas de
      interés de sistemas cuánticos.
  \end{itemize}
  
  \newpage
  \section{Metodología}

  \newpage
  \section{Cronograma de actividades}

  \newpage
  \section{Bibliografía}
  
\end{document}

\documentclass[a4paper]{article}
\usepackage[margin=1.4in]{geometry}
\usepackage[utf8]{inputenc}
\usepackage[T1]{fontenc}
\usepackage{textcomp}
\usepackage{amsmath, amssymb}
\usepackage{amsthm}
\usepackage{braket}

\vfuzz=30pt
\hfuzz=30pt

\DeclareMathOperator{\R}{\mathbb{R}}
\DeclareMathOperator{\C}{\mathbb{C}}
\DeclareMathOperator{\N}{\mathbb{N}}
\DeclareMathOperator{\Z}{\mathbb{Z}}
\DeclareMathOperator{\Tr}{Tr}

\newtheorem{definition}{Definition}
\newtheorem{theorem}{Theorem}
\newtheorem{proposition}{Proposition}
\newtheorem{lemma}{Lemma}
\newtheorem{example}{Example}

\title{Finite and profinite quantum systems (Vourdas)}
\author{Ernesto Camacho Ramírez}
\begin{document}
  \maketitle
  
  \section{Quantum systems with variables in $\Z(d)$ }

  In this section Vourdas only talks about odd integer
  dimesion.

  \subsection{Fourier transforms in $\Sigma[\Z(d)]$}

  Let us consider the quantum system $\Sigma[\Z(d)]$ with
  variables in $\Z(d)$. It is describe by a $d$-dimensional
  Hilbert space $H[\Z(d)]$, that contains complex functions
  $f(m)$ where $m \in \Z(d)$.

  In $H[\Z(d)]$ consider an orthonormal basis of
  \textit{position states}, which we denote as $\ket{X; m}$ 
  where $m \in \Z(d)$. 

  \begin{definition}
    The Fourier transform, is given by
    \[
      F = 
      \frac{1}{\sqrt{d}}
      \sum_{m,n}^{} \omega(mn) \ket{X;m} \bra{X;n},
      \quad
      \omega(\alpha) = \exp\left( 2\pi i \alpha / d \right).
    \] 
  \end{definition}

  Using the Fourier transform we define the basis of
  \textit{momentum states} as
  \[
    \ket{P; m}
    = F\ket{X; m}
    = \frac{1}{\sqrt{d}} \sum_{n}^{} \omega(mn) \ket{X; n}.
  \]

  The position and momentum operators $x$ and $p$ are given
  by
  \[
    x = \sum_{n=0}^{d-1} n \ket{X; n} \bra{X; n};
    \quad
    p = \sum_{n=0}^{d-1} n \ket{P; n} \bra{P; n}.
  \] 
  The $x$ and $p$ are defined modulo $d \bf{1}$, because $n$ 
  are integers modulo $d$. We can obtain the momentum
  operator from the position operator and vice versa using
  the Fourier transform:
  \[
    p = FxF^{*},
    \quad FpF^{*} = -x.
  \] 

  Vourdas, for convience, defines the functions
  \[
    \Delta_0(x)
    = \frac{1}{d} \sum_{n=0}^{d-1} \omega(nx)
  \] 
  and generally
  \[
    \Delta_m(x)
    = \partial_x^{m} \Delta_0(x)
    = \frac{1}{d} \sum_{n=0}^{d-1} \left( i \frac{2\pi n}{d}
    \right)^{m} \omega(nx);
    \quad
    \Delta_m(x+d) = \Delta_m(x).
  \] 
  Extra care is needed in numerical calculations, due to the
  circular nature of the variables. 

  The commutator $[x,p]$ is an important quantity in the
  case of continuous Heisenberg-Weyl groups, related to
  infinitesimal displacements in phase space. In the finite
  case, the Heisenberg-Weyl group is discrete and the
  quantity is of less importance.

  \subsection{The Heisenberg-Weyl group $HW[\Z(d)]$}

  \begin{definition}
    The Heisenberg-Weyl group has elements
    $g(\alpha,\beta,\gamma)$, where $\alpha, \beta, \gamma$ 
    are elements of some ring, and the multiplication rule:
    \[
      g(\alpha_1,\beta_1,\gamma_1)
      g(\alpha_2,\beta_2,\gamma_2)
      = g(\alpha_1+\alpha_2,\beta_1+\beta_2,
      \gamma_1+\gamma_2+2^{-1}
      (\alpha_1\beta_2-\alpha_2\beta_1)).
    \] 
  \end{definition}

  \begin{definition}
    The displacement operators $X, Z$ are the unitary
    operators
    \[
      X = \exp\left( -i \frac{2\pi}{d} p \right);
      \quad
      Z = \exp\left( i \frac{2\pi}{d} x \right).
    \] 
  \end{definition}

  They act on the position and momentum states as follows:
  \begin{align*}
    &Z^{\alpha} \ket{P; m} = \ket{P; m + \alpha};
    \quad
    Z^{\alpha} \ket{X; m} = \omega(\alpha m) \ket{X; m}\\
    &X^{\beta} \ket{P; m} = \omega(-m\beta) \ket{P; m};
    \quad
    X^{\beta} \ket{X; m} = \ket{X; m+\beta}
  \end{align*}
  where $\alpha, \beta \in \Z(d)$.

  They also obey the important relations
  \[
    X^{d} = Z^{d} = 1;
    \quad
    X^{\beta}Z^{\alpha} =
    Z^{\alpha}X^{\beta}\omega(-\alpha\beta);
    \quad
    \Tr(X) = \Tr(Z) = 0.
  \] 
  And
  \[
    FXF^{*} = Z;
    \quad
    FZF^{*} = X^{*}
  \] 

  The position-momentum phase space is the toroidal lattice
  $\Z(d) \times \Z(d)$, and the relations $X^{d} = Z^{d} =
  \bf 1$ indicate the toroidal nature of the phase space.

  General displacement operators are the unitary operators
  \[
    D(\alpha,\beta)
    = Z^{\alpha} X^{\beta} \omega(-2^{-1}\alpha\beta);
    \quad
    \alpha,\beta \in \Z(d).
  \] 

  The operators $D(\alpha,\beta)\omega(\gamma)$ form a
  representation of the $HW[\Z(d)]$ Heisenberg-Weyl group.
  The displacement operators act on the position and
  momentum states as follows:
  \begin{align*}
    D(\alpha,\beta) \ket{X; m}
    = \omega(2^{-1}\alpha\beta + \alpha m) \ket{X;
    m+\beta};\\
    D(\alpha,\beta) \ket{P; m}
    = \omega(-2^{-1}\alpha\beta - \beta m) \ket{P; m+\alpha}
  .\end{align*}

  \subsection{Partiy operators}

  Parity operators have been studied extensively in the
  context of the harmonic oscilator. In the preset context
  they are defined as follows.

  \begin{definition}
    The partiy operator around the origin in phase space, is
    given by
    \[
      P(0,0) = F^2;
      \quad
      P(0,0)^2 = 1;
      \quad
      P(0,0)^{*} = P(0,0).
    \] 
    The parity operator around the point $(\alpha,\beta)$ in
    phase space (a.k.a. the displaced parity operator) is
    given by
    \[
      P(\alpha,\beta)
      = D(\alpha,\beta) P(0,0) D(\alpha,\beta)^{*};
      \quad
      P(\alpha,\beta)^2 = 1.
    \] 
  \end{definition}

  The parity operators act on the position and momentum
  states in the following way
  \begin{align*}
    P(\alpha,\beta) \ket{X; m}
    &=
    \omega(2\alpha\beta - 2\alpha m) \ket{X; 2\beta - m}\\
    P(\alpha,\beta) \ket{P; m}
    &= 
    \omega(-2\alpha\beta-2\beta m) \ket{P; 2\alpha - m}.
  \end{align*}

  An important set of properties of both the displacement
  operators and the partiy oeprators, are the
  \textit{marginal properties}. Vourdas uses these to later
  prove marginal properties of the Wigner function.

  \begin{proposition}
    Let $\Pi_X(\alpha)$ and $\Pi_P(\alpha)$, be projectors
    to position and momentum states:
    \[
      \Pi_X(\alpha) = \ket{X; \alpha} \bra{X; \alpha};
      \quad
      \Pi_P(\alpha) = \ket{P; \alpha} \bra{P; \alpha}.
    \] 
  \end{proposition}

  \begin{itemize}
    \item The displacement operators obey the following
      \textit{marginal relations} along the horizontal and
      vertical lines in the phse space $\Z(d) \times \Z(d)$:
      \[
        \frac{1}{d} \sum_{\beta=0}^{d-1} D(\alpha,\beta)
        = \Pi_P(2^{-1}\alpha)P(0,0).
      \] 
      \[
        \frac{1}{d} \sum_{\alpha,\beta}^{} D(\alpha,\beta) =
        P(0,0).
      \] 
    \item THe partiy operators obey the following
      \textit{marginal relations} along the horizontal and
      vertical lines in the phase space $\Z(d) \times
      \Z(d)$:
      \begin{align*}
        \frac{1}{d} \sum_{\beta=0}^{d-1} P(\alpha,\beta)
        &= \Pi_P(\alpha) \\
        \frac{1}{d} \sum_{\alpha=0}^{d-1} P(\alpha,\beta)
        &= \Pi_X(\beta) \\
        \frac{1}{d} \sum_{\alpha,\beta}^{} P(\alpha,\beta)
        &= 1.
      \end{align*}
  \end{itemize}

  As an analog to the continuous case, the displacement and
  parity operators are related through Fourier transforms.

  \subsection{Wigner and Weyl functions}

  The Wigner and Weyl functions play a central role in phase
  space methods in quantum mechanics. In particular the
  subject of tomography constructs the Wigner function.

  \begin{definition}
    Let $\theta$ be an arbitrary operator. The corresponding
    Wigner function $W(\alpha,\beta; \theta)$ is defined as:
    \[
      W(\alpha,\beta; \theta) = \Tr\left( \theta
      P(\alpha,\beta) \right).
    \] 
  \end{definition}
  If $\theta$ is a density matrix, the Wigner function can
  be interpreted as a pseudo-probability distribution of the
  particle in the position-momentum phase space. The
  following proposition is based on the marginal properties
  of the parity operators.

  \begin{proposition}
    For an arbitrary operator $\theta$, let
    \[
      \theta_X(m,n) = \bra{X; m}\theta\ket{X; n};
      \quad
      \theta_P(m,n) = \bra{P; m}\theta\ket{P; n}.
    \] 
    The Wigner function obeys the following marginal
    properties:
    \begin{align*}
      \frac{1}{d} \sum_{\beta=0}^{d-1} W(\alpha,\beta;
      \theta)
      &= \theta_P(\alpha,\alpha) \\
      \frac{1}{d} \sum_{\alpha=0}^{d-1} W(\alpha,\beta;
      \theta)
      &= \theta_X(\beta,\beta) \\
      \frac{1}{d} \sum_{\alpha,\beta}^{} W(\alpha,\beta;
      \theta) 
      &= \Tr \theta.
    \end{align*}
  \end{proposition}

  \section{Finite geometries and mutually unbiased bases}

  Vourdas then discusses the space $\Z(d) \times \Z(d)$ as a
  finite geometry and its link to the subject of mutually
  unbiased bases. There are deep mathematical problems
  realted to these bases, and they also have important
  applications in quantum communications and quantum
  cryptography.

  As is usual, he makes a distinction:
  \begin{itemize}
    \item When $d = p$ is a prime number, $\Z(p)$ is a
      field. $\Z(p) \times \Z(p)$ is what he calls a
      near-linear finite geometry, based on the axiom that
      two lines have at most one point in common. The number
      of mutually unbiased bases is $p+1$ and there is a
      duality between the finite geometry and the mutually
      unbiased bases. These results can be extended to the
      case that $d = p^{e}$, using the Galois field
      $GF(p^{e})$.
    \item When $d$ is not a prime number, then $\Z(d)$ is a
      ring. Then $\Z(d) \times \Z(d)$ is a non-near-linear
      finite geometry, and two lines might have more than
      one point in common. The number of mutually unbiased
      baess is not known, but it is probably smaller than
      $d+1$. Here there is no duality between the finite
      geometry and the mutually unbiased bases.
  \end{itemize}

  He starts with non-near-linear finite geometries, which is
  the case where $d$ is not a prime or prime power. He
  denotes a line throught the point $(\alpha,\beta)$ as the
  set of points
  \[
    L(\rho,\sigma | \alpha,\beta)
    = \{(\tau \rho + \alpha, \tau \sigma + \beta) : \tau \in
    \Z(d)\},
    \quad
    \rho,\sigma,\alpha,\beta \in \Z(d).
  \] 

  He also mentions that the lines through the origin
  $(0,0)$, denoted as $L(\rho,\sigma)$ are a cyclic module
  generated by $(\rho,\sigma)$, but that he will use the
  intuitive term line. The geometry becomes slightly more
  complicated, for example, the number of points in
  $L(\rho,\sigma)$ is $d / GCD(\rho,\sigma,d)$, if $\lambda$ 
  is an ivnertible element in $\Z(d)$ then
  $L(\rho\lambda,\sigma\lambda) = L(\rho,\sigma)$, and the
  intersection of two lines is a line, which he calls a
  subline, with the number of common points being a divisor
  of $d$.

  In the case $d = p$ where $p$ is a prime number, the
  $\Z(p)$ is a field. In this case the only divisor of $p$ 
  is 1, and two lines through the origin have one point in
  common.

  \subsection{Mutually unbiased bases}

  In Vourdas notation, a set of orthonormal bases in
  $H[\Z(d)]$ are called mutually unbiased, if the vectors in
  any two of these bases obey the relation
  \[
    |\braket{X; m|Y; n}|^2 = \frac{1}{d}, 
    \quad m,n \in \Z(d),
  \] 
  for all $m,n$. He mentions the basic result that the
  number of mutually unbiased bases in $H[\Z(d)]$ is less
  then or equal to $d+1$. In systems with prime, or power of
  prime dimension, the inequality becomes equality. 

  \subsection{Mutually unbiased bases in $H[\Z(p)]$}

  Now he considers systems with prime dimension $p$, and
  constructs a set of $p+1$ MUBs. His construction is based
  on symplectic transformations. This method is then
  generalized to systems with power of prime dimension
  $p^{e}$ with variables in the Galois field.

  In $H[\Z(p)]$ where $p$ is an odd prime, consider the $p$ 
  orthonormal bases
  \[
    \ket{\mathcal{X}(v); m} = S(0,-1|1,v)\ket{X; m};
    \quad
    v,m \in \Z(p),
  \] 
  where $S(0,-1|1,v)$ are symplectic matrices. In the case
  $v = 0$, then we obtain momentum states:
  \[
    \ket{\mathcal{X}(0); m}
    = S(0,-1|1,0) \ket{X; m}
    = F^{*}\ket{X; m}
    = \ket{P; -m}.
  \] 
  As notation he also uses $\ket{\mathcal{X}(-1); m} =
  \ket{X; m}$. And so he obtains $p+1$ orthonormal bases
  \[
    \ket{\mathcal{X}(v); m};
    \quad 
    v \in \{-1\} \cup \Z(p).
  \] 
  \begin{proposition}
    For  $v \neq v'$, 
    \[
      |\braket{\mathcal{X}(v'); n|\mathcal{X}(v); m}|^2
      = \frac{1}{p};
      \quad
      v,v' \in \{-1\} \cup \Z(p).
    \] 
  \end{proposition}

  \begin{proposition}
    There is a duality between the $\psi(p) = p+1$ lines
    through the origin in the near-linear finite geometry
    $\Z(p) \times \Z(p)$, and the $\psi(p) = p+1$ mutually
    unbiased bases in the Hilbert space $H[\Z(p)]$, where
    \[
      \mathcal{L}(v) \leftrightarrow
      \{\ket{\mathcal{X}(v); m}\};
      \quad
      v = -1,\ldots,p-1.
    \] 
    Where $p$ points in the line $\mathcal{L}(v)$ correspond
    to the $p$ vectors in the basis $\{\ket{\mathcal{X}(v);
    m}\}$.
  \end{proposition}
  The duality between mutually unbiased bases and finite
  geometries, does not hold for non-prime dimensions. This
  motivates the revision of the concept of mutually unbiased
  bases into another concept (weak mutually unbiased bases).

  Vourdas notes (among other topics), that acting with a
  unitary transformation on a set of mutually unbiased
  bases, we get another set of mutually unbiased bases. But
  there are unitarily inequivalent mutually unbiased bases
  which have been discussed.

  \subsection{Galois theory}

  Vourdas makes the statement that some theorems in Galois
  theory are valid only for a prime $p \neq 2$, and so it is
  the case which he considers in the chapter.

  First consider the ring of pylonomials $[\Z(p)](\epsilon)$ 
  with coefficients in the field $\Z(p)$. Let
  $\mathcal{p}(\epsilon)$ be an irreducible polynomial of
  degree $e$. The ring $[\Z(p)](\epsilon) /
  \mathcal{p}(\epsilon)$ has as elements plynomials in
  $[\Z(p)](\epsilon)$ which are defined modulo
  $\mathcal{p}(\epsilon)$. The $[\Z(p)](\epsilon) /
  \mathcal{p}(\epsilon)$ is a representation of the Galois
  field $GF(p^{e})$. Different irreducible polynomials of
  the same degree $e$, lead to isomorphic finite fields.
  
  Addition and multiplication of two Galois numbers is the
  standard addition and multiplication of polynomials, and
  the result is defined modulo the polynomial
  $\mathcal{p}(\epsilon)$. The Galois number $\alpha$ can be
  viewed as a vector
  $(\alpha_0,\alpha_1,\ldots,\alpha_{e-1})$ in $[\Z(p)]^{e}$ 
  in the basis $\{1,\epsilon,\ldots,\epsilon^{e-1}\}$. In
  this way, the addition of two Galois numbers is the
  standard addition of vectors in $[\Z(p)]^{e}$.

  \begin{proposition}
    \begin{itemize}
      \item The elements of $GF(p^{e})$ obey the relation
        \[
          \alpha^{p^{e}} = \alpha.
        \] 
      \item The Frobenius transformation
        \[
          \sigma(\alpha) = \alpha^{p}
        \] 
        defines an automorphism in $GF(p^{e})$ and leads to
        Galois conjugates
        \[
          \alpha \to \alpha^{p} \to \ldots \to
          \alpha^{p^{e-1}} \to \alpha.
        \] 
        Elements in the subfield $\Z(p)$ of $GF(p^{e})$ are
        self-conjugates, becase for $\alpha \in \Z(p)$ we
        get $\alpha^{p} = \alpha$.
    \end{itemize}
  \end{proposition}

  The trace of $\alpha \in GF(p^{e})$ is the sum of all its
  conjugates, and it is an element of $\Z(p)$:
  \begin{align*}
    &\Tr(\alpha) = \Tr_{e / 1}(\alpha) = \alpha + \alpha^{p}
    + \alpha^{p^2} + \ldots + \alpha^{p^{e-1}} \\
    &\Tr(\alpha) \in \Z(p); \quad \alpha \in GF(p^{e}).
  \end{align*}
  This is the trace with respect to the extension from
  $\Z(p)$ to $GF(p^{e})$. All conjugates have the same
  trace. The trace is linear, $\Tr(\alpha+\beta) =
  \Tr(\alpha)+\Tr(\beta)$. If $\alpha \in \Z(p)$ and $\beta
  \in GF(p^{e})$, then $\Tr(\alpha\beta) = \alpha
  \Tr(\beta)$.

  Vourdas finally gives me a way to calculate the dual field
  basis. He does this by defining two symmetric and
  ivnertible $e \times e$ matrices, $g$ and $G$, with
  elements in $\Z(p)$:
  \[
    g_{ij} := \Tr\left( \epsilon^{i+j} \right);
    \quad
    G = g^{-1},
    \quad
    i,j = 0,\ldots,e-1.
  \] 
  The elements depend on the choice of the irreducible
  polynomial $\mathcal{p}(\epsilon)$. He can then define
  $\{E_0,E_1,\ldots,E_{e-1}\}$ as the dual basis to
  $\{1,\epsilon,\epsilon^2,\ldots,\epsilon^{e-1}\}$, defined
  as
  \[
    E_i = \sum_{j}^{} G_{ij} \epsilon^{j}.
  \] 
  Defined in this way we have $\Tr(\epsilon^{i} E_j)
  \delta(i,j)$. We can then use the two basis to express any
  element $\alpha \in GF(p^{e})$.

  Additive characters in $GF(p^{e})$ are defined as
  \[
    \chi(\alpha) = \omega\left( \Tr(\alpha) \right);
    \quad
    \alpha \in GF(p^{e}),
  \] 
  where $\omega(r) = \exp\left( i 2\pi r / p \right)$. All
  conjugates have the same character. The Pontryagin dual
  group to $GF(p^{e})$, i.e., the group of its characters,
  is isomorphic to $GF(p^{e})$.

  \section{Quantum systems with variables in $GF(p^{e})$}

  Vourdas considers a quantum system with variables in
  $GF(p^{e})$, where $p$ is a prime number ($p \neq 2$),
  which he calls a Galois quantum system and denotes it as
  $\Sigma[GF(p^{e})]$. For comparison, he also considers an
  $e$-partite system where each component is a quantum
  system with variables in $\Z(p)$, a system which he
  denotes as $\Sigma \{[\Z(p)]^{e}\}$.

  \subsection{Fourier transforms in $\Sigma[GF(p^{e})]$}

  He starts by using the notation
  \begin{align*}
    \omega(r) &= \exp\left( i \frac{2\pi r}{p} \right);
    \quad r \in \Z(p)\\
    \Omega(s) &= \exp\left( i \frac{2\pi s}{e} \right);
    \quad s \in \Z(e).
  \end{align*}
  The Hilbert space for the system $\Sigma[GF(p^{e})]$ is
  the $p^{e}$-dimensional space $H[GF(p^{e})]$, of complex
  wavefunctions $f(m)$, where $m \in GF(p^{e})$. A basis
  for this Hilbert space consists of the position states
  $\ket{X; m}$ where $m \in GF(p^{e})$.

  The Fourier transform is then given in terms of the
  characters:
  \[
    F = \frac{1}{\sqrt{p^{e}}} \sum_{m,n \in
    GF(p^{e})}^{} \chi(mn) \ket{X; m} \bra{X; n},
  \] 
  and so the momentum states are
  \[
    \ket{P; m}
    = F\ket{X; m}
    = \frac{1}{\sqrt{p^{e}}} \sum_{n \in GF(p^{e})}^{}
    \chi(mn) \ket{X; n}.
  \] 
  If $m = \sum_{}^{} m_i \epsilon^{i}$, the bijective map
  \[
    m \leftrightarrow (m_0,\ldots,m_{e-1}),
  \] 
  implies that there is a bijective map
  \[
    H[GF(p^{e})]
    \leftrightarrow
    H[\Z(p)] \otimes \cdots \otimes H[\Z(p)]
  \] 
  where
  \[
    \ket{X; m_0 + m_1\epsilon + \ldots + m_{e-1}
    \epsilon^{e-1}} 
    \mapsto
    \ket{X; m_0} \otimes \cdots \otimes \ket{X; m_{e-1}}.
  \] 
  This of course depends on the chosen basis for
  $GF(p^{e})$.

  In this way an alternative way to express the Fourier
  transform is by
  \[
    F = \frac{1}{\sqrt{p^{e}}}
    \sum_{}^{} \omega\left( \sum_{}^{} g_{ij} m_i n_j \right) 
    \ket{X; m_0} \bra{X; n_0} \otimes \cdots \otimes
    \ket{X; m_{e-1}} \bra{X; n_{e-1}}.
  \] 
  And then the momentums tates can be written in terms of
  the dual components as a tensor product.

  \subsection{The Heisenberg-Weyl group $HW[GF(p^{e})]$}

  Vourdas starts by defining the displacement operators
  again, he does this explicitly this time, no need for
  exponentials of the position or momentum operators.
  \begin{align*}
    Z(\alpha)
    &= \sum_{n \in GF(p^{e})}^{} \chi(\alpha n) \ket{X; n}
    \bra{X; n};
    \quad
    \alpha, \beta \in GF(p^{e}) \\
    X(\beta)
    &= F Z(\alpha) F^{*}
    = \sum_{n \in GF(p^{e})}^{} \chi(-\beta n) \ket{P; n}
    \bra{P; n}.
  \end{align*}

  They can also be expressed as tensor products:
  \begin{align*}
    Z(\alpha)
    &= \sum_{}^{} \omega_p\left( \sum_{}^{} g_{ij} \alpha_i
    n_j\right) \ket{X; n_0} \bra{X; n_0} \otimes \cdots
    \otimes \ket{X; n_{e-1}} \bra{X; n_{e-1}} \\
    X(\beta)
    &= \sum_{}^{} \omega_p\left( \sum_{}^{} g_{ij} \beta_i
    n_j \right) \ket{P; n_0} \bra{P; n_0} \otimes \cdots
    \otimes \ket{P; n_{e-1}} \bra{P; n_{e-1}},
  \end{align*}
  in the basis $\{\epsilon^{i}\}$ for $GF(p^{e})$. And as
  before, general displacement operators are given by
  \[
    D(\alpha,\beta)
    = Z(\alpha) X(\beta) \chi(-2^{-1}\alpha\beta).
  \] 
  The operators $Z(\alpha)$, $X(\beta)$ act on position and
  momentum states as follows:
  \begin{align*}
    &Z(\alpha) \ket{P; m} = \ket{P; m+\alpha}; 
    \quad
    Z(\alpha) \ket{X; m} = \chi(\alpha m) \ket{X; m} \\
    &X(\beta) \ket{P; m} = \chi(-m \beta) \ket{P; m};
    \quad
    X(\beta) \ket{X; m} = \ket{X; m+\beta}.
  \end{align*}
  They follow the commutation rule:
  \[
    X(\beta) Z(\alpha) = Z(\alpha) X(\beta) \chi(-\alpha
    \beta).
  \] 
  Importantly, the $Z(\alpha)$, $X(\beta)$ are $p^{e} \times
  p^{e}$ complex matrices, and they have the $\omega_p(i)$ 
  as eigenvalues (with various multiplicities). In general
  we have
  \begin{align*}
    D(\alpha,\beta) \ket{X; m}
    &= \chi(2^{-1} \alpha \beta + \alpha m) \ket{X; m+\beta}
    \\
    D(\alpha,\beta) \ket{P; m}
    &= \chi(-2^{-1} \alpha\beta - \beta m) \ket{P;
    m+\alpha}.
  \end{align*}

  In the context of the isomorphism, the displacement
  operators acting on $H[GF(p^{e})]$ are expressed in terms
  of the displacement operators $\mathcal{D}$ acting on the
  various $H[\Z(p)]$ as
  \[
    D(\alpha,\beta)
    = D(\hat \alpha_0, \beta_0) \otimes \cdots \times D(\hat
    \alpha_{e-1},\beta_{e-1}).
  \] 

  \subsection{Parity operators and Wigner functions}

  The material that was presented in the case of finite
  field quantum systems is also valid for Galois quantum
  systems by replacing the characters $\omega(\alpha)$ where
  $\alpha \in \Z(d)$, with $\chi(\alpha)$ where $\alpha \in
  GF(p^{e})$. The definition of the partiy operators is the
  same just with the character replacements.

  \begin{proposition}
    The partiy operators acting on $H[GF(p^{e})]$ are
    expressed in terms of the displacement operators
    $\mathcal{P}$ acting on the various $H[\Z(p)]$ as:
    \[
      P(\alpha,\beta)
      = P(\hat \alpha_0,\beta_0) \otimes \cdots \otimes
      P(\hat \alpha_{e-1}, \beta_{e-1}).
    \] 
  \end{proposition}

  \subsection{Mutually unbiased baess in $H[GF(p^{e})]$ and
  duality to $GF(p^{e}) \times GF(p^{e})$}

  Vourdas now extends the discussion on mutually unbiased
  bases to systems with variables in Galois fields. There
  are $p^{e} + 1$ mutually unbiased bases in $H[GF(p^{e})]$,
  which he constructs explicitly in this section.

  In $H[GF(p^{e})]$ consider the $p^{e}$ orthonormal bases
  \[
    \ket{\mathcal{X}(v); m}
    = S(0,-1|1,v) \ket{X; m};
    \quad
    v,m \in GF(p^{e}),
  \] 
  where $S(0,-1|1,v)$ are symplectic matrices. In the case
  $v = 0$, this is the basis of momentum states:
  \[
    \ket{\mathcal{X}(0); m}
    = S(0,-1|1,0) \ket{X; m}
    = F^{*}\ket{X; m}
    = \ket{P; m}.
  \] 
  Denote the position states by the convention
  \[
    \ket{\mathcal{X}(-1); m}
    = \ket{X; m}.
  \] 
  \begin{proposition}
    For $v \neq v'$,
    \[
      |\braket{\mathcal{X}(v'); n|\mathcal{X}(v); m}|^2
      \frac{1}{p^{e}};
      \quad
      v, v' \in \{-1\} \cup GF(p^{e}).
    \] 
    Therefore these are a set of $p^{e}+1$ mutually unbiased
    bases.
  \end{proposition}

  \subsection{The finite geometry $GF(p^{e}) \times
  GF(p^{e})$}

  The $GF(p^{e}) \times GF(p^{e})$ has the $p^{e}+1$ lines
  through the origin given by
  \[
    L(0,1) = \mathcal{L}(-1);
    \quad
    g(0,-1|1,v) \circ L(0,1) = \mathcal{L}(-1);
    \quad
    v \in GF(p^{e}).
  \] 

  \begin{proposition}
    There is a duality between the $p^{e}+1$ mutually
    unbiased bases in $H[GF(p^{e})]$, and the $p^{e}+1$ 
    lines through the origin in the finite geometry
    $GF(p^{e}) \times GF(p^{e})$, where
    \[
      \mathcal{B}(v) \leftrightarrow \mathcal{L}(v).
    \] 
  \end{proposition}
\end{document}

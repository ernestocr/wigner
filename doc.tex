\documentclass[a4paper]{report}
\usepackage[margin=1.8in]{geometry}

\usepackage[utf8]{inputenc}
\usepackage[T1]{fontenc}
\usepackage{textcomp}
\usepackage[spanish]{babel}
\decimalpoint

\usepackage{csquotes}
\usepackage[sorting=none]{biblatex}
\addbibresource{refs.biblatex}

\usepackage{amsmath, amssymb}
\usepackage{amsthm}
\usepackage{bm}
\usepackage{braket}

\usepackage{booktabs}

\DeclareMathOperator{\R}{\mathbb{R}}
\DeclareMathOperator{\C}{\mathbb{C}}
\DeclareMathOperator{\N}{\mathbb{N}}
\DeclareMathOperator{\Z}{\mathbb{Z}}

\let\H\relax
\DeclareMathOperator{\H}{\mathcal H}
\DeclareMathOperator{\Sz}{\mathcal S}

\DeclareMathOperator{\dom}{Dom}
\DeclareMathOperator{\prob}{Prob}
\DeclareMathOperator{\id}{id}
\DeclareMathOperator{\Tr}{Tr}
\DeclareMathOperator{\Op}{Op}
\DeclareMathOperator{\W}{W}
\DeclareMathOperator{\F}{\mathcal{F}}

\newtheorem{definition}{Definición}
\newtheorem{theorem}{Teorema}
\newtheorem{proposition}{Proposición}
\newtheorem{lemma}{Lema}
\newtheorem{corollary}{Corolario}
\newtheorem{example}{Ejemplo}
\newtheorem{axiom}{Axioma}
\newtheorem{remark}{Observación}

\hfuzz=50pt

\title{Funciones de Wigner en el Espacio de Fase Discreto}
\author{Ernesto Camacho Ramírez}

\begin{document}
  \maketitle

  \section{Introducción}

  El número de elementos de un conjunto general de
  operadores cuánticos unitarios sobre estados de $n$-qudit
  generalmente crece exponencialmente con $n$. Una excepción
  importante a ésta regla involucra el conjunto de
  operadores de Clifford que actúan sobre estados
  estabilizadores. Éstos estados juegan un papel importante
  en la corrección de errores cuánticos
  \cite{gottesmanHeisenbergRepresentationQuantum1998} y son
  cerrados bajo la acción de las compuertas de Clifford. La
  simulación eficiente de dichos sistemas con una
  computadora clásica se demostró con el algoritmo tableau
  de Aaronson y Gottesman \cite{
    aaronsonImprovedSimulationStabilizer2004,
  gottesmanHeisenbergRepresentationQuantum1998} para qubits
  ($d=2$). La busqueda de un explicación de por qué un
  algoritmo tan eficiente es posible para la simulación de
  circuitos de Clifford ha sido un objeto de mucho estudio
  \cite{gottesmanFaultTolerantQuantumComputation1999,
    howardContextualitySuppliesMagic2014,
  mariPositiveWignerFunctions2012}. El progreso reciente ha
  sido resultado del trabajo de Wooters
  \cite{woottersWignerFunctionFormulationFiniteState1987},
  Eisert \cite{mariPositiveWignerFunctions2012}, Gross
  \cite{grossHudsonTheoremFinitedimensional2006} y Emerson
  \cite{howardContextualitySuppliesMagic2014}, quienes han
  formulado una nueva perspectiva basada en los espacios de
  fase discretos de estados y operadores en espacios de
  Hilbert finitos, utilizando funciones discretas de Wigner.
  Se ha demostrado que los estados estabilizadores tienen
  funciones de Wigner discretas definidas positivas y que
  los operadores de Clifford son mapeos definidos positivos.
  Esto implica que los circuitos de Clifford son simulables
  eficientemente en computadores clásicas. En sistemas de
  dimensiones impares, se ha demostrado que los estados
  estabilizadores son análogos discretos a los estados
  gaussianos en sistemas continuos
  \cite{grossHudsonTheoremFinitedimensional2006} y se ha
  demostrado que las compuertas del grupo de Clifford tienen
  Hamiltonianos armónicos subyacentes que conservan los
  puntos discretos del espacio de fase
  \cite{kociaSemiclassicalFormulationGottesmanKnill2017}.
  Esto significa que los circuitos de Clifford se pueden
  expresar mediante integrales de trayectoria truncadas en
  el orden $\hbar^{0}$ y por lo tanto son expresamente
  clásicos
  \cite{kociaSemiclassicalFormulationGottesmanKnill2017,
  kohComputingQuopitClifford2017}.

  En este trabajo, buscamos construir, de una manera
  explícita, los operadores \textit{puntuales} de fase
  discretos (núcleos de la transformación de Wigner) para
  qubits y qutrits. Estrictamente hablando, buscamos la
  construcción \textit{estándar}
  \cite{woottersWignerFunctionFormulationFiniteState1987} y
  \textit{no estandar} (contribución del trabajo) de los
  núcleos que permitan dos formas de la función de Wigner,
  utilizando conjuntos de bases no equivalentes bajo
  transformaciones unitarias. La motivación para realizar
  éste trabajo es que, los núcleos que no son equivalentes
  conservan la propiedad tomográfica básica, lo que permite
  expresar la función de Wigner de cualquier estado como una
  combinación lineal de probabilidades medidas y la
  inequivalencia conduce a la posibilidad de encontrar
  estados no estabilizadores con funciones de Wigner no
  negativas, lo que contrasta con resultados previos para el
  caso discreto
  \cite{grossHudsonTheoremFinitedimensional2006,
  galvaoDiscreteWignerFunctions2005,
  cormickInterferenceDiscreteWigner2006}.

  \chapter{Preliminares}

  Para poder estudiar la función de Wigner discreta es
  necesario presentar la función de Wigner en el caso
  continuo. Para ésto requerimos un conocimiento básico de
  la mecánica cuántica, la cual es una teoría física que
  nació debido a la incapacidad de la mecánica clásica de
  explicar algunos fenómenos físicos que se observaban a
  nivel atómico. A diferencia de la mecánica clásica, la
  teoría cuántica es probabilística naturalmente y sus
  formulaciones matemáticas son bastante distintas. En la
  mecánica clásica, cuando se fija un estado de un sistema
  físico, el valor especificado por un observable (algo se
  puede medir del sistema) está completamente determinado.
  En la mecánica cuántica ésto ya no es cierto, los
  observables solo nos brindan distribuciones de los
  posibles valores. Antes de plantear los principios de la
  mecánica cuántica, repasamos rápidamente el concepto del
  espacio de fase en la mecánica clásica, ya que unas de las
  ideas principales de la distribución de Wigner es vincular
  éstas dos teorías físicas.

  \subsection{La mecánica clásica y el espacio de fase}

  El concepto del espacio de fase es una herramienta de la
  mecánica clásica, la cual describe la evolución temporal
  de un sistema físico. Dicho de una manera muy sencilla, la
  mecánica clásica estudia partículas y sus trayectorias,
  las cuales se rigen de acuerdo a las leyes de Newton.  Se
  considera que la partícula se `mueve' en un espacio
  euclideano, es decir, su \textit{posición} está dado por
  $\bm{x} = (x_1,\ldots,x_n) \in \R^{n}$. El
  \textit{momentum} es una cantidad dada por $p_j = m \dot
  x_j$, donde $\bm{\dot x}$ es la derivada respecto al
  tiempo de la posición, es decir, la velocidad de la
  partícula, y $m$ es la \textit{masa} de la partícula. Las
  cantidades que uno desea medir de nuestro sistema físico
  se les llama \textit{observables}, y en la mecánica
  clásica, son las funciones continuas que tienen como
  argumentos las cantidades $\bm{x},\bm{p}$ y $m$. Ejemplos
  de ellos son el momentum, la energía cinética, la energía
  potencial, etc. La función de energía más usual es la que
  está dada como la suma de la \textit{energía cinética} y
  \textit{energía potencial}:
  \begin{equation}
    H(\bm{x},\bm{p})
    = \frac{1}{2m} \sum_{j=1}^{n} p_j^2 + V(x).
  \end{equation}
  A la energía del sistema se le conoce como el
  \textit{Hamiltoniano}.  Utilizando ésta función de
  energía, la ley de Newton nos brinda las ecuaciones de
  movimiento de la partícula en cuestión:
  \begin{equation}
    \frac{dx_j}{dt}
    = \frac{\partial H}{\partial p_j},
    \quad
    \frac{dp_j}{dt}
    = -\frac{\partial H}{\partial x_j}.
  \end{equation}
  Expresada como un sistema de ecuaciones diferenciales, a
  la ley de Newton se le conoce como las \textit{ecuaciones
  de Hamilton}.  

  Con ésto, es natural representar el estado del sistema
  clásico considerando el par $(\bm{x},\bm{p})$. Al espacio
  $\R^{2n}$ correspondiente le llamamos el \textit{espacio
  de fase}. A las soluciones de las ecuaciones de Hamilton
  se les conoce como \textit{trayectorias}, y son curvas que
  viven en el espacio de fase.
  \begin{definition}
    El espacio de fase de una partícula que se mueven en
    $\R^{n}$ es $\R^{2n}$, considerado como el conjunto de
    las $(2n)$-tuplas de la forma
    \[
      \left(
        x_1, \ldots, x_n, p_1, \ldots, p_n
      \right),
    \] 
    donde $x_j$ y $p_j$ son elementos de $\R$.
  \end{definition}

  Cabe mencionar que el tratado moderno de la mecánica
  clásica está fundamentado en la geometría diferencial de
  las variedades simplécticas
  \cite{mcinerneyFirstStepsDifferential2013}, en donde el
  espacio de fase se define como el espacio cotangente del
  espacio de configuraciones $T^{*} \R_x \cong \R_x \times
  \R_p$, donde el espacio de configuraciones  $\R_x$ es el
  espacio de posición y el $\R_p$ es el espacio del
  momentum.  Para nuestros objetivos basta con desginar el
  espacio de fase como el espacio $\R^{2n}$.

  Notemos que por el momento no ha surgido ninguna
  interpretación probabilística en la mecánica clásica, pues
  es una teoría determinística. Dado que la teoría cuántica
  es probabilística, es natural preguntarnos ¿qué
  características de la mecánica clásica serían deseables en
  la teoría cuántica? y ¿qué beneficios habría en hacer un
  vínculo entre la mecánica clásica y la cuántica?
  \cite{schroeckQuantumMechanicsPhase1996}, especialmente
  cuando sabemos que la teoría cuántica (y sus derivados) es
  nuestra teoría más precisa. Un esfuerzo por vincular las
  descripciones clásicas y cuánticas del mundo es la
  representación de Wigner-Weyl de la mecánica cuántica.
  Ésta es una formulación que intenta usar la noción del
  espacio de fase en la dinámica cuántica y la idea básica
  es la construcción de \textit{cuasi-distribuciones}
  real-valuadas que representan a los sistemas cuánticos.

  \section{La teoría cuántica}

  La teoría cuántica ha tomado distintas direcciones despues
  de su concepción en los años viente y existen distintas
  formulaciones matematicamente equivalentes que surgieron
  despues de las teorias iniciales de Schrödinger (mecánica
  de ondas) y de Heisenberg (mecánica matricial).  La más
  común hoy en dia es la formulación en el espacio de
  Hilbert, la cual fue desarrollada de manera rigurosa por
  Von Neumann en 1932.  La segunda formulación más común,
  especialmente en la teoría cuantica de campos, es la
  formulación de la integral de trayectoria de Feynman
  desarrollada en 1948.  Otra formulación, de particular
  interés para nuestro trabajo, es la formulación en el
  espacio de fase, que tiene sus inicios en 1932 por Wigner,
  pero que solo fue desarrollada como una descripción
  completa de la mecánica cuántica despues de la segunda
  guerra mundial. 

  En cualquiera de las formulaciones, la mecánica cuántica
  como toda teoría física, permite el cálculo del
  comportamiento y las propiedades de sistemas físicos. Dado
  un sistema físico, se definen los \textit{observables}
  como las cantidades que podemos medir sobre el sistema,
  por ejemplo la temperatura de algún cuerpo. En la
  formulación de Schrödinger, a los sistemas físicos se le
  asocia un espacio de Hilbert separable. Los observables
  físicos son representados por los operadores auto-adjuntos
  definidos en algún subespacio del espacio de Hilbert. El
  estado de un sistema representa toda la información del
  sistema en algún momento y está dado por operadores
  auto-adjuntos que satisfacen ciertas condiciones
  adicionales. Cuando no hay incertidumbre sobre el estado
  en el que está el sistema, podemos representar el estado
  por un vector del espacio de Hilbert, y decimos que el
  sistema está en un estado puro.

  El ejemplo físico de mayor importancia para éste trabajo
  es el de una partícula moviendose en un espacio
  Euclideano. El espacio de Hilbert asociado a éste sistema
  generalmente es el espacio de las funciones
  cuadráticamente integrables $L^2(\R^{n})$. Los estados
  puros del sistema son los elementos de éste espacio y el
  operador correspondiente al observable de la posición de
  la partícula es el operador que multiplica una función por
  la coordenada. La mecánica cuántica nos dice que no
  podemos medir la posición exacta de una partícula en un
  estado arbitrario, lo único que podemos hacer es obtener
  la \textit{probabilidad} de encontrar a la partícula en
  algún subconjunto de $\R^{n}$.  Estadísticamente, nos
  interesa obtener el \textit{valor esperado} de la posición
  de la partícula en un estado en específico, así como el
  valor esperado de otros operadores de interés como lo son
  la energía del sistema, el momentum, el momentum angular,
  entre otros. La mecánica cúantica nos permite estudiar los
  sistemas y sus observables de manera probabilística y
  también nos permite describir su evolución temporal por
  medio de la ecuación de Schrödinger. 

  \subsubsection{Sobre el rigor matemático}

  La mayoría de los artículos y libros que se consultaron en
  éste trabajo sobre la mecánica cuántica en el espacio de
  fase y las cuasi-distribuciones en general, no son lo
  suficientemente rigurosos. Ésto tiene sentido porque la
  audiencia principal de éstos trabajos la conforman los
  físicos, los cuales aligeran el rigor con la finalidad de
  ser más eficientes y rápidos en la resolución de sus
  problemas.  Como mencionamos anteriormente von Neumann,
  entre otros, crearon un sistema matemático que describe a
  la mecánica cuántica de una manera completamente rigurosa,
  por medio del análisis funcional y en especial en la
  teoría espectral de los operadores lineales.  A pesar de
  ésto, es muy común no hacer uso de ella explícitamente, y
  en la experiencia del autor, su uso es casi nulo en la
  literatura sobre la función de Wigner. En su lugar, el
  físico utiliza la notación de Dirac, la cual más que
  notación, es una maquinaria para hacer cálculos la cual no
  es rigurosa en su formulación clásica pero sí brinda los
  resultados correctos. Como ejemplo de ésta falta de rigor,
  consideremos de nuevo a una partícula que se mueve en una
  sola dimensión:

  \begin{quote}
    Es útil para el físico hacer uso de una representación
    de un estado de la partícula, conocida como la
    representación en el espacio de posición o de momentum.
    Específicamente, consideremos el espacio de Hilbert $\H
    = L^2(\R)$.  El operador asociado a la posición de la
    partícula, $\hat{x}$, está dado por
    \[
      \hat{x} \psi = x \psi,
      \quad \psi \in L^2(\R).
    \] 
    La representación en el espacio de posición del vector
    $\psi$, está dada por
    \[
      \psi(x) = \langle x, \psi \rangle,
    \] 
    donde $x$ representa a su vez el `eigenvalor' y
    `eigenvector' del operador de posición. Existen
    varios detalles, unos más obvios que otros que nos
    pueden causar varios problemas. Para empezar el operador
    de posición $\hat{x}$ no es un operador acotado, no está
    definido en todo el espacio de Hilbert, y ni siquiera
    tiene eigenvectores verdaderos. La justificación del
    físico para usar ésta representación es que el
    eigenvector $\ket x$ es un \textit{eigenvector
    generalizado} del operador, dado por la función de
    Dirac, $\ket x := \delta(x - x')$.  Ésto le permite al
    físico hacer los cálculos necesarios y funciona! Pero
    evidemente ésto es un problema para el matemático por
    que la delta de Dirac no es una función, sino una
    distribución, así que ni siquiera es un elemento de
    $L^2(\R)$, mucho menos es un eigenvector del operador de
    posición en el sentido correcto del concepto. El
    espectro del operador de posición es todo $\R$ y el
    físico considera el conjunto de eigenvectores
    generalizados como una `base continua' de $\H$. Ésto se
    hace para poder expresar operadores en términos de ésta
    supuesta base continua de $L^2(\R)$, por ejemplo,
    definen la resolución de la identidad como
    \[
      \id_{\H}
      = \int_{\R} \ket x \bra x \, dx.
    \] 
    Donde el integrando, $\ket x \bra x$, representa el
    operador de proyección en la dirección de la eigenvector
    $\ket x$. La resolución espectral y otras identidades
    similares se introducen en los cálculos de las
    mediciones sobre los sistemas. Con la introducción de
    ficciones matemáticas como la delta de Dirac, se pueden
    expresar los operadores diferenciales como operadores
    integrales. 
  \end{quote}

  La realidad es que la teoría cuántica no requiere de la
  maquinaria de Dirac expuesta en el ejemplo anterior para
  hacer mediciones, la teoría espectral nos brinda todo lo
  que necesitamos para ésto. Además, en caso de querer hacer
  uso de la notación de Dirac forzosamente, existe una
  manera de hacerla rigurosa utilizando lo que se conoce
  como espacios de Hilbert \textit{equipados}. A pesar de
  ésto hemos decidido que para la primera parte de éste
  trabajo, en donde se introduce de manera didáctica a la
  función de Wigner en el caso continuo, no seremos
  totalmente rigurosos y haremos uso de la notación de Dirac
  en ocasiones y de la delta de Dirac muy seguido,
  ateniendónos a su validez debido a la teoría de
  distribuciones de Scwhartz. Ésto se debe a que en la parte
  principal de éste trabajo, estaremos haciendo uso de
  espacios de Hilbert de dimensión \textit{finita}, en donde
  todo se vuelve más sencillo en cierto sentido.

  \subsection{La formulación en el espacio de Hilbert}

  Con lo anterior aclarado, comencemos definiendo los
  conceptos y algunos de los axiomas que necesitaremos de la
  mecánica cuántica. No será necesario estudiar la evolución
  temporal del sistema. Para un repaso al análisis
  funcional, teoría de medida y algebra lineal que
  necesitaremos, el lector puede consultar el apéndice (?).

  \begin{axiom}
    A cada sistema cúantico le corresponde un espacio de
    Hilbert $\H$. Los estados del sistema son todos los
    operadores lineales $\rho : \H \to \H$,
    definidos-positivos y de traza finita, tales que $\Tr(
    \rho) = 1$.
  \end{axiom}

  Un estado cuántico $\rho$ se dice \textit{puro}, is
  existe un elemento $\psi \in \H$, tal que para todo
  $\alpha \in \H$ se cumple
  \[
    \rho(\alpha)
    = \frac{\braket{\psi | \alpha}}{\braket{\psi | \psi}}
    \psi
  \] 
  De ésta manera, cuando hablemos de un estado puro,
  podremos referirnos a un elemento $\psi \in \H$, algo que
  casi siempre sucede en la literatura física.

  \begin{axiom}
    A cada observable físico, $A$, sobre el espacio de fase
    clásico, le corresponde un observable cuántico
    representado por un operador auto-adjunto $\hat{A} :
    \mathcal D_{\hat{A}} \to \H$.  
  \end{axiom}

  Es importante aclarar que el dominio de un operador no es
  un detalle pequeño, ya que generalmente no están definidos
  en todo el espacio de Hilbert. Muchos problemas surgen del
  ignorar los detalles de los dominios.

  \begin{axiom}
    La probabilidad que una medición de una observable $A$ 
    sobre un sistema en el estado $\rho$ de un
    resultado en el conjunto de Borel $E \subset \R$ está
    dado por
    \[
      \mu_{\rho}^{A}(E)
      = \Tr\left( P_A(E) \circ \rho \right)
    \] 
    donde el mapa $P_A : \mathcal B(\R) \to \mathcal L(\H)$,
    es la medida proyección valuada con el mapa $A$ dado por
    el teorema espectral.
  \end{axiom}

  El teorema espectral nos dice que para cualquier operador
  auto-adjunto $A$, existe una medida
  proyección-valuada $P_{A}$ tal que $A$ puede
  ser representado por la integral
  \[
    A
    = \int_{\R} \lambda dP_{A}(\lambda).
  \] 
  Para operadores de espacios de dimensión finita ésto se
  reduce a la diagonalización de la matrices Hermíticas.

  Si un sistema cuántico está en un estado descrito por un
  vector unitario $\psi \in \H$, entonces el valor esperado
  de un observable $A$ satisface
  \[
    \langle \hat{A} \rangle_\psi
    := \langle \psi, \hat{A}\psi \rangle,
  \] 
  donde $\langle \cdot, \cdot \rangle$ es el producto
  interno del espacio de Hilbert en cuestion. De manera
  general, dado un estado $\rho$, el valor esperado de
  un observable $\hat{A}$ está dado por
  \[
    \Tr\left(\hat{A}\rho\right)
    = \Tr\left( \rho\hat{A} \right),
  \]
  donde $\Tr$ es la traza del operador.

  Folland.

  El estado de un sistema cuántico es el espacio de Hilbert
  proyectivo $\mathbb P\H$, es decir, el conjunto de todas
  las rectas complejas que pasan por el origin de un espacio
  de Hilbert $\H$. Los observables son las medidas
  proyección-valuadas de Borel sobre $\R$, es decir, los
  mapeos $\Pi$ de los conjuntos de Borel de $\R$ hacía las
  proyecciones ortogonales sobre $\H$ tales que $\Pi(\R) =
  \id_{\H}$ y si $E_1,E_2,\ldots$ son conjuntos disjuntos de
  Borel, entonces
  \[
    \Pi(E_j)\Pi(E_k) = 0, \quad j \neq k,
  \] 
  y
  \[
    \sum_{j}^{} \Pi(E_j) = \Pi\left( \bigcup_j E_j \right).
  \] 
  Si $\Pi$ es una medida que satisface lo anterior y $u \in
  \H$ es un vector unitario, entonces $E \mapsto \langle
  \Pi(E)u, u \rangle$ es un medida de probabilidad ordinaria
  sobre $\R$, y es la distribución de probabilidad del
  observable $\Pi$ en el estado $u$.

  Por medio del teorema espectral, las medidas
  proyección-valuadas $\Pi$ están en correspondencia
  uno-a-uno con los operadores auto-adjuntos $A$ sobre $\H$, 
  \[
    A = \int \lambda \, d\Pi(\lambda), 
    \quad \Pi(E) = \xi_E(A).
  \] 
  De ésta manera, podemos pensar que los observables son los
  operadores auto-adjuntos. Si $A$ es un operador
  auto-adjunto y $u$ es un vector unitario en el dominio de
  $A$, entonces el valor esperado del observable $A$ en el
  estado $u$ es
  \[
    \int \lambda \langle d\Pi(\lambda)u, u \rangle
    = \langle Au, u \rangle.
  \] 
  La distribución de probabilidad de $A$ en el estado $u$ 
  está concentrada en un punto en particular $\lambda$ 
  precisamente cuando $u$ es un eigenvector de $A$ con
  eigenvalor $\lambda$. Si el espectro de $A$ es solamente
  discreto, de tal manera que $A$ tiene una base ortonormal
  $\{e_j\}$ correspondiente a sus eigenvectores con
  eigenvalores $\{\lambda_j\}$, entonces la distribución de
  probabilidad de $A$ en cualquier estado está dado por
  \[
    E \mapsto \sum_{\lambda_j \in E}^{} |\langle u, e_j
    \rangle|^2.
  \] 

  \subsection{Los operadores de posición y de momentum}

  Los dos operadores indispensables para éste trabajo son
  los operadores de posición y de momentum. En particular
  pensemos en el sistema de una partícula moviendose en el
  espacio euclideano de dimensión $n$. El espacio de Hilbert
  es $L^2(\R^{n})$. Si $f \in L^2(\R^{n})$ es un vector
  unitario, entonces $|f|^2$ puede ser interpretado como la
  densidad de probabilidad de la posición de la partícula en
  el estado $f$, es decir, que la probabilidad de que la
  partícula se encuentre en el subconjunto $B \subset R^{n}$
  es $\int_B |f|^2$. 

  Con ésto identificamos a los operadores $Q_1,\ldots,Q_n$ 
  correspondientes a funciones coordenadas clásicas
  $q_1,\ldots,q_n$. En partícular, si $E \subset \R$ 
  entonces la probabilidad de que la $j$-ésima coordenada
  $x_j$ de la partícula se encuentre en $E$ está dada por
  \[
    \int_{x_j \in E} |f(x)|^2 \, dx.
  \] 
  Por lo tanto la medida proyección-valuda $\Pi_j$ para
  dicho observable será dada por
  \[
    \Pi_j(E) = 
    \text{la multiplicación por la función característica de
    } \{x : x_j \in E\},
  \] 
  y se sigue que el operador 
  \[
    Q_j = \int \lambda d\Pi_j(\lambda)
  \] 
  es la multiplicación por la $j$-ésima función coordenada,
  generalmente denotado por $X_j$, 
  \[
    Q_jf(x) = X_jf(x) = x_jf(x),
  \] 
  definido para todo $f \in L^2(\R^{n})$ tal que $x_jf \in
  L^2(\R^{n)}$.

  Notemos que no existen estados $f \in L^2(\R^{n})$ para
  los cuales los observables $Q_j$ tienen valores
  definitivos. Las eigenfunciones de los $Q_j$ son las
  deltas de Dirac $\delta_{x_0}(x) = \delta(x - x_0)$. Éstas
  se pueden considerar como un conjunto de estados `estados
  idealizados' y que forman una `base ortonormal continua'
  \[
    \langle \delta_{x_1}, \delta_{x_2} \rangle
    = \int \delta(x-x_1)\delta(x-x_2) \, dx
    = \delta(x_1-x_2),
  \] 
  \[
    f = \int f(x)\delta_x \, dx
    = \int \langle f, \delta_x \rangle \delta_x \, dx,
  \] 
  donde las integrales se interpretan en el sentido de
  distribuciones.


  .

  Informalmente, los eigenvectores del operador de posición
  forman un conjunto completo de vectores, es decir, podemos
  expandir cualquier estado $\ket \psi$ en los eigenvectores de
  posición. La relación de completitud es
  \[
    \id_{\H}
    = \int_{\R} \ket x \bra x \, dx, 
  \] 
  y así podemos representar a $\ket \psi$ como
  \begin{equation}
    \ket \psi
    = \id_{\H} \ket \psi
    = \int_{\R} \ket x \bra x \ket \psi \, dx
    = \int_{\R} \psi(x) \ket x \, dx
  \end{equation}
  donde $\psi(x)$ es la \textit{función de onda} 
  \begin{equation}
    \psi(x)
    = \braket{x|\psi},
  \end{equation}
  del estado $\ket \psi$ en el \textit{espacio de posición}.
  Similarmente podemos expresar el estado $\ket \psi$ 
  respecto a los eigenvectores del operador de momentum
  utilizando la relación de completitud:
  \[
    \id_{\H}
    = \int_{\R} \ket p \bra p \, dp.
  \] 
  Analogamente, ésto nos brinda
  \begin{equation}
    \ket \psi
    = \int_{\R} \ket p \bra p \ket \psi \, dp
    = \int_{\R} \psi(p) \ket p \, dp,
  \end{equation}
  donde la función onda en el \textit{espacio de posición}
  es
  \begin{equation}
    \psi(p)
    = \braket{p|\psi}.
  \end{equation}
  Las funciones $\psi(x)$ y $\psi(p)$ son dos
  representaciones del \textit{mismo} estado $\ket \psi$.
  Más adelante veremos que la transformada de Fourier
  conecta la función de posición $\psi(x)$ con la del
  momentum $\psi(p)$, 
  \begin{equation}
    \psi(x)
    = \frac{1}{\sqrt{2\pi\hbar}} \int_{\R} \psi(p)e^{ixp /
    \hbar} \, dp.
  \end{equation}

  \subsection{El principio de incertidumbre}

  A los operadores de posición y de momentum se les conoce
  como operadores canónicos y satisfacen la siguiente
  relación
  \begin{equation}
    \hat{x} \hat{p} - \hat{p} \hat{x} 
    = -i\hbar.
  \end{equation}
  El hecho de que no conmutan tiene consecuencias que
  permean toda la teoría cuántica. . . 

  \subsection{En dimensión finita}

  %%%-----------------------------------------------------

  \chapter{Función de Wigner}

  \section{Mecánica cuántica en el espacio de fase}

  El principio de la incertidumbre hace que el concepto de
  un espacio de fase de la mecánica cuántica, análogo al de
  la mecánica clásica, sea problemático. Ésto se debe a que
  la posición y el momentum de una partícula no ésta bien
  definida de manera simultánea, así que no se puede definir
  una densidad de probabilidad conjunta verdadera. Aún así
  exiten funciones que se \textit{asemejan} a distribuciones
  sobre el espacio de fase, llamadas funciones de
  distribución quasi-probabilísticas, las cuales son útiles
  de manera práctica, además de vincular de cierta manera a
  la mecánica clásica y la cuántica. Ésto sucede porque
  dichas quasi-distribuciones nos permiten calcular los
  valores esperados de los operadores cuánticos de manera
  muy similar a los promedios clásicos de la mecánica
  estadística. La primera y más conocida de éstas funciones
  es la cuasi-distribución ó función de Wigner.

  La función de Wigner tiene una larga historia que inicia
  con un artículo publicado por Eugene P. Wigner en 1932
  \cite{wignerQuantumCorrectionThermodynamic1932} sobre
  correcciones cuánticas del equilibrio termodinámico. La
  idea principal de Wigner fue introducir una
  quasi-distribución que le permite calcular valores
  esperados cuánticos de una manera análoga a los valores
  esperados de la mecánica estadística. Para el caso de un
  estado puro $\psi$ obtenemos la expresión que expuso
  Wigner originalmente:
  \begin{equation}
    \label{eqn:wigners_original}
    W(x,p)
    = \frac{1}{2\pi\hbar} \int
    \psi(x+\tfrac{1}{2}y)\overline{\psi(x-\tfrac{1}{2}y)}
    e^{\frac{i}{\hbar} p y} \, dy,
  \end{equation}
  donde la integral es sobre todo $\R$ y $\psi$ está en su
  representación en el espacio de posición. Wigner partió de
  la mecánica estadística clásica, la cual nos dice que
  cuando se tiene un conjunto de partículas, la evolución
  del sistema puede ser estudiado de manera probabilística
  mediante la ecuación de Liouville, la cual nos brinda una
  distribución sobre el espacio de fase [?]. La idea es
  considerar una distribución en el espacio de fase que nos
  permita hacer las mismas interpretaciones probabilísticas
  de la mecánica cuántica que podemos hacer con la
  formulación en el espacio de Hilbert.

  Unos años antes, Hermann Weyl formuló una manera de mapear
  observables clásicos a observables cuánticos, lo cual se
  conoce como la cuantización de Weyl [?]. Debido a la no
  conmutatividad de los operadores canónicos $\hat{x}$ y
  $\hat{p}$, la cuantización de un observable clásico no es
  único y es necesario introducir un orden específico. Para
  el mapeo de Weyl, el orden utilizado para los operadores
  canónicos es el orden simétrico ejemplificado por
  \[
    xp \mapsto
    \frac{1}{2}\left( \hat{x} \hat{p} + \hat{p} \hat{x}
    \right).
  \]
  con éste acorde, podemos mapear polinomios de las
  variables $x$ y $p$ a operadores lineales en algún espacio
  de Hilbert correspondiente. Más aún, utilizando a la
  transofrmación de Fourier, el mapeo Weyl nos produce un
  operador a partir de alguna función de $x$ y $p$ en el
  espacio de fase con ciertas condiciones de regularidad. Se
  verá más adelante que la transformación de Weyl y de
  Wigner son inversos para una clase de funciones y
  operadores.  Weyl y Wigner presentaron sus mapas con
  distintas intenciones y al parecer ninguno tuvo el interés
  en buscar el mapa inverso, ni buscaron una forma para
  desarollar un producto no conmutativo en el espacio de
  fase. La formulación de la mecánica cuántica en el espacio
  de fase, equivalente a las otras formulaciones, fue
  trabajo de Enrique Moyal y de Groenewold de manera
  independiente \cite{curtrightQuantumMechanicsPhase2012}.
  El trabajo de Groenewold (en forma de su tésis de 1946)
  desarrolló los fundamentos de la mecánica cuántica en el
  espacio de fase. Su trabajo fue el primero en reconocer
  que el mapa de Weyl es realmente una transformación
  invertible, y \textit{no} solo una regla de cuantización
  \cite{todorovQuantizationMystery2012}. Por otro lado Moyal
  desarrolló sus ideas sobre la naturaleza estadística de la
  mecánica cuántica, y en su formulación se introduce un
  producto en el espacio de fase correspondiente al producto
  de operadores en el espacio de Hilbert, tal producto se le
  conoce como el producto-$\star$ ó producto de Moyal.

  

  \section{Transformación de Wigner}

  \subsection{Motivación}

  Lo que sigue es un intento de justificar y motivar de
  manera informal la idea detrás de las cuasi-distribuciones
  y en particular la de Wigner. Luego daremos las
  definiciones con las cuales estaremos trabajando con el
  fin de probar las propiedades de la transformación de
  Wigner que nos interesa preservar al pasar al espacio
  discreto. En caso de que el lector esté interesado en la
  formulación completa de la mecánica cuántica en el espacio
  de fase, le invtamos a consultar los trabajos de Moyal
  [?], Groenwold [?], Cohen [?] y Gossman [?].

  Dada una densidad probabilística de Liouville la cual se
  obtuvo a partir de resolver la ecuación de Liouville
  respecto a un conjunto de partículas, y dado un observable
  $A : \R^{2n} \to \R$, que depende de la posición $x$ y del
  momentum $p$ de una partícula, podemos calcular el valor
  esperado del observable como
  \begin{equation}
    \mathbb E[a]
    = \iint A(x,p) F(x,p) \, dx \, dp,
  \end{equation}
  donde $F$ es la densidad de probabilidad de Liouville. La
  intención, dada la naturaleza probabilística de la mecánica
  cuántica, es poder calcular valores esperados
  de observables cuánticos de manera análoga, es decir,
  mediante una integración en el espacio de fase. 

  En lo que sigue nos enfocamos a una partícula en una sola
  dimensión, pero cabe mencionar que se puede extender a
  varias dimensiones sin mucho problema. Sea $A$ un
  observable y $\rho$ un estado de un sistema cuántico,
  entonces siguiendo la idea anterior, nos gustaría poder
  hacer el siguiente cálculo
  \begin{equation}
    \Tr\left(\rho \hat{A}\right)
    = \iint A(x,p)F(x,p) \, dx \, dp,
  \end{equation}
  donde $F$ es una densidad probabilística en el espacio de
  fase relacionada al estado $\rho$.

  Evidentemente lo primero que necesitamos es una
  correspondencia entre operadores (incluyendo el operador
  de densidad) en el espacio de Hilbert y funciones
  integrables en el espacio de fase. Asumiendo que nuestro
  operador depende de los operadores $\hat{x}$ y $\hat{p}$,
  es natural intentar el simple reemplazo formal de los
  operadores por las variables $x$ y $p$,
  \[
    \hat{A}(\hat{x},\hat{p})
    \mapsto 
    A(x,p).
  \] 
  Inmediatamente nos econtramos en un problema con el
  ordenamiento de los operadores, ya que en el espacio de
  fase las variables $x$ y $p$ conmutan, mientras que los
  operadores correspondientes no son conmutativos. Por lo
  tanto corremos el riesgo de asignar la misma función
  escalar $A(x,p)$ a distintos operadores
  $\hat{A}(\hat{x},\hat{p})$.

  Para ver ésto, consideremos en particular al operador
  $e^{i\left( \xi \hat{x} + \eta \hat{p} \right)}$, donde
  $\xi, \eta \in \R$. Reemplazando los operadores por las
  variables, el valor esperado sería de la forma
  \begin{equation}
    \Tr\left( \rho e^{i\left( \xi \hat{x} + \eta
    \hat{p} \right)} \right)
    = \iint e^{i\left( \xi x + \eta p \right)} F(x,p) \,
    dx \, dp
  \end{equation}
  con una densidad particular $F(x,p)$. Ahora consideremos
  el operador $e^{i\xi \hat{x}} e^{i\eta\hat{p}}$, siguiendo
  el mismo razonamiento obtenemos
  \begin{equation}
    \Tr\left(\rho e^{i\xi \hat{x}} e^{i\eta
    \hat{p}}\right)
    = \iint e^{i\xi x} e^{\eta p} \tilde F(x,p) \, dx \,
    dp
    = \iint e^{i\left( \xi x + \eta p \right)} \tilde
    F(x,p) \, dx \, dp.
  \end{equation}
  Notemos que la función en el espacio de fase que aparece
  en el integrando es la misma para ambos operadores. De
  hecho, utilizando la identidad Baker-Campbell-Hausdorf
  sabemos que
  \begin{equation}
    e^{i\left( \xi \hat{x} + \eta \hat{p} \right)}
    = e^{i\xi \hat{x}}e^{i\eta \hat{p}}e^{-\frac{1}{2}i\hbar
    \xi \eta}
    \neq e^{i\xi \hat{x}}e^{i\eta \hat{p}}.
  \end{equation}
  Así que en general las distribuciones $F$ y $\tilde F$
  deben ser distintas si no obtendremos los mismos valores
  esperados para distintos operadores en un mismo estado.
  Distintas reglas asociando funciones de operadores no
  conmutativos con sus correspondientes funciones escalares
  nos darán distintas distribuciones
  \cite{leeTheoryApplicationQuantum1995}. En el caso del
  ordenamiento de Weyl, se puede probar que 
  \[
    e^{\xi x + \eta p} \mapsto e^{\xi \hat{x} + \eta
    \hat{p}},
  \] 
  por lo tanto eligiendo ésta asignación podemos definir la
  densidad $F$ mediante la ecuación
  \begin{equation}
    \Tr\left(
      \rho e^{i(\xi \hat{x} + \eta \hat{p})}
    \right)
    = \iint e^{i(\xi x + \eta p)}F(x,p) \, dx \, dp.
  \end{equation}
  Nos interesa obtener una expresión para $F$, observando el
  integrando, utilizamos la transformación de Fourier (en
  dos dimensiones) para obtener una expresión de la densidad
  en términos de la traza:
  \begin{equation}
    \label{eqn:density_from_trace}
    F(x,p)
    = \frac{1}{(2\pi)^2} \iint \Tr\left( \rho 
      e^{i\left( \xi \hat{x} + \eta \hat{p} \right)} \right)
      e^{-i\left( \xi x + \eta p\right)} \, d\xi \, d\eta.
  \end{equation}

  Enseguida buscamos expresar el integrando sin la traza y
  sobre todo sin los operadores $\hat{x}$ y $\hat{p}$.
  Hacemos esto expresando a la traza mediante la
  ``expansión'' de los operadores $\hat{A}$ y $\rho$ en
  términos de la posición y luego hacioendo explicíto la
  aplicación del operador exponencial sobre los vectores
  $\ket x$. Primero
  \begin{equation}
    \Tr\left( \rho \hat{A} \right) 
    = \iint \braket{x|\rho|x'} \braket{x'|\hat{A}|x} \, dx
    \, dx'.
  \end{equation}
  Luego, el operador $e^{i\eta \hat{p}}$ actúa sobre $\ket
  x$ como una traslación en el espacio de posición, es decir
  \[
    e^{i\eta \hat{p}} \ket x
    = \ket{x - \eta \hbar},
  \] 
  y el operador $e^{i\xi \hat{x}}$ actúa sobre un $\ket x$ 
  simplemente como una multiplicación por un factor de fase,
  \[
    e^{i\xi \hat{x}} \ket x
    = e^{i\xi x} \ket x.
  \] 
  Finalmente recordando que $e^{i(\xi \hat{x} + \eta
  \hat{p})} = e^{i \xi \hat{x}} e^{i \eta \hat{p}}
  e^{-\frac{1}{2}i\hbar \xi \eta}$, obtenemos
  \begin{align*}
    \Tr\left( \rho e^{i(\xi \hat{x} + \eta \hat{p})} \right) 
    &= \iint \braket{x|\rho|x'} \braket{x'| e^{i(\xi \hat{x}
    + \eta \hat{p})}|x} \, dx \, dx' \\
    &= \iint \braket{x|\rho|x'} \braket{x'|e^{i\eta
    \hat{p}}|x} e^{i\xi x}e^{-\frac{1}{2} i \hbar \xi \eta}
    \, dx \, dx' \\
    &=  \iint \braket{x|\rho|x'} \braket{x'|x - \eta \hbar}
    e^{i\xi x}e^{-\frac{1}{2} i \hbar \xi \eta} \, dx \, dx'
    \\
    &= \int \braket{x|\rho|x - \eta \hbar} e^{i\xi (x
    -\frac{1}{2} \hbar \eta)} \, dx.
  \end{align*}

  Sustituyendo ésta expresión de la traza en el integrando
  de (\ref{eqn:density_from_trace}), y haciendo el cambio de
  variable $x'' = x' - \frac{1}{2}\eta\hbar$, obtenemos una
  expresión que ya no incluye a los operadores $\hat{x}$ y
  $\hat{p}$:
  \begin{align*}
    F(x,p)
    &= \frac{1}{(2\pi)^2} \iiint \braket{x'|\rho|x' -
    \eta\hbar} e^{i\xi(x' - \frac{1}{2}\hbar \eta)}
    e^{-i(\xi x + \eta p)} \, d\xi \, d\eta \, dx'\\
    &= \frac{1}{(2\pi)^2}\iiint \braket{x'|\rho|x'-\eta
    \hbar} e^{i\xi(x' - \frac{1}{2}\hbar\eta - x)} e^{-i\eta
    p} \, d\xi \, d\eta \, dx' \\
    &= \frac{1}{(2\pi)^2}\iiint \braket{x'' + \tfrac{1}{2}
    \eta \hbar|\rho|x'' - \tfrac{1}{2} \eta \hbar}
    e^{i\xi(x'' - x)} e^{-i\eta p} \, d\xi \, d\eta \, dx''.
  \end{align*}

  Utilizando la representación integral de la función delta
  de Dirac (respecto a $\xi$) y luego integrando sobre $x'$,
  las cosas se simplifican aún más:
  \begin{align}
    F(x,p)
    &= \frac{1}{2\pi} \iint \braket{
    x'+\tfrac{1}{2}\eta\hbar | \rho|
    x'-\tfrac{1}{2}\eta\hbar} \delta(x'-x) e^{-i\eta p} \,
    d\eta \, dx' \\
    &= \frac{1}{2 \pi} \int \braket{x +
    \tfrac{1}{2}\eta\hbar | \rho | x - \tfrac{1}{2}
    \eta\hbar}e^{-i \eta p} \, d\eta.
  \end{align}
  Haciendo un cambio de variable más, $y = -\eta \hbar$, por
  fin llegamos a lo que definimos como la transformación de
  Wigner de un estado arbitrario:
  \begin{equation}
    W(x,p)
    := \frac{1}{2\pi\hbar} \int \braket{x -
      \tfrac{1}{2}y|\rho|x+\tfrac{1}{2}y} e^{\frac{i}{\hbar}
    y p} \, dy.
  \end{equation}
  Si el estado $\rho = \ket \psi \bra \psi$ es puro,
  inmediatamente llegamos a la expresión de Wigner
  (\ref{eqn:wigners_original}) utilizando la representación
  de la función de onda. Formalmente se puede extender ésta
  idea para transformar operadores que no son operadores de
  densidad, aunque justificar ésto de manera rigurosa no es
  tan trivial. En ésto caso la transformación de Wigner de
  un operador lineal $\hat{A}$ es
  \begin{equation}
    A(x,p)
    = \frac{1}{2 \pi \hbar} \int
    \braket{x-\tfrac{1}{2}y|\hat{A}|x+\tfrac{1}{2}y}
    e^{\frac{i}{\hbar}p y} \, dy.
  \end{equation}

  Integrando el producto de las funciones de Wigner de un
  estado $\rho$ y de un observable $\hat{A}$ sobre el
  espacio de fase se puede demostrar fácilmente que
  efectivamente hemos calculado el valor esperado del
  observable, cumpliendo con nuestro objetivo inicial, es
  decir,
  \[
    \Tr\left( \rho\hat{A} \right) 
    = \iint A(x,p)W(x,p) \, dx \, dp.
  \] 

  Hemos encontrado (al menos de una manera formal) una
  distribución sobre el espacio de fase que describe un
  sistema cuántico, desafortunadamente, la distribución de
  Wigner que obtuvimos no nos brinda una distribución
  probabilística verdadera, pues puede tomar valores
  negativos. Sin embargo, nos permite calcular los valores
  esperados de los operadores cuánticos y otras cantidades
  probabilísticas como las densidades de posición y
  momentum, y por ésto se le otorga el nombre de una
  cuasi-distribución. En la siguiente sección definimos el
  concepto de la transformación de Wigner dentro de ciertos
  espacios de funciones más restringidos para poder darle un
  pococ de validez matemática al procedimiento anterior.

  \subsection{La transformación de Wigner}

   Lo que sigue está
  principalmente basado en los libros de Gosson
  \cite{gossonWignerTransform2017} y de Folland
  \cite{follandHarmonicAnalysisPhase1989}, quienes
  introducen la función de Wigner de manera general, como
  una transformación integral entre ciertos espacios de
  funciones. Despues de definir y mostrar ciertas
  propiedades importantes de la transformación de Wigner,
  introducimos la transformación de Weyl la cual nos dará la
  inversa de la transformación de Wigner bajo ciertas
  condiciones. Despues de formalizar la correspondencia
  entre operadores sobre espacios de Hilbert y las funciones
  en el espacio de fase, particularizamos los conceptos en
  el contexto de la mecánica cuántica. 

  Iniciamos definiendo una versión más general de la
  transformación de Wigner, la cual se conoce como la
  \textit{transformación de Wigner cruzada}. Ésta
  transformación ya no está actuando sobre operadores
  lineales, sino que actúa sobre el espacio de Schwartz de
  la funciones rápidamente decrecientes $\Sz(\R^{n}) \subset
  L^2(\R^{n})$. Ésto es un detalle que rara vez se menciona
  en los artículos físicos sobre la función de Wigner.
  \begin{definition}
    Sean $\psi, \phi \in \Sz(\R^{n})$, definimos la
    transformación de Wigner cruzada como
    \begin{equation}
      \label{eqn:cross_wigner_transform}
      W(\psi,\phi)(x,p)
      = (2\pi\hbar)^{-n} \int_{\R^{n}} e^{-\frac{i}{\hbar} p
      \cdot y} \psi(x + \tfrac{1}{2}y) \overline{\phi(x -
      \tfrac{1}{2}y)} \, dy,
    \end{equation}
    para $x,p \in \R^{n}$.
  \end{definition}
  En partícular, definimos la transformación de Wigner de un
  elemento $\psi$ de $\Sz(\R^{n})$ como $W(\psi,\psi)$ y la
  denotamos $W\psi$ por brevedad. Podemos extender la
  definición de la transformación de Wigner de manera única
  al espacio de las funciones cuadráticamente integrables.
  \begin{proposition}
    La transformación de Wigner $W : \Sz(\R^{n}) \times
    \Sz(\R^{n}) \to \Sz(\R^{2n})$ se puede extender de
    manera única al operador sesquilineal
    \[
      W : L^2(\R^{n}) \times L^2(\R^{n}) \to L^2(\R^{2n}),
    \] 
    tal que
    \[
      \|W(\psi,\phi)\|_{L^2(\R^{2n})}
      = \|f\|_{L^2(\R^{n})} \|g\|_{L^2(\R^{n})}
    \] 
    para todo $\psi, \phi$ en  $L^2(\R^{n})$.
  \end{proposition}
  Aún más, podemos extender la transformación de Wigner
  cruzada a las distribuciones templadas $\Sz'(\R^{n})$, las
  cuales forman un espacio dual a $\Sz(\R^{n})$.
  \begin{proposition}
    La transformación de Wigner cruzada $W : \Sz(\R^{n})
    \times \Sz(\R^{n}) \to \Sz(\R^{2n})$ se extiende en un
    mapa sesquilineal
    \[
      W : \Sz'(\R^{n}) \times \Sz'(\R^{n}) \to \Sz'(\R^{n})
    \] 
    por medio de la definición
    \[
      W(\psi,\phi)
      = (2\pi\hbar)^{-n / 2} (I_d \otimes \mathcal F_2)
      V(\psi \otimes \overline{\phi}),
    \] 
    donde
    \[
      (f \otimes g)(x,y) = f(x)g(y),
    \] 
    y donde $\mathcal F_2$ es la transformada de Fourier
    parcial sobre las variables $p$ y $V$ es un operador de
    cambio de coordenadas sobre $\R^{n}$ definido como
    \[
      Vf(x,y) = f\left( x + \frac{1}{2}y, x - \frac{1}{2}y
      \right).
    \] 
  \end{proposition}
  Las pruebas de las dos proposiciones anteriores se pueden
  encontrar en el libro de Gosson. Lo importante para
  nosotros es que ahora podemos definir la transformación de
  Wigner sobre distribuciones templadas, lo cual será
  necesario para vincular la transformación de Wigner con la
  de Weyl, lo que a su vez nos permite calcular la
  transformación de una clase amplia de operadores.
  Enseguida enunciamos algunas de las propiedades
  importantes de la transformación de Wigner. 

  Ahora enunciamos a varias propiedades de la transformación
  de Wigner cruzada, que son esenciales para usarla en la la
  mecánica cuántica. Como se mencionó anteriormente, la
  transformación cruzada es un mapa sesquilineal, y en
  general no es aditiva, es decir $W(\psi + \phi) \neq W\psi
  + W\phi$. Aparecen algunos términos cruzados que según
  Gosson tienen que ver con la interferencia:
  \begin{proposition}
    Supongamos que $\psi \in L^2(\R^{n})$ es una combinación
    lineal de elementos de $L^2(\R^{n})$, i.e.,
    \begin{equation}
      \psi = \sum_{j=1}^{m} \lambda_j \psi_j,
    \end{equation}
    entonces
    \begin{equation}
      W\psi
      = \sum_{j=1}^{m} |\lambda_j|^2 W\psi_j
      + 2 \Re\left(\sum_{k=1,k > l}^{m} \sum_{l=1}^{m}
      \lambda_k \overline{\lambda_l} W(\psi_k,
    \psi_l)\right).
    \end{equation}
  \end{proposition}

  \begin{proposition}
    Si $\psi, \phi \in L^2(\R^{n})$, entonces $W(\psi,\phi)
    = \overline{W}(\phi,\psi)$. En particular
    \[
      W\psi
      = W(\psi,\psi)
      = \overline{W}(\psi,\psi)
      = \overline{W\psi},
    \] 
    por lo tanto $W\psi$ es real-valuada. 
  \end{proposition}

  A pesar de ser real, no siempre es positivo\ldots
  Existe un resultado de Soto y Claviere [?], que dice que
  $W\psi$ es no-negativa sí y solo sí, $\psi$ es una
  Gaussiana generalizada.
  \begin{proposition}
    Sean $\psi \in L^2(\R^{n})$ y supongamos que $W\psi \in
    L^{1}(\R^{2n})$. Entonces $W\psi \in L^{1}(\R^{2n})$ y
    \begin{equation}
      \int_{\R^{2n}} (W\psi)(x,p) \, dx \, dp 
      = \|\psi\|^2.
    \end{equation}
  \end{proposition}

  \begin{proposition}
    Supongamos que $\psi, \phi \in L^2(\R^{n})$. La función
    $z \mapsto W(\psi,\phi)(z)$ es acotada y continua en
    $\R^{2n}$.
  \end{proposition}

  \begin{proposition}
    Sean $\psi,\phi \in \Sz'(\R^{n})$. La función $z
    \mapsto W(\psi,\phi)(z)$ es continua en $\R^{2n}$.
  \end{proposition}

  \begin{proposition}
    Supongamos que $\psi, \phi \in L^{1}(\R^{n}) \cap
    L^2(\R^{n})$. Entonces
    \begin{equation}
      \int_{\R^{n}} W(\psi,\phi)(x,p) \, dp
      = \psi(x)\overline{\phi}(x),
    \end{equation}
    \begin{equation}
      \int_{\R^{n}} W(\psi,\phi)(x,p) \, dx
      = \F\psi(p) \overline{\F\phi}(p).
    \end{equation}
    En particular tenemos
    \begin{equation}
      \int_{\R^{n}} W\psi(x,p) \, dp
      = |\psi(x)|^2,
      \quad
      \int_{\R^{n}} W\psi(x,p) \, dx
      = |\F\psi(p)|^2.
    \end{equation}
  \end{proposition}
  De las proposiciones anteriores ya podemos ver la utilidad
  que nos puede brindar la transformación de Wigner, ya que
  por medio de integrales sobre algún eje podemos recuperar
  las supuestas densidades de la posición y momentum de las
  funciones de onda. Aún con éste primer acercamiento, no
  hemos hablado sobre la transformación de Wigner de estados
  cuánticos ya sean puros o mixtos, ni de operadores en
  general. Para ésto nos conviene definir la transformación
  de Weyl, la cual nos permitirá asociar observables
  clásicos en el espacio de fase con operadores lineales de
  un espacio de Hilbert.

  \subsection{La transformación de Weyl}

  El problema de la cuantización consiste en encontrar una
  correspondencia entre funciones sobre el espacio de fase
  $\R^{2n}$ y operadores auto-adjuntos sobre $L^2(\R^{n})$,
  tales que las propiedades de los observables clásicos se
  reflejen lo más posible en sus correspondientes
  observables cuánticos, en una manera consistente con la
  interpretación probabilística de la mecánica cuántica.
  Dada la no conmutatividad de los operadores de posición y
  de momentum, no hay una correspondencia única, por lo
  tanto se restringe el mapeo de una manera ad-hoc, buscando
  satisfacer ciertas propiedades razonables. Entre ellas,
  deseamos que los operadores correspondientes a las
  coordenadas de posición y de momentum $q_j$ y $p_j$ deben
  ser los operadores $Q_j$ y $P_j$, ésto incluye a los
  polinomios. Otras propiedades que sea desean tener es que
  si $f,g$ son observables clásicos, entonces el valor
  esperado de $A_{f+g}$ en cualquier estado debe ser la suma
  de los valores esperados  $A_f$ y $A_g$, y que el operador
  identidad corresponda a la función $f(x,p) = 1$. Entre más
  restricciones imponemos menos probable es que encontremos
  una correspondencia que satisfaga todo, y de hecho se
  puede probar que no existe una cuantización que satisface
  todas las propiedades que los físicos desean. 

  El procedimiento de cuantización más natural y más
  estudiado es el que se conoce como la cuantización de
  Weyl.  La idea consiste en primero considerar la
  representación de Fourier de un observable clásico $A :
  \R^{2n} \to \R$ en el espacio de fase,
  \begin{equation}
    A(x,p)
    = (2\pi\hbar)^{-n} \int_{\R^{2n}} \F[A](\xi,
    \eta) e^{\frac{i}{\hbar} \left( \xi \cdot x + \eta \cdot
    p\right) } \, d\xi \, d\eta,
  \end{equation}
  donde $\F[A]$ es la transformada de Fourier de $A$. Luego
  lo que hacemos es reemplazar de manera formal a las
  variables $x$ y $p$ por los operadores $\hat{x}$ y
  $\hat{p}$, siguiendo el ordenamiento de Weyl. Bajo ciertas
  condiciones de regularidad que se le piden a $A$,
  obtenemos al operador $\hat{A}$ correspondiente:
  \begin{equation}
    \label{eqn:weyl_quant_1}
    \hat{A}(\hat{x},\hat{p})
    = (2\pi\hbar)^{-n} \int_{\R^{2n}} \F[A](\xi,\eta)
      e^{\frac{i}{\hbar} \left( \xi \cdot \hat{x} + \eta
      \cdot \hat{p}\right) } \, d\xi \, d\eta.
  \end{equation}
  La integral anterior se puede considerar como una integral
  de Bochner, pero enseguida le daremos una definición
  operacional que nos será más útil.

  La hipotesis de que el observable clásico $A$ es Fourier
  transformable no es un requisito muy estricto, y
  naturalmente no nos garantiza mucho sobre la naturaleza
  del operador de Weyl correspondiente. De una manera más
  restrictiva, se ha probado que la transformación de Weyl
  de una función cuadraticamente integrable es un operador
  Hilbert-Schmidt sobre el espacio $L^2(\R^{n})$ [?]. Los
  operadores de Hilbert-Schmidt son operadores lineales
  acotados con norma de Hilbert-Schmidt finita. Un resultado
  más general es el que obtuvo Wong [WONG], quien ha probado
  que si el observable es un elemento de $L^{p}(\R^{2n})$
  con $p \in [1,2]$, entonces el operador es acotado.
  
  Principalmente se ha estudiado un caso partícular de las
  funciones lisas sobre el espacio de fase. Siguiendo a
  Gosson de nuevo, en el siguiente análisis consideramos el
  espacio de Schwartz de las funciones rapidamente
  decrecientes $\Sz(\R^{2n})$ y su dual. Para el espacio de
  Schwartz (y su dual por extensión), Gosson prueba que la
  transformación de Weyl es una biyección entre los
  observables y los operadores Hilbert-Schmidt. En general
  podemos definir una cuantización como un mapeo $\Op :
  \Sz(\R^{2n}) \to \mathcal L\left(\mathcal S(\R^{n}),
  \Sz(\R^{n})\right)$, que asocia a una función $A \in
  \Sz(\R^{2n})$ generalmente llamada \textit{símbolo}, un
  operador continuo $\hat{A} = \Op(A)$ definido en algún
  subespacio denso de $L^2(\R^{n})$, por ejemplo
  $\Sz(\R^{n})$.

  Comenzamos dándole un sentido riguroso a la exponencial
  que aparece en el integrando de (\ref{eqn:weyl_quant_1}).
  Ésto lo podemos hacer definiendo el siguiente operador
  \begin{equation*}
    \hat{M}(z_0)
    = e^{\frac{i}{\hbar} \left( x_0 \cdot \hat{x} + p_0
    \cdot \hat{p} \right) },
  \end{equation*} 
  el cual actúa sobre alguna función $\psi$ mediante
  \begin{equation}
    \hat{M}(\xi,\eta)\psi(x)
    = e^{\frac{i}{\hbar} \left( \xi \cdot x + \frac{1}{2}
    \eta \cdot \xi \right)} \psi(x + \eta).
  \end{equation} 
  Con ésto podemos definir la cuantización de un símbolo de
  manera operacional:
  \begin{definition}
    Sea $A \in \Sz(\R^{2n})$. El operador de Weyl, $\hat{A}
    = \Op_W(A)$ del símbolo $A$ se define para $\psi \in
    \Sz(\R^{n})$ como
    \begin{equation}
      \label{eqn:weyl_quant_2}
      \hat{A}\psi(x)
      = \frac{1}{(2\pi\hbar)^{n}}
      \int_{\R^{2n}} \F[A](\xi,\eta) \hat{M}(\xi,\eta)
      \psi(x) \, d\xi \, d\eta,
    \end{equation}
    donde $\F$ es la transformación de Fourier y $\hat{M}$ es
    el operador definido anteriormente.
  \end{definition}
  La expresión anterior se puede escribir en términos de la
  función $A$ directamente al introducir el operador de
  Grossmann-Royer.
  \begin{definition}
    El operador de Grossmann-Royer $\hat{R}(\xi,\eta)$ se
    define como
    \[
      \hat{R}(\xi,\eta)\psi(x)
      = e^{\frac{2i}{\hbar} \eta \cdot (x - \xi)} \psi(2\xi
      - x).
    \] 
  \end{definition}
  \begin{definition}
    Sea $A \in \Sz(\R^{2n})$. El operador de Weyl $\hat{A} =
    \Op_W(A)$ está dado para todo $\psi \in \Sz(\R^{n})$ 
    como
    \begin{equation}
      \left( \hat{A}\psi \right)(x)
      = (\pi\hbar)^{-n} \int_{\R^{2n}}
      A(\xi,\eta)\hat{R}(\xi,\eta)\psi(x) \, d\xi \, d\eta,
    \end{equation}
    donde $\hat{R}(\xi,\eta)$ es el operador de
    Grossmann-Royer.
  \end{definition}
  
  La utilidad de expresar la transformación de Weyl
  utilizando el operador de Grossmann-Royer es para expresar
  la definición del operador de Weyl en la forma integral
  más conocida. Utilizando la definición de
  $\hat{R}(\xi,\eta)$ obtenemos
  \[
    \hat{A}\psi(x)
    = (\pi\hbar)^{-n} \int_{\R^{2n}} A(\xi,\eta)
    e^{\frac{2i}{\hbar} \eta \cdot (x - \xi)} \psi(2\xi - x)
    \, d\xi \, d\eta.
  \] 
  Haciendo el cambio de variable $y = 2\xi - x$ obtenemos $x
  - \xi = \xi - y$, entonces
  \begin{equation}
    \label{eqn:weyl_quant_k}
    \hat{A}\psi(x)
    = (2\pi\hbar)^{-n} \int_{\R^{2n}} e^{\frac{i}{\hbar} p
    \cdot (x - y)} A\left( \frac{x+y}{2}, p \right) \psi(y)
    \, dy \, dp.
  \end{equation}
  La definición anterior es rigurosa bajo ciertas
  condiciones, en partícular para los operadores que tienen
  una expresión integral (como los operadores
  Hilbert-Schmidt), es decir
  \[
    \hat{A}\psi(x) = \int_{\R^{n}} K(x,y) \psi(y) \, dy
  \] 
  donde $K$ se conoce como el \textit{núcleo distribucional}
  de $\hat{A}$. Observando el integrando de la ecuación
  (\ref{eqn:weyl_quant_k}), podemos ver que para un operador
  de Weyl, el núcleo distribucional está dado por
  \begin{equation}
    K(x,y)
    = (2\pi\hbar)^{-n} \int_{\R^{n}} e^{\frac{i}{\hbar} p
    \cdot (x - y)}A\left( \frac{x+y}{2}, p \right) \, dp.
  \end{equation}
  Por medio de la fórmula de inversión de Fourier y otro
  cambio de variable, podemos expresar al símbolo $A$ en
  términos del núcleo $K$,
  \begin{equation}
    A(x,p)
    = \int_{\R^{n}} e^{-\frac{i}{\hbar} p \cdot y} K\left( x
    + \frac{1}{2}y, x - \frac{1}{2}y\right) \, dy.
  \end{equation}
  Al símbolo obtenido a partir de operador de Weyl se conoce
  como símbolo de Weyl.  Lo anterior nos dice que existe una
  correspondencia entre el símbolo de Weyl y los núcleos
  distribucionales de los operadores de Weyl. En partícular,
  si $K(x,y) = (2\pi\hbar)^{-n} \psi(x)\overline{\psi(y)}$,
  entonces el símbolo de Weyl del operador con núcleo $K$ es
  precisamente la transformación de Wigner de $\psi$,
  $W\psi$! De hecho, la representación integral caracteriza
  al espacio de operadores de Hilbert-Schmidt, pues éstos
  operadores son aquellos que tienen núcleos cuadraticamente
  integrables.

  \textbf{Observación.} Notemos que las formulas expuestas
  funcionan de manera simbólica en la mayoría de los casos,
  pero solo son rigurosas cuando se trata de una
  `decuantización' de operadres que se comportan muy bien.
  En el caso en donde no se puede justificar con rigor el
  uso de tal fórmula, debemos tener mucho cuidado a la hora
  de concluir cosas sobre los operadores obtenidos por el
  procedimiento de cuantización de Weyl.
  
  Trabajando de manera formal, podemos utilizar la
  formulación (\ref{eqn:weyl_quant_k}) para obtener los
  operadores correctos correspondientes a las funciones
  coordenadas $x_j$ y $p_j$, los cuales notamos que no
  pertenecen al espacio de Schwartz.
  \begin{equation}
    \Op_W(x_j)\psi = x_j\psi,
    \quad
    \Op_W(p_j)\psi = -i\hbar \partial_{x_j}\psi.
  \end{equation}
  La transformación de Weyl también nos da los operadores
  correctos correspondientes a los polinomios involucran a
  las variables $x$ y $p$. Ahora mostraremos la relación
  entre la transformación de Wigner y la de Weyl, y porque
  se consideran mapeos inversos.

  \subsubsection{La transformación de Wigner-Weyl}

  Consideremos el operador de Weyl de un observable adecuado
  sobre el espacio de fase. Entonces podemos utilizar a la
  transformación cruzada de Wigner para calcular el producto
  interior $\langle \psi, \hat{A}\phi \rangle$, para
  funciones de Schwartz.
  \begin{proposition}
    \label{prop:wigner-weyl}
    Sea $A \in \Sz(\R^{2n})$. Tenemos que
    \begin{equation}
      \langle \psi, \hat{A}\phi \rangle
      = \int_{\R^{2n}} A(x,p)W(\psi,\phi)(z) \, dx \, dp.
    \end{equation}
    para todo $\psi, \phi \in \Sz(\R^{n})$, donde $\hat{A} =
    \Op_W(A)$.
  \end{proposition}
  Utilizando ésto, finalmente podemos expresar el valor
  esperado de un operador de Weyl en un estado $\psi$
  mediante una integral en el espacio de fase.
  \begin{proposition}
    El valor esperado de un operador de Weyl $\hat{A} =
    \Op_W(A)$ en un estado $\psi$ está dado por
    \begin{equation}
      \langle \hat{A} \rangle_\psi
      = \frac{1}{\langle \psi, \psi \rangle} 
      \int_{\R^{2n}} A(z) W\psi(z) \, dz.
    \end{equation}
  \end{proposition}  
  En efecto tenemos una decuantización de los operadores
  obtenidos a partir de la cuantización de Weyl de un
  símbolo en el espacio de fase. Tenemos una expresión
  formal que nos es interesante
  \begin{equation}
    A(x,p) = \Tr\left( \hat{M}(x,p) \Op_W(A) \right),
  \end{equation}
  donde $(\hat{M}(x,p)\psi)(y) = 2e^{2i p (y - x)}\psi(2x -
  y)$. Ésta expresión solo se puede calcular en ciertas
  situaciones, por suerte la que nos interesa calcular es la
  del caso en que $\Op_W(A) = \rho$ es un operador de
  densidad, pues éste es de clase traza.

  Con las propiedades que se probaron en la sección de la
  transformación de Wigner respecto a la integración en los
  ejes del espacio de fase, y con el resultado anterior que
  nos permite calcular el valor esperado de un operador de
  Weyl con la transformación de Wigner, podemos formular
  parte de la mecánica cuántica en el espacio de fase.

  \subsection{En mecánica cuántica}

  \begin{definition}
    Un operador $\hat{A}$ acotado sobre $L^2(\R^{n})$ se
    dice que es de clase de traza, si existen dos bases
    ortonormales $\{\psi_j\}$ y $\{\phi_k\}$ tales que
    \begin{equation}
      \sum_{j,k}^{} \left|
      \langle \hat{A} \psi_j, \phi_k \rangle
      \right| < \infty.
    \end{equation}
  \end{definition}

  \begin{definition}
    Si $\hat{A}$ es un operador de clase de traza, entonces
    definimos su traza como la serie absolutamente
    convergente
    \begin{equation}
      \Tr\left( \hat{A} \right) 
      = \sum_{j=1}^{} \langle \hat{A}\psi_j, \psi_j \rangle,
    \end{equation}
    y mencionamos que ésto es independiente de la elección
    de la base.
  \end{definition}

  Ahora consideremos el caso de los operadores de Weyl.

  \begin{proposition}
    Sean $\hat{A} = \Op_W(a)$ y $\hat{B} = \Op_W(b)$
    operadores de Hilbert-Schmidt. Entonces
    \begin{equation}
      \Tr\left( \hat{A}\hat{B} \right) 
      = \Tr\left( \hat{B}\hat{A} \right) 
      = \frac{1}{(2\pi\hbar)^{n}} \int_{\R^{2n}} a(z)b(z) \,
      dz.
    \end{equation}
  \end{proposition}

  \begin{proposition}
    Sea $\hat{A} = \Op_W(a)$ un operador de clase de traza.
    Si adicionalmente $a \in L^{1}(\R^{n})$, entonces
    \begin{equation}
      \Tr\left( \hat{A} \right) 
      = \frac{1}{(2\pi\hbar)^{n}} \int_{\R^{2n}} a(z) \, dz.
    \end{equation}
  \end{proposition}

  Si $\psi \in L^2(\R^{n})$ es una función con norma
  unitaria, la información de $\psi$ es equivalente a la
  información de su transformación de Wigner [GOSSON]. Se
  puede probar que la proyección ortogonal $\hat{\Pi}_\psi$ 
  sobre el rayo $\{\lambda \psi : \lambda \C\}$ es el
  operador de Weyl con símbolo
  \[
    \pi_\psi = (2\pi\hbar)^{n}W\psi.
  \] 
  Con ésto podemos identificar un estado cuántico $\psi$ con
  su función de Wigner.

  \begin{definition}
    Consideremos el estado mixto $\{(\psi_j,\alpha_j)\}$,
    con $\|\psi_j\| = 1$, $\alpha_j \geq 0$ y $\sum_{j}^{}
    \alpha_j = 1$. El operador de densidad $\rho$ de
    éste estado es el operador de Weyl
    \begin{equation}
      \rho
      = (2\pi\hbar)^{n} \sum_{j}^{} \alpha_j \Op_W(W\psi_j).
    \end{equation}
    La transformación de Wigner de éste estado es la función
    \begin{equation}
      \rho = \sum_{j}^{} \alpha_j W\psi_j.
    \end{equation}
  \end{definition}

  \begin{proposition}
    Sea $\{\hat{\Pi}_j\}$ una familia de proyecciones
    ortogonales con rango uno y $\{\alpha_j\}$ una familia
    de números reales no negativos tales que $\sum_j
    \alpha_j = 1$. Entonces $\rho = \sum_{j}^{}
    \alpha_j \hat{\Pi}_j$ es el operador de densidad de
    algún estado mixto $\{(\psi_j,\alpha_j)\}$.
  \end{proposition}

  \begin{proposition}
    Sea $\rho$ un operador acotado en $L^2(\R^{n})$.
    \begin{itemize}
      \item Sea $\rho$ un operador de densidad.
        Entonces es auto-adjunto, semi-definido positivo, y
        tiene traza igual a uno.
      \item Supongamos que $\rho$ es auto-adjunto,
        semi-definido positivo y tiene traza igual a uno,
        entonces es un operador de densidad.
    \end{itemize}
  \end{proposition}

  \section{Ejemplos}

  \subsection{El oscilador armónico}

  Dos caminos: resolver el problema de Schrödinger, luego
  obtener la función de Wigner ó, resolver el
  problema-$\star$. Funciones de onda del problema de
  Schrödinger son los polinomios Hermíticos, transformando
  nos dan los de Laguerre! . . .

  \chapter{Funciones de Wigner en el Espacio Fase Discreto}

  \chapter{Construcción no-estándar}

  \newpage
  \appendix

  \chapter{Preliminares matemáticos}

  \section{Espacio de Schwartz y su dual}

  \begin{definition}
    Sea $\Sz(\R^{n})$ el espacio vectorial complejo
    de las funciones de prueba de Schwartz. La función
    $\psi$ es un elemento de $\Sz(\R^{n})$, si es
    lisa y si las funciones $x^{\beta} \partial_x^{\alpha}
    \psi$ son acotadas para todo los índices $\alpha =
    (\alpha_1, \ldots, \alpha_n)$ y $\beta = (\beta_1,
    \ldots, \beta_n)$ en $\N^{n}$.
  \end{definition}

  \begin{definition}
    El espacio dual de $\Sz(\R^{n})$ es el espacio de
    las distribuciones (ó funciones generalizadas)
    templadas, y lo denotaremos como $\Sz'(\R^{n})$.
  \end{definition}

  \begin{definition}
    El espacio de Hilbert de las (clases de) funciones
    complejas cuadraticamente-integrables sobre $\R^{n}$ se
    denota como $L^2(\R^{n})$, y está equipado con el
    producto interno
    \[
      \langle f, g \rangle
      = \int_{\R^{n}} \overline{f}(x) g(x) \, dx.
    \] 
  \end{definition}
 
  Los elementos del espacio de Schwartz sobre $\R^{n}$
  tienen derivadas parciales que decrecen rápidamente. El
  dual del espacio de Schwartz se conoce como las
  \textit{distribuciones templadas} sobre $\R^{n}$. Existe
  una inclusión natural de las funciones cuadraticamente
  integrables en el espacio de distribuciones. Además, el
  espacio de Schwartz es un subespacio denso de
  $L^2(\R^{n})$, podemos expresar éstas relaciones con un
  poco de abuso de notación,
  \[
    \Sz(\R^{n})
    \subset L^2(\R^{n})
    \subset \Sz'(\R^{n}).
  \]


  Sea $\mathcal D(\R^{n})$ el espacio de las funciones de
  prueba.

  \begin{definition}
    Supongamos que $\{\phi_n\}_{n \in \N}$ es una sucesión
    convergente en $\mathcal D(\R^{n})$ con límite $\phi \in
    \mathcal D(\R^{n})$. Si $T : \mathcal D(\R^{n}) \to \C$
    es un funcional lineal continuo, es decir
    \[
      \forall \psi, \phi \in \mathcal D(\R^{n}) : T(\phi +
      \psi) = T(\phi) + T(\psi),
    \] 
    \[
      \forall \phi \in \mathcal D(\R^{n}) : \forall \alpha
      \in \C : T(\alpha \phi) = \alpha T(\phi),
    \] 
    y
    \[
      \phi_n \to \phi \implies T(\phi_n) \to T(\phi).
    \] 
    Entonces $T$ es una distribución.
  \end{definition}

  \begin{definition}
    Definimos a la distribución de Dirac $\delta_a$ como la
    distribución que satisface
    \[
      \forall \phi \in \mathcal D(\R^{n}) : \delta_a(\phi) =
      \phi(a).
    \] 
  \end{definition}

  $C^{\infty}_0(\R^{n})$ es el conjunto de todas las función
  sobre $\R^{n}$ lisas y con soporte compacto. Los espacios
  $C^{\infty}_0(\R^{n})$ y $\Sz(\R^{n})$ son densos en
  $L^{r}(\R^{n})$, $1 \leq r < \infty$.

  Iniciamos definiendo
  la transformación de Wigner sobre el espacio de las
  funciones de Schwartz $\Sz(\R^{n})$, para ésto primero
  recordamos la transformación de Fourier.

  \begin{definition}
    La transformada de Fourier se denotará por $\F$,
    y usaremos una versión que incluye una constante de
    normalización dada por la constante de Planck,
    \[
      \F\psi(x)
      = \frac{1}{(2\pi\hbar)^{n / 2}} \int_{\R^{n}}
      e^{-\frac{i}{\hbar} x \cdot y} \psi(y) \, dy,
    \] 
    para $\psi \in \Sz(\R^{n})$.
  \end{definition}
  La transformada de Fourier es un automorfismo del espacio
  de Schwartz, y se puede extender a un automorfismo
  unitario sobre $L^2(\R^{n})$, cuya inversa está dada por
  \[
    \F^{-1}\psi(x)
    = \frac{1}{(2\pi\hbar)^{n / 2}} \int_{\R^{n}}
    e^{\frac{i}{\hbar} x \cdot y} \psi(y) \, dy.
  \] 
  Además se puede extender por medio de la dualidad a un
  automorfismo de $\Sz'(\R^{n})$.

  \begin{definition}
    La transformada de Fourier de una función en
    $L^{1}(\R^{n})$ es la función $\hat{f}$ sobre $\R^{n}$ 
    definida como
    \[
      \F[f](\xi) = (2\pi)^{-n / 2} \int_{\R^{n}} e^{-i x
      \cdot \xi} f(x) \, dx, 
      \quad \xi \in \R^{n}.
    \] 
  \end{definition}

  \begin{theorem}
    La transformada de Fourier es biyectiva en
    $\Sz(\R^{n})$.
  \end{theorem}

  \begin{theorem}[Plancherel]
    La biyección $\F : \Sz(\R^{n}) \to \Sz(\R^{n})$ 
    se puede extender únicamente a un operador unitario
    sobre $L^2(\R^{n})$.
  \end{theorem}

  \begin{proposition}
    El espacio $\Sz(\R^{n})$ es denso en $\Sz'(\R^{n})$.
  \end{proposition}

  \section{Wong}

  Sea $f$ una función medible sobre $\R^{n}$. Definimos la
  función $\rho(\xi,\eta)f$ sobre $\R^{n}$ como
  \begin{equation}
    (\rho(\xi,\eta)f)(x)
    = e^{i\xi \cdot x + \frac{1}{2} i \xi \cdot \eta}
    f(x+\eta),
    \quad x \in \R^{n}.
  \end{equation}
  El operador $\rho(\xi,\eta) : L^2(\R^{n}) \to L^2(\R^{n})$
  es unitario para todo $\xi, \eta \in \R^{n}$. Si nos
  enfoncamos en el subespacio de $L^2(\R^{n})$ de las
  funciones de Schwartz, entonces podemos definir la
  siguiente función sobre el espacio de fase $\R^{2n}$. Sean
  $f,g \in \Sz(\R^{n})$, entonces
  \begin{equation}
    V(f,g)(\xi,\eta)
    = (2\pi)^{-n / 2} \langle \rho(\xi,\eta)f, g \rangle.
  \end{equation}
  A ésta función Wong la llama la transformación
  Fourier-Wigner de $f$ y de $g$. De manera explícita
  tenemos 
  \begin{equation}
    V(f,g)(\xi,\eta)
    = (2\pi)^{-n / 2} \int_{\R^{n}} e^{i \xi \cdot y}
    f\left( f + \frac{\eta}{2} \right) \overline{g\left( y -
    \frac{\eta}{2}\right) } \, dy.
  \end{equation}

  El operador $V : \Sz(\R^{n}) \times \Sz(\R^{n}) \to
  \Sz(\R^{2n})$ es un mapeo lineal.

  Wong menciona que la transformada de Wigner $W(f)$ de una
  función en $L^2(\R^{n})$, introducida por Wigner en 1932,
  es una herramienta para el estudio de la distribución de
  probabilidad conjunta no existente de la posición y
  momentum de un estado $f$.

  \begin{theorem}
    Sean $f$ y $g$ elementos de $\Sz(\R^{n})$. Entonces
    \begin{equation}
      \F[V(f,g)](x,\xi)
      = (2\pi)^{-n / 2} \int_{\R^{n}} e^{-i \xi \cdot p
      } f\left( x + \frac{p}{2} \right)
      \overline{g\left( x - \frac{p}{2} \right)} \, dp.
    \end{equation}
  \end{theorem}

  Con ésto Wong define a la transformación de Wigner de dos
  funciones $f$ y $g$ en $\Sz(\R^{n})$ como
  \begin{equation}
    W(f,g)(x,\xi)
    = (2\pi)^{-n / 2} \int_{\R^{n}} e^{-i\xi \cdot p}
    f\left( x + \frac{p}{2} \right) \overline{g\left( x -
    \frac{p}{2} \right) } \, dp.
  \end{equation}

  La transformada de Wigner $W : \Sz(\R^{n}) \times
  \Sz(\R^{n}) \to \Sz(\R^{2n})$ puede ser extendida a un
  operador sesquilineal
  \[
    W : L^2(\R^{n}) \times L^2(\R^{n}) \to L^2(\R^{2n}),
  \] 
  tal que
  \[
    \|W(f,g)\|_{L^2(\R^{2n})}
    = \|f\|_{L^2(\R^{n})} \|g\|_{L^2(\R^{n})}.
  \] 

  De acuerdo a Wong, podemos obtener un operador lineal
  acotado $Q : L^2(\R^{2n}) \to B(L^2(\R^{n}))$ a partir de
  cualquier función $\sigma \in L^2(\R^{2n})$.

  En la mecánica cúantica, los observables deben ser
  representados por operadores auto-adjuntos, la
  transformación de Weyl nos permite hacer ésto.

  La correspondencia resulta ser entre funciones en el
  espacio de fase y los operadores Hilbert-Schmidt.

  \begin{definition}
    Sea $h \in L^2(\R^{2n})$. Definimos el operador $S_h :
    L^2(\R^{n}) \to L^2(\R^{n})$ como
    \begin{equation}
      (S_hf)(x)
      = \int_{\R^{n}} h(x,y) f(y) \, dy,
    \end{equation}
    para todo $f \in L^2(\R^{n})$.
  \end{definition}
  Al operador $S_h$ se le conoce como el operador
  Hilbert-Schmidt correspondiente al núcleo $h$.

  Para probar que el conjunto de transformaciones de Weyl
  con símbolos en $L^2(\R^{2n})$ es igual al conjunto de
  operadores Hilbert-Schmidt sobre $L^2(\R^{n})$, Wong
  define y utiliza el producto tensorial de funciones de
  $L^2(\R^{n})$.

  \begin{theorem}
    Sea $\sigma \in L^2(\R^{2n})$. Entonces $W_\sigma :
    L^2(\R^{n}) \to L^2(\R^{n})$ es un operador
    Hilbert-Schmidt con núcleo $(2\pi)^{-n / 2}K \sigma$
    donde $K : L^2(\R^{2n}) \to L^2(\R^{2n})$ se define como
    \[
      (Kf)(x,y)
      = (T^{-1}\F_2 f)(y,x).
    \] 
  \end{theorem}

  Entonces ahora sabemos que si $\sigma \in L^2(\R^{2n})$,
  entonces $W_\sigma$ es un operador Hilbert-Schmidt con
  núcleo $(2\pi)^{-n / 2} K \sigma$. El converso es cierto
  también, si $A$ es un operador Hilbert-Schmidt arbitrario,
  entonces tiene un representación integral $A = S_h$, y
  $\sigma = (2\pi)^{-n / 2} K^{-1}h$.

  Se puede definir la transformación de Weyl para símbolos
  en $\Sz'(\R^{2n})$. Resulta que para $2 < r < \infty$,
  existe una función $\sigma \in L^{r}(\R^{2n})$ tal que la
  transformación de Weyl $W_\sigma$ no es un operador lineal
  acotado sobre $L^2(\R^{n})$.

  \section{Resumen de Folland}

  

  

  Para Folland, la transformación de Wigner de dos funciones
  $f$ y $g$ es la transformación de Fourier de la
  transformación de Fourier-Wigner:
  \[
    W(f,g)(\xi, \eta)
    = \int_{\R^{2n}} e^{-2\pi i (\xi q + \eta p} V(f,g)(q,x)
    \, dp \, dq.
  \] 
  Donde
  \[
    V(f,g)(q,p)
    = \int e^{2\pi i q y} f(y + \frac{1}{2}p)\overline{g(y -
    \frac{1}{2}p)} \, dy.
  \] 
  Así que
  \[
    W(f,g)(\xi, \eta) = \int e^{-2\pi i \eta p }
    f(x+\frac{1}{2}p)\overline{g(y-\frac{1}{2}p)} \, dp.
  \] 

  Folland muestra que $W$ mapea $\Sz(\R^{n}) \times
  \Sz(\R^{n})$ a $\Sz(\R^{2n})$ y se puede extender al caso
  de las distribuciones templadas.

  \newpage
  \printbibliography

\end{document}

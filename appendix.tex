\chapter{Apéndices}

 \section{La mecánica clásica y el espacio de fase}

  El concepto del espacio de fase es una herramienta de la
  mecánica clásica, la cual describe la evolución temporal
  de un sistema físico. Dicho de una manera muy sencilla, la
  mecánica clásica estudia partículas y sus trayectorias,
  las cuales se rigen de acuerdo a las leyes de Newton.  Se
  considera que la partícula se `mueve' en un espacio
  euclideano, es decir, su \textit{posición} está dado por
  $x = (x_1,\ldots,x_n) \in \R^{n}$. El
  \textit{momentum} es una cantidad dada por $p_j = m \dot
  x_j$, donde $\dot x$ es la derivada respecto al
  tiempo de la posición, es decir, la velocidad de la
  partícula, y $m$ es la \textit{masa} de la partícula. Las
  cantidades que uno desea medir de nuestro sistema físico
  se les llama \textit{observables}, y en la mecánica
  clásica, son las funciones continuas que tienen como
  argumentos las cantidades $x,p$ y $m$. Ejemplos
  de ellos son el momentum, la energía cinética, la energía
  potencial, etc. La función de energía más usual es la que
  está dada como la suma de la \textit{energía cinética} y
  \textit{energía potencial}:
  \begin{equation}
    H(x,p)
    = \frac{1}{2m} \sum_{j=1}^{n} p_j^2 + V(x).
  \end{equation}
  A la energía del sistema se le conoce como el
  \textit{Hamiltoniano}.  Utilizando ésta función de
  energía, la ley de Newton nos brinda las ecuaciones de
  movimiento de la partícula en cuestión:
  \begin{equation}
    \frac{dx_j}{dt}
    = \frac{\partial H}{\partial p_j},
    \quad
    \frac{dp_j}{dt}
    = -\frac{\partial H}{\partial x_j}.
  \end{equation}
  Expresada como un sistema de ecuaciones diferenciales, a
  la ley de Newton se le conoce como las \textit{ecuaciones
  de Hamilton}. Con ésto, es natural representar el estado
  del sistema clásico considerando el par $(x,p) \in
  \R^{2n}$. Al espacio $\R^{2n}$ se le conoce como el
  \textit{espacio de fase}. A las soluciones de las
  ecuaciones de Hamilton se les conoce como
  \textit{trayectorias}, y son curvas que viven en el
  espacio de fase.
  \begin{definition}
    El espacio de fase de una partícula que se mueve en
    $\R^{n}$ es $\R^{2n}$, considerado como el conjunto de
    las $(2n)$-tuplas de la forma
    \[
      \left(
        x_1, \ldots, x_n, p_1, \ldots, p_n
      \right),
    \] 
    donde $x_j$ y $p_j$ son elementos de $\R$.
  \end{definition}

  Cabe mencionar que el tratado moderno de la mecánica
  clásica está fundamentado en la geometría diferencial de
  las variedades simplécticas
  \cite{mcinerneyFirstStepsDifferential2013}, en donde el
  espacio de fase se define como el espacio cotangente del
  espacio de configuraciones $T^{*} \R_x \cong \R_x \times
  \R_p$, donde el espacio de configuraciones  $\R_x$ es el
  espacio de posición y el $\R_p$ es el espacio del
  momentum.  Para nuestros objetivos basta con desginar el
  espacio de fase como el espacio $\R^{2n}$.

  Notemos que por el momento no ha surgido ninguna
  interpretación probabilística en la mecánica clásica, pues
  es una teoría determinística. Dado que la teoría cuántica
  es probabilística, es natural preguntarnos ¿qué
  características de la mecánica clásica serían deseables en
  la teoría cuántica? y ¿qué beneficios habría en hacer un
  vínculo entre la mecánica clásica y la cuántica?
  \cite{schroeckQuantumMechanicsPhase1996}, especialmente
  cuando sabemos que la teoría cuántica (y sus derivados) es
  nuestra teoría más precisa. Un esfuerzo por vincular las
  descripciones clásicas y cuánticas del mundo es la
  representación de Wigner-Weyl-Moyal de la mecánica
  cuántica.  Ésta es una formulación que intenta usar la
  noción del espacio de fase en la dinámica cuántica y la
  idea básica es la construcción de
  \textit{cuasi-distribuciones} real-valuadas que
  representan a los sistemas cuánticos.

  \section{Espacio de Schwartz y su dual}

  \begin{definition}
    Sea $\Sz(\R^{n})$ el espacio vectorial complejo
    de las funciones de prueba de Schwartz. La función
    $\psi$ es un elemento de $\Sz(\R^{n})$, si es
    lisa y si las funciones $x^{\beta} \partial_x^{\alpha}
    \psi$ son acotadas para todo los índices $\alpha =
    (\alpha_1, \ldots, \alpha_n)$ y $\beta = (\beta_1,
    \ldots, \beta_n)$ en $\N^{n}$.
  \end{definition}

  \begin{definition}
    El espacio dual de $\Sz(\R^{n})$ es el espacio de
    las distribuciones (ó funciones generalizadas)
    templadas, y lo denotaremos como $\Sz'(\R^{n})$.
  \end{definition}

  \begin{definition}
    El espacio de Hilbert de las (clases de) funciones
    complejas cuadraticamente-integrables sobre $\R^{n}$ se
    denota como $L^2(\R^{n})$, y está equipado con el
    producto interno
    \[
      \langle f, g \rangle
      = \int_{\R^{n}} \overline{f}(x) g(x) \, dx.
    \] 
  \end{definition}
 
  Los elementos del espacio de Schwartz sobre $\R^{n}$
  tienen derivadas parciales que decrecen rápidamente. El
  dual del espacio de Schwartz se conoce como las
  \textit{distribuciones templadas} sobre $\R^{n}$. Existe
  una inclusión natural de las funciones cuadraticamente
  integrables en el espacio de distribuciones. Además, el
  espacio de Schwartz es un subespacio denso de
  $L^2(\R^{n})$, podemos expresar éstas relaciones con un
  poco de abuso de notación,
  \[
    \Sz(\R^{n})
    \subset L^2(\R^{n})
    \subset \Sz'(\R^{n}).
  \]

  Sea $\mathcal D(\R^{n})$ el espacio de las funciones de
  prueba.

  \begin{definition}
    Supongamos que $\{\phi_n\}_{n \in \N}$ es una sucesión
    convergente en $\mathcal D(\R^{n})$ con límite $\phi \in
    \mathcal D(\R^{n})$. Si $T : \mathcal D(\R^{n}) \to \C$
    es un funcional lineal continuo, es decir
    \[
      \forall \psi, \phi \in \mathcal D(\R^{n}) : T(\phi +
      \psi) = T(\phi) + T(\psi),
    \] 
    \[
      \forall \phi \in \mathcal D(\R^{n}) : \forall \alpha
      \in \C : T(\alpha \phi) = \alpha T(\phi),
    \] 
    y
    \[
      \phi_n \to \phi \implies T(\phi_n) \to T(\phi).
    \] 
    Entonces $T$ es una distribución.
  \end{definition}

  \begin{definition}
    Definimos a la distribución de Dirac $\delta_a$ como la
    distribución que satisface
    \[
      \forall \phi \in \mathcal D(\R^{n}) : \delta_a(\phi) =
      \phi(a).
    \] 
  \end{definition}

  $C^{\infty}_0(\R^{n})$ es el conjunto de todas las función
  sobre $\R^{n}$ lisas y con soporte compacto. Los espacios
  $C^{\infty}_0(\R^{n})$ y $\Sz(\R^{n})$ son densos en
  $L^{r}(\R^{n})$, $1 \leq r < \infty$.

  Iniciamos definiendo
  la transformación de Wigner sobre el espacio de las
  funciones de Schwartz $\Sz(\R^{n})$, para ésto primero
  recordamos la transformación de Fourier.

  \begin{definition}
    La transformada de Fourier se denotará por $\F$,
    y usaremos una versión que incluye una constante de
    normalización dada por la constante de Planck,
    \[
      \F\psi(x)
      = \frac{1}{(2\pi\hbar)^{n / 2}} \int_{\R^{n}}
      e^{-\frac{i}{\hbar} x \cdot y} \psi(y) \, dy,
    \] 
    para $\psi \in \Sz(\R^{n})$.
  \end{definition}
  La transformada de Fourier es un automorfismo del espacio
  de Schwartz, y se puede extender a un automorfismo
  unitario sobre $L^2(\R^{n})$, cuya inversa está dada por
  \[
    \F^{-1}\psi(x)
    = \frac{1}{(2\pi\hbar)^{n / 2}} \int_{\R^{n}}
    e^{\frac{i}{\hbar} x \cdot y} \psi(y) \, dy.
  \] 
  Además se puede extender por medio de la dualidad a un
  automorfismo de $\Sz'(\R^{n})$.

  \begin{definition}
    La transformada de Fourier de una función en
    $L^{1}(\R^{n})$ es la función $\hat{f}$ sobre $\R^{n}$ 
    definida como
    \[
      \F[f](\xi) = (2\pi)^{-n / 2} \int_{\R^{n}} e^{-i x
      \cdot \xi} f(x) \, dx, 
      \quad \xi \in \R^{n}.
    \] 
  \end{definition}

  \begin{theorem}
    La transformada de Fourier es biyectiva en
    $\Sz(\R^{n})$.
  \end{theorem}

  \begin{theorem}[Plancherel]
    La biyección $\F : \Sz(\R^{n}) \to \Sz(\R^{n})$ 
    se puede extender únicamente a un operador unitario
    sobre $L^2(\R^{n})$.
  \end{theorem}

  \begin{proposition}
    El espacio $\Sz(\R^{n})$ es denso en $\Sz'(\R^{n})$.
  \end{proposition}

  \section{Wong}

  Sea $f$ una función medible sobre $\R^{n}$. Definimos la
  función $\rho(\xi,\eta)f$ sobre $\R^{n}$ como
  \begin{equation}
    (\rho(\xi,\eta)f)(x)
    = e^{i\xi \cdot x + \frac{1}{2} i \xi \cdot \eta}
    f(x+\eta),
    \quad x \in \R^{n}.
  \end{equation}
  El operador $\rho(\xi,\eta) : L^2(\R^{n}) \to L^2(\R^{n})$
  es unitario para todo $\xi, \eta \in \R^{n}$. Si nos
  enfoncamos en el subespacio de $L^2(\R^{n})$ de las
  funciones de Schwartz, entonces podemos definir la
  siguiente función sobre el espacio de fase $\R^{2n}$. Sean
  $f,g \in \Sz(\R^{n})$, entonces
  \begin{equation}
    V(f,g)(\xi,\eta)
    = (2\pi)^{-n / 2} \langle \rho(\xi,\eta)f, g \rangle.
  \end{equation}
  A ésta función Wong la llama la transformación
  Fourier-Wigner de $f$ y de $g$. De manera explícita
  tenemos 
  \begin{equation}
    V(f,g)(\xi,\eta)
    = (2\pi)^{-n / 2} \int_{\R^{n}} e^{i \xi \cdot y}
    f\left( f + \frac{\eta}{2} \right) \overline{g\left( y -
    \frac{\eta}{2}\right) } \, dy.
  \end{equation}

  El operador $V : \Sz(\R^{n}) \times \Sz(\R^{n}) \to
  \Sz(\R^{2n})$ es un mapeo lineal.

  Wong menciona que la transformada de Wigner $W(f)$ de una
  función en $L^2(\R^{n})$, introducida por Wigner en 1932,
  es una herramienta para el estudio de la distribución de
  probabilidad conjunta no existente de la posición y
  momentum de un estado $f$.

  \begin{theorem}
    Sean $f$ y $g$ elementos de $\Sz(\R^{n})$. Entonces
    \begin{equation}
      \F[V(f,g)](x,\xi)
      = (2\pi)^{-n / 2} \int_{\R^{n}} e^{-i \xi \cdot p
      } f\left( x + \frac{p}{2} \right)
      \overline{g\left( x - \frac{p}{2} \right)} \, dp.
    \end{equation}
  \end{theorem}

  Con ésto Wong define a la transformación de Wigner de dos
  funciones $f$ y $g$ en $\Sz(\R^{n})$ como
  \begin{equation}
    W(f,g)(x,\xi)
    = (2\pi)^{-n / 2} \int_{\R^{n}} e^{-i\xi \cdot p}
    f\left( x + \frac{p}{2} \right) \overline{g\left( x -
    \frac{p}{2} \right) } \, dp.
  \end{equation}

  La transformada de Wigner $W : \Sz(\R^{n}) \times
  \Sz(\R^{n}) \to \Sz(\R^{2n})$ puede ser extendida a un
  operador sesquilineal
  \[
    W : L^2(\R^{n}) \times L^2(\R^{n}) \to L^2(\R^{2n}),
  \] 
  tal que
  \[
    \|W(f,g)\|_{L^2(\R^{2n})}
    = \|f\|_{L^2(\R^{n})} \|g\|_{L^2(\R^{n})}.
  \] 

  De acuerdo a Wong, podemos obtener un operador lineal
  acotado $Q : L^2(\R^{2n}) \to B(L^2(\R^{n}))$ a partir de
  cualquier función $\sigma \in L^2(\R^{2n})$.

  En la mecánica cúantica, los observables deben ser
  representados por operadores auto-adjuntos, la
  transformación de Weyl nos permite hacer ésto.

  La correspondencia resulta ser entre funciones en el
  espacio de fase y los operadores Hilbert-Schmidt.

  \begin{definition}
    Sea $h \in L^2(\R^{2n})$. Definimos el operador $S_h :
    L^2(\R^{n}) \to L^2(\R^{n})$ como
    \begin{equation}
      (S_hf)(x)
      = \int_{\R^{n}} h(x,y) f(y) \, dy,
    \end{equation}
    para todo $f \in L^2(\R^{n})$.
  \end{definition}
  Al operador $S_h$ se le conoce como el operador
  Hilbert-Schmidt correspondiente al núcleo $h$.

  Para probar que el conjunto de transformaciones de Weyl
  con símbolos en $L^2(\R^{2n})$ es igual al conjunto de
  operadores Hilbert-Schmidt sobre $L^2(\R^{n})$, Wong
  define y utiliza el producto tensorial de funciones de
  $L^2(\R^{n})$.

  \begin{theorem}
    Sea $\sigma \in L^2(\R^{2n})$. Entonces $W_\sigma :
    L^2(\R^{n}) \to L^2(\R^{n})$ es un operador
    Hilbert-Schmidt con núcleo $(2\pi)^{-n / 2}K \sigma$
    donde $K : L^2(\R^{2n}) \to L^2(\R^{2n})$ se define como
    \[
      (Kf)(x,y)
      = (T^{-1}\F_2 f)(y,x).
    \] 
  \end{theorem}

  Entonces ahora sabemos que si $\sigma \in L^2(\R^{2n})$,
  entonces $W_\sigma$ es un operador Hilbert-Schmidt con
  núcleo $(2\pi)^{-n / 2} K \sigma$. El converso es cierto
  también, si $A$ es un operador Hilbert-Schmidt arbitrario,
  entonces tiene un representación integral $A = S_h$, y
  $\sigma = (2\pi)^{-n / 2} K^{-1}h$.

  Se puede definir la transformación de Weyl para símbolos
  en $\Sz'(\R^{2n})$. Resulta que para $2 < r < \infty$,
  existe una función $\sigma \in L^{r}(\R^{2n})$ tal que la
  transformación de Weyl $W_\sigma$ no es un operador lineal
  acotado sobre $L^2(\R^{n})$.

  \section{Resumen de Folland}

  Para Folland, la transformación de Wigner de dos funciones
  $f$ y $g$ es la transformación de Fourier de la
  transformación de Fourier-Wigner:
  \[
    W(f,g)(\xi, \eta)
    = \int_{\R^{2n}} e^{-2\pi i (\xi q + \eta p} V(f,g)(q,x)
    \, dp \, dq.
  \] 
  Donde
  \[
    V(f,g)(q,p)
    = \int e^{2\pi i q y} f(y + \frac{1}{2}p)\overline{g(y -
    \frac{1}{2}p)} \, dy.
  \] 
  Así que
  \[
    W(f,g)(\xi, \eta) = \int e^{-2\pi i \eta p }
    f(x+\frac{1}{2}p)\overline{g(y-\frac{1}{2}p)} \, dp.
  \] 

  Folland muestra que $W$ mapea $\Sz(\R^{n}) \times
  \Sz(\R^{n})$ a $\Sz(\R^{2n})$ y se puede extender al caso
  de las distribuciones templadas.

\section{Anillos y campos}

  Si intentamos definir el espacio de fase discreto sobre un
  anillo como $\Z_4$ tendremos problemas a la hora de
  definir las rectas. Por ejemplo si consideramos la recta
  \[
    x + 2p = 0,
  \]
  que tiene como solución al conjunto de puntos
  \[
    \{(0,0), (2,1), (0,2), (2,3)\}.
  \] 
  Ahora, la recta 
  \[
    x = 0,
  \] 
  tiene como solución al conjunto
  \[
    \{(0,0), (0,1), (0,2), (0,3)\}.
  \] 
  Notemos que los dos conjuntos tiene una intersección con
  más de un elemento, algo que contradice nuestra noción
  geoemétrica de una linea en el espacio.

  Sea $\F_N$ un campo de $N$ elementos donde $N = r^{k}$ es
  una potencia de un número primo $r$. El primo $p$ se
  conoce como la característica de $\F_N$ y se define como
  el número entero más pequeño tal que
  \[
    1 + 1 + \ldots + 1 = 0.
  \] 
  $\F_r$ es simplemente $\Z_r$ pero $\F_N$ no es $\Z_N$.
  Para construir a $\F_N$ requerimos de un polinomio $f(x)$ 
  de grado $k$ que es irreducible en $\F_r$. Si $\alpha$ es
  una raíz de $f(x)$, el campo que obtenemos al adjuntar
  $\alpha$ a $\F_r$ es
  \[
    \F_N
    = \F_r(\alpha) \cong \F_r[x] / \langle f(x) \rangle.
  \] 
  \begin{example}
    Consideremos el anillo $\Z_4$ y el polinomio $f(x) = x^2
    + x + 1$. El campo de Galois es
    \[
      \F_2[x] / (x^2 + x + 1)
      \cong \F_2(\alpha)
      = \{0, 1, \alpha, \alpha + 1\},
    \] 
    donde $\alpha^2 + \alpha + 1 = 0$. En el campo $\F_2 =
    {0,1}$ no hay soluciones al polinomio $f(x)$ y definimos
    su solución como $\alpha$. Todo elemento de
    $\F_2(\alpha)$ tiene la forma $a_1 \alpha + a_0$, donde
    $a_j \in \F_2$, por lo tanto $\{1, \alpha\}$ es una base
    de espacio vectorial de $\F_4$ sobre $\F_2$.
  \end{example}

  El mapeo $\sigma : \alpha \to \alpha^r$ donde $\alpha \in
  \F_N$ es un automorfismo lineal de $\F_N$ llamado el
  \textit{automorfismo de Frobenius} que nos da los
  conjugados de Galois. Elementos del campo primo son
  invariantes bajo $\sigma$. Definimos la operación de traza
  de un elemento $\alpha \in \F_N$ (distinta a la traza de
  un operador lineal) como:
  \[
    \tr(\alpha) 
    = \alpha + \alpha^{r} + \alpha^{r^2} + \ldots +
    \alpha^{r^{k-1}}
    = \sum_{m=0}^{k-1} \sigma^{m}(\alpha).
  \] 
  La operación $\tr$ mapea a todo elemento del campo finito
  a un elemento del campo primo $\tr : \F_N \to \F_r$. Para
  cualquier base $E = \{e_0,e_1,\ldots,e_{N-1}\}$, existe
  una única base de campo $\tilde E = \{\tilde
  e_0,\ldots,\tilde e_{N-1}\}$ tal que $\tr(\tilde e_j e_k)
  = \delta_{jk}$. La base $\tilde E$ se llama la base dual a
  $E$. Podemos utilizar la base dual para encontrar la
  expansión única en coeficientes respecto a la base $E$ 
  para cualquier elemento $x$ del campo. Para obtener el
  componente $x_s$ del elemento $x$, hacemos
  \[
    \tr(x \tilde e_s)
    = \sum_{r}^{k} x_r \tr(e_r \tilde e_s)
    = x_s.
  \]

  \section{Klappenecker, Godsil, Roy}

  \subsection{Característica impar}

  Consideremos un espacio de Hilbert de dimensión $d =
  p^{n}$ donde $p$ es un primo impar, y sea $\{\ket{u}\}$ la
  base estándar para el espacio $\C^{d}$ indexado por los
  elementos de la extensión de Galois $\mathbb F_{d} =
  \GF(p,n)$. Dado un $a \in \mathbb F_d$, definamos los
  operadores
  \begin{align}
    X(a) &: \ket u \mapsto \ket{u+a} \\
    Z(a) &: \ket u \mapsto \omega^{\tr(au)} \ket{u}.
  \end{align}
  Notemos que son los operadores de desplazamiento básicos
  que definimos anteriormente en la metodología de Wootters
  y Gibbons, pero ésta vez las operaciones dentro de los
  kets y en las potencias de la raíz primitiva son las del
  campo finito y no solo operaciones módulo $p$. Recordamos
  que la base estándar es un conjunto completo de
  eigenvectores de $Z(a)$. Además la base `Fourier
  conjugada'
  \[
    \ket{\tilde u}
    = \sum_{v \in \mathbb F_d}^{} \omega^{\tr(uv)} \ket{v},
    \quad u \in \mathbb F_d,
  \] 
  es un conjunto completo de eigenvectores del operador
  $X(a)$. De nuevo definamos a los operadores de
  desplazamiento generales como
  \[
    D(a,b) = X(a)Z(b).
  \] 
  Cada operador de desplazamiento es unitario y monomial.
  Bandyopadhyay et al particionan dichas matrices en
  conjuntos mutuamente conmutativos y demuestra que los
  eigenvectores simultáneaos de cada conjunto forman bases
  mutuamente insesgadas. La siguiente proposición nos dice
  como construir a esos eigenvectores.
  \begin{proposition}
    Sea $D(a,b)$ un operador de desplazamiento donde $a,b
    \in \mathbb F_d$. Si $c$ y $n$ son elementos del campo
    tales que $c = \frac{b}{2a}$, entonces el vector
    \begin{equation}
      \ket{a,b; n}
      = \sum_{x \in \mathbb F_d}^{} \omega^{\tr(cx^2+2nx)}
      \ket{u},
    \end{equation}
    es un eigenvector de $D(a,b)$.
  \end{proposition}
  \begin{proof}
    .
  \end{proof}

  \subsection{Característica par}

  El caso donde el campo primo es de característica par es
  un poco más complicado. ¿Por qué? En éste caso es
  necesario hacer operaciones en un \textit{anillo} de
  Galois. Sea $\Z_4$ el anillo de los enteros módulo 4.
  Decimos que un polinomio $h(x) \in \Z_4[x]$ es un
  \textit{primitivo básico} si y solo si su imágen en
  \[
    \Z_4[x] / \langle 2 \rangle
    \cong
    \Z_2[x]
  \] 
  bajo el mapeo canónico es un polinomio primitivo en
  $\Z_2[x]$. Sea $h(x)$ un polinomio primitivo básico mónico
  de grado $n$, entonces el anillo
  \begin{equation}
    GR(4,n)
    = \Z_4[x] / \langle h(x) \rangle    
  \end{equation}
  se cono como el anillo de Galois de grado $n$ sobre
  $\Z_4$.
  Sea $\xi$ el elemento primitivo de orden $2^{n}-1$ del
  anillo y consideremos el conjunto Teichmüller
  \[
    T_n
    = \{0,1,\xi,\ldots,\xi^{2n-2}\},
  \] 
  del anillo. Klappenecker [CITE] nos menciona que todo
  elemento $r \in GR(4,n)$ puede ser expresado de manera
  única en la forma $r = a + 2b$ donde $a,b \in T_n$. Con
  ésta caracterización de los elementos del anillo en
  términos de los elementos del Teichmüller podemos definir
  el automorfismo de Frobenius (ver apéndice ()) $\sigma :
  GR(4,n) \to \Z_4$ definido como
  \[
    \sigma(a+2b)
    = a^2+2b^2.
  \] 
  Con ésto definamos a la traza del anillo de Galois $\tr :
  GR(4,n) \to \Z_4$ utilizando el mapa de Frobenius:
  \[
    \tr(r)
    = \sum_{k=0}^{n-1} \sigma^{k}(r).
  \] 
  Con éstos acordes Klappenecker construye las siguientes
  bases mutuamente insesgadas.
  \begin{proposition}
    Sea $GR(4,n)$ el anillo de Galois de $4^{n}$ elementos y
    $T_n$ su Teichmüller. Entonces el conjunto de vectores
    definidos como
    \begin{equation}
      \ket{a; b}
      = 2^{-n / 2} \left(
        \eta^{\tr\left( ax + 2bx \right)}
      \right)_{x \in T_n},
    \end{equation}
    donde $\eta = e^{2\pi i / 4}$, son bases mutuamente
    insesgadas del espacio $\C^{2^{n}}$.
  \end{proposition}

  Godsil y Roy definen las matrices de Pauli para $a, u \in
  T$ como
  \begin{align}
    X(a) &: \ket u \mapsto \ket{u+a+2 \sqrt{ua}} \\
    Z(a) &: \ket u \mapsto (-1)^{\tr(au)} \ket{u}
    = i^{\tr(2au)}\ket u.
  \end{align}
  \begin{proposition}
    Si $c,d \in T$, los vectores
    \begin{equation}
      \ket{c,d}
      = \sum_{x \in T}^{} i^{\tr(cx^2+2dx)} \ket x,
    \end{equation}
    son eigenvectores de los operadores de desplazamiento
    $D(a,b)$, donde $a,b \in T$.
  \end{proposition}

  \section{Campos y anillos de Galois}

  \subsection{Extensiones de campo}

  Como hemos visto, la estructura algebráica del campo
  finito es necesaria para poder construir una función de
  Wigner discreta con todas las propiedades deseables. La
  artimética $d$-modular no es suficiente ya que no se
  preservan ciertas propiedades importantes de una geometría
  finita. Los campos $\Z_p$ donde $p$ es un número primo son
  los más sencillos de manejar porque las operaciones se
  hacen módulo $p$, pero ésto se vuelve inpráctico porque no
  todo sistema cuántico puede ser modelado por un espacio de
  Hilbert de dimensión prima. El concepto de una
  \textit{extensión de campos} nos permite augmentar el
  campo $\Z_p$ a un campo finito de orden $p^{n}$. A la
  extensión de campo se le dice \textit{extensión de Galois}
  y son de característica $p$.

  Consideremos al anillo de polinomios $\Z_p[x]$ con
  coeficientes en el campo primo $\Z_p$. Sea $p(x)$ un
  polynomio mónico irreducible de grado $n$:
  \begin{equation}
    p(x)
    = c_0 + c_1 x + \cdots + c_{n-1} x^{n-1} + x^{n},
    \quad c_n \in \Z_p.
  \end{equation}
  El cociente $\Z_p[x] / p(x)$ es un representación del
  campo de Galois $\GF(p^{n})$. Distintos polinomios
  irreducibles del mismo grado nos brinda campos finitos
  isomorfos. Las operaciones de la extensión de campo son
  las operaciones de suma y multiplicación usuales, módulo
  el polinomio $p(x)$. Es importante notar que un elemento
  $\alpha \in \GF(p^{n})$ se puede ver como un vector
  $(\alpha_0, \alpha_1, \ldots, \alpha_{n-1})$ en
  $(\Z_p)^{n}$, utilizando la base
  $\{1,\omega,\ldots,\omega^{n-1}\}$ donde $\omega$ es una
  raíz del polinomio irreducible. De ésta manera la adición
  en la extensión se reduce a la adición usual del espacio
  vectorial $(\Z_p)^{n}$. Enseguida definimos algunos
  conceptos necesarios para trabajar con el espacio de fase
  discreto y para desarrollar la metodología de Wootters
  para la construcción de la función de Wigner.

  \begin{definition}
    Sea $\GF(p^{n})$ un campo de Galois, definimos la
    transformación de Frobenius como
    \begin{equation}
      \sigma(\alpha) = \alpha^{p};
      \quad
      \sigma^{n} = 1.
    \end{equation}
    Dicha transformación define un automorfismo en
    $\GF(p^{n})$ y produce los \textit{conjugados de
    Galois}.
  \end{definition}

  \begin{definition}
    La traza de un elemento $\alpha \in \GF(p^{n})$ es la
    suma de todos sus conjugados, y es un elemento del campo
    primo $\Z_p$:
    \begin{equation}
      \tr(\alpha)
      = \tr_{n / 1}(\alpha)
      = \alpha + \alpha^{p} + \alpha^{p^2} + \cdots +
      \alpha^{p^{n-1}}.
    \end{equation}
  \end{definition}
  La traza así definida es la traza \textit{respecto} a la
  extensión de $\Z_p$ a $\GF(p^{n})$. La traza es lineal
  respecto a la suma en los elementos de la extensión, y si
  $\alpha \in \Z_p$ entonces $\tr(\alpha\beta) =
  \alpha\tr(\beta)$ para todo $\beta \in \GF(p^{n})$.

  \subsection{Bases de una extensión de Galois}

  . . .

  Para construir la base dual y para el cálculo de la traza de
  los elementos de la extensión, resulta ventajoso definir
  la siguiente matriz con elementos en el campo primo.

  \begin{proposition}
    Sean $g,G$ las siguientes matrices simétricas e
    invertibles $n \times n$ con elementos en $\Z_p$:
    \begin{equation}
      g_{ij}
      = \tr(\omega^{i+j});
      \quad
      G = g^{-1},
      \quad
      i,j = 0,\ldots,n-1.
    \end{equation}
    El conjunto de elementos $\{E_0,E_1,\ldots,E_{n-1}\}$ 
    definido como
    \begin{equation}
      E_i = \sum_{j}^{} G_{ij} \omega^{j},
    \end{equation}
    es una base dual a
    $\{1,\omega,\omega^2,\ldots,\omega^{n-1}\}$.
  \end{proposition}

  Con lo anterior, podemos calcular los
  componentes de cualquier elemento $\alpha$ de la extensión
  de campos, en términos de la base o de su dual:
  \begin{align}
    \alpha_i
    &= \tr(\alpha E E_i);
    \quad
    \overline{\alpha}_i
    = \tr(\alpha \omega^{i}) \\
    \alpha_i 
    &= \sum_{j}^{} G_{ij} \overline{\alpha}_j;
    \quad
    \overline{\alpha}_i
    = \sum_{j}^{} g_{ij} \alpha_j.
  \end{align} 

  \subsection{Sistemas compuestos y factorización tensorial}

  Sabemos que una extensión de Galois $\GF(p^{n})$ se puede
  ver como un espacio vectorial sobre el campo primo $\Z_p$
  respecto a la adición del campo. Pero también tiene más
  estructra, como la multiplicación y las transformaciones
  de Frobenius. Por ésto, Vourdas [CITE] nota que
  representar un sistema cuántico con una extensión de
  Galois de grado $n$ no es lo mismo que representarlo con
  una suma directa del campo primo. 

  Sea $\ket{\alpha}$ un base ortonormal de un espacio de
  Hilbert $\H$ de dimensión $p^{n}$, indexada por los
  elementos de una extensión de Galois. Además sea $\H_p$ un
  espacio de dimensión $p$ y $\ket{k}$ un base ortonormal de
  $\H_p$ indexado por los elementos del campo primo $\Z_p$.
  Utilizando la base natural de la extensión de campo,
  podemos expresar a cualquier elemento $\alpha \in
  \GF(p^{n})$ como $\alpha = \sum_{i}^{} \alpha_i \omega^{i}
  $, ésto nos da un mapeo biyectivo:
  \begin{equation}
    \label{eqn:extension_field_map}
    \alpha \mapsto (\alpha_0,\ldots,\alpha_{n-1}),
  \end{equation} 
  lo cual induce una correspondencia entre los elementos de
  la base de $\H$ y $\H_p$:
  \begin{equation}
    \ket{\alpha}
    = \ket{\alpha_0 + \alpha_1 \omega + \ldots +
      \alpha_{n-1}\omega^{n-1}}
    \mapsto \ket{\alpha_0} \otimes \ket{\alpha_1} \otimes
    \cdots \otimes \ket{\alpha_{n-1}},
  \end{equation}
  la cual a su vez nos da una correspondencia entre los
  espacios de Hilbert $\H$ y $\H_p \otimes \cdots \otimes
  \H_p$. Notemos que la correspondencia depende de la
  elección de base del campo, por lo que la construcción de
  otros objetos por medio del producto tensorial también
  serán afectados. Por ejemplo los operadores de
  desplazamiento que actuán sobre el espacio de Hilbert de
  dimensión $p^{n}$ pueden ser expresados en términos de
  productos tensoriales de operadores de desplazamiento en
  los subsistemas. El mapeo(\ref{eqn:extension_field_map})
  nos permite hacer ésto como:
  \begin{equation}
    D(\alpha,\beta)
    = D(\alpha_0, \beta_0) \otimes \cdots \otimes
    D(\alpha_{n-1}, \beta_{n-1}),
  \end{equation}
  donde $\alpha_i$ son los componentes de $\alpha$ en la
  base elegida, y $\beta_i$ son los componentes respecto a
  la base dual, y los operadores $D(\alpha_i,\beta_i)$ son
  los operadores de desplazamiento actuando sobre los
  subsistemas $\H_p$.

  Klimov. . .

\chapter{Apéndices}

\section{La mecánica clásica y el espacio de fase}

\label{sec:classical-phase-space}

El concepto del espacio de fase es una herramienta de la
mecánica clásica, la cual describe la evolución temporal de
un sistema físico. Dicho de una manera muy sencilla, la
mecánica clásica estudia partículas y sus trayectorias, las
cuales se rigen de acuerdo a las leyes de Newton.  Se
considera que la partícula se `mueve' en un espacio
euclideano, es decir, su \textit{posición} está dado por $x
= (x_1,\ldots,x_n) \in \R^{n}$. El \textit{momentum} es una
cantidad dada por $p_j = m \dot x_j$, donde $\dot x$ es la
derivada respecto al tiempo de la posición, es decir, la
velocidad de la partícula, y $m$ es la \textit{masa} de la
partícula. Las cantidades que uno desea medir de nuestro
sistema físico se les llama \textit{observables}, y en la
mecánica clásica, son las funciones continuas que tienen
como argumentos las cantidades $x,p$ y $m$. Ejemplos de
ellos son el momentum, la energía cinética, la energía
potencial, etc. La función de energía más usual es la que
está dada como la suma de la \textit{energía cinética} y
\textit{energía potencial}:
\begin{equation}
  H(x,p)
  = \frac{1}{2m} \sum_{j=1}^{n} p_j^2 + V(x).
\end{equation}
A la energía del sistema se le conoce como el
\textit{Hamiltoniano}.  Utilizando ésta función de energía,
la ley de Newton nos brinda las ecuaciones de movimiento de
la partícula en cuestión:
\begin{equation}
  \frac{dx_j}{dt}
  = \frac{\partial H}{\partial p_j},
  \quad
  \frac{dp_j}{dt}
  = -\frac{\partial H}{\partial x_j}.
\end{equation}
Expresada como un sistema de ecuaciones diferenciales, a
la ley de Newton se le conoce como las \textit{ecuaciones de
Hamilton}. Con ésto, es natural representar el estado del
sistema clásico considerando el par $(x,p) \in \R^{2n}$. Al
espacio $\R^{2n}$ se le conoce como el \textit{espacio de
fase}. A las soluciones de las ecuaciones de Hamilton se les
conoce como \textit{trayectorias}, y son curvas que viven en
el espacio de fase.
\begin{definition}
  El espacio de fase de una partícula que se mueve en
  $\R^{n}$ es $\R^{2n}$, considerado como el conjunto de
  las $(2n)$-tuplas de la forma
  \[
    \left(
      x_1, \ldots, x_n, p_1, \ldots, p_n
    \right),
  \] 
  donde $x_j$ y $p_j$ son elementos de $\R$.
\end{definition}

Cabe mencionar que el tratado moderno de la mecánica clásica
está fundamentado en la geometría diferencial de las
variedades simplécticas \cite{mcinerney2013}, en donde el
espacio de fase se define como el espacio cotangente del
espacio de configuraciones $T^{*} \R_x \cong \R_x \times
\R_p$, donde el espacio de configuraciones  $\R_x$ es el
espacio de posición y el $\R_p$ es el espacio del momentum.
Para nuestros objetivos basta con desginar el espacio de
fase como el espacio $\R^{2n}$.

Dado que la mecánica cuántica es una teoría estadística, es
más apropiado comparar la mecánica de una sola partícula
cuántica con un \textit{conjunto} de partículas clásicas. La
maquinaría utilizada para éste propósito es la mecánica
estadística, especialmente la ecuación de Liouville para la
distribución de un conjunto de partículas.

%Podemos considerar que una partícula ocupa un punto
%geométrico $x$ en el espacio euclideano $\R^{n}$ conocido
%como el espacio de configuración o e posición, y con un
%momentum $p$ en otro espacio euclideano $\R^{n}$ conocido
%como el espacio de momentum. Idealmente los valores $x$ y
%$p$ puede ser medidos de manera precisa en cualquier tiempo
%$t$. Los observables clásicos son funciones de las
%cantidades $x$ y $p$, y el conjunto de todos los pares
%$(x,p)$ es el espacio de fase $\R^{2n}$. Las ecuaciones de
%movimiento de Newton poseen varias simetrías, por ejemplo
%las ecuaciones pueden ser invariantes bajo cambios en
%posición (shift) y cambios en momentum (boost).

%Un sistema físico constituido por varias partículas
%clásicas se describe por un conjunto de parámetros
%$(x_i,p_i,t)$ los cuales en principio son observables con
%total precisión.

%Notemos que por el momento no ha surgido ninguna
%interpretación probabilística en la mecánica clásica, pues
%es una teoría determinística. Dado que la teoría cuántica
%es probabilística, es natural preguntarnos ¿qué
%características de la mecánica clásica serían deseables en
%la teoría cuántica? y ¿qué beneficios habría en hacer un
%vínculo entre la mecánica clásica y la cuántica?
%\cite{schroeck1996}, especialmente cuando sabemos que la
%teoría cuántica (y sus derivados) es nuestra teoría más
%precisa. Un esfuerzo por vincular las descripciones
%clásicas y cuánticas del mundo es la representación de
%Wigner-Weyl-Moyal de la mecánica cuántica.  Ésta es una
%formulación que intenta usar la noción del espacio de fase
%en la dinámica cuántica y la idea básica es la construcción
%de \textit{cuasi-distribuciones} real-valuadas que
%representan a los sistemas cuánticos.

\section{Campos finitos}
\label{sec:fields}

Si intentamos definir el espacio de fase discreto sobre un
anillo como por ejemplo $\Z_4$, tendremos problemas a la
hora de trabajar con rectas y otras estructuras geométricas.
Por ejemplo, la recta
\[
  x + 2p = 0,
\]
tiene como solución al conjunto de puntos
\begin{equation}
  \label{eqn:ap_fields_1}
  \{(0,0), (2,1), (0,2), (2,3)\}.
\end{equation} 
Similarmente, la recta 
\[
  x = 0,
\] 
tiene como solución al conjunto
\begin{equation}
  \label{eqn:ap_fields_2}
  \{(0,0), (0,1), (0,2), (0,3)\}.
\end{equation} 
Notemos que los dos conjuntos tiene una intersección con
\textit{más} de un elemento, algo que contradice nuestra
noción geoemétrica de una linea en el espacio, ya que ambas
lineas son distintas. Ésto es un ejemplo de la falta de
estructura geométrica de un espacio de fase discreto cuando
solo utilizamos un anillo. Para poder preservar las nociones
geométricas del espacio Euclideano es necesario utilizar una
estructura algebráica más \textit{rica}, especificamente la
de un campo finito.

Lo que sigue es un resumén breve de los conceptos y
resultados que fundamentan el uso de los campos finitos para
trabajar con la geometría necesaria del espacio de fase
discreto. La mayoría está basado en las notas de Zhe-Xian
Wan \cite{wan} y en el libro de Rudolf Lidl y Harald
Niederreiter \cite{lidl1994}.

\begin{definition}
  Un anillo $(R,+,\cdot)$ es un conjunto $R$, con dos
  operaciones binarias, denotadas por $+$ y $\cdot$, tales
  que:
  \begin{enumerate}
    \item $R$ es un grupo abeliano respecto a la suma $+$.
    \item La operación $\cdot$ es asociativa.
    \item Las leyes distributivas se cumplen, es decir, para
      todo $a,b,c \in R$ tenemos que
      \[
        a \cdot (b+c) = a \cdot b + a \cdot c,
        \quad
        \text{y}
        \quad
        (b+c) \cdot a = b \cdot a + c \cdot a.
      \] 
  \end{enumerate}
\end{definition}
Un anillo $R$ se llama \textit{anillo con identidad} si
tiene una identidad multiplicativa. Si la operación $\cdot$
es conmutativa entonces el anillo es conmutativo. Decimos
que el anillo es un \textit{dominio integral} si es un
anillo conmutatitivo con identidad $1 \neq 0$, tal que $ab =
0$ implica que $a = 0$ o $b = 0$. Un anillo es un
\textit{anillo de división} si los elementos no cero de $R$
forman un grupo bajo $\cdot$. En particular un anillo de
división conmutativo es un \textit{campo}. Visto de otra
manera tenemos que:
\begin{definition}
  Sea $\F$ un conjunto con dos operaciones binarias, una
  llamada suma $+$ y otra llamada multiplicación $\cdot$.
  Decimos que $\F$ es un campo si
  \begin{enumerate}
    \item $(\F,+)$ es un grupo abeliano.
    \item $(\F^{*},\cdot)$ es un grupo abeliano, donde
      $\F^{*} = \F \setminus \{0\}$ y $0$ es el cero del
      grupo aditivo.
    \item Se satisfacen las propiedades distributivas:
      \[
        a(b+c) = ab+ac,
        \quad
        (b+c)a = ba+ca,
      \] 
      para todo $a,b,c \in \F$.
  \end{enumerate}
\end{definition}
El inverso aditivo de un elemento $a \in \F$ se denota como
$-a$ y el inverso multiplicativo se denota $a^{-1}$.
\begin{definition}
  Sea $\F$ un campo. Si el número de elementos de $\F$ es
  infinito, entonces $\F$ es un \textit{campo infinito}. Si
  el número de elementos es finito, decimos que $\F$ es un
  \textit{campo finito}.
\end{definition}
Todos los campos son en particular un dominio integral, pero
el converso no es siempre cierto, al menos que el dominio
integral sea finito.
\begin{example}
  Sea $n$ un número entero. El conjunto $\Z_n = \Z / n\Z$
  consiste de $n$ clases residuales
  \[
    [a] = a + n\Z = \{a + nk : k \in \Z\}.
  \] 
  Como de costumbre, omitimos la notación de clase de
  equivalencia $[\cdot]$. En $\Z_n$ tenemos que
  \[
    \underbrace{1 + 1 + \cdots + 1}_{n \text{ veces}} 
    = 0.
  \] 
  Notemos que $\Z_n$ simplemente consiste de los enteros
  módulo $n$. En el caso en $n$ es un número primo $p$,
  entonces el anillo de clases residuales de los enteros
  módulo el ideal principal generado por $p$, $\Z_p$, es un
  campo.
\end{example}
\begin{definition}
  Sea $\F$ un campo y $1$ su identidad multiplicativa. Si
  para cualquier entero positivo $m$ tenemos que $m \cdot 1
  \neq 0$, entonces la \textit{característica} de  $\F$ es
  0. Si existe un entero positivo $m$ tal que $m \cdot 1 =
  0$, entonces el entero $p$ más pequeño que satisface $p
  \cdot 1 = 0$ se llama la \textit{característica} de $\F$.
\end{definition}
Los campos $\Q, \R$ y $\C$ son de característica 0, mientras
que $\Z_p$ es de característica prima $p$. Es fácil probar
que la característica de un campo es 0 o es un número primo.
Si $\F$ es un campo de característica $p$ entonces
\[
  \Pi = \{0, 1, 2 \cdot 1,\ldots,(p-1) \cdot 1\},
\] 
es un \textit{subcampo} de $\F$. De hecho, es el subcampo
más pequeño de $\F$ y se le conoce como el \textit{campo
primo} ya que es isomorfo a $\Z_p$.
\begin{lemma}
  Sea $\F$ un campo finito de característica $p \neq 0$. Sea
  $n$ un entero no negativo, entonces el mapeo $\sigma_n :
  \F \to \F$ dado por
  \[
    \sigma_n : a \mapsto a^{p^{n}},
    \quad a \in \F
  \] 
  es un automorfismo de $\F$.
\end{lemma}

Es natural preguntarnos como podemos construir campos
finitos de orden no primo. Para ésto investiguemos
brevemente a los campos de clases residuales. Sea $\F$ un
campo y $m$ un elemento fijo de $\F$. Entonces para
cualesquiera dos elementos $a,b \in \F$, decimos que
\begin{equation}
  a \cong b \, (\mod m)
\end{equation}
si y solo si $m | (a-b)$. Ésto es una clase de equivalencia
en $\F$ y cada clase de equivalencia se conoce como
\textit{clase residual} módulo $m$.
\begin{definition}
  Sea $\F$ un campo, $\F[x]$ el anillo de polinomios sobre
  $\F$ y $f(x)$ un polinomio en $\F[x]$. El anillo
  \[
    F[x] / \langle f(x) \rangle
  \] 
  se conoce como el anillo de clases residuales del anillo
  polinomial $\F[x]$ módulo el polinomio $f(x)$. Más aún, si
  $f(x)$ es un polinomio irreducible sobre $\F$, entonces
  $\F[x] / \langle f(x) \rangle$ es un campo.
\end{definition}
\begin{theorem}
  Sea $\F$ un campo, $\E$ un subcampo de $\F$ y $\alpha \in
  \E$. Sea $p(x)$ un polinomio irreducible de grado $n$ 
  sobre $\F_p$ y supongamos que $p(\alpha) = 0$. Entonces
  $\F[\alpha]$ es un subcampo de $\E$,
  \begin{equation}
    \F[\alpha]
    = \{a_0 + a_1\alpha + a_2 \alpha^2 + \cdots a_{n-1}
    \alpha^{n-1} : \alpha_i \in \F\},
  \end{equation}
  y todo elemento de $\F[\alpha]$ puede ser expresado de
  manera única como
  \begin{equation}
    a_0 + a_1\alpha + a_2 \alpha^2 + \cdots + a_{n-1}
    \alpha^{n-1}.
  \end{equation}
  Además $F[\alpha]$ es isomofo a la clase residual $\F[x] /
  \langle p(x) \rangle$.
\end{theorem}
\begin{theorem}
  Sea $\F$ un campo y $p(x)$ un polinomio irreducible de
  grado $n$ sobre $\F$. Denotemos la clase residual de $x
  \mod p(x)$ por $\alpha$. Entonces
  \begin{equation}
    F[x] / \langle p(x) \rangle
    \cong F[\alpha].
  \end{equation}
  Además si $\F$ es un campo finito con $p$ elementos,
  entonces el orden de  $F[x] / \langle p(x) \rangle$ es
  $p^{n}$.
\end{theorem}
El teorema anterior nos dice que para construir a $\F_d$
donde $d = p^{n}$ requerimos de un polinomio $f(x)$ de grado
$n$ que es irreducible en $\F_p$.  Si $\alpha$ es una raíz
de $f(x)$, el campo que obtenemos al adjuntar $\alpha$ a
$\F_p$ es
\begin{equation}
  \F_N
  = \F_r(\alpha) \cong \F_r[x] / \langle f(x) \rangle.
\end{equation} 
\begin{example}
  Consideremos el campo $\Z_2$ y el polinomio $f(x) = x^2
  + x + 1$. Notemos que $f(0) = f(1) = 1$, por lo tanto
  $f(x)$ no tiene raíces en $\Z_2$, i.e., $f(x)$ es
  irreducible sobre $\Z_2$. De acuerdo al teorema anterior,
  la extensión de Galois $\Z_2 / \langle f(x) \rangle$ tiene
  $2^2$ elementos y está dada por
  \[
    \F_4
    := \Z_2[x] / (x^2 + x + 1)
    \cong \Z_2[\alpha]
    = \{0, 1, \alpha, \alpha + 1\},
  \] 
  donde $\alpha$ es la clase residual de $x \mod x^2+x+1$.
  La tabla de adición está dada por
  \begin{table}[ht]
    \centering
    \label{tab:F4_addition}
    \begin{tabular}{c | c c c c}
      $+$ & 0 & 1 & $\alpha$ & $\alpha+1$ \\
      \hline
      0 & 0 & 1 & $\alpha$ & $\alpha+1$ \\
      1 & 1 & 0 & $\alpha + 1$ & $\alpha$ \\
      $\alpha$ & $\alpha$ & $\alpha+1$ & 0 & 1 \\
      $\alpha + 1$ & $\alpha+1$ & $\alpha$ & 1 & 0
    \end{tabular}
    \caption{Tabla de adición de $\F_4$.}
  \end{table}
  Similarmente podemos calcular la tabla de multiplicación.
  \begin{table}[ht]
    \centering
    \label{tab:label}
    \begin{tabular}{c | c c c c}
      $\cdot$ & 0 & 1 & $\alpha$ & $\alpha+1$ \\
      \hline
      0 & 0 & 0 & 0 & 0 \\
      1 & 0 & 1 & $\alpha$ & $\alpha+1$ \\
      $\alpha$ & 0 & $\alpha$ & $\alpha+1$ & 1 \\
      $\alpha+1$ & 0 & $\alpha+1$ & 1 & $\alpha$\\
    \end{tabular}
    \caption{Tabla de multiplicación de $\F_4$.}
  \end{table}
  Cualquier campo de cuatro elementos es isomorfo a $\F_4$.
\end{example}
Si $\F_d$ es un campo finito de $d$ elementos, el grupo
multiplicativo $\F_d^{*}$ es de orden $d - 1$. El orden de
cada elemento de éste grupo es un divisor de $d-1$, por lo
tanto $a^{d-1} = 1$ para todo $a \in \F_d^{*}$. Además se
puede probar que el grupo multiplicativo de cualquier campo
finito es cíclico. A los generadores del grupo cíclico
$\F_d^{*}$ se les conoce como \textit{elementos primitivos}.


\subsection{Automorfismos de campos finitos}

Sea $d$ una potencia de un primo y $\F_d$ un campo finito.
Sea $n$ un entero positivo, entonces podemos ver a $\F_d$
como un subcampo del campo $\F_{d^{n}}$. El mapeo
\begin{equation}
  \sigma : \alpha \mapsto \alpha^{d}
\end{equation}
de $\F_{d^{n}}$ a si mismo es un automorfismo de
$\F_{d^{n}}$. Resulta que $\sigma(a) = a$ si y solo si $a
\in \F_d$, i.e., los elementos del subcampo son invariantes
bajo $\sigma$. Al automorfismo $\sigma$ de $\F_d$ se le
conoce como el \textit{automorfismo de Frobenius}. Denotemos
por $\sigma^2$ a la composición $\sigma \circ \sigma$,
entonces $\sigma^2$ también es un automorfismo de
$\F_{d^{n}}$ sobre $\F_d$. En general, si $\sigma^{0} = 1$
donde $1$ en éste caso significa el mapeo identidad sobre
$\F_{d^{n}}$, tenemos que
\begin{equation}
  \sigma^{i+1}
  = \sigma \circ \sigma^{i},
  \quad i = 1,2,\ldots,
\end{equation}
son automorfismos sobre $\F_d$. Notemos que $\sigma^{n} =
1$. Además de que $\sigma$ fija a todo elemento de $\F_d$,
también tenemos que $\sigma^{1}$ y $\sigma^{k} \neq 1$ para
$1 \leq k < n$, y $\sigma^{0} = 1, \sigma, \sigma^2, \ldots,
\sigma^{n-1}$ son $n$ automorfismos \textit{distintos} de
$\F_{d^{n}}$ sobre $\F_d$. El conjunto de automorfismos
de $\F_{d^{n}}$ sobre $\F_d$ forman un grupo respecto a la
composición, llamado el \textit{grupo de Galois} de
$\F_{q^{n}}$ sobre $\F_d$, denotado por
$\Gal\left(\F_{d^{n}} / \F_d\right)$.  Se puede demostrar
que
\begin{equation}
  \Gal\left( \F_{d^{n}} / \F_d \right) 
  = \langle \sigma \rangle,
\end{equation}
i.e., todo automorfismo puede ser expresado como una
potencia del automorfismo de Frobenius. En otras palabras,
los automorfismos $\sigma^{0} = 1$, $\sigma, \ldots,
\sigma^{n-1}$ son \textit{todos} los automorfismos de
$\F_{d^{n}}$ sobre $\F_d$.

\begin{definition}
  Sea $d$ una potencia de un primo y $n$ un entero positivo.
  Podemos asumir que $\F_d$ es un subcampo de $\F_{d^{n}}$.
  Sea $\sigma$ el automorfismo de Frobenius de $\F_{d^{n}}$ 
  sobre $\F_d$. Si $\alpha \in \F_{d^{n}}$, su traza
  relativa a $\F_q$ es
  \begin{equation}
    \tr_{\F_{d^{n}} / \F_d}(\alpha)
    = \sum_{i = 0}^{n-1} \sigma^{i}(\alpha)
    = \alpha + \alpha^{d} + \alpha^{d^2} + \cdots +
    \alpha^{d^{n-1}}.
  \end{equation}
\end{definition}
Si el campo y subcampo son claros dentro del contexto en que
se maneja la traza, simplemente la denotamos por $\tr$. La
operación $\tr$ mapea a todo elemento del campo finito a un
elemento del subcampo $\tr : \F_{d^{n}} \to \F_d$. La traza
satisface la siguientes propiedades.
\begin{theorem}
  Para $\alpha,\beta \in \F_{d^{n}}$ y $a \in \F_d$ tenemos
  \begin{itemize}
    \item $\tr(\alpha) \in \F_d$.
    \item $\tr(\alpha+\beta) = \tr(\alpha) + \tr(\beta)$;
    \item $\tr(a\alpha) = a \tr(\alpha)$ y $\tr(a) = n
      \tr(a)$.
    \item $\tr\left( \alpha^{d} \right) = \tr(\alpha)$.
    \item La traza es suprayectiva y para $\alpha \in
      \F_{d^{n}}$, $\tr(\alpha) = 0$ si y solo si existe un
      elemento $\beta \in \F_{d^{n}}$ tal que $\alpha =
      \beta - \beta^{d}$.
    \item Para cualquier $a \in \F_d$, el número de
      elementos $\alpha \in \F_{d^{n}}$ tales que
      $\tr(\alpha) = a$ es $d^{n-1}$.
  \end{itemize}
\end{theorem}

\subsection{Bases de campos}

Sean $\F_{d^{n}}$ y $\F_d$ campos finitos, donde $d$ es una
potencia de un primo y $n$ es un entero positivo. Supongamos
además que $\F_d$ es un subcampo de $\F_{d^{n}}$. Podemos
ver al campo $\F_{d^{n}}$ como un espacio vectorial sobre
$\F_d$ al definir una multiplicación escalar de la siguiente
manera:
\begin{align*}
  \F_d \times \F_{d^{n}} &\to \F_{d^{n}} \\
  (a, \alpha) &\mapsto a\alpha.
\end{align*}
Supongamos que $\dim \F_{d^{n}} = m$ y sea $\alpha_1,
\alpha_2, \ldots, \alpha_m$ una base del espacio vectorial
$\F_{d^{n}}$ sobre $\F_d$. Todo elemento $\beta \in
\F_{d^{n}}$ puede ser expresado como una combinación lineal
de elementos de la base con coeficientes en $\F_d$:
\[
  \beta = b_1 \alpha_1 + \ldots + b_n \alpha_m.
\] 
Por lo tanto $\left|\F_{d^{n}}\right|$ lo que implica que $m
= n$. En general, $\F_{d^{n}}$ es un espacio vectorial sobre
$\F_d$ de dimensión $n$. A la cardinalidad de la base se le
conoce grado de la extensión $\F_{d^{n}}$ sobre $\F_d$ y es
común denotarla como
\[
  [\F_{d^{n}} : \F_d] = n.
\] 
Sumar y restar elementos de la base se puede hacer
utilizando la base de manera natural. Para obtener las
coordenadas de una suma $\beta + \gamma$, basta con sumar
las coordenadas de $\beta$ y de $\gamma$ en la base. Podemos
hacer algo similar para la multiplicación de los elementos.
Podemos obtener más bases por medio de matrices no
singulares.
\begin{theorem}
  Sea $\{\alpha_1,\alpha_2,\ldots,\alpha_n\}$ una base de
  $\F_{d^{n}}$ sobre $\F_d$ y 
  \[
    (a_{ij}), 
    \quad 1 \leq i,j \leq n,
  \] 
  una matriz $n \times n$ no singular sobre $\F_d$. Los
  elementos
  \[
    \beta_j
    = \sum_{i=1}^{n} a_{ij} \alpha_i,
    \quad j = 1,2,\ldots,n
  \] 
  forman una base de $\F_{d^{n}}$ sobre $\F_d$.
\end{theorem}
Existen distintos criterios para identificar cuando un
conjunto de elementos del campo es una base para el espacio
vectorial, pero éstos criterios no serán importantes para
nosotros. Por ahora, consideremos un elemento $\alpha \in
\F_{d^{n}}$ de grado $n$ sobre $\F_d$. Entonces
\[
  1, \alpha, \alpha^2, \ldots, \alpha^{n-1}
\] 
es una base del campo. Ésta base comunmente se conoce como
la \textit{base polinomial}. Enseguida introducimos el
concepto de una base dual del campo finito.
\begin{definition}
  Sean $\{\alpha_1,\alpha_2,\ldots,\alpha_n\}$ y
  $\{\beta_1,\beta_2,\ldots,\beta_n\}$ dos bases de
  $\F_{d^{n}}$ sobre $\F_d$. Si
  \[
    \tr(\alpha_i \beta_j)
    = \delta_{ij},
    \quad
    \text{para todo}
    \quad
    i,j = 1,2,\ldots,n,
  \] 
  entonces decimos que las bases son duales una respecto a
  la otra, en particular
  $\{\beta_1,\beta_2,\ldots,\beta_n\}$ es una base dual a
  $\{\alpha_1,\alpha_2,\ldots,\alpha_n\}$.
\end{definition}
Toda base de un campo tiene una \textit{única} base dual.
Para construir la base dual y para el cálculo de la traza de
los elementos de la extensión, resulta ventajoso definir la
siguiente matriz con elementos en el campo primo.
\begin{proposition}
  Sean $g,G$ las siguientes matrices simétricas e
  invertibles $n \times n$ con elementos en $\Z_p$:
  \begin{equation}
    g_{ij}
    = \tr(\omega^{i+j});
    \quad
    G = g^{-1},
    \quad
    i,j = 1,\ldots,n.
  \end{equation}
  El conjunto de elementos
  $\{\beta_1,\beta_2,\ldots,\beta_n\}$ definido como
  \begin{equation}
    \beta_i = \sum_{j}^{} G_{ij} \omega^{j},
  \end{equation}
  es una base dual a
  $\{1,\alpha,\alpha^2,\ldots,\alpha^{n-1}\}$.
\end{proposition}
Con lo anterior, podemos calcular los componentes de
cualquier elemento $\alpha$ de la extensión de campos, en
términos de la base o de su dual:
\begin{align}
  \alpha_i
  &= \tr(\alpha \beta_i);
  \quad
  \overline{\alpha}_i
  = \tr(\alpha \alpha^{i}) \\
  \alpha_i 
  &= \sum_{j}^{} G_{ij} \overline{\alpha}_j;
  \quad
  \overline{\alpha}_i
  = \sum_{j}^{} g_{ij} \alpha_j.
\end{align} 

Entre las bases que podemos elegir de un campo finito,
a parte de la base polinomial, existen bases que resultan
ser muy útiles, especificamente para la construcción de
Wootters de los operadores de Pauli generalizados. Ésto es
el concepto de una base \textit{auto-dual}.
\begin{definition}
  Una base $\{\alpha_1,\alpha_2,\ldots,\alpha_n\}$ de
  $\F_{d^{n}}$ sobre $\F_d$ es una base auto-dual si
  \begin{equation}
    \tr_{\F_{d^{n}} / \F_d}(\alpha_i \alpha_j)
    = \delta_{ij},
    \quad
    \text{para todo}
    \quad
    i,j = 1,2,\ldots,n.
  \end{equation}
\end{definition}
La existencia de una base auto-dual no es garantizada para
toda potencia de primos. Pero, para $d = 2^{r}$ sí existe
una base auto-dual de $\F_{d^{n}}$ sobre $\F_d$. El caso de
característica impar es más sutil. Si $d$ es una potencia de
un primo impar, entonces existe una base auto-dual de
$\F_{d^{n}}$ sobre $\F_d$ si y solo si $n$ también es impar.
Es natural preguntarnos ¿cuántas bases auto-duales existen
para un campo finito? La respuesta a ésta pregunta no es
trivial pero si definitiva, y resulta que también es
distinta para los casos de característica par e impar.
\begin{example}
  Sea $\alpha \in \F_{2^3}$ una raíz del polinomio
  irreducible $x^3+x^2+1$ en $\F_2[x]$. Entonces el conjunto
  $\{\alpha,\alpha^2,1+\alpha+\alpha^2\}$ es una base de
  $\F_{2^3}$ sobre $\F_2$. Además es fácil verificar que es
  una base auto-dual.
\end{example}

%\subsection{Diferencias entre característica par e impar}
%
%Ahora mencionamos algunas sutilizas o distinciones entre
%campos finitos de característica par e impar. Sea $\F_d$ un
%campo finito de $d$ elementos donde $d = p^{n}$ y $p$ es un
%número primo. Consideremos los conjuntos 
%\[
%  \F_d^2 := \{x^2 : x \in \F_d\}
%  \quad \text{y} \quad
%  \left(\F_d^*\right)^2 := \{x^2 : x \in \F_d^{*}\}.
%\] 
%Si $d = 2^{n}$, entonces todo elemento de $\F_d$ es un
%elemento cuadrado, es decir $\F_d = \F_d^{2}$ y $\F_d^{*} =
%\left( \F_d^{*} \right)^2$. Además, todo elemento de $\F_d$
%tiene una \textit{única} raíz en $\F_d$. Ésto no es el caso
%para característica impar. No todo elemento del grupo
%multiplicactivo tiene una raíz cuadrada. Cuando $p$ es
%impar, los elementos de $\left( \F_d^{*} \right)^2$ son
%elementos \textit{cuadrados} de $\F_d^{*}$ y los elementos
%$\F_d^{*} \setminus \left( \F_d^{*} \right)^2$ son elementos
%\textit{no-cuadrados}.
%
%Otra diferencia es la resolución de la ecuación cuadrática
%\begin{equation}
%  ax^2 + bx + c = 0,
%\end{equation}
%sobre un campo finito $\F$. Si $\F$ no es de característica
%dos, entonces la fórmula
%\begin{equation}
%  x = \frac{-b \pm \sqrt{b^2 - 4ac}}{2a}
%\end{equation}
%nos brinda sus soluciones. Ésta fórmula no funciona para
%característica par, pero ésto no significa que no se puede
%resolver para un campo $\F_{d^{n}}$. De hecho el siguiente
%teorema nos dice cuando existe una solución de la ecuación
%cuadrática.
%\begin{theorem}
%  Sea $ax^2 + bx + c = 0$ una ecuación cuadrática sobre
%  $\F_{2^{n}}$. Cuando $b = 0$, entonces la ecuación tiene
%  una única solución
%  \[
%    x = \left( \frac{c}{a} \right)^{2^{n-1}}.
%  \] 
%  Cuando $b \neq 0$, entonces hay dos casos:
%  \begin{itemize}
%    \item Si $\tr(ac / b^2) = 1$, no tiene solución.
%    \item Si $\tr(ac / b^2) = 0$, tiene dos soluciones.
%  \end{itemize}
%\end{theorem}

\subsection{Sumas exponenciales}
\label{subsec:exp_sums}

El uso de sumas exponenciales para campos finitos resulta
ser útil para varias aplicaciones. Hacemos un resumen breve
de algunos resultados ya que mucha de la literatura que
utilizamos para la construcción explícita de las MUBs
requiere de manera directa o indirecta de éstos resultados.
En particular exponemos (sin prueba) dos resultados que
utilizamos para demostrar la equivalencia de las sumas que
aparecen en el ejemplo de los tres qutrits. Las
demostraciones y detalles se pueden encontrar en el libro de
Lidl y Niederreiter \cite{lidl1997}.

Las sumas exponenciales están formadas por un grupo especial
de homomorfismos llamados \textit{caracteres}.
\begin{definition}
  Sea $G$ un grupo abeliano finito de orden $|G|$ con
  identidad $1_G$. Un \textit{caracter} $\chi$ de $G$ es un
  homomorfismo de $G$ a el grupo multiplicativo $U$ de los
  números complejos de valor absoluto unitario. Notemos que
  \[
    \left( \chi(g) \right)^{|G|}
    = \chi\left( g^{|G|} \right) 
    = \chi\left( 1_G \right) 
    = 1
  \] 
  para todo $g \in G$, por lo que los valores $\chi$ son las
  $|G|$-ésimas raíces de la unidad. 
\end{definition}
El conjunto $\hat G$ de todos los caracteres de $G$ forma un
grupo abeliano bajo la multiplicación de caracteres. Ahora
veamos algunas propiedades interesantes de los caracteres.
\begin{theorem}
  \label{thm:lidl_char_sum}
  Si $\chi$ es un caracter no trivial de un grupo finito
  abeliano $G$, entonces
  \begin{equation}
    \sum_{g \in G}^{} \chi(g) = 0.
  \end{equation}
  Además, si $g \in G$ tal que $g \neq 1_G$, entonces
  \begin{equation}
    \sum_{\chi \in \hat G}^{} \chi(g) = 0.
  \end{equation}
\end{theorem}
%De éste teorema se sigue que el número de caracteres de un
%grupo abeliano finito $G$ es igual a la cardinalidad de $G$,
%$|G|$. Éstas propiedades nos brindan unas relaciones de
%\textit{ortogonalidad} de los caracteres. Si $\chi$ y $\psi$
%son caracteres de $G$, entonces
%\begin{equation}
%  \frac{1}{|G|}
%  \sum_{g \in G}^{} \chi(g) \overline{\psi(g)}
%  = \begin{cases}
%    0 & \text{si } \chi \neq \psi, \\
%    1 & \text{si } \chi = \psi.
%  \end{cases}
%\end{equation}
Recordemos que en un campo finito $\F_d$, existen dos grupos
finitos abelianos de importancia, el grupo aditivo y el
grupo multiplicativo. En ambos casos podemos obtener
expresiones explicítas de los caracteres del grupo.
Consideremos el grupo aditivo de $\F_d$. Sea $p$ la
característica de $\F_d$, e identificamos al subcampo primo
de $\F_d$ con $\Z_p$. Definamos a la función $\chi_1$ de la
siguiente manera:
\begin{equation}
  \chi_1(c)
  = e^{2\pi i \tr(c) / p},
  \quad
  \text{para todo } c \in \F_d.
\end{equation}
Es trivial probar que $\chi_1$ es un caracter del grupo
aditivo, y le llamamos un \textit{caracter aditivo} de
$\F_d$. Cuando el contexto es claro, denotaremos a $\chi_1$
simplemente por $\chi$. Todos los caracteres aditivos de
$\F_d$ pueden ser expresados en términos del caracter
$\chi_1$. Para los caracteres aditivos $\chi_a$ y $\chi_b$
tenemos que
\begin{equation}
  \sum_{c \in \F_d}^{}
  \chi_a(c) \overline{\chi_b(c)}
  = \begin{cases}
    0 & \text{para } a \neq b \\
    d & \text{para } a = b,
  \end{cases}
\end{equation}
y en particular
\begin{equation}
  \sum_{c \in \F_d}^{} \chi_a(c) = 0,
  \quad
  \text{para } a \neq 0.
\end{equation}
Similarmente podemos formar los caracteres del grupo
multiplicativo $\F_d^{*}$ del campo finito $\F_d$, los
cuales se llaman \textit{caracteres multiplicativos}. Éstos
caracteres se pueden expresar de la siguiente manera.
\begin{theorem}
  Sea $g$ un elemento primitivo fijo de $\F_d$. Para todo $j
  = 0,1,\ldots,d-2$, la función $\psi_j$ dada por
  \begin{equation}
    \psi_j(g^{k})
    = e^{2\pi i j k / (d-1)},
    \quad
    \text{para todo } k = 0,1,\ldots,d-2,
  \end{equation}
  define un caracter multiplicativo en $\F_d$ y además todo
  caracter multiplicativo se obtiene de ésta manera.
\end{theorem}
Muchos de los resultados de las sumas Gaussianas y de Weil
que veremos enseguida dependen de ambos caracteres, pero
para nuestro trabajo basta con enfocarnos en el grupo de
caracteres aditivos. Sin embargo, definimos el concepto de
una suma Gaussiana de forma general, la cual involucra a los
caracteres multiplicativos. Sean $\psi$ un caracter
multiplicativo y $\chi$ un caracter aditivo de $\F_d$. La
suma Gaussiana $G(\psi,\chi)$ se define como
\begin{equation}
  G(\psi,\chi)
  = \sum_{c \in \F_d^{*}}^{} \psi(c) \chi(c).
\end{equation}
Las sumas Gaussianas satisfacen las siguientes propiedades
en cuanto a los posibles valores que pueden tomar.
\begin{theorem}
  Sean $\psi$ y $\chi$ caracteres multiplicativos y aditivos
  de $\F_d$ respectivamente. Entonces la suma Gaussiana
  satisface
  \begin{equation}
    G(\psi,\chi)
    = \begin{cases}
      d-1 & \text{para } \psi = \psi_0, \chi = \chi_0, \\
      -1 & \text{para } \psi = \psi_0, \chi \neq \chi_0, \\
      0 & \text{para } \psi \neq \psi_0, \chi = \chi_0,
    \end{cases}
  \end{equation}
  y además, si $\psi \neq \psi$ y $\chi \neq \chi_0$,
  entonces
  \begin{equation}
    |G(\psi,\chi)| = \sqrt{d}.
  \end{equation} 
\end{theorem}
%Por medio de las sumas Gaussianas podemos expresar a los
%caracteres multiplicativos en términos de los caracteres
%aditivos (y vice versa). Ésto asemeja a una expansión
%en términos de Fourier. En éste sentido las sumas Gaussianas
%vinculan ambos tipos de caracteres.
El teorema anterior es fundamental en muchas de las pruebas
de las construcciones de MUBs que investigamos ya que por
definición del producto interno del espacio de Hilbert $\H$,
el producto interno de dos elementos se expresa en términos
de sumas exponenciales. En nuestro trabajo no hicimos uso
explícito de éstos resultados ya que nos apoyamos en los
teoremas de Kanat y de Bandyophyay et al.

Otro tipo de sumas exponenciales importante para éste
trabajo son las sumas de caracteres con argumentos
polinomiales.
\begin{definition}
  Sea $\chi$ un caracter aditivo no trivial de $\F_d$ y sea
  $f \in \F_d[x]$ un polinomio de grado positivo. La suma de
  la forma
  \begin{equation}
    \sum_{c \in \F_d}^{} \chi\left( f(c) \right),
  \end{equation}
  se conoce como una suma de Weil.
\end{definition} 
Como mencionamos en el capítulo anterior, la evaluación de
éstas sumas generalmente es dificil. En ciertos casos es
posible tratar con éstas sumas de manera directa, por
ejemplo para polinomios lineales tenemos el siguiente
resultado.
\begin{theorem}
  \label{thm:lidl_linear_sum}
  Si $\chi$ es un caracter aditivo no trivial de $\F_d$ y
  $\gcd(n,d-1) = 1$, entonces
  \begin{equation}
    \sum_{c \in \F_d}^{} \chi\left( a c^{n} + b \right) 
    = 0,
  \end{equation}
  para todo $a,b \in \F_d$ con $a \neq 0$.
\end{theorem}
Para el siguiente caso particular necesitamos el concepto de un
$p$-polinomio afín. Un polinomio de la forma
\begin{equation}
  \label{eqn:affine_poly}
  L(x)
  = \sum_{i = 0}^{n} \alpha_i x^{p^{i}},
\end{equation}
con coeficientes en la extensión de campos $\F_{p^{n}}$ de
$\F_p$ se conoce como un $p$-polinomio sobre $\F_{p^{n}}$.
Un polinomio de la forma $A(x) = L(x) - \alpha$, donde
$L(x)$ es un $p$-polinomio sobre $\F_{p^{n}}$ y $\alpha \in
\F_{p^{n}}$ se conoce como un $p$-polinomio afín sobre
$\F_{p^{n}}$.
\begin{theorem}[Teorema 5.32 de \cite{lidl1997}]
  \label{thm:lidl_weil_poly}
  Sea $\F_d$ un campo finito de característica $p$ y sea
  \begin{equation}
    f(x) = a_r x^{p^{r}} + a_{r-1} x^{p^{r-1}} + \cdots +
    a_1 x^{p} + a_0 x + a
  \end{equation}
  un $p$-polinomio afín sobre $\F_d$. Además sea $\chi_b$
  con $b \in \F_d^{*}$ un caracter aditivo no trivial de
  $\F_d$ donde
  \begin{equation}
    \chi_b(c) = \chi(bc).
  \end{equation}
  Para simplificar la notación, definamos 
  \[
    g(x)
    = x a_r + x^{p} a_{r-1}^{p} + \cdots +
    x^{p^{r-1}}a_1^{p^{r-1}} + x^{p^{r}}a_0^{p^{r}}.
  \] 
  Entonces
  \begin{equation}
    \sum_{c \in \F_d}^{} \chi_b\left( f(c) \right) 
    = 
    \begin{cases}
      \chi_b(a) d & \text{si } g(b) = 0, \\
      0 & \text{en otro caso}.
    \end{cases}
  \end{equation}
\end{theorem}

%\subsection{Geometría finita y planos afínes}
%
%Como hemos visto, la estructura algebráica del campo
%finito es necesaria para poder construir una función de
%Wigner discreta con todas las propiedades deseables. La
%artimética $d$-modular no es suficiente ya que no se
%preservan ciertas propiedades importantes de una geometría
%finita. Los campos $\Z_p$ donde $p$ es un número primo son
%los más sencillos de manejar porque las operaciones se
%hacen módulo $p$, pero ésto se vuelve inpráctico porque no
%todo sistema cuántico puede ser modelado por un espacio de
%Hilbert de dimensión prima. El concepto de una
%\textit{extensión de campos} nos permite augmentar el
%campo $\Z_p$ a un campo finito de orden $p^{n}$. A la
%extensión de campo se le dice \textit{extensión de Galois}.

%\subsection{Formas cuadráticas y formas alternantes sobre
%campos finitos}
%
%Sea $\F_d$ una campo finito de $d$ elementos donde $d$ es
%una potencia de un primo. El polinomio de grado dos en $n$
%indeterminadas $x_1,x_2,\ldots,x_n$ sobre $\F_d$ 
%\begin{equation}
%  Q(x_1,x_2,\ldots,x_n)
%  = \sum_{1 \leq i \leq k \leq n}^{} b_{ik} x_i x_k,
%  \quad b_{ik} \in \F_q,
%\end{equation}
%es una \textit{forma cuadrática} en $x_1,x_2,\ldots,x_n$
%sobre $\F_d$. Se dice que es \textit{definida} si
%$Q(a_1,a_2,\ldots,a_n) = 0$ para $a_1,a_2,\ldots,a_n \in
%\F_d$ implica que $a_1 = a_2 = \ldots = a_n = 0$. En caso
%contrario se dice que es \textit{indefinida}. Ahora
%supongamos que $d$ es de característica impar.
%
%Por otro lado, la forma bilineal en $2n$ indeterminadas
%$x_1,x_2,\ldots,x_n$ y $y_1,y_2,\ldots,y_n$ 
%\begin{equation}
%  K(x_1,x_2,\ldots,x_n; y_1,y_2,\ldots,y_n)
%  = \sum_{i,j = 1}^{n} k_{ij} x_i y_j,
%  \quad k_{ij} \in \F_d,
%\end{equation}
%se llama \textit{forma alternante} sobre $\F_d$ si
%$K(x_1,\ldots,x_n; x_1,\ldots,x_n) = 0$ para todo
%$(x_1,\ldots,x_n) \in \F_d^{n}$. 

\subsection{Sistemas compuestos y factorización tensorial}

Sabemos que una extensión de Galois $\GF(p^{n})$ se puede
ver como un espacio vectorial sobre el campo primo $\Z_p$
respecto a la adición del campo. Pero también tiene más
estructra, como la multiplicación y las transformaciones
de Frobenius. Por ésto, Vourdas [CITE] nota que
representar un sistema cuántico con una extensión de
Galois de grado $n$ no es lo mismo que representarlo con
una suma directa del campo primo. 

Sea $\ket{\alpha}$ un base ortonormal de un espacio de
Hilbert $\H$ de dimensión $p^{n}$, indexada por los
elementos de una extensión de Galois. Además sea $\H_p$ un
espacio de dimensión $p$ y $\ket{k}$ un base ortonormal de
$\H_p$ indexado por los elementos del campo primo $\Z_p$.
Utilizando la base natural de la extensión de campo,
podemos expresar a cualquier elemento $\alpha \in
\GF(p^{n})$ como $\alpha = \sum_{i}^{} \alpha_i \omega^{i}
$, ésto nos da un mapeo biyectivo:
\begin{equation}
  \label{eqn:extension_field_map}
  \alpha \mapsto (\alpha_0,\ldots,\alpha_{n-1}),
\end{equation} 
lo cual induce una correspondencia entre los elementos de
la base de $\H$ y $\H_p$:
\begin{equation}
  \ket{\alpha}
  = \ket{\alpha_0 + \alpha_1 \omega + \ldots +
    \alpha_{n-1}\omega^{n-1}}
  \mapsto \ket{\alpha_0} \otimes \ket{\alpha_1} \otimes
  \cdots \otimes \ket{\alpha_{n-1}},
\end{equation}
la cual a su vez nos da una correspondencia entre los
espacios de Hilbert $\H$ y $\H_p \otimes \cdots \otimes
\H_p$. Notemos que la correspondencia depende de la
elección de base del campo, por lo que la construcción de
otros objetos por medio del producto tensorial también
serán afectados. Por ejemplo los operadores de
desplazamiento que actuán sobre el espacio de Hilbert de
dimensión $p^{n}$ pueden ser expresados en términos de
productos tensoriales de operadores de desplazamiento en
los subsistemas. El mapeo(\ref{eqn:extension_field_map})
nos permite hacer ésto como:
\begin{equation}
  D(\alpha,\beta)
  = D(\alpha_0, \beta_0) \otimes \cdots \otimes
  D(\alpha_{n-1}, \beta_{n-1}),
\end{equation}
donde $\alpha_i$ son los componentes de $\alpha$ en la
base elegida, y $\beta_i$ son los componentes respecto a
la base dual, y los operadores $D(\alpha_i,\beta_i)$ son
los operadores de desplazamiento actuando sobre los
subsistemas $\H_p$.

\section{Anillos de Galois}

En la construcción de las MUBs para campos finitos de
característica par fue necesario `salirnos' del campo y
hacer operaciones en el \textit{anillo de Galois}. Aquí
enunciamos varios de las definiciones y resultados que se
utilizan para tal construcción. La teoría de anillos de
Galois fue desarrollada por W. Krull en 1924. 
\begin{definition}
  Un anillo de Galois se define como un anillo finito con
  identidad $1$ tal que el conjunto de sus divisores de 0,
  más el 0, forma un ideal principal $\langle p \cdot 1
  \rangle$ para algún primo $p$.
\end{definition}
\begin{example}
  Consideremos el anillo $\Z_{p^{s}}$ donde $p$ es un número
  primo y $s$ es un entero primo. Identificamos a $n \cdot
  1$ con $n$ para todo $n \in \Z_{p^{n}}$. El conjunto de
  divisores de 0 más el 0 forma el ideal principal $\langle
  p\rangle$. Por lo tanto $\Z_{p^{s}}$ es un anillo de
  Galois de $p^{s}$ elementos. Cuando $s = 1$, $\Z_{p^{s}} =
  \F_p$ es el campo de $p$ elementos y $\langle p \rangle =
  \langle 0 \rangle$.
\end{example}
\begin{example}
  Sea $h(x)$ un polinomio irreducible básico mónico de grado
  $m$ en $\Z_{p^{s}}[x]$. Ahora consideremos el anillo de
  clases residuales
  \[
    \Z_{p^{s}}[x] / \langle h(x) \rangle.
  \] 
  La clase residual
  \[
    a_0 + a_1 x + \cdots + a_{m-1} x ^{m-1} + \langle h(x)
    \rangle,
  \] 
  donde $a_0,a_1,\ldots,a_{m-1} \in \Z_{p^{s}}$ son
  distintos elementos de $\Z_{p^{s}}[x] / \langle h(x)
  \rangle$. El orden de éste anillo es $p^{sm}$. El elemento
  $1 + \langle h(x) \rangle$ es la identidad multiplicativa
  de $\Z_{p^{s}}[x] / \langle h(x) \rangle$ y $\langle h(x)
  \rangle$ es el cero, por lo tanto dicho anillo es un
  anillo conmutativo con uno. Todos los elementos del ideal
  principal $\langle p + \langle h(x) \rangle \rangle$ son
  divisores del cero o son cero. Ésto demuestra que
  $\Z_{p^{s}}[x] / \langle h(x) \rangle$ es un anillo de
  Galois.

  Ahora consideremos $\xi = x + \langle h(x) \rangle$.
  Entonces $h(\xi) = 0$ y
  \[
    a_0+\cdots+a_{m-1}x^{m-1} + \langle h(x) \rangle
    = a_0+a_1\xi+\cdots+a_{m-1}\xi^{m-1}.
  \] 
  Por lo tanto
  \[
    \Z_{p^{s}}[x] / \langle h(x) \rangle
    = \Z_{p^{s}}[\xi],
  \] 
  y todos los elementos del anillo de Galois pueden ser
  representados por las potencias de $\xi$. Ésta
  representación se conoce como la \textit{representación
  aditiva} de los elementos del anillo de Galois
  $\Z_{p^{s}}[\xi]$.
\end{example}
\begin{definition}
  Para un anillo conmutativo con identidad $1$, el orden de
  $1$ en el grupo aditivo del anillo se llama la
  característica del anillo.
\end{definition}
Conmunmente denotamos al anillo de Galois $\Z_{p^{s}}[x] /
\langle h(x) \rangle$ donde $h(x)$ es un polinomio básico
irreducible sobre $\Z_{p^{s}}$ de grado $m$, por
\[
  \GR(p^{s}, p^{sm}),
\] 
donde $p^{s}$ es la característica y $p^{sm}$ su
cardinalidad. Los elementos de un anillo de Galois pueden
ser representados en términos de combinaciones polinomiales
de un subgrupo particular de la extensión de anillos.
\begin{theorem}
  En el anillo de Galois $\GR(p^{s}, p^{sm})$ existe un
  elemento no cero $\xi$ de orden $p^{m}-1$, el cual es raíz
  de un polinomio básico primitivo $h(x)$ de grado $m$ sobre
  $\Z_{p^{s}}$ que divide a $x^{p^{m}-1} - 1$ en
  $Z_{p^{s}}[x]$ tal que
  \begin{equation}
    \GR(p^{s},p^{sm})
    = \Z_{p^{s}}[\xi]
    = \{a_0+a_1\xi+\ldots+a_{m-1}\xi^{m-1} : a_i \in
    \Z_{p^{s}}\}.
  \end{equation}
  Sea
  \begin{equation}
    \mathcal T
    = \{0,1,\xi,\xi^2,\ldots,\xi^{p^{m}-2}\},
  \end{equation}
  entonces cualquier elemento $c \in \GR(p^{s},p^{sm})$ 
  puede ser escrito de manera única como
  \[
    c = a_0 + a_1p + \cdots a_{s-1}p^{s-1},
  \] 
  donde $a_0,a_1,\ldots,a_{s-1} \in \mathcal T$.
\end{theorem}
La representación de un elemento de $\GR(p^{s},p^{sm})$ en
términos de los elementos de $\mathcal T$ se conoce como la
\textit{representación $p$-ádica} del elemento. En
particular utilizamos la representación $2$-ádica para hacer
los cálculos requeridos en la construcción de las MUBs en
característica dos. Los detalles de dicha construcción se
pueden consultar en el trabajo de Calderbank, Cameron,
Kantor y Seidel sobre códigos de Kerdock sobre $\Z_4$
\cite{calderbank1997}.

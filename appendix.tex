\chapter{Apéndices}

  \section{Espacio de Schwartz y su dual}

  \begin{definition}
    Sea $\Sz(\R^{n})$ el espacio vectorial complejo
    de las funciones de prueba de Schwartz. La función
    $\psi$ es un elemento de $\Sz(\R^{n})$, si es
    lisa y si las funciones $x^{\beta} \partial_x^{\alpha}
    \psi$ son acotadas para todo los índices $\alpha =
    (\alpha_1, \ldots, \alpha_n)$ y $\beta = (\beta_1,
    \ldots, \beta_n)$ en $\N^{n}$.
  \end{definition}

  \begin{definition}
    El espacio dual de $\Sz(\R^{n})$ es el espacio de
    las distribuciones (ó funciones generalizadas)
    templadas, y lo denotaremos como $\Sz'(\R^{n})$.
  \end{definition}

  \begin{definition}
    El espacio de Hilbert de las (clases de) funciones
    complejas cuadraticamente-integrables sobre $\R^{n}$ se
    denota como $L^2(\R^{n})$, y está equipado con el
    producto interno
    \[
      \langle f, g \rangle
      = \int_{\R^{n}} \overline{f}(x) g(x) \, dx.
    \] 
  \end{definition}
 
  Los elementos del espacio de Schwartz sobre $\R^{n}$
  tienen derivadas parciales que decrecen rápidamente. El
  dual del espacio de Schwartz se conoce como las
  \textit{distribuciones templadas} sobre $\R^{n}$. Existe
  una inclusión natural de las funciones cuadraticamente
  integrables en el espacio de distribuciones. Además, el
  espacio de Schwartz es un subespacio denso de
  $L^2(\R^{n})$, podemos expresar éstas relaciones con un
  poco de abuso de notación,
  \[
    \Sz(\R^{n})
    \subset L^2(\R^{n})
    \subset \Sz'(\R^{n}).
  \]

  Sea $\mathcal D(\R^{n})$ el espacio de las funciones de
  prueba.

  \begin{definition}
    Supongamos que $\{\phi_n\}_{n \in \N}$ es una sucesión
    convergente en $\mathcal D(\R^{n})$ con límite $\phi \in
    \mathcal D(\R^{n})$. Si $T : \mathcal D(\R^{n}) \to \C$
    es un funcional lineal continuo, es decir
    \[
      \forall \psi, \phi \in \mathcal D(\R^{n}) : T(\phi +
      \psi) = T(\phi) + T(\psi),
    \] 
    \[
      \forall \phi \in \mathcal D(\R^{n}) : \forall \alpha
      \in \C : T(\alpha \phi) = \alpha T(\phi),
    \] 
    y
    \[
      \phi_n \to \phi \implies T(\phi_n) \to T(\phi).
    \] 
    Entonces $T$ es una distribución.
  \end{definition}

  \begin{definition}
    Definimos a la distribución de Dirac $\delta_a$ como la
    distribución que satisface
    \[
      \forall \phi \in \mathcal D(\R^{n}) : \delta_a(\phi) =
      \phi(a).
    \] 
  \end{definition}

  $C^{\infty}_0(\R^{n})$ es el conjunto de todas las función
  sobre $\R^{n}$ lisas y con soporte compacto. Los espacios
  $C^{\infty}_0(\R^{n})$ y $\Sz(\R^{n})$ son densos en
  $L^{r}(\R^{n})$, $1 \leq r < \infty$.

  Iniciamos definiendo
  la transformación de Wigner sobre el espacio de las
  funciones de Schwartz $\Sz(\R^{n})$, para ésto primero
  recordamos la transformación de Fourier.

  \begin{definition}
    La transformada de Fourier se denotará por $\F$,
    y usaremos una versión que incluye una constante de
    normalización dada por la constante de Planck,
    \[
      \F\psi(x)
      = \frac{1}{(2\pi\hbar)^{n / 2}} \int_{\R^{n}}
      e^{-\frac{i}{\hbar} x \cdot y} \psi(y) \, dy,
    \] 
    para $\psi \in \Sz(\R^{n})$.
  \end{definition}
  La transformada de Fourier es un automorfismo del espacio
  de Schwartz, y se puede extender a un automorfismo
  unitario sobre $L^2(\R^{n})$, cuya inversa está dada por
  \[
    \F^{-1}\psi(x)
    = \frac{1}{(2\pi\hbar)^{n / 2}} \int_{\R^{n}}
    e^{\frac{i}{\hbar} x \cdot y} \psi(y) \, dy.
  \] 
  Además se puede extender por medio de la dualidad a un
  automorfismo de $\Sz'(\R^{n})$.

  \begin{definition}
    La transformada de Fourier de una función en
    $L^{1}(\R^{n})$ es la función $\hat{f}$ sobre $\R^{n}$ 
    definida como
    \[
      \F[f](\xi) = (2\pi)^{-n / 2} \int_{\R^{n}} e^{-i x
      \cdot \xi} f(x) \, dx, 
      \quad \xi \in \R^{n}.
    \] 
  \end{definition}

  \begin{theorem}
    La transformada de Fourier es biyectiva en
    $\Sz(\R^{n})$.
  \end{theorem}

  \begin{theorem}[Plancherel]
    La biyección $\F : \Sz(\R^{n}) \to \Sz(\R^{n})$ 
    se puede extender únicamente a un operador unitario
    sobre $L^2(\R^{n})$.
  \end{theorem}

  \begin{proposition}
    El espacio $\Sz(\R^{n})$ es denso en $\Sz'(\R^{n})$.
  \end{proposition}

  \section{Wong}

  Sea $f$ una función medible sobre $\R^{n}$. Definimos la
  función $\rho(\xi,\eta)f$ sobre $\R^{n}$ como
  \begin{equation}
    (\rho(\xi,\eta)f)(x)
    = e^{i\xi \cdot x + \frac{1}{2} i \xi \cdot \eta}
    f(x+\eta),
    \quad x \in \R^{n}.
  \end{equation}
  El operador $\rho(\xi,\eta) : L^2(\R^{n}) \to L^2(\R^{n})$
  es unitario para todo $\xi, \eta \in \R^{n}$. Si nos
  enfoncamos en el subespacio de $L^2(\R^{n})$ de las
  funciones de Schwartz, entonces podemos definir la
  siguiente función sobre el espacio de fase $\R^{2n}$. Sean
  $f,g \in \Sz(\R^{n})$, entonces
  \begin{equation}
    V(f,g)(\xi,\eta)
    = (2\pi)^{-n / 2} \langle \rho(\xi,\eta)f, g \rangle.
  \end{equation}
  A ésta función Wong la llama la transformación
  Fourier-Wigner de $f$ y de $g$. De manera explícita
  tenemos 
  \begin{equation}
    V(f,g)(\xi,\eta)
    = (2\pi)^{-n / 2} \int_{\R^{n}} e^{i \xi \cdot y}
    f\left( f + \frac{\eta}{2} \right) \overline{g\left( y -
    \frac{\eta}{2}\right) } \, dy.
  \end{equation}

  El operador $V : \Sz(\R^{n}) \times \Sz(\R^{n}) \to
  \Sz(\R^{2n})$ es un mapeo lineal.

  Wong menciona que la transformada de Wigner $W(f)$ de una
  función en $L^2(\R^{n})$, introducida por Wigner en 1932,
  es una herramienta para el estudio de la distribución de
  probabilidad conjunta no existente de la posición y
  momentum de un estado $f$.

  \begin{theorem}
    Sean $f$ y $g$ elementos de $\Sz(\R^{n})$. Entonces
    \begin{equation}
      \F[V(f,g)](x,\xi)
      = (2\pi)^{-n / 2} \int_{\R^{n}} e^{-i \xi \cdot p
      } f\left( x + \frac{p}{2} \right)
      \overline{g\left( x - \frac{p}{2} \right)} \, dp.
    \end{equation}
  \end{theorem}

  Con ésto Wong define a la transformación de Wigner de dos
  funciones $f$ y $g$ en $\Sz(\R^{n})$ como
  \begin{equation}
    W(f,g)(x,\xi)
    = (2\pi)^{-n / 2} \int_{\R^{n}} e^{-i\xi \cdot p}
    f\left( x + \frac{p}{2} \right) \overline{g\left( x -
    \frac{p}{2} \right) } \, dp.
  \end{equation}

  La transformada de Wigner $W : \Sz(\R^{n}) \times
  \Sz(\R^{n}) \to \Sz(\R^{2n})$ puede ser extendida a un
  operador sesquilineal
  \[
    W : L^2(\R^{n}) \times L^2(\R^{n}) \to L^2(\R^{2n}),
  \] 
  tal que
  \[
    \|W(f,g)\|_{L^2(\R^{2n})}
    = \|f\|_{L^2(\R^{n})} \|g\|_{L^2(\R^{n})}.
  \] 

  De acuerdo a Wong, podemos obtener un operador lineal
  acotado $Q : L^2(\R^{2n}) \to B(L^2(\R^{n}))$ a partir de
  cualquier función $\sigma \in L^2(\R^{2n})$.

  En la mecánica cúantica, los observables deben ser
  representados por operadores auto-adjuntos, la
  transformación de Weyl nos permite hacer ésto.

  La correspondencia resulta ser entre funciones en el
  espacio de fase y los operadores Hilbert-Schmidt.

  \begin{definition}
    Sea $h \in L^2(\R^{2n})$. Definimos el operador $S_h :
    L^2(\R^{n}) \to L^2(\R^{n})$ como
    \begin{equation}
      (S_hf)(x)
      = \int_{\R^{n}} h(x,y) f(y) \, dy,
    \end{equation}
    para todo $f \in L^2(\R^{n})$.
  \end{definition}
  Al operador $S_h$ se le conoce como el operador
  Hilbert-Schmidt correspondiente al núcleo $h$.

  Para probar que el conjunto de transformaciones de Weyl
  con símbolos en $L^2(\R^{2n})$ es igual al conjunto de
  operadores Hilbert-Schmidt sobre $L^2(\R^{n})$, Wong
  define y utiliza el producto tensorial de funciones de
  $L^2(\R^{n})$.

  \begin{theorem}
    Sea $\sigma \in L^2(\R^{2n})$. Entonces $W_\sigma :
    L^2(\R^{n}) \to L^2(\R^{n})$ es un operador
    Hilbert-Schmidt con núcleo $(2\pi)^{-n / 2}K \sigma$
    donde $K : L^2(\R^{2n}) \to L^2(\R^{2n})$ se define como
    \[
      (Kf)(x,y)
      = (T^{-1}\F_2 f)(y,x).
    \] 
  \end{theorem}

  Entonces ahora sabemos que si $\sigma \in L^2(\R^{2n})$,
  entonces $W_\sigma$ es un operador Hilbert-Schmidt con
  núcleo $(2\pi)^{-n / 2} K \sigma$. El converso es cierto
  también, si $A$ es un operador Hilbert-Schmidt arbitrario,
  entonces tiene un representación integral $A = S_h$, y
  $\sigma = (2\pi)^{-n / 2} K^{-1}h$.

  Se puede definir la transformación de Weyl para símbolos
  en $\Sz'(\R^{2n})$. Resulta que para $2 < r < \infty$,
  existe una función $\sigma \in L^{r}(\R^{2n})$ tal que la
  transformación de Weyl $W_\sigma$ no es un operador lineal
  acotado sobre $L^2(\R^{n})$.

  \section{Resumen de Folland}

  Para Folland, la transformación de Wigner de dos funciones
  $f$ y $g$ es la transformación de Fourier de la
  transformación de Fourier-Wigner:
  \[
    W(f,g)(\xi, \eta)
    = \int_{\R^{2n}} e^{-2\pi i (\xi q + \eta p} V(f,g)(q,x)
    \, dp \, dq.
  \] 
  Donde
  \[
    V(f,g)(q,p)
    = \int e^{2\pi i q y} f(y + \frac{1}{2}p)\overline{g(y -
    \frac{1}{2}p)} \, dy.
  \] 
  Así que
  \[
    W(f,g)(\xi, \eta) = \int e^{-2\pi i \eta p }
    f(x+\frac{1}{2}p)\overline{g(y-\frac{1}{2}p)} \, dp.
  \] 

  Folland muestra que $W$ mapea $\Sz(\R^{n}) \times
  \Sz(\R^{n})$ a $\Sz(\R^{2n})$ y se puede extender al caso
  de las distribuciones templadas.

\documentclass[a4paper]{article}

\usepackage[margin=1.2in]{geometry}
\usepackage[utf8]{inputenc}
\usepackage[T1]{fontenc}
\usepackage{textcomp}
\usepackage{amsmath, amssymb}
\usepackage{amsthm}
\usepackage{braket}

\DeclareMathOperator{\R}{\mathbb{R}}
\DeclareMathOperator{\C}{\mathbb{C}}
\DeclareMathOperator{\N}{\mathbb{N}}
\DeclareMathOperator{\Z}{\mathbb{Z}}
\DeclareMathOperator{\Tr}{Tr}

\newtheorem{definition}{Definition}
\newtheorem{theorem}{Theorem}
\newtheorem{proposition}{Proposition}
\newtheorem{lemma}{Lemma}
\newtheorem{corollary}{Corrollary}
\newtheorem{example}{Example}

\title{Some notes}
\author{Ernesto Camacho Ramírez}
\begin{document}
  \maketitle
  
  \section{Investigations on discrete Wigner functions of
  multi-qubit systems}

  \subsection{K. Srinivasan}

  Uses Gibbons and Wootters method for constructing discrete
  Wigner functions. This means they assign mutually unbiased
  bases to straight line striations of the phase space. They
  acknowledge that by doing this, there is no unique Wigner
  function associated to any density matrix. 

  A Hilbert space of dimension $N$ is associated with a
  discrete phase space of $N \times N$ cardinality. The
  horizontal and vertical axes of the discrete phase space
  are asociated with two non-commuting observables. The
  points are ordered pairs of elements of a Galois field of
  size $N$. Given the equation for a line $aq + bp = c$, if
  we fix the values of $a$ and $b$ and vary $c$ over the
  field, then a set of $N$ parallel lines called a striation
  is generated. Each striation contains exactly one ray, and
  for fixed values of $x$ and $y$, varying  $s$ from the
  Galois field, then the set of $N$ points $(sa,sb)$ gives a
  ray.

  For prime power dimensions there are $N+1$ striations in
  the discrete phase space and $N+1$ MUBs. In order to give
  lines and striations a physical meaning, Wootters
  associates the lines of the phase space with pure states:
  if $\lambda_j^i$ is the $i$-th line of the $j$-th
  striation, then $Q(\lambda_j^i) =
  \ket{\lambda_j^i}\bra{\lambda_j^i}$ is the rank one
  projector associated with the line.

  This guy mentions that, just like the continuous case, one
  can define a set of $N^2$ translations $\mathcal
  T_{(q,p)}$ in discrete phase space, which shifts each
  point in a line by the amount $(q,p)$. For every
  striation, there exists $N-1$ translation operators which
  leave the lines in the striation invariant and those
  translation operators are formed by the points of the ray
  of the corresponding striation. So in order to achieve
  translational covariance, Wootters idea is to associate to
  each phase space translation a corresponding translation
  (displacement) operator, such that $T_\alpha T_\beta =
  \eta T_{\alpha+\beta}$ and such that we can decompose the
  translation operator as a tensor product of
  \textit{basic} translation operators that act upon the
  individual subystems of the composite Hilbert space:
  \[
    T_{(q,p)}
    = T_{(q_1,p_1)} \otimes T_{(q_2,p_2)} \otimes \cdots
    \otimes T_{(q_n,p_n)}.
  \] 

  For a prime dimension system, the generalized Pauli
  matrices $X$ and $Z$ can be associated to the unit
  horizontal and vertical translations $\mathcal T_{(1,0)}$ 
  and $\mathcal T_{(0,1)}$. Such operators obey the
  commutation relation $ZX = \eta XZ$. And so the unitary
  operators are then defined up to a phase factor as
  \[
    T_{(q,p)}
    = X^{q_1} Z^{p_1} \otimes \cdots \otimes X^{q_n}
    Z^{p_n}.
  \] 

  The condition of translational covariance is ensured by
  satisfying the equation
  \[
    Q(\mathcal T_\alpha \lambda)
    = T_\alpha Q(\lambda) T_\alpha^{*}.
  \] 
  Which in turn requires the translation operators
  corresponding to a striation (that is those which leave
  the parallel lines of the striation invariant), commute.

  Then in order to construct the Wigner function, we start
  with its fundamental property, that is the sum (integral)
  along a particular line $\lambda$ is equal to the
  expectation value of the rank one projector $Q(\lambda)$.
  The sum of the Wigner function $W$ along this line is
  equal to the probability $p(\lambda) = \Tr(Q(\lambda)
  \rho) = \sum_{\alpha \in \lambda}^{} W_\alpha$, so we can
  reconstruct the Wigner function by the $N+1$ projective
  measurements. Note that for any given point $\alpha$ in
  the discrete phase space, there are $N+1$ lines which pass
  through it. Then, the sum of Wigner elements $S(\alpha)$ 
  along the $N+1$ lines wich contains the point is given by
  \[
    S(\alpha)
    = \sum_{\lambda \ni \alpha}^{} \sum_{\beta \in
    \lambda}^{} W_\beta
    = NW_\alpha + \sum_{\gamma}^{} W_\gamma.
  \] 
  Then
  \[
    W_\alpha = \frac{1}{N} \left( \sum_{\lambda \ni
    \alpha}^{} \sum_{\beta \in \lambda}^{} W_\beta -
  \sum_{\gamma}^{} W_\gamma \right), 
  \] 
  where $\sum_{\beta \in \lambda}^{} W_\beta$ is the
  probability to measure the systems along the projector
  $Q(\lambda)$. And so
  \[
    W_\alpha
    = \frac{1}{N} \Tr(A_\alpha \rho),
  \] 
  where the $A_\alpha$ are the so called phase-point
  operators, given by
  \[
    A_\alpha = \sum_{\lambda \ni \alpha}^{} Q(\lambda) - I.
  \] 

  The author then goes about providing the phase-point
  operator properties along with Wigner function properties
  (note the Phd thesis does not even prove them). Then he
  talks about a labeling scheme for lines and striations in
  the particular case of multiqubit Wigner functions.
  Identifying the origin with the element $(0,0)$, there
  exists exactly one line in each striation passing through
  the origin called a ray. The rays are formed by the points
  $(sx,sy)$ for fixed $(x,y)$. In this case the rays of the
  horizontal and vertical striations are given by the points
  $(1,0)$ and $(0,1)$. The other rays are generated by
  $(1,\omega), (1,\omega^2), \ldots,(1,\omega^{N-2})$. And
  so he associates a \textit{striation number} $j$ with a
  fixed point and the ray $\lambda_j^0$. Given the ray of
  the vertical striation $\lambda_1^0$, the other lines of
  the striation are obtained using the set of $N-1$ 
  horizontal translation operators, $\mathcal T_{(s,0)}$,
  such that $\lambda_1^i = \mathcal T_{(i,0)}\lambda_1^0$.
  The author \textit{chooses} the translations $\mathcal
  T_{(0,s)}$ for obtaining the other lines by acting on the
  ray of the rest of the striations.

  The author continues by given a concrete example: a single
  qubit system. As we know, the horizontal and the vertical
  axes are associated with the Pauli matrices $\sigma_x$ and
  $\sigma_z$, which are non-commuting. There are $6$ lines
  in the phase space, which can be grouped into 3 sets of
  parallel lines. The corresponding translation operators
  are associated to the points $(0,s)$, $(s,0)$ and $(s,s)$.
  The translation operators are traceless and mutually
  orthogonal, and their eigenvectors forms the MUBs. The
  asociation of the rank one operators and the lines is not
  unique, in fact there are $2^{2+1} = 8$ possible ways to
  do the assignment. After a choice of quantum net, we can
  build the phase-point operators and then calculate the
  Wigner function.

  For the case of the two qubit system, for each striation,
  there exists a set of $N-1 = 3$ translations which leave
  the lines of the striation invariant. The translations
  associated to the five striations are $\mathcal T_{(0,s)},
  \mathcal T_{(s,0)}, \mathcal T_{(s,s)}, \mathcal
  T_{(s,s\omega}$ and $\mathcal T_{(s,s\omega^2)}$. There
  are 15 unitary translation operators along with the
  identity. The oeprators can be equally partitioned into
  $5$ sets and each set contains $3$ operators. The common
  eigenstates of the $3$ traceless, mutually orthogonal,
  commuting operators are associated with the lines of the
  striation. In this case there are $4^{4+1} = 1024$ quantum
  nets.

  After this, the author talks about quantum nets and
  equivalence classes as studied by Gibbons and Wootters.
  Then he starts talking about the advantages and
  limitations of the Gibbons construction. The main
  disadvantage is that the Wigner function as constructed by
  Wootters is not unique, as there are $N^{N+1}$ versions of
  it. As such, the non-classicality of the reconstructed
  state is not obvious. Interestingly he shows the results
  of an experiment of the tomographic reconstruction of the
  experimental data.

  \section{Geometrical Representations of Finite Dimensional
  Quantum Systems}

  \subsection{Daniël Marjenburgh}

  The authors plan to introduce a discrete Wigner function
  is the same as before: to preserve as many of the
  continuous version's properties as possible. He starts
  from first principles and the phase point operators, using
  the traditional definition of the Wigner function as the
  trace of product of the density matrix and the displaced
  parity operators. The properties that are imposed are that
  of translation covariance and the tomographic property.

  The author decides to use the $N \times N$ grid of points
  as the natural phase space, noting that there is no
  redundency in the case of quantum tomagraphy (unlike the
  DWF that use a $2N \times 2N$ phase space).

  This guy gives a very intuitive explaination for the
  natural analog of the continuous, the conjugate bases and
  the property of mutual unbiasedness, along with the
  Fourier transform.

  Interesting conclusions on the even and odd cases, noting
  that for $N$ odd, there is a unique expression for the
  phase point operators such the WF satisfies the
  tomographic property and translation covariance.

  \section{The interplay between quantum contextuality and
    Wigner negativity}
  
  \subsection{Pierre-Emmanuel Emeriau}

  The author's thesis focuses on two nonclassical
  behaviours: quantum contextuality and Wigner negativity.
  The is a notion superseding nolocality that can be
  exhibited by quantum systems. In discrete settings,
  contextuality has been shown to be necessary and
  sufficient for advantages in some cases. This seems to be
  a continous Wigner function focused thesis. 

  \section{Hudson's theorem for finite-dimensional quantum
  systems}

  \subsection{D. Gross}

  Gross shows that for a system of odd dimension, only the
  stabilizer states posses a non-negative Wigner function.
  The give an axiomatic characterization which completely
  fixes the definition of the Wigner function.

  Gross notes that partly triggered by the advent of quantum
  information theory, considerable work has been undertaken
  to explore Wigner functions for finite-dimensional quantum
  systems. Finally someone that agrees with me, Gross notes
  that there seem to be to clear distinct approaches that
  can be identified in the leterature. The first tries to
  cast the definition of the Wigner function into a form
  that can be interpreted for both continuous variable and
  discrete systems, these are the works of Vourdas (2004),
  Paz (2002), Klimov and Gross. The second approach,
  introduced by Gibbons, Hoffman and Wootters, is based on
  the properties of the continuous Wigner function.

  Gross actually proposes a definition for odd dimensions
  that highly resembles the continuous version and
  supposedly is most sensible judged in both approaches. He
  defines it as:
  \[
    W_\psi(p,q)
    = d^{-1} \sum_{\xi \in \Z_d}^{} e^{-\frac{2\pi}{d} i \xi p}
    \bar{\psi}(q-2^{-1}\xi) \psi(q+2^{-1}\xi).
  \] 
  Such definition is actually the disrete symplectic Fourier
  transform of the discrete characteristic function (like
  Vourdas' approach) and he shows that it is the unique
  choice that mimics certain desirable properties.

  He proceeds by talking about stabilizer states, which are
  joint eigenvectors of certain sets of elements of the
  qubit Pauli group. There is of course, stabilizer states
  for higher-dimensional quantum systems. Although they
  display complex features such as multi-particle
  entanglement, such states allow for an efficient classical
  description. The implications of the Gottesman-Knill
  theorem are quite important. He then proposes the
  following theorem:

  \begin{theorem}
    Let $d$ be odd and $\psi \in L^2(\Z_d^n)$ be a state
    vector. The Wigner function of $\psi$ is non-negative if
    and only if $\psi$ is a stabilizer state. Furthermore,
    since $\psi(q) \neq 0$ for all $q$, a vector $\psi$ is a
    stabilizer state if and only if it is of the form
    \[
      \psi(q) \propto e^{\frac{2\pi}{d}i (q\theta q + x q)},
    \] 
    where $q,x \in \Z_d^{n}$ and $\theta$ is a symmetric
    matrix with entries in $\Z_d$.
  \end{theorem}

  Gross' theorem shows that if the \textit{right}
  definitions are employed, the continuous and discrete case
  behave very similarly. Also one can further the idea that
  stabilizer states are like Gaussian states.

  In order to vinculate Galvao's research on the
  non-negativity of the DWF, Gross comments on Wootters
  approach. The axioms such definition should meet are:
  \begin{itemize}
    \item $W$ is a linear mapping sending operators to
      functions on a $d \times d$ lattice, called the phase
      space.
    \item The Wigner function is covariant under the action
      of the Weyl operators.
    \item There exists a function $Q(\lambda)$ that assigns
      a pure quantum state to every line $\lambda$ in phase
      space. If $\psi$ is a state vector, then the sum of
      its Wigner function along $\lambda$ must be equal to
      the overlap $|\braket{Q(\lambda)|\psi}|^2$.
  \end{itemize}

  Gross notes that for a $d$-dimensional Hilbert space,
  there exist $d^{d+1}$ distinct generalized Wigner
  functions, and that such approach is only described for $d
  = p^{n}$ where $p$ is a prime number, beacuase the of the
  necessary finite geometry given by finite fields.

  Next up is the notion of the stabilizer states. Consider a
  composite system, build of $n$ $d$-level particles.
  Considering the system as a single $d^{n}$-dimensional
  space or as a composite system gives distints notions of
  stabilizer states. Galvao's work concerns quantum states
  in prime power dimensions that are non-negative with
  respect to all possible definitions of generalized Wigner
  functions. Such states are shown to be mixtures of
  single-particle stabilizer states. If the Wigner function
  of a quantm computer is positive at all times, then it
  operates only with stabilizer states. For the case of
  non-qubit pure states, Gross' theorem implies Galvao's
  results, and favorably, there is only one definition to
  study. Although Gross' theorem does not address qubits or
  mixed states.

  The next section gives a review of the Weyl
  representation. Start with an odd $d$-dimensional quantum
  system. Choose the computational basis labeled by the
  elements of the ring $\Z_d$. The main objects in the phase
  space formulation are the Weyl operators (generalized
  Pauli operators). Define (the character) $\chi(q) =
  e^{\frac{2\pi}{d}iq}$ and define the shift and boost
  operatros:
  \[
    \hat x(q) \ket x = \ket{x+q}
    \quad
    \hat z(p) \ket x = \chi(px) \ket x.
  \] 
  The Weyl operators are then defined as (just like Vourdas)
  \[
    w(p,q) = \chi(-2^{-1}pq) \hat z(p) \hat x(q),
  \] 
  for $p,q,t \in Q = \Z_d$. Gross notes the choice of phase
  factors ensures that the symplectic inner product appears
  in the composition law, thus making the connection between
  the Weyl operators and symplectic geometry manifest. Other
  definitions (like no phase factor) carry the same
  dependence in a less obvious manner. The set of Weyl
  operators is closed under multiplication, up to phase
  factors. Writing  $w(v) = w(v_p,v_q)$ for elements $v =
  (v_p,v_q) \in \Z_d^2$, then $V := \Z_d \times \Z_d$ with
  the symplectic inner product is called the phase space.

  Such construction can be generalized to multi particle
  systems. Consider $\Z_d^{n}$ where the product is given by
  the inner product. The Weyl operators are now given by the
  tensor products
  \[
    w(p,q)
    = w(p_1,\ldots,p_n,q_1,\ldots,q_n)
    = w(p_1,q_1) \otimes \cdots \otimes w(p_n,q_n).
  \] 

  Finally he gives some remarks that were left unmentioned.
  A state vector $\ket \psi$ can be identified with a
  complex function on $\Z_d$ using the inner product
  $\psi(q) = \braket{q|\psi}$.

  Next Gross introduces the Clifford group, as the set of
  unitary operators such that
  \[
    Uw(v)U^{*} = c(v)w(S(v))
  \] 
  for some maps $c: V \to \C$ and $S : V \to V$. For prime
  values of $d$, $\Z_d$ has a finite field structure, and
  $\Z_d^{n}$ is a finite vector space. He goes into some
  detail about the Clifford group and affine
  transformations, the metaplectic group.

  Next he sets up the Fourier transforms on the space
  $\Z_d^n$ for a complex function $f: Q \to \C$. He defines
  it as
  \[
    \mathcal Ff(p)
    = (d^{n})^{-1 / 2} \sum_{q \in \Z_d^{n}}^{}
    \hat\chi(pq)f(q).
  \] 
  Then he defines a more robust version using the characters
  $\zeta$.

  Finally he proceeds to define and give properties of the
  Wigner function. The Weyl operators $\{w(p,q)\}$ form an
  orthonormal basis in the space of operators on $\mathcal
  H$ with respect to the trace scalar product. Just like in
  the continuous case, the \textit{characteristic function}
  of the operator $\rho$ is given by its expansion
  coefficients with respect to the Weyl basis:
  \[
    \Xi_\rho(\xi,x)
    = d^{-n} \Tr(w(\xi,x)^{*}\rho).
  \] 
  And now he completes the analogy with the continuous case,
  as he has defined the symplectic Fourier transform and the
  characteristic function. And so for $d$ odd, $Q =
  \Z_d^{n}$, let $V, \mathcal H$ be as usual and let $\rho$ 
  be a quantum state. Then the Wigner function $W_\rho$ is
  the symplectic Fourier transformation of the
  characteristic function $\Xi_\rho$. An explicit
  calculation gives the usual definition:
  \[
    W_\rho(a)
    = d^{-n} \Tr(A(a)\rho),
  \] 
  where the $A(a)$ supposedly coincides with Wootters phase
  space point operators. He then states the usual properties
  of the Wigner function: the phase point operators have
  unit trace and form an orthonormal basis in the space of
  Hermitian operators on $\mathcal H$. And so the Wigner
  function of a Hermitian operator is real and we have the
  overlap and normalization properties. Constructed in this
  way, the Wigner function for a pure state is given by the
  equation at the beginning, which looks like Wigner
  original. The marginal properties hold as well with
  respect to the conjugate variables. He also notes that the
  phase point operators factor as a tensor product
  \[
    A(p_1,\ldots,p_n,q_1,\ldots,q_n)
    = \bigotimes_i^n A^{(i)}(p_i,q_i)
  \] 
  and hence so does the Wigner function. The origin point
  operator is the parity operator and he even gives the
  Moyal product.

  Important for his theorem is the proof the Wigner function
  defined in his manner is Clifford covariant. Also, Gross
  mentions that his version coincides with Vourdas and some
  other ones, plus Klimovs version for a specific example.
  Also, his construction satisfies the axioms put forth by
  Wootters and Gibbons, noting that Wootters mentions that
  some of the generalized Wigner functions are special in
  some sense of symmetry. In his language, the symmetry is
  showed by the Clifford covariance. Importantly he notes
  that if we require a definition of a Wigner function to be
  Clifford covariant, then this condition practically picks
  out his construction, giving a sense a uniqueness.

  Gross goes back to work with the stabilizer states, first
  noting that using the composition law of the Heisenberg
  group, two Weyl operators $w(v_1),w(v_2)$ commute if and
  only iff $[v_1,v_2] = 0$. And then considering the image
  of an entire subspace $M$ under the Weyl representation
  $w$. The set
  \[
    w(M) = \{w(m) | m \in M\}
  \] 
  consists of mutually commuting operators if and only if
  the symplectic form vanishes on $M$:
  \[
    [m_1,m_2] = 0,
    \quad m_i \in M.
  \] 
  Such spaces are called \textit{isotropic}. If $M$ is
  isotropic, then the operators $w(M)$ can be simultaneously
  diagonalized. If $M$ is maximally isotropic subspace of
  $V$ and $v \in V$, then up to a global phase, there is a
  unique state vector $\ket{M,v}$ that fulfills the
  eigenvalue equations
  \[
    \chi([v,m])w(m)\ket{M,v} = \ket{M,v}
  \] 
  for all $m \in M$. Such states are called stabilizer
  states.

  I think the important thing to note here is that for every
  odd prime dimension, the Wootters discrete Wigner function
  is uniquely determined by Clifford covariance. And that no
  Clifford covariant operator basis or discrete Wigner
  function exists in any even prime power dimension.

  \section{Gross' thesis}

  Gross' approach is the same as Vourdas. First he discusses
  the notions of the Heisenberg group, the characteristic
  function and Wigner function. Then he contributes to an
  open problem in the theory of stabilizer codes. His way of
  doing it is to center around the connection to the
  continuous way.

  He starts by defining characters. If $G$ is a finite
  abelian group, a character $\chi$ of $G$ is a homomorphism
  from $G$ into the circle group $S^{1}$, that is the set of
  complex numbers of modulus one. The set of all characters
  becomes a group of its own, called $G$ 's dual group $\hat
  G$. Finite abelian groups are isomorphic to their
  respective dual groups but there is no canonic way of
  identifying  $G$ with $\hat G$.

  As a fundamental example he defines the characters on
  finite fields. For prime dimension we have
  \[
    a \mapsto \chi_a(\cdot) := \omega^{a}.
  \] 
  For a finite extension of a prime field $\mathbb F_p$ we
  have
  \[
    b \mapsto \chi_b(\cdot) := \chi_{\mathbb F_p}(\Tr(b
    \cdot)).
  \] 

  Next he introduces the notion of a finite symplectic
  geometry, which reduces to the theory of finite vector
  spaces with a symplectic form. A finite vector space $V$ 
  is called symplectic if it posseses a non-degenerate
  bilinear form $[\cdot,\cdot]$ which is anti-symmetric
  \[
    [v,w] = -[w,v].
  \] 
  Should $V$ be defined over a field of characteristic two,
  then we additionally demand the form fulfills
  \[
    [v,v] = 0.
  \] 
  Given a subspace $M$ of $V$, the symplectic complement
  $M^{\perp}$ is the set of all $v \in V$ such that $[v,m] =
  0$ for all $m \in M$. $M$ is \textit{isotropic} if the
  form vanishes on $M$. Of note is that fact that if $M$ is
  isotropic then $M \subset M^{\perp}$. The maximum
  dimension of an isotropic space is $\dim V / 2$. A space
  that reaches this limit is maximal isotropic. Gross then
  subsequently refers to symplectic vector spaces as phase
  spaces.

  Next he defines the Fourier transform and the symplectic
  Fourier transform for abelian groups. In particular, he
  writes the Fourier transform of an $f \in L^{1}(\mathbb
  F)$ as a function on $\mathbb F$ itself by choosing a
  faithful character $\chi$ of $\mathbb F$:
  \[
    \hat f(a)
    = \hat f(\chi_a)
    = \frac{1}{\sqrt{d}} \sum_{b \in \mathbb F}^{}
    \braket{\chi|ab}^{*}f(b).
  \] 
  By choosing a character $\chi$ of $\mathbb F$, then for a
  symplectic vector space $V$ over the finite field, the
  character induces an isomorphism from $V$ to $\hat V$ by
  \[
    \chi_a(\cdot) = \chi([a,\cdot]).
  \] 
  He then defines the symplectic Fourier transform. 

  In the next section, Gross starts by saying that if a
  quantum system posses a symmetry, then there should exist
  a unitary, irreducible, possibly projective representation
  of the symmetry group. He starts with a field $\mathbb F$ 
  that is not of characteristic two and defines the
  Heisenberg group $H(\mathbb F)$ by its composition law
  using the standard symplectic form on the vector space
  $\mathbb F^2$. The Heisenberg group enters the quantum
  scene through the Weyl representation which maps the group
  to operators on the Hilbert space $\C^{d}$ where $d$ is
  the order of the finite field. The representation is
  constructed by fixing a character $\chi$ of $\mathbb F$ 
  and choosing a basis in $\mathcal H$. Next he defines what
  he calls the shift and clock operators
  \begin{align*}
    x(q) &: \ket{\phi_k} \mapsto \ket{\phi_{k+q}} \\
    z(p) &: \ket{\phi_k} \mapsto \chi(pk) \ket{\phi_k}.
  \end{align*} 
  And so the Weyl representation is given by
  \[
    w(p,q,t) := \chi(t - 2^{-1}pq) z(p)x(q).
  \] 
  The image of the $w$ is the set of Weyl operators. Two
  Weyl operatos commute if the symplectic inner product
  vanishes. 

  The definition of the Heisenberg group extends to finite
  vector spaces $\mathbb F^{n}$ in a natural way,
  \[
    \mathcal J_{\mathbb F^{2n}} = \bigotimes_{i=1}^{n}
    \mathcal J_{\mathbb F^2}.
  \] 
  Using this we can define the group $H^{n}(\mathbb F)$. The
  Weyl representation of $H^{n}(\mathbb F)$ is then defined
  as
  \[
    (p,q,t)
    \mapsto
    \chi(t)w(p_1,q_1) \otimes \cdots \otimes w(p_n,q_n)
  \] 
  where $p_i$ and $q_i$ are the components of $p$ and $q$ 
  with respect to the natural basis in $\mathbb F^{n}$.

  Now, for the qubit case, the fields that should be used
  are of characteristic two, and so we cannot define the
  Heisenberg group which requires the use of the symbol
  $2^{-1}$. However, Gross shows us that there is still a
  projective representation of $\mathbb F_2$ defined in an
  analogous way:
  \[
    w(p,q) := i^{-pq}z(p)x(q),
  \] 
  for $p,q \in \mathbb F_2$. The group generated by
  $\{w(p,q)\}_{p,q}$ is called the Pauli group, or the
  Heisenberg group for qubits even though it does not
  satisfy the composition law definition. He also mentions
  that the Heisenberg group for qubits is an extension of
  $\Z_4$ by $\mathbb F_2^{2n}$ while the non-binary
  Heisenberg group extends $\mathbb F_d$ by $\mathbb
  F_d^{2n}$.

  He proceeds as in the other article, to define the
  characteristic function of an operator, given by the
  coefficients of the expansion in terms of the Weyl
  operators.

  Next he defines the Wigner function of a hermitian
  operator $A$ to be the symplectic Fourier transform of its
  characteristic function:
  \[
    W := \hat{\mathcal F} \circ \Xi.
  \] 
  Considering the Hilbert space $\mathcal H = \C^{nd}$, $d =
  p^{r}$, $V = \mathbb F_d^{2n}$, the Wigner function for
  all $a \in V$ is given by
  \[
    \frac{1}{d^{n}}
    \Tr\left( 
      \frac{1}{d^{n}} \sum_{b \in V}^{}
      \chi([a,b])^{*}w(b)^{*} A
    \right) .
  \] 
  This leads us the definition of the phase space point
  operators
  \[
    A(a)
    = \frac{1}{d^{n}} \sum_{b \in V}^{}
    \chi([a,b])^{*}w(b)^{*}.
  \] 
  These phase space point operators are hermitian, they have
  unit trace and are an orthonormal basis with respect to
  the Hilbert-Schmidt inner product. These properties
  immediately gives us an interpretation of the Wigner
  function. Furthermore, the Wigner function constructed in
  this manner has symplectic covariance. In the case of the
  qubit, the Wigner function looses its covariance under
  Clifford operations.

  Next he handles the case of extension fields. He considers
  the one-dimensional Heisenberg group on an extension field
  $\mathbb F_{p^{r}}$. Defining the character as
  $\chi(\cdot) = \omega^{\Tr a}$, he defines the reduced
  Heisenberg group with its own composition law. The Weyl
  representation of $H(\mathbb F_{p^{r}})$ is defined on
  \[
    \mathcal H = \C^{p^{r}}.
  \] 
  Introducing a tensor structure on $\mathcal H$, he defines
  the shift and clock oeprators, where these composed of
  operators that act only on specific subsystems:
  \[
    w(p,q,t) 
    = \bigotimes_i w^{(i)}(p_i,q^{i}),
  \] 
  and hence the Weyl representation factors with respect to
  the tensor structure.

  As an intereseting note, he mentions a natural question.
  Given an $p^{r}$ dimensional Hilbert space, is it better
  to associate it with a two dimensional phase space over
  $\mathbb F_{p^{r}}$ or a with a $2r$-dimensional one over
  $\mathbb F_p$? The latter choice seems to be more natural
  since all relevent structures (subspaces, isotropic
  spaces, symplectic mappings) can be mapped from $\mathbb
  F_{p^{r}}^{2}$ to $\mathbb F_p^{2r}$ but not the other way
  around. However, certain constructions in quantum state
  tomography and in the theory of mutually unbiased bases
  rely on the geometry of a two dimensional vector space
  which is of course more present in $\mathbb F_{p^{r}}^2$.
\end{document}

\documentclass[a4paper]{article}

\usepackage[margin=1.1in]{geometry}
\usepackage[utf8]{inputenc}
%\usepackage[T1]{fontenc}
%\usepackage{textcomp}
\usepackage[spanish]{babel}
\usepackage{amsmath, amssymb}
\usepackage{amsthm}
\usepackage{braket}
\decimalpoint

\DeclareMathOperator{\R}{\mathbb{R}}
\DeclareMathOperator{\C}{\mathbb{C}}
\DeclareMathOperator{\N}{\mathbb{N}}
\DeclareMathOperator{\Z}{\mathbb{Z}}
\DeclareMathOperator{\Tr}{Tr}
\DeclareMathOperator{\tr}{tr}

\title{Abelian curves}
\author{Ernesto Camacho Ramírez}
\begin{document}
  \maketitle

  We will use the basis $\{\sigma^3,\sigma^5,\sigma^6\}$ for
  the field $GF(2^3)$ where $\sigma$ is the generating
  element. We solve the recurrence relation for the basis
  elements and choose the all positive solution for all
  rays. From this we can decompose any element of the field
  as a linear combination of said basis elements and use the
  recurrence relation to obtain the general formula:
  \begin{equation}
    c_{\alpha,\mu}
    = c_{\sum_{j=1}^{} a_j \sigma_j,\mu}
    = \chi\left[
    \sum_{k=1}^{N-1} \left(
      \mu a_k \sigma_k
      \sum_{j=k+1}^{N} a_j \sigma_j
    \right) \right]
    \prod_{l=1}^N c_{a_l \sigma_l, \mu}.
  \end{equation}
  Using the rotation coefficients we can calculate the
  corresponding phase $\phi(\alpha,\beta)$ for any point in
  phase space. Finally we can assign a displacement operator
  $D(\alpha,\beta)$ at every point in the following manner:
  \begin{equation}
    D(\alpha,\beta)
    = \phi(\alpha,\beta) Z_\alpha X_\beta.
  \end{equation}

  For this fixed choice of phases for the discrete phase
  space we will see which curves out of all possible types
  of curves are abelian in the sense that they satisfy the
  following equations:
  \begin{equation}
    D(\alpha_1,f(\alpha_1)) D(\alpha_2,f(\alpha_2))
    = D(\alpha_1+\alpha_2, f(\alpha_1+\alpha_2)),
  \end{equation}
  or more generaly for a curve given by a set of points
  $\{(\alpha_i,\beta_i)\}$:
  \begin{equation}
    \label{eqn:prop}
    D(\alpha_1,\beta_1) D(\alpha_2,\beta_2)
    = D(\alpha_1+\alpha_2, \beta_1+\beta_2).
  \end{equation}

  The fixed phase space for the rays is given by the
  following matrix where the rows are indexed by $\alpha$
  and the columns by $\beta$, where the origin is the
  top-left most entry and the field elements are ordered by
  the powers of the generating element $\sigma$:
  \begin{align}
    \Gamma
    &=
    \displaystyle \left(\begin{array}{rrrrrrrr}
    1 & 1 & 1 & 1 & 1 & 1 & 1 & 1 \\
    1 & -1 & -i & 1 & -i & i & i & -1 \\
    1 & -i & -1 & i & -i & i & 1 & -1 \\
    1 & 1 & i & i & i & 1 & 1 & i \\
    1 & -i & -i & i & -1 & 1 & i & -1 \\
    1 & i & i & 1 & 1 & i & 1 & i \\
    1 & i & 1 & 1 & i & 1 & i & i \\
    1 & -1 & -1 & i & -1 & i & i & -i
    \end{array}\right) \\
  \end{align}

  Following the Annals paper we know how to obtain all types
  of additive curves for three qubits. There are 100 regular
  curves, 21 doubly degenerated exceptional curves of the
  form (5.18) and 7 exceptional curves of the form (5.24),
  which are quadruply degenerate in one direction and doubly
  in the other. 

  In addition to this, we can form \textit{bundles} of
  curves with four possible factorization schemes: $(3,0,6),
  (1,6,2), (2,3,4)$ and $(0,9,0)$. We can actually obtain
  different bundles for each factorization scheme, and
  to simplify the search we will focus on the examples given
  in the Annals paper.

  \section{$(3,0,6)$ curves}

  Four bundles of nine regular curves each
  \begin{equation}
    \beta = \phi_0 \alpha + \phi^2 \alpha^2 + \phi
    \alpha^{4},
    \quad
    \alpha = 0,
  \end{equation}
  where $\phi_0 \in F$ and $\tr(\phi) = 0$ have $(3,0,6)$ 
  factorization. The simplest choice is of course given by
  the rays ($\phi = 0$) which were used to fix the phase
  space, and therefore these curves are trivially abelian.
  The other bundles of curves can be obtained by using the
  values $\sigma, \sigma^2$ and $\sigma^{4}$ for $\phi$.
  These bundles can in fact be obtained from the rays by
  local transformations and one would hope that all of them
  are abelian but it seems that this is not the case. 

  Consider the bundle given by $\phi = \sigma$. The curves
  are given by the equation:
  \begin{equation}
    \beta = \phi_0 \alpha + \sigma^2 \alpha^2 + \sigma
    \alpha^{4},
  \end{equation}
  along with the vertical line. For the specific choice of
  $\Gamma$ (all positive solutions for all rays), we
  actually get \textit{five} abelian curves (including the
  vertical ray which is trivial):
  \begin{align}
    \beta = \sigma \alpha + \sigma^2 \alpha^2 + \sigma
    \alpha^{4} 
    &\to \{
      (0,0), (\sigma,\sigma^6), (\sigma^2,\sigma),
      (\sigma^3,\sigma^{7}), (\sigma^{4},\sigma^{5}),
      (\sigma^{5}, \sigma^{3}), (\sigma^{6},\sigma^{4}),
      (\sigma^{7}, \sigma^2)
    \} \\
    \beta = \sigma^2 \alpha + \sigma^2 \alpha^2 + \sigma
    \alpha^{4}
    &\to \{
      (0,0), (\sigma,\sigma), (\sigma^2,\sigma^{5}),
      (\sigma^3,0), (\sigma^{4},\sigma^{6}),
      (\sigma^{5}, \sigma^{5}), (\sigma^{6},\sigma^{6}),
      (\sigma^{7}, \sigma^1)
    \} \\
    \beta = \sigma^3 \alpha + \sigma^2 \alpha^2 + \sigma
    \alpha^{4} 
    &\to \{
      (0,0), (\sigma,\sigma^5), (\sigma^2,\sigma^{4}),
      (\sigma^3,\sigma), (\sigma^{4},\sigma^{7}),
      (\sigma^{5}, \sigma^{2}), (\sigma^{6},\sigma^{3}),
      (\sigma^{7}, \sigma^6)
    \} \\
    \beta = \sigma^6 \alpha + \sigma^2 \alpha^2 + \sigma
    \alpha^{4} 
    &\to \{
      (0,0), (\sigma,0), (\sigma^2,\sigma^{3}),
      (\sigma^3,\sigma^3), (\sigma^{4},\sigma^{3}),
      (\sigma^{5}, 0), (\sigma^{6},0),
      (\sigma^{7}, \sigma^3)
    \}.
  \end{align}
  I have not tested with the other possible values of
  $\phi$, but something similar is bound to happen.

  \section{$(1,6,2)$ curves}

  We can get another four bundles of nine curves of the form
  \begin{equation}
    \beta = \phi_0 \alpha + \phi^2 \alpha^2 + \phi
    \alpha^{4},
    \quad
    \alpha = 0,
  \end{equation}
  where $\phi_0 \in F$ and $\tr(\phi) = 1$. Interestingly
  for $\phi = 1$ we don't get any abelian curves
  (except $\alpha=0$). But by choosing $\phi = \sigma^3$ we
  obtain two non-trivial abelian curves:
  \begin{align}
    \beta = \sigma^{6} \alpha^2 + \sigma^3 \alpha^{4}
    &\to 
    \{
      (0,0), (\sigma,\sigma^3), (\sigma^2,\sigma^{6}),
      (\sigma^3,\sigma^{6}), (\sigma^{4},\sigma^{4}), 
      (\sigma^{5},0), (\sigma^{6},\sigma^3), (1,\sigma^{4})
    \} \\
    \beta = \alpha + \sigma^{6} \alpha^2 + \sigma^3
    \alpha^{4}
    &\to
    \{
      (0,0), (\sigma,1), (\sigma^2,1),
      (\sigma^3,\sigma^{4}), (\sigma^{4},0),
      (\sigma^{5},\sigma^{5}), (\sigma^{6},\sigma^{4}), 
      (1,\sigma^{5})
    \}.
  \end{align}
  Again all the bundles with $\tr(\phi) = 1$ can be obtained
  from the bundle given by $\phi = 1$ by local
  transformations. The paper mentions that we can actually
  use \textit{nonlocal} transformations to obtain bundles of
  factorization $(3,0,6)$ from these bundles.

  \section{$(2,3,4)$ curves}

  There is a third factorization that consists of bundles
  of regular curves. They obtain this by choosing three
  appropriate non-intersecting regular curves and construct
  five additional curves from these ones. By adding the
  vertical line they obtain a complet set of regular curves
  of factorization $(2,3,4)$. One such example of these
  bundles is given by equation (7.17) in the Annals paper.
  Using this example we find that there are four abelian
  curves, only one of which is non-trivial since $\alpha=0$,
  $\beta=0$ and $\beta = \sigma^{3}\alpha$ are obviuosly
  abelian:
  \begin{align}
    \beta = \sigma \alpha + \sigma^2 \alpha^2 + \sigma
    \alpha^{4} 
    &\to 
    \{
      (0,0), (\sigma,\sigma^{6}), (\sigma^2,\sigma), 
      (\sigma^3,1), (\sigma^{4},\sigma^{5}), 
      (\sigma^{5},\sigma^3), (\sigma^{6},\sigma^{4}),
      (1,\sigma^2)
    \}.
  \end{align}

  \section{$(0,9,0)$ curves}

  Bundles of this last factorization type always contain
  exceptional curves. We study the example given by the
  equations (7.18) and (7.19) in the Annals paper. This set
  includes seven regular curves and two exceptional curves.
  Both exceptional curves are doubly degenerate in both
  directions. We have found that \textit{six} of these
  curves are abelian, including both exceptional curves:
  \begin{align}
    \alpha = \beta + \sigma^{6} \beta^2 + \sigma^3 \beta^{4}
    &\to \{
      (0,0), (1, \sigma), (1,\sigma^2),
      (\sigma^{4},\sigma^3), (0,\sigma^{4}),
      (\sigma^{5},\sigma^{5}), (\sigma^{4},\sigma^{6}), 
      (\sigma^{5},1)
    \} \\
    \alpha = \sigma^3 \beta^2 + \sigma^{5} \beta^{4}
    &\to \{
      (0,0), (\sigma^3,\sigma), (\sigma^2,\sigma^2),
      (\sigma^{5},\sigma^3), (\sigma^{5},\sigma^{4}),
      (\sigma^3,\sigma^{5}), (0,\sigma^{6}), (\sigma^2,1)
    \} \\
    \beta = \sigma^6 \alpha^2 + \sigma^3 \alpha^{4}
    &\to \{
      (0,0), (\sigma,\sigma^3), (\sigma^2,\sigma^{6}),
      (\sigma^3,\sigma^{6}), (\sigma^{4},\sigma^{4}),
      (\sigma^{5},0), (\sigma^{6},\sigma^3), (1,\sigma^{4})
    \} \\
    \beta = \alpha + \sigma^3 \alpha^2 + \sigma^{5}
    \alpha^{4} 
    &\to \{
      (0,0), (\sigma,1), (\sigma^2,0), (\sigma^3,\sigma^2),
      (\sigma^{4},1), (\sigma^{5},\sigma^2),
      (\sigma^{6},\sigma^{6}), (1,\sigma^{6})
    \} \\
    \beta^2 + \sigma^{5} \beta = \sigma^2\alpha^2 +
    \sigma^{6}\alpha
    &\to \{
      (0,0), (0,\sigma^{5}), (\sigma^3,\sigma^{4}),
      (\sigma^3,1), (\sigma^{4},0), (\sigma^{4},\sigma^{5}),
      (\sigma^{6},\sigma^{4}), (\sigma^{6},1)
    \} \\
    \beta^2 + \sigma^2 \beta = \sigma^{6} \alpha^2 +
    \sigma^{5} \alpha
    &\to \{
      (0,0), (0,\sigma^2), (\sigma^2,\sigma^3),
      (\sigma^2,\sigma^{5}), (\sigma^{6},0),
      (\sigma^{6},\sigma^2), (1,\sigma^{3}), (1, \sigma^{5})
    \}.
  \end{align}

  \textbf{Note}. Although in the factorization $(0,9,0)$
  example both curves were doubly degenerate in both
  directions, the following curve given by equation (5.25)
  in the Annals paper is doubly degenerate in one direction
  and quadruply degenerate in the other and but it is also
  abelian. This curve has the points:
  \begin{equation}
    \{
      (0,0), (\sigma^3,0), (\sigma^{5},0), (\sigma^2,0),
      (\sigma^3,\sigma^6), (\sigma^5,\sigma^6), 
      (\sigma^2, \sigma^6), (0,\sigma^6)
    \}.
  \end{equation}

  In fact the Annals paper states that there are 21
  exceptional curves of type doubly degenerated in both
  directions, and 7 of the type doubly degenerated in one
  direction and quadruply in the other (section 5.2.2). For
  our $\Gamma$, out of the 21 curves of the first type, 6
  were abelian, and out of the 7 of the other type, 3 were
  abelian.

\end{document}

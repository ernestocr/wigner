\documentclass[a4paper]{article}

\usepackage[margin=1.2in]{geometry}
\usepackage[utf8]{inputenc}
%\usepackage[T1]{fontenc}
%\usepackage{textcomp}
\usepackage[spanish]{babel}
\usepackage{amsmath, amssymb}
\usepackage{amsthm}
\usepackage{braket}
\decimalpoint

\DeclareMathOperator{\R}{\mathbb{R}}
\DeclareMathOperator{\C}{\mathbb{C}}
\DeclareMathOperator{\N}{\mathbb{N}}
\DeclareMathOperator{\Z}{\mathbb{Z}}
\DeclareMathOperator{\Tr}{Tr}
\DeclareMathOperator{\tr}{tr}

\title{Parametrization of curves}
\author{Ernesto Camacho Ramírez}
\begin{document}
  \maketitle

  We can find all of the parametrizations of any additive
  curve (closed under addition) in the following way: we
  first choose three generating elements $p_1,p_2,p_3$ of
  the curve (viewed as subgroup), then using a self-dual
  basis, e.g.  $\{\sigma^3,\sigma^5,\sigma^6\}$ in the case
  of $F = GF(2^3)$, we can express the curve as:
  \begin{equation}
    \label{eqn:param}
    (\alpha(\tau), \beta(\tau))
    = \Tr(\tau \sigma^3) p_1 + \Tr(\tau \sigma^{5}) p_2 +
    \Tr(\tau \sigma^{6}) p_3.
  \end{equation}

  Any exceptional curve that is doubly degenerate in both
  directions can be expressed as:
  \begin{align*}
    (0,0), (\alpha_1,0), (\alpha_1,\beta_1),
    (\alpha_2,\beta_1(\alpha_1^{-1}\alpha_2+1)),
    (\alpha_2,\beta_1\alpha_1^{-1}\alpha_2),\\
    (\alpha_1+\alpha_2,\beta_1\alpha_1^{-1}\alpha_2),
    (\alpha_1+\alpha_2,\beta_1(\alpha^{-1}\alpha +
    1)),(0,\beta_1),
  \end{align*}
  where $\alpha_1$ and $\alpha_2$ are not $0$ and $\alpha_1
  + \alpha_2 = 0$.

  Consider the example $\alpha_1 = \sigma^{4}, \alpha_2 =
  \sigma^3$ (curve 5.19 in the Annals paper which belongs to
  the (0,9,0) bundle):
  \begin{equation}
    (0,0), (\sigma^{4},0), (\sigma^{4},\sigma^{5}),
    (\sigma^3,\sigma^{7}), (\sigma^3,\sigma^{4}), 
    (\sigma^{6},\sigma^{4}), (\sigma^{6},\sigma^{7}),
    (0,\sigma^{5}).
    \label{eqn:curve}
  \end{equation}
  We can choose the generating elements
  \begin{equation}
    (\sigma^4,0), \quad
    (\sigma^3,\sigma^7), \quad
    (\sigma^4,\sigma^5),
  \end{equation}
  and obtain the curve:
  \[
    \begin{array}{c|c}
      \tau & (\alpha(\tau), \beta(\tau)) \\[2mm]
      \hline
      0 & (0,0)\\
      1 & (\sigma^6,\sigma^4) \\
      \sigma & (\sigma^{6},\sigma^{7}) \\
      \sigma^2 & (\sigma^{4},0) \\
      \sigma^3 & (0,\sigma^{5}) \\
      \sigma^4 & (\sigma^{3},\sigma^{7})\\
      \sigma^5 & (\sigma^{4},\sigma^{5})\\
      \sigma^6 & (\sigma^3,\sigma^{4})\\
    \end{array}
  \] 
  which is precisely the curve \eqref{eqn:curve}. For every
  ordered choice of generating elements we obtain a distinct
  parametrization of the curve. There are in total 168
  parametrizations for this curve, which corresponds to the
  amount of isomorphisms between the additive groups of $F$
  and the curve in question.

  In a more general manner we can choose the generating
  elements
  \begin{equation}
    (\alpha_1,0), \quad 
    (0,\beta_1), \quad
    (\alpha_2,\beta_1\alpha_1^{-1}\alpha_2),
  \end{equation}
  and use the equation \eqref{eqn:param} to generate
  parametrizations of \textit{any} $2 \times 2$ degenerate
  curve. Given the parametrization
  $(\alpha(\tau),\beta(\tau))$ I have verified that the
  formula of the eigenstates
  \begin{equation}
    \ket{\psi_\kappa^{\alpha,\beta}}\bra{\psi_{\kappa}^{\alpha,\beta}},
  \end{equation}
  in terms of the displacement operators
  $D(\alpha(\tau),\beta(\tau))$ does gives us the whole
  eigenbasis corresponding to the curve, as it should.

\end{document}

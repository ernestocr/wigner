\documentclass[a4paper]{article}

\usepackage[margin=1.1in]{geometry}
\usepackage[utf8]{inputenc}
%\usepackage[T1]{fontenc}
%\usepackage{textcomp}
\usepackage[spanish]{babel}
\usepackage{amsmath, amssymb}
\usepackage{amsthm}
\usepackage{braket}
\decimalpoint

\DeclareMathOperator{\R}{\mathbb{R}}
\DeclareMathOperator{\C}{\mathbb{C}}
\DeclareMathOperator{\N}{\mathbb{N}}
\DeclareMathOperator{\Z}{\mathbb{Z}}
\DeclareMathOperator{\Tr}{Tr}
\DeclareMathOperator{\tr}{tr}

\newtheorem{theorem}{Theorem}

\title{Delta function for $GF(p^{n})$}
\author{Ernesto Camacho Ramírez}
\begin{document}
  \maketitle

  Recall the definition of an eigenstate of an abelian curve
  $\Gamma = \{(\alpha(\tau),\beta(\tau)) : \tau \in F\}$
  using the displacement operators:
  \begin{equation}
    \ket{\psi_\kappa^{\alpha,\beta}}
    \bra{\psi_\kappa^{\alpha,\beta}}
    = \frac{1}{p^{n}}
    \sum_{\tau}^{} \chi(\kappa\tau) 
    D(\alpha(\tau),\beta(\tau)).
  \end{equation}
  By fixing a phase space and using the definition of the
  kernel in terms of the displacement operators:
  \begin{equation}
    w(\alpha,\beta)
    = \frac{1}{p^{n}} \sum_{\gamma,\delta}^{} 
    \chi(\gamma\beta - \delta\alpha) D(\gamma,\delta),
  \end{equation}
  we can calculate the Wigner function for any eigenstate of
  commuting displacement operators.  Running through the
  calculations and using the fact the displacement operators
  have the correct phases (for odd characteristic the phases
  are unique) we obtain the following simplified expression
  for the Wigner function at the arbitrary point $(a,b)$:
  \begin{equation}
    \braket{
      \psi_{\kappa}^{\alpha,\beta}|w(a,b)|
      \psi_{\kappa}^{\alpha,\beta}
    }
    = \frac{1}{2^{n}}
    \sum_{\tau}^{} 
    \chi\left( \alpha(\tau)b - \beta(\tau)a \right) 
    \chi(\kappa\tau).
  \end{equation}
  Ignoring the cases $\kappa \neq 0$ for the moment, our
  task becomes evaluating the Gauss sum:
  \begin{equation}
    \sum_{\tau}^{} 
    \chi\left( \alpha(\tau)b - \beta(\tau)a \right).
  \end{equation}

  We will use theorem (5.34) from the book
  \textit{Finite Fields} by Lidle and Niederreiter, to show
  that this Gauss sum evaluates to $p^{n}$ precisely at the
  points of the curve, and $0$ anywhere else. We now state
  the the theorem:
  \begin{theorem}
    Let $F$ be a finite field of $p^{n}$ elements where $p$
    is a prime number and let
    \begin{equation}
      f(x)
      = a_r x^{p^{r}} + a_{r-1}x^{p^{r-1}} + \ldots
      + a_1x^{p} + a_0x + a,
    \end{equation}
    be an affine $p$-polynomial over $F$. If $b \in F^{*}$,
    then
    \begin{equation}
      \sum_{\tau}^{} 
      \chi(bf(\tau))
      = \begin{cases}
        \chi(ba) p^{n} & \text{ if } 
        ba_r + b^{p}a_{r-1}^p + \ldots +
        b^{p^{r-1}}a_1^{p^{r-1}} + b^{p^{r}}a_0^{p^{r}} = 0,
        \\
        0 & \text{ otherwise. }
      \end{cases}
    \end{equation}
  \end{theorem}

  Now let us consider the general parametric form of an
  additive curve for the Galois field $F$ of $p^{n}$
  elements:
  \begin{equation}
    \alpha(\tau)
    = \sum_{m=0}^{n-1} \alpha_m \tau^{p^{m}},
    \quad
    \quad
    \beta(\tau)
    = \sum_{m=0}^{n-1} \beta_m \tau^{p^{m}}.
  \end{equation}
  We say that a curve $\Gamma$ is \textit{commutative} if
  for any two points $(\alpha,\beta), (\alpha',\beta')$ on
  the curve we have $\tr(\alpha \beta' - \beta \alpha') =
  0$, which of course imposes certain restrictions on the
  coefficients $\alpha_m$ and $\beta_m$. 
  %This definition imposes the following restriction to
  %the coefficients $\alpha_m$ and $\beta_m$ of the
  %parametrization:
  %\begin{equation}
  %  \label{eqn:commutativity}
  %  \sum_{m=0}^{n-1} \alpha_{n-m}^{p^{m}}
  %  \beta_{n-m+q}^{p^{m}}
  %  =
  %  \sum_{m=0}^{n-1} \beta_{n-m}^{p^{m}}
  %  \alpha_{n-m+q}^{p^{m}},
  %  \quad
  %  \text{for } q = 1,2,\ldots,n-1,
  %\end{equation}
  %where the indices must be understood modulo $n$. 
  Now for a curve of this form, the Gauss sum becomes:
  \begin{align}
    \sum_{\tau}^{} 
    \chi\left( \alpha(\tau)b - \beta(\tau)a \right) 
    &= \sum_{\tau}^{} 
    \chi\left[
    \left(
      \sum_{m=0}^{n-1} \alpha_m \tau^{p^{m}}
    \right)b
    - \left(
      \sum_{m=0}^{n-1} \beta_m \tau^{p^{m}}
    \right)a
    \right] \\
    &= \sum_{\tau}^{} 
    \chi\left[
      \sum_{m=0}^{n-1} \left(
        \alpha_m \tau^{p^{m}} b -
        \beta_m \tau^{p^{m}} a
      \right)
    \right] \\
    &= \sum_{\tau}^{} 
    \chi\left[
      \sum_{m=0}^{n-1} 
      (\alpha_m b - \beta_m a) \tau^{p^{m}}
    \right].
  \end{align}
  Notice that the argument of the additive character has the
  required form for the use of Lidl's theorem, for $b = 1$,
  $r = n-1$ and $a = 0$.  Therefore the gauss sum will be
  equal to $p^{n}$ if and only if the point $(a,b)$ 
  satisfies the following condition:
  \begin{equation}
    \label{eqn:lidl_cond}
      \sum_{m=0}^{n-1} 
      (\alpha_{n-1-m} b - \beta_{n-1-m} a)^{p^{m}}
      %= \sum_{m=0}^{n-1} 
      %\left(
      %  \alpha_{n-1-m}^{p^{m}} b^{p^{m}}
      %  - \beta_{n-1-m}^{p^{m}} a^{p^{m}}
      %\right)
      = 0.
  \end{equation}

  %Let us now consider the particular case in which the point
  %$(a,b) = (\alpha(\tau), \beta(\tau))$ belongs to the curve
  %$\Gamma$. Substituting in the left hand side of equation
  %\eqref{eqn:lidl_cond} we obtain:

  %\begin{align}
  %  \sum_{m=0}^{n-1} 
  %  (\alpha_{n-1-m} \beta(\tau)
  %  - \beta_{n-1-m} \alpha(\tau))^{p^{m}}
  %  &= \sum_{m=0}^{n-1} 
  %  \left[
  %    \alpha_{n-1-m} \left( 
  %      \sum_{k=0}^{n-1} \beta_{k} \tau^{p^{k}}
  %    \right)  - \beta_{n-1-m} \left( 
  %      \sum_{k=0}^{n-1} \alpha_{k} \tau^{p^{k}}
  %    \right) 
  %  \right]^{p^{m}} \\
  %  &= \sum_{m=0}^{n-1} 
  %  \left[
  %    \left( 
  %      \sum_{k=0}^{n-1}
  %      \alpha_{n-1-m} \beta_{k} \tau^{p^{k}}
  %    \right)  - \left( 
  %      \sum_{k=0}^{n-1}
  %      \beta_{n-1-m} 
  %      \alpha_{k} \tau^{p^{k}}
  %    \right) 
  %  \right]^{p^{m}} \\
  %  &= \sum_{m=0}^{n-1} 
  %  \left[
  %    \sum_{k=0}^{n-1}
  %    (\alpha_{n-1-m} \beta_{k}
  %    -
  %    \beta_{n-1-m} \alpha_{k}) \tau^{p^{k}}
  %  \right]^{p^{m}} \\
  %  &= \sum_{m=0}^{n-1}\left[
  %    \sum_{k=0}^{n-1} \left( 
  %    \alpha_{n-1-m}^{p^{m}} \beta_k^{p^{m}}
  %    - \beta_{n-1-m}^{p^{m}} \alpha_k^{p^{m}}
  %  \right) \tau^{p^{k+m}}\right] \\
  %  %= \left[(\alpha_2\beta_0 + \beta_2\alpha_0)\tau
  %  % (\alpha_2\beta_1 + \beta_2\alpha_1)\tau^{p}
  %  % (\alpha_2\beta_2 +
  %  %beta_2\alpha_2)\tau^{p^{2}}\right]\\
  %  %+ \left[(\alpha_1^{p}\beta_0^{p} +
  %  %beta_1^{p}\alpha_0^{p})\tau^{p} +
  %  %\alpha_1^{p}\beta_1^{p} +
  %  %beta_1^{p}\alpha_1^{p})\tau^{p^2}
  %  % (\alpha_1^{p}\beta_2^{p} +
  %  %beta_1^{p}\alpha_2^{p})\tau^{p^3} \right] \\
  %  %+ \left[
  %  % (\alpha_0^{p^{2}}\beta_0^{p^2} +
  %  % \beta_0^{p^2}\alpha_0^{p^2})\tau^{p^2}
  %  % + (\alpha_0^{p^2}\beta_1^{p^2} +
  %  % \beta_0^{p^2}\alpha_1^{p^2})\tau^{p^{3}}
  %  % + (\alpha_0^{p^2}\beta_2^{p^2} +
  %  % \beta_0^{p^2}\alpha_2^{p^2})\tau^{p^{4}}
  %  %right] \\
  %  %= \left[(\alpha_2\beta_0 + \beta_2\alpha_0)\tau
  %  % (\alpha_2\beta_1 + \beta_2\alpha_1)\tau^{p}
  %  % (\alpha_2\beta_2 +
  %  %beta_2\alpha_2)\tau^{p^{2}}\right]\\
  %  %+ \left[(\alpha_1^{p}\beta_0^{p} +
  %  %beta_1^{p}\alpha_0^{p})\tau^{p} +
  %  %\alpha_1^{p}\beta_1^{p} +
  %  %beta_1^{p}\alpha_1^{p})\tau^{p^2}
  %  % (\alpha_1^{p}\beta_2^{p} +
  %  %beta_1^{p}\alpha_2^{p})\tau \right] \\
  %  %+ \left[
  %  % (\alpha_0^{p^{2}}\beta_0^{p^2} +
  %  % \beta_0^{p^2}\alpha_0^{p^2})\tau^{p^2}
  %  % + (\alpha_0^{p^2}\beta_1^{p^2} +
  %  % \beta_0^{p^2}\alpha_1^{p^2})\tau
  %  % + (\alpha_0^{p^2}\beta_2^{p^2} +
  %  % \beta_0^{p^2}\alpha_2^{p^2})\tau^{p}
  %  %right] \\
  %  %= \left( 
  %  % \alpha_2\beta_0 + \beta_2\alpha_0 +
  %  % \alpha_1^{p}\beta_2^{p} + \beta_1^{p}\alpha_2^{p} +
  %  % \alpha_0^{p^2}\beta_1^{p^2} +
  %  % \beta_0^{p^2}\alpha_1^{p^2}
  %  %right)\tau\\
  %  %+ \left( 
  %  % \alpha_2\beta_1 + \beta_2\alpha_1 
  %  % + \alpha_1^{p}\beta_0^{p} + \beta_1^{p}\alpha_0^{p}
  %  % + \alpha_0^{p^2}\beta_2^{p^2} +
  %  % \beta_0^{p^2}\alpha_2^{p^2}
  %  %right)\tau^{p} \\
  %  %+ \left( 
  %  % \alpha_2\beta_2 + \beta_2\alpha_2
  %  % + \alpha_1^{p}\beta_1^{p} + \beta_1^{p}\alpha_1^{p}
  %  % + \alpha_0^{p^2}\beta_0^{p^2} +
  %  % \beta_0^{p^2}\alpha_0^{p^2}
  %  %right)\tau^{p^2} \\
  %  &= \sum_{m=0}^{n-1} 
  %  \left[
  %    \sum_{k=0}^{n-1} 
  %    \left(
  %      \alpha_{n-1-k}^{p^{k}}
  %      \beta_{m-k}^{p^{k}}
  %      - 
  %      \beta_{n-1-k}^{p^{k}}
  %      \alpha_{m-k}^{p^{k}}
  %    \right)
  %  \right]\tau^{p^{m}} \\
  %  &= \sum_{m=0}^{n-1} 
  %  \left[
  %    -\sum_{k=0}^{n-1} \left(
  %    \alpha_{m-k}^{p^{k}}\beta_{n-1-k}^{p^{k}} 
  %    - \beta_{m-k}^{p^{k}} \alpha_{n-1-k}^{p^{k}} \right)
  %  \right]
  %  \tau^{p^{m}}.\label{eqn:sign}
  %\end{align}

  %First notice that for the case $m = n-1$, the coefficient
  %of the term $\tau^{p^{n-1}}$ is equal to zero. For all
  %other cases the coefficient of the term
  %$\tau^{p^{m}}$ in equation \eqref{eqn:sign} can be made
  %equal to a case of equation \eqref{eqn:commutativity} by
  %exponentiating to an appropriate power. In particular for
  %a given $m$, we have that the coefficient in
  %\eqref{eqn:sign} is equal to the $p^{m}$-th power of
  %\eqref{eqn:commutativity}, for $q = n-1-m$:
  %\begin{align}
  %  \left[
  %  \sum_{l=0}^{n-1} \left( 
  %    \alpha_{n-l}^{p^{l}} \beta_{n-l+q}^{p^{l}}
  %    - \beta_{n-l}^{p^{l}} \alpha_{n-l+q}^{p^{l}}
  %  \right) 
  %  \right]^{p^{m}}
  %  &= \sum_{l=0}^{n-1} \left( 
  %    \alpha_{n-l}^{p^{l+m}} \beta_{n-l+n-1-m}^{p^{l+m}}
  %    - \beta_{n-l}^{p^{l+m}} \alpha_{n-l+n-1-m}^{p^{l+m}}
  %  \right) \\
  %  &= \sum_{k=0}^{n-1} \left( 
  %    \alpha_{n-k+m}^{p^{k}}\beta_{n-k+m+n-1-m}^{p^{k}}
  %    - \beta_{n-k+m}^{p^{k}}\alpha_{n-k+m+n-1-m}^{p^{k}}
  %  \right)  \\
  %  &= \sum_{k=0}^{n-1} \left( 
  %    \alpha_{m-k}^{p^{k}}\beta_{n-1-k}^{p^{k}}
  %    - \beta_{m-k}^{p^{k}}\alpha_{n-1-k}^{p^{k}}
  %  \right), 
  %\end{align}
  %where we have used the change of variables $k = l + m$.
  %Therefore for a point on the curve, all such coefficients
  %vanish and we obtain
  %\begin{equation}
  %  \sum_{m=0}^{n-1} 
  %  \left( \alpha_{n-1-m}\beta(\tau) -
  %  \beta_{n-1-m}\alpha(\tau) \right)^{p^{m}}
  %  = 0,
  %\end{equation}
  %i.e., Lidl's theorem condition is satisfied and the Gauss
  %sum evaluates to $p^{n}$.  Now we only need to prove that
  %the only points that satisfy condition
  %\eqref{eqn:lidl_cond} are precisely the ones that belong
  %to the curve. If this is the case then the Gauss sum
  %evaluates to $0$ for every point \textit{not} in the
  %curve, which will finally give us a Wigner function
  %represented as delta function.

  %No other point $(a,b)$ can satisfy the commutativity
  %condition, but it is not so easy to see that the
  %overall sum \eqref{eqn:lidl_cond} does not vanish for an
  %arbitrary point. We proceed by contradiction.

  We will now prove that the only points that satisfy this
  condition for a given curve $\Gamma$ are precisely the
  points that belong to it. Let $(a,b)$ be a point in phase
  space that satisfies condition \eqref{eqn:lidl_cond} for a
  parametrization given by the functions $\alpha(\tau)$ and
  $\beta(\tau)$, and now consider the following identity:
  \begin{align}
    \sum_{m=0}^{n-1}  
    \tr\left[
      (\alpha_m b - \beta_m a) \tau^{p^{m}}
    \right]
    &= \sum_{m=0}^{n-1} 
    \left[
      \sum_{k=0}^{n-1} 
      (\alpha_m b - \beta_m a)^{p^{k}}
      \tau^{p^{m+k}}
    \right] \\
    &= \sum_{m=0}^{n-1} \left[
    \sum_{k=0}^{n-1} \left( 
      \alpha_{m-k} b - \beta_{m-k} a
    \right)^{p^{k}} \right] \tau^{p^{m}} \\
    &= \sum_{m=0}^{n-1} \left[
    \sum_{k=0}^{n-1}
    (\alpha_{n-1-k} b - \beta_{n-1-k} a)^{p^{m+k+1}}
    \right] \tau^{p^{m}} \\
    &= \sum_{m=0}^{n-1} \left[
    \sum_{k=0}^{n-1}
    (\alpha_{n-1-k} b - \beta_{n-1-k} a)^{p^{k}}
    \right]^{p^{m+1}} \tau^{p^{m}}.
  \end{align}
  Notice that the coefficient of the term $\tau^{p^{m}}$ is
  simply the $(m+1)$-th power of the left hand side of
  condition \eqref{eqn:lidl_cond}. Therefore since by
  hypothesis the point $(a,b)$ satisfies the condition
  \eqref{eqn:lidl_cond}, each of the coefficients of the
  terms $\tau^{p^{m}}$ vanish, and therefore
  \begin{equation}
    \sum_{m=0}^{n-1} 
    \tr\left[
      (\alpha_m b - \beta_m a) \tau^{p^{m}}
    \right] = 0,
  \end{equation}
  for all $\tau \in F$. On the other hand notice that
  \begin{align}
    \sum_{m=0}^{n-1} 
    \tr\left[
      (\alpha_m b - \beta_m a) \tau^{p^{m}}
    \right]
    &= \tr\left[
      \sum_{m=0}^{n-1} 
      (\alpha_m b - \beta_m a) \tau^{p^{m}}
    \right] \\
    &= \tr\left[
      \left(\sum_{m=0}^{n-1} 
      \alpha_m \tau^{p^{m}}\right) b
      - \left(\sum_{m=0}^{n-1} 
      \beta_m \tau^{p^{m}}\right) a
    \right] \\
    &= \tr\left[\alpha(\tau) b - \beta(\tau) a\right].
  \end{align}
  Therefore $\tr\left( \alpha(\tau) b - \beta(\tau) a
  \right) = 0$ for all $\tau \in F$, i.e., the point $(a,b)$
  \textit{commutes} with all of the points of the curve
  $\Gamma$. We know that the number of points on a
  commutative curve is $p^{n}$, therefore the only points
  whose Gauss sum is equal to $p^{n}$ are the points that
  are precisely elements of curve $\Gamma$.  So for an
  arbitrary point $(a,b)$ and $\kappa = 0$ we obtain the
  desired delta function:
  \begin{align}
    \braket{
      \psi_{0}^{\alpha,\beta}|w(a,b)|
      \psi_{0}^{\alpha,\beta}
    }
    &= \frac{1}{2^{n}}
    \sum_{\tau}^{} 
    \chi\left( \alpha(\tau)b + \beta(\tau)a \right)  \\
    &= \frac{1}{2^{n}}\begin{cases}
      2^{n} & \text{ if } (a,b) =
      (\alpha(\nu),\beta(\nu)) \quad \text{for some } \nu
      \in F,\\
      0 & \text{ otherwise }
    \end{cases} \\
    &= \sum_{\nu}^{}
    \delta_{a,\alpha(\nu)} \delta_{b,\beta(\nu)}.
  \end{align}

  Now we consider the case $\kappa \neq 0\ldots$

\end{document}

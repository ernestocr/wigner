\documentclass[a4paper]{article}

\usepackage[margin=1.1in]{geometry}
\usepackage[utf8]{inputenc}
%\usepackage[T1]{fontenc}
%\usepackage{textcomp}
\usepackage[spanish]{babel}
\usepackage{amsmath, amssymb}
\usepackage{amsthm}
\usepackage{braket}
\decimalpoint

\DeclareMathOperator{\R}{\mathbb{R}}
\DeclareMathOperator{\C}{\mathbb{C}}
\DeclareMathOperator{\N}{\mathbb{N}}
\DeclareMathOperator{\Z}{\mathbb{Z}}
\DeclareMathOperator{\Tr}{Tr}
\DeclareMathOperator{\tr}{tr}

\title{Rotation coefficients}
\author{Ernesto Camacho Ramírez}
\begin{document}
  \maketitle

  We will try to study an example where the substitution of
  the indices in the coefficients is not trivial. To get up
  to speed let's recall the definition of the rotation
  operators. For a given slope $\xi$ the ray rotation
  operator is defined by its action on the horizontal
  $Z_\alpha$ operators:
  \begin{equation}
    V_\xi Z_\alpha V_\xi^{*}
    = c_{\alpha,\xi}^* \chi(\xi \alpha^2) Z_\alpha
    X_{\xi\alpha},
  \end{equation}
  where $V_\xi$ is unitary and can be expressed diagonally
  in the Fourier basis:
  \begin{equation}
    V_\xi
    = \sum_{\kappa}^{} c_{\kappa,\xi} \ket{\tilde\kappa}
    \bra{\tilde\kappa},
  \end{equation}
  with the coefficients satisfying the recurrence relation
  \begin{equation}
    c_{\alpha,\xi} c_{\kappa,\xi}
    = c_{\alpha+\kappa,\xi} \chi(\xi\kappa\alpha).
  \end{equation}
  This condition must be satisfied in order for the Wigner
  kernel to have the tomographic property. In order to
  explícitly calculate these operators we must solve the
  recurrence relation. In order to match up with the results
  of Klimov, Muñoz, Sanchez-Soto we can follow their
  methodology which consists in first noting that for
  $\alpha = \kappa$ we obtain:
  \begin{equation}
    c_{\alpha,\xi}^2 = \chi(\xi\alpha^2).
  \end{equation}
  By choosing a basis $\{\sigma_j\}$ of the Galois field
  $GF(2^N)$, we can solve this equation (non-uniquely) for
  each basis element $\sigma_j$. Of course we have multiple
  sets of solutions considering the square root, so the
  authors \textit{fix} a choice of signs. In the 2017 paper
  they decided to choose the all positive solutions. Having
  fixed a solution for the basis, we can decompose any
  element of the field as a linear combination of said basis
  elements and use the recurrence relation to obtain the
  general formula:
  \begin{equation}
    c_{\alpha,\xi}
    = c_{\sum_{j=1}^{} a_j \sigma_j,\xi}
    = \chi\left[
    \mu \sum_{k=1}^{N-1} \left(
      a_k \sigma_k
      \sum_{j=k+1}^{N} a_j \sigma_j
    \right) \right]
    \prod_{l=1}^N c_{a_l \sigma_l, \mu}.
  \end{equation}

  Given this solution for the recurrence equation we can now
  obtain the explicit expression of the ray rotation
  operators.

  In the 2017 paper, the authors use the same solution
  method for curves more general than those of the rays.  A
  curve of the form $\beta = f(\alpha)$ for suitable $f$ can
  always be transformed into the horizontal ray. In the
  operator viewpoint this can be accomplished by a
  symplectic operator $P_f$ such that
  \begin{equation}
    P_f Z_\alpha P_f^* \sim Z_\alpha X_{f(\alpha)}.
  \end{equation}
  This operator can once again be expressed diagonally in
  the Fourier basis:
  \begin{equation}
    P_f = \sum_{\kappa}^{} c_{\kappa,f}
    \ket{\tilde\kappa}\bra{\tilde\kappa},
  \end{equation}
  where the coefficients $c_{\kappa,f}$ now satisfy the
  recurrence relation:
  \begin{equation}
    c_{\kappa,f} c_{\alpha,f}
    = \chi(\alpha f(\kappa)) c_{\kappa+\alpha,f},
    \quad c_{0,f} = 1.
  \end{equation}
  Again, assuming that $\alpha = \kappa$ we obtain the
  non-uniquely solvable equation:
  \begin{equation}
    c_{\kappa,f}^2 = \chi(\kappa f(\kappa)).
  \end{equation}
  We employ the same solution method as before, i.e., we
  solve this equation for the basis elements and fix a sign
  convention, which in our case is the all positive
  solutions. Then we use the recurrence relation to obtain
  the following general formula:
  \begin{equation}
    c_{\alpha,f}
    = c_{\sum_{j=1} a_j \sigma_j, f}
    = \chi\left[
    \sum_{k=1}^{N-1} \left(\sum_{j=k+1}^N a_j
    \sigma_j\right) f(a_k \sigma_k) \right] \prod_{l=1}^N
    c_{a_l \sigma_l, f}  , \quad c_{0,f} = 1.
  \end{equation}
  Which is, of course, very similar to the ray formula. We
  can use this rotation operator $P_f$ to obtain new sets of
  orthonormal basis from the standard set
  $\ket{\psi_{\xi,\nu}}$. The nature of the $P_f$ implies
  that these new sets preserve the mutual unbiasedness of
  the standard sets. We label these elements as:
  \begin{equation}
    \ket{\psi_{\xi,\nu}^f}
    = P_f \ket{\psi_{\xi,\nu}},
    \quad
    \ket{\tilde\nu^f}
    = P_f \ket{\tilde\nu}.
  \end{equation}

  Somewhere along the calculation of the Wigner function of
  some rotated state, we substitute some indices of the
  rotation coefficients hoping for some form of cancelation
  of a product of coefficients.  There is no problem with
  the substitution of indices using the delta function, the
  problem stems from \textit{how} the rotation coefficients
  are calculated.  For the particular case of $\xi' = 0$.
  The product of coefficients that we wish to cancel is
  given by:
  \begin{equation}
    c_{\eta,f} c_{\eta,\xi}^* 
    = c_{\eta,f} c_{\eta,\eta^{-1}f(\eta)}^* 
    = 1,
  \end{equation}
  for $\xi = \eta^{-1}f(\eta)$.

  The problem seems to be that even though $f(\eta) = \eta
  \left( \eta^{-1}f(\eta) \right)$ (which means that the
  curve and the ray intersect at the point $\eta$ in phase
  space), the coefficients are calculated slightly
  differently, since the general formula uses the curve/ray
  information in a \textit{multiplicative} manner and not
  just linearly. We try to clarify this using the following
  example. Recall that we are working with the Galois field
  $GF(2^3)$ and the curve in question is given by $f(\alpha)
  = \alpha + \alpha^2 + \alpha^4$. The chosen self-dual
  basis is $\{x^3, x^5, x^6\}$. Now consider the example
  $\eta = 1$.  In the self-dual basis we can express $\eta$
  as the sum 
  \begin{equation}
    \eta = 1 = x^3 + x^5 + x^6,
  \end{equation}
  and so its components in $\Z_2$ are $a_k = 1$ for
  $k=1,2,3$. Now we calculate the solution of the equation
  () (for the ray coefficients) for the chosen basis
  elements: 
  \begin{equation}
    c_{\sigma_j,1}^2
    = \chi(1 \cdot \sigma_j^2),
  \end{equation}
  for which we obtain, after choosing the
  positive solutions:
  \begin{equation}
    c_{\sigma_j,1} = i,
    \quad j = 1,2,3.
  \end{equation}
  Similarly, for the curve coefficients we solve the
  equation ():
  \begin{equation}
    c_{\sigma_j, f}^2
    = \chi(\sigma_j f(\sigma_j)),
  \end{equation}
  and again choosing the positive solutions we obtain:
  \begin{equation}
    c_{\sigma_j, f} = i,
    \quad j = 1,2,3.
  \end{equation}
  Considering the components of $\eta = 1$, we see that
  \begin{equation}
    c_{a_l \sigma_l, 1}
    = c_{a_l \sigma_l, f}
    = i.
  \end{equation}
  and so the product term in both formulas for the general
  coefficients is the same, namely
  \begin{equation}
    \prod_{l=1}^N c_{a_l \sigma_l, 1}
    = \prod_{l=1}^N c_{a_l \sigma_l, f}
    = i^3
    = -i.
  \end{equation}
  Seeing that the coefficients for the basis elements is the
  same for both the ray and curve coefficients, one would
  like to assume that the general coefficients would also be
  the same, considering we can express any field element as
  a linear combination of the basis elements. But as we
  shall see this is not the case at least for our example.
  Let us now calculate the sums inside the group character
  for both cases. For the ray coefficients:
  \begin{align}
    \mu \sum_{k=1}^{N-1} \left( 
      a_k \sigma_k \sum_{j=k+1}^{N} a_j \sigma_j
    \right) 
    &= \sum_{k=1}^{N-1} \left( 
      \sigma_k \sum_{j=k+1}^{N} \sigma_j
    \right) \\
    &= \sigma_1 \sum_{j=2}^{3} \sigma_j
    + \sigma_2 \sum_{j=3}^{3} \sigma_j \\
    &= \sigma_1 \left( \sigma_2 + \sigma_3 \right) 
    + \sigma_2 \sigma_3 \\
    &= x^3 (x^{5} + x^{6}) + x^5 x^6 \\
    &= 0.
  \end{align}
  For the curve coefficients we have:
  \begin{align}
    \sum_{k=1}^{N-1} \left( 
      a_k \sigma_k \sum_{j=k+1}^{N} f(a_j \sigma_j)
    \right) 
    &= \sum_{k=1}^{N-1} \left( 
      \sigma_k \sum_{j=k+1}^{N} f(\sigma_j)
    \right) \\
    &= \sigma_1 \sum_{j=2}^{3} f(\sigma_j)
    + \sigma_2 \sum_{j=3}^{3} f(\sigma_j) \\
    &= \sigma_1 \left( f(\sigma_2) + f(\sigma_3) \right) 
    + \sigma_2 f(\sigma_3) \\
    &= x^3 \left( f(x^5) + f(x^6) \right) 
    + x^5 f(x^6) \\
    &= x^3 \left( 1 + 1 \right) + x^5 \cdot 1 \\
    &= x^5.
  \end{align}
  As we can see the sum is distinct and it happens that the
  group character is also different:
  \begin{equation}
    \chi(0) = 1,
    \quad
    \text{and}
    \quad
    \chi(x^5) = -1,
  \end{equation}
  therefore even though $\xi \eta = f(\eta)$ for $\eta = 1$
  and $\xi = \eta^{-1}f(\eta) = 1$, we obtain distinct
  coefficients:
  \begin{equation}
    c_{\eta,\xi} = c_{1,1} = -i,
    \quad
    \text{and}
    \quad
    c_{\eta,f} = c_{1,f} = i.
  \end{equation}
  It follows that the desired identity does not hold for all
  $\eta$ in this example.

\end{document}

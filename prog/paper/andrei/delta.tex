\documentclass[a4paper]{article}

\usepackage[margin=1.1in]{geometry}
\usepackage[utf8]{inputenc}
%\usepackage[T1]{fontenc}
%\usepackage{textcomp}
\usepackage[spanish]{babel}
\usepackage{amsmath, amssymb}
\usepackage{amsthm}
\usepackage{braket}
\decimalpoint

\DeclareMathOperator{\R}{\mathbb{R}}
\DeclareMathOperator{\C}{\mathbb{C}}
\DeclareMathOperator{\N}{\mathbb{N}}
\DeclareMathOperator{\Z}{\mathbb{Z}}
\DeclareMathOperator{\Tr}{Tr}
\DeclareMathOperator{\tr}{tr}

\newtheorem{theorem}{Theorem}

\title{Fixed phase space example}
\author{Ernesto Camacho Ramírez}
\begin{document}
  \maketitle

  Recall the definition of an eigenstate of an abelian curve
  $\{(\alpha(\tau),\beta(\tau) : \tau \in F\}$ using the
  displacement operators:
  \begin{equation}
    \ket{\psi_\kappa^{\alpha,\beta}}
    \bra{\psi_\kappa^{\alpha,\beta}}
    = \frac{1}{2^{n}}
    \sum_{\tau}^{} \chi(\kappa\tau) 
    D(\alpha(\tau),\beta(\tau)).
  \end{equation}
  By fixing a phase space and using the definition of the
  kernel in terms of the displacement operators:
  \begin{equation}
    w(\alpha,\beta)
    = \frac{1}{2^{n}} \sum_{\gamma,\delta}^{} 
    \chi(\gamma\beta +\delta\alpha) D(\gamma,\delta),
  \end{equation}
  we can calculate the Wigner function for any eigenstate of
  commuting displacement operators.
  Running through the calculations and using the fact the
  displacement operators have the correct phases we obtain
  the following simplified expression for the Wigner
  function at the arbitrary point $(a,b)$:
  \begin{equation}
    \braket{
      \psi_{\kappa}^{\alpha,\beta}|w(a,b)|
      \psi_{\kappa}^{\alpha,\beta}
    }
    = \frac{1}{2^{n}}
    \sum_{\tau}^{} 
    \chi\left( \alpha(\tau)b + \beta(\tau)a \right) 
    \chi(\kappa\tau).
  \end{equation}
  Ignoring the cases $\kappa \neq 0$ for the moment our task
  becomes evaluating the gauss sum:
  \begin{equation}
    \sum_{\tau}^{} 
    \chi\left( \alpha(\tau)b + \beta(\tau)a \right).
  \end{equation}

  We will try to use Lidl's theorem (5.34) to show that this
  gauss sum evaluates to $2^{n}$ precisely at the points of
  the curve, and 0 anywhere else.

  \begin{theorem}
    For a polynomial of the form
    \begin{equation}
      f(x)
      = a_r x^{2^{r}} + a_{r-1}x^{2^{r-1}} + \ldots
      + a_1x^{2} + a_0x + a.
    \end{equation}
    Then 
    \begin{equation}
      \sum_{\tau}^{} 
      \chi(f(\tau))
      = \begin{cases}
        \chi(a) 2^{n} & \text{ if } 
        a_r + a_{r-1}^2 + \ldots + a_1^{2^{r-1}} +
        a_0^{2^{r}} = 0 \\
        0 & \text{ otherwise. }
      \end{cases}
    \end{equation}
  \end{theorem}

  Now let us consider the general form of an additive curve
  for the field $F = GF(2^3)$:
  \begin{equation}
    \alpha(\tau)
    = \sum_{m=0}^{2} \alpha_m \tau^{2^{m}}
    = \alpha_0\tau + \alpha_1\tau^2 + \alpha_2\tau^{4},
    \quad
    \beta(\tau)
    = \sum_{m=0}^{2} \beta_m \tau^{2^{m}}
    = \beta_0\tau + \beta_1\tau^2 + \beta_2\tau^{4}.
  \end{equation}
  If this curve is to be commutative then we require:
  \begin{equation}
    \sum_{m=0}^{2} \alpha_{3-m}^{2^{m}} \beta_{3-m+q}^{2^{m}}
    =
    \sum_{m=0}^{2} \beta_{3-m}^{2^{m}}
    \alpha_{3-m+q}^{2^{m}},
    \quad
    q = 1,2,
  \end{equation}
  where the indices must be understood modulo $n=3$. Now for
  a curve of this form, the gauss sum becomes:
  \begin{align}
    \sum_{\tau}^{} 
    \chi\left( \alpha(\tau)b + \beta(\tau)a \right) 
    &= \sum_{\tau}^{} 
    \chi\left(
    (\alpha_0\tau + \alpha_1\tau^2 + \alpha_2\tau^{4})b
    + (\beta_0\tau + \beta_1\tau^2 + \beta_2\tau^{4})a
    \right) \\
    &= \sum_{\tau}^{} 
    \chi\left( 
      (\alpha_0 b + \beta_0 a)\tau
      + (\alpha_1 b + \beta_1 a)\tau^2
      + (\alpha_2 b + \beta_2 a)\tau^{4}
    \right).
  \end{align}
  Notice that the argument of the additive character has the
  required form of Lidl's theorem, for $r = 2$ and $a = 0$.
  Then the gauss sum will be equal to $2^{n}$ if and only if
  \begin{equation}
    \label{eqn:lidl_cond}
      (\alpha_2 b + \beta_2 a)
      + (\alpha_1 b + \beta_1 a)^2
      + (\alpha_0 b + \beta_0 a)^{4}
      = \alpha_2 b + \beta_2 a
      + \alpha_1^2 b^2 + \beta_1^2 a^2
      + \alpha_0^{4} b^{4} + \beta_0^{4} a^{4}
      = 0.
  \end{equation}
  Using the commutativity condition we can see that the
  points $(a,b)$ which belong to the curve
  $\{(\alpha(\tau),\beta(\tau)) : \tau \in F\}$ satisfy this
  last equation and therefore we obtain $2^{n}$ for the
  Gauss sum, to see this, consider some $\tau \in F$
  and substitute $a = \alpha(\tau)$ and $b = \beta(\tau)$ in
  equation \eqref{eqn:lidl_cond} (I only show for the linear
  term, but the quadratic and quartic term also disappear): 
  \begin{align}
    \alpha_2 b + \beta_2 a
    + \alpha_1^2 b^2 + \beta_1^2 a^2
    + \alpha_0^{4} b^{4} + \beta_0^{4} a^{4}
    &= \left( 
      \alpha_0 \beta_2 + \alpha_2^2 \beta_1^2
      + \alpha_1^{4} \beta_0^{4}
      + \beta_0 \alpha_2 + \beta_2^2\alpha_1^2
      + \beta_1^{4}\alpha_0^{4}
    \right)\tau + \ldots \\
    &= \left( 
      \sum_{m=0}^{2}
      \alpha_{3-m}^{2^{m}} \beta_{3-m+2}^{2^{m}}
      + 
      \sum_{m=0}^{2}
      \beta_{3-m}^{2^{m}} \alpha_{3-m+2}^{2^{m}}
    \right) \tau + \ldots \\
    &= 0 \cdot \tau + \ldots \\
    &= 0.
  \end{align}

  So for any point on the curve, Lidl's condition is
  satisfied and so the sum evaluates to $2^{n}$. This isn't
  saying much since we could have used the trace property of
  the points on a commutative curve to obtain the same
  result easier, but, given Lidl's theorem we  now only need to
  prove that the only points that satisfy condition
  \eqref{eqn:lidl_cond} are the ones that belong to the
  curve. If this is the case then the gauss
  sum evaluates to $0$ for every point not in the curve. 

  \begin{equation}
    \vdots
  \end{equation}

  So for an arbitrary point $(a,b)$ we have
  \begin{align}
    \braket{
      \psi_{0}^{\alpha,\beta}|w(a,b)|
      \psi_{0}^{\alpha,\beta}
    }
    &= \frac{1}{2^{n}}
    \sum_{\tau}^{} 
    \chi\left( \alpha(\tau)b + \beta(\tau)a \right)  \\
    &= \frac{1}{2^{n}}\begin{cases}
      2^{n} & \text{ if } (a,b) =
      (\alpha(\nu),\beta(\nu)), \quad \nu \in F \\
      0 & \text{ otherwise, }
    \end{cases} \\
    &= \delta_{a,\alpha(\nu)} \delta_{b,\beta(\nu)}.
  \end{align}

\end{document}

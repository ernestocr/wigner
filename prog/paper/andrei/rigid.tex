\documentclass[a4paper]{article}

\usepackage[margin=1.1in]{geometry}
\usepackage[utf8]{inputenc}
%\usepackage[T1]{fontenc}
%\usepackage{textcomp}
\usepackage[spanish]{babel}
\usepackage{amsmath, amssymb}
\usepackage{amsthm}
\usepackage{braket}
\decimalpoint

\DeclareMathOperator{\R}{\mathbb{R}}
\DeclareMathOperator{\C}{\mathbb{C}}
\DeclareMathOperator{\N}{\mathbb{N}}
\DeclareMathOperator{\Z}{\mathbb{Z}}
\DeclareMathOperator{\Tr}{Tr}
\DeclareMathOperator{\tr}{tr}

\title{Rotation coefficients}
\author{Ernesto Camacho Ramírez}
\begin{document}
  \maketitle

  To get us up to speed, we recall the method for
  calculating the rotation operator coefficients. We can
  solve the following recurrence relation:
  \begin{equation}
    c_{\alpha,\xi} c_{\kappa,\xi}
    = c_{\alpha+\kappa,\xi} \chi(\xi\kappa\alpha).
  \end{equation}
  by first noting that for $\alpha = \kappa$ we obtain:
  \begin{equation}
    c_{\alpha,\xi}^2 = \chi(\xi\alpha^2).
  \end{equation}
  By choosing a basis $\{\sigma_j\}$ of the Galois field
  $GF(2^N)$, we can solve this equation (non-uniquely) for
  each basis element $\sigma_j$. Of course we have multiple
  sets of solutions considering the square root, so the
  authors \textit{fix} a choice of signs. In the 2017 paper
  they decided to choose the all positive solutions. Having
  fixed a solution for the basis, we can decompose any
  element of the field as a linear combination of said basis
  elements and use the recurrence relation to obtain the
  general formula:
  \begin{equation}
    c_{\alpha,\xi}
    = c_{\sum_{j=1}^{} a_j \sigma_j,\xi}
    = \chi\left[
    \sum_{k=1}^{N-1} \left(
      a_k \sigma_k
      \sum_{j=k+1}^{N} \mu a_j \sigma_j
    \right) \right]
    \prod_{l=1}^N c_{a_l \sigma_l, \mu}.
  \end{equation}

  Given this solution for the recurrence equation we can now
  obtain the explicit expression of the ray rotation
  operators. We can do the same for a commutative additive
  curve $f$, where the coefficients of the rotation operator
  satisfy:
  \begin{equation}
    c_{\kappa,f} c_{\alpha,f}
    = \chi(\alpha f(\kappa)) c_{\kappa+\alpha,f},
    \quad c_{0,f} = 1.
  \end{equation}
  Again, assuming that $\alpha = \kappa$ we obtain the
  non-uniquely solvable equation:
  \begin{equation}
    c_{\kappa,f}^2 = \chi(\kappa f(\kappa)).
  \end{equation}
  We employ the same solution method as before, i.e., we
  solve this equation for the basis elements and fix a sign
  convention, which in our case is the all positive
  solutions. Then we use the recurrence relation to obtain
  the following general formula:
  \begin{equation}
    c_{\alpha,f}
    = c_{\sum_{j=1} a_j \sigma_j, f}
    = \chi\left[
    \sum_{k=1}^{N-1} \left(\sum_{j=k+1}^N a_j
    \sigma_j\right) f(a_k \sigma_k) \right] \prod_{l=1}^N
    c_{a_l \sigma_l, f}  , \quad c_{0,f} = 1.
  \end{equation}
 
  What we are trying to see is if there exists some fixed
  geometrical structure on the discrete phase space such
  that for a set of rays and set of curves, the displacement
  operator phases match for each point. The sets of curves
  in question are the rays:
  \begin{equation}
    \beta = \mu \alpha,
    \quad \mu \in F,
  \end{equation}
  and the set of curves given by the equation:
  \begin{equation}
    \beta = \phi_0 \alpha + \sigma^3 \alpha + \sigma^{5}
    \alpha,
    \quad \phi_0 \in F,
  \end{equation}
  where $\sigma \in F$ is the field generator. If the sets
  of curves cover the whole phase space (and for now
  considering mutually non-intersecting curves), then for a
  given point $(\alpha,\beta)$ there exists $\mu$ and $\nu$ 
  such that $\beta = \mu \alpha = f_\nu(\alpha)$. The
  question is, if there is a way to calculate the
  coefficients such that $c_{\alpha,\mu} = c_{\alpha,f_\nu}$
  for all possible combinations? We must first ask ourselves
  how much freedom we have in the calculation of the
  coefficients.  After fixing a self-dual basis, e.g.,
  $\{\sigma^3, \sigma^{5}, \sigma^{6}\}$, and fixing a curve
  parameter $\mu$, we have two possible solutions for each
  basis element, and therefore $2^3$ possible solution sets
  for each $\mu$. After fixing these coefficients there
  doesn't seem to be anymore choices to be made. So for each
  element of the field $F$ we have $2^{3}$ possible basis
  solutions, and so we have $8^8$ possible ``geometric
  structures'' for a given set of curves.

  The next question is, how can we compare two different
  geometric structures? For each point $(\alpha,\beta)$ in
  phase space we can obtain the displacement operator phase
  by finding the appropriate ray and curve parameters and
  calculating the rotation coefficients:
  \begin{align}
    \phi(\alpha,\beta)
    &= c_{\alpha,\alpha^{-1}\beta} \\
    \phi'(\alpha,\beta)
    &= c_{\alpha,\alpha^{-1}\phi_0 + \sigma^2 + \sigma^{5}}.
  \end{align}
  For example, by choosing the all positive basis solutions
  for each ray and each curve we obtain the following
  matrices (we fix the order the of the field by the powers
  of the generating element $\sigma$):
  \begin{align}
    \beta = \mu \alpha
    &\rightarrow
    \displaystyle \left(\begin{array}{rrrrrrrr}
    1 & 1 & 1 & 1 & 1 & 1 & 1 & 1 \\
    1 & -1 & -i & 1 & -i & i & i & -1 \\
    1 & -i & -1 & i & -i & i & 1 & -1 \\
    1 & 1 & i & i & i & 1 & 1 & i \\
    1 & -i & -i & i & -1 & 1 & i & -1 \\
    1 & i & i & 1 & 1 & i & 1 & i \\
    1 & i & 1 & 1 & i & 1 & i & i \\
    1 & -1 & -1 & i & -1 & i & i & -i
    \end{array}\right) \\
    \beta = \phi_0 \alpha + \sigma^3 \alpha + \sigma^5 \alpha
    &\rightarrow
    \displaystyle \left(\begin{array}{rrrrrrrr}
    1 & 1 & 1 & 1 & 1 & 1 & 1 & 1 \\
    1 & -1 & i & 1 & i & -i & -i & -1 \\
    1 & i & -1 & -i & i & -i & 1 & -1 \\
    1 & 1 & i & i & i & 1 & 1 & i \\
    1 & i & i & -i & -1 & 1 & -i & -1 \\
    1 & i & i & 1 & 1 & i & 1 & i \\
    1 & i & 1 & 1 & i & 1 & i & i \\
    1 & 1 & -1 & -i & 1 & -i & i & -i
    \end{array}\right)
  \end{align}
  We can now compare each entry in both matrices to see if
  the phases are the same. For the all positive choice for
  each curve parameter we can clearly see that we do
  \textit{not} obtain the same phases at the same points.
  For example, consider the entries at the second row and
  third column. The second row corresponds to the element
  $\sigma$, and so we can easily obtain the ray parameter
  which is $\mu = \sigma$, and the curve parameter $\phi_0 =
  \sigma^2 + \sigma$. It can be verified quickly that
  \begin{equation}
    \mu \sigma = \sigma^2 = f_{\phi_0}(\sigma).
  \end{equation}
  But, as we can see from the matrices the rotation
  coefficients are \textit{not} the same:
  \begin{equation}
    c_{\sigma,\mu} = -i \neq i = c_{\sigma,\phi_0}.
  \end{equation}

  The problem is now comparing each of the $8^8$ possible
  $64$ element matrices between the sets of curves.  There
  must be a large amount of redundency in these geometric
  structures but I don't see an easy way to reduce the size
  of the combinatorial problem from a practical point of
  view.


  From an analytical view, we are trying to see if there is
  a set of basis solutions such that:
  \begin{equation}
    \phi(\alpha,\beta) = \phi'(\alpha,\beta),
  \end{equation}
  for all $\alpha,\beta \in F$, i.e., for all $\mu,\nu \in
  F$ there exists $\{c_{\sigma_j,\mu}\}$ and
  $\{c_{\sigma_j,f_\nu}\}$ such that . . . 

\end{document}

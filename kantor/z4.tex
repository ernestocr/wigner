\documentclass[a4paper]{article}

\usepackage[utf8]{inputenc}
\usepackage[T1]{fontenc}
\usepackage{textcomp}
\usepackage[spanish]{babel}
\usepackage{amsmath, amssymb}
\usepackage{amsthm}
\decimalpoint

\DeclareMathOperator{\R}{\mathbb{R}}
\DeclareMathOperator{\C}{\mathbb{C}}
\DeclareMathOperator{\N}{\mathbb{N}}
\DeclareMathOperator{\Z}{\mathbb{Z}}
\DeclareMathOperator{\GF}{GF}
\DeclareMathOperator{\diag}{diag}
\DeclareMathOperator{\GL}{GL}

\newtheorem{definition}{Definition}
%\newtheorem{theorem}{Teorema}
%\newtheorem{proposition}{Proposición}
\newtheorem{lemma}{Lemma}
%\newtheorem{corollary}{Corolario}
\newtheorem{example}{Example}

\title{$\Z_4$-Kerdock Codes, Orthogonal Spreads, and
Extremal Euclidean Line-Sets}

\author{Kantor}
\begin{document}
  \maketitle
  
  One of the main goals of the paper was to imitate a
  construction of equiangular lines using binary Kerdock
  codes, by using $\Z_4$-linear codes instead. Kantor in
  particular has previously investigated connections between
  orthogonal spreads and Kerdock sets. When the vector space
  $\Z_2^{2m+2}$ is equipped with the quadratic form
  \[
    x_1x_{m+2} + x_2x_{m+3}+\cdots + x_{m+1}x_{2m+2},
  \] 
  there are $(2^{m+1}-1)(2^{m}+1)$ singular points. A
  partition of this set of points into $2^{m}+1$ totally
  singular $(m+1)$-spaces is called an \textit{orthogonal
  spread}. From these spreads, Kantor constructed Kerdock
  codes and vice-versa. In this paper, they find a way to
  pass between the binary world of orthogonal spreads and
  the real or complex world of frames and line-sets. The
  bridge between them are extraspecial 2-groups. Each such
  group $E$ has order $2^{1+2k}$ for some integer $k$,
  arises as a group of isometries of the Euclidean space
  $\R^{2^{k}}$. On the other hand, $E / Z(E)$ inherits a
  quadratic form, producing the desired binary geometry.

  They go further and modify the groups with which they work
  with in order to connect symplectic spreads,
  $\Z_4$-Kerdock codes and extremal complex lines-sets.
  Further still, they consider the analogue of the binary
  case using extraspecial $p$-groups for odd primes $p$, in
  order to relate symplectic spreads and extremal line-sets
  in $\C^{p^{m}}$.

  \section{Extraspecial 2-groups}

  Let $k$ be a fixed positive integer and let $N = 2^{k}$.
  Recall that for a prime $p$, a $p$-group $P$ is
  \textit{extraspecial} if the centre $Z(P)$ has order $p$ 
  and if $P / Z(P)$ is elementary abelian. A group is
  elementary abelian if it is abelian and all of its
  elements besides the identity have the same (prime) order.
  It follows the the quotient $P / Z(P)$ is a vector space
  over the galois field $\GF(p)$.

  Let  $E = E_k$ be an extraspecial 2-group of order
  $2^{1+2k}$. It is described as an irreducible group of
  orthogonal $N \times N$ matrices with real entries. A
  representation of a group $G$ is said to be irreducible if
  it has only trivial subrepresentations. A
  subrepresentation is the co-restriction of the
  representation to the general linear group of a
  $G$-invariant subspace. Extraspecial groups are analogues
  of the Heisenberg group over finite fields. Since $E$ has
  $2^{2k}$ distinct linear characters and
  \[
    2^{1+2k} = 2^{2k} \cdot 1^2 + (2^{k})^2,
  \] 
  it is a faithful irreducible representation of $E$. A
  group representation is faithful when the group
  homomorphism is injective.

  A group $L$ of real orthogonal transformations containing
  $E$ as a normal subgroup is needed as well. The elements
  of $L$ act on $E$ by conjugation, fixing the centre $Z(E)$ 
  of order 2. Therefore there is a well-defined action on
  the elementary abelian group $\overline{E} = E / Z(E)$ of
  order $2^{2k}$. This action on $\overline{E}$ preserves a
  non-singular quadratic form $Q$, which provides us with
  the bridge between binary orthogonal and real orthogonal
  geometry.

  Now we explicitly construct the group $E$. Let $V$ denote
  the vector space $\Z_2^{k}$ and $x \cdot y$ denote the
  usal dot producto for $x,y \in V$. Now consider $\R^{N}$ 
  with its natural inner product and let $O(\R^{N})$ denote
  the group of the orthogonal linear transformations. Label
  the standard basis of $\R^{N}$ as $e_v$ where $v \in V$.
  For $b \in V$, define the permutation matrix $X(b)$ and
  the diagonal matrix $Y(b)$ as follows:
  \[
    X(b) : e_v \mapsto e_{v+b},
    \quad
    \text{and}
    \quad
    Y(b) := \diag[(-1)^{b \cdot v}].
  \] 
  The groups $X(V) := \{X(b) | b \in V\}$ and $Y(V) :=
  \{Y(b) | b \in V\}$ are contained in $O(\R^{N})$ and are
  ismorphic to the additive group of $V$. Now consider the
  generated group $E := \langle X(V), Y(V)\rangle$. It is an
  irreducible subgroup of $O(\R^{N})$. 

  \begin{lemma}
    The group $E = \langle X(V), Y(V) \rangle$ is
    extraspecial of order $2^{1+2k}$, and $Z(E) = \langle -I
    \rangle$. Moreover, $E = X(V)Y(V)\langle -I \rangle$,
    and every elemento of $E$ can be written uniquely in the
    form $X(a)Y(b)(-I)^{\gamma}$ for some $a,b \in V$ and
    $\gamma \in \Z_2$.
  \end{lemma}

  A consequence of this lemma is that $E / Z(E) \cong V
  \oplus V$. The natural map from $E$ to $\overline{E} = E /
  Z(E)$ will be denoted by the overline, so that $X(a) \in
  E$ maps to $\overline{X}(a) \in V \oplus V$. The centre
  $Z(E)$ is identified with $\Z_2$ and they consider the map
  $Q : \overline{E} \mapsto \Z_2$ defined by
  \[
    Q(\overline{e}) = e^2,
  \] 
  for any $\overline{e} \in \overline{E}$ and any preimage
  $e$ of $\overline{e}$ in $E$. As an example consider,
  $\overline{X}(a)\overline{Y}(b) \in \overline{E}$, then
  \begin{align*}
    Q\left( \overline{X}(a)\overline{Y}(b) \right) 
    &= \left(X(a)Y(b)(-I)^{\gamma}\right)
    \left(X(a)Y(b)(-I)^{\gamma}\right) \\
    &= X(a+a)Y(b+b)(-I)^{\gamma + \gamma + a \cdot b} \\
    &= X(0)Y(0)(-I)^{a \cdot b} \\
    &= a \cdot b.
  \end{align*} 
  The quadratic form $Q$ is non-singular on $\overline{E}$,
  which means the the associated bilinear form is not
  singular. Such form on $\overline{E}$ is given by 
  \[
     \left( \overline{e}_1, \overline{e}_2 \right) 
     = [e_1,e_2],
  \] 
  where
  \[
    [x,y] = x^{-1} y^{-1} x y
  \] 
  denotes the commutator of $x$ and $y$. It follows from the
  multiplication law in $E$ that
  \begin{align*}
    \left( 
      \overline{X}(a) \overline{Y}(b),
      \overline{X}(a') \overline{Y}(b')
    \right) 
    &= \left(X(a)Y(b)(-I)^{\gamma}\right)^{-1}
    \left(X(a')Y(b')(-I)^{\gamma'}\right)^{-1}
    \left(X(a)Y(b)(-I)^{\gamma})\right)
    \left(X(a')Y(b')(-I)^{\gamma'}\right) \\
    &= Y(b)^{-1}X(a)^{-1}  
    Y(b')^{-1}X(a')^{-1}
    X(a)Y(b)X(a')Y(b') \\
    &= Y(b)X(a)Y(b')X(a')X(a)Y(b)X(a')Y(b') \\
    &= Y(b)X(a)Y(b')X(a') X(a+a')Y(b+b')(-I)^{a' \cdot b} \\
    &= Y(b) X(a)Y(b') X(a)Y(b+b')(-I)^{a' \cdot b} \\
    &= Y(b) X(0) Y(b) (-I)^{a' \cdot b} (-I)^{a \cdot b'} \\
    &= (-I)^{a \cdot b'} (-I)^{a' \cdot b} \\
    &= a \cdot b' - a' \cdot b.
  \end{align*}

  The $k$-spaces $\overline{X}(V)$ and $\overline{Y}(V)$ are
  \textit{totally singular}, i.e., the quadratic form $Q$ 
  vanishes on each of the them. To see this, take any
  element $\overline{X}(a) \in \overline{X}(V)$ and compute
  \[
    Q(\overline{X}(a))
    = Q(\overline{X}(a)\overline{Y}(0))
    = a \cdot 0
    = 0.
  \]
  The same goes for $\overline{Y}(V)$. Kantor trivially
  notes that $\overline{X}(V) \cap \overline{Y}(V) = 0$.
  Therefore $\overline{E}$ is an $\Omega^{+}(2k,2)$ space,
  it has maximal Witt index. Unsure what this means but may
  not be important.

  Kantor et al then say they require isometries of $\R^{N}$
  that normalize $E$ (i.e., the normalizer of $E$ in $L$).
  They obviously normalize the centre $Z(E)$ as well, they
  act on $\overline{E}$ by conjugation, and induce
  isometries of $\overline{E}$. They then define a
  particularly interesting matrix:
  \begin{equation}
    H
    = \frac{1}{\sqrt{N}}
    \left[(-1)^{u \cdot v}\right]_{u,v \in V}.
  \end{equation}
  The matrix $\sqrt{N}H$ is the $k$-fold Kronecker producto
  of the $2 \times 2$ Hadamard matrix
  \[
    \begin{pmatrix} 1 & 1 \\ 1 & -1 \end{pmatrix},
  \] 
  this implies that $H^2 = I$. The matrix $H$ connects the
  matrix  $X$ and $Y$ by conjugation. First observe that
  \begin{align}
    e_a H X(b) H
    &= \left(
      \frac{1}{\sqrt{N}} \sum_{v \in V}^{}
      (-1)^{a \cdot v} e_v
    \right) X(b) H \\
    &= \left( 
      \frac{1}{\sqrt{N}}
      \sum_{v \in V}^{} (-1)^{a \cdot v} e_{v+b} 
    \right) H \\
    &= \frac{1}{N} \sum_{v, w \in V}^{} 
    (-1)^{a \cdot v} (-1)^{(v+b) \cdot w} e_{w} \\
    &= \frac{1}{N} \sum_{v,w \in V}^{} 
    (-1)^{(a + w) \cdot v + b \cdot w} e_{w} \\
    &= \frac{1}{N} \sum_{v, w \in V}^{} 
    (-1)^{(a+w) \cdot v + b \cdot w + a \cdot b + a \cdot
    b}e_w \\
    &= \frac{1}{N} \sum_{v,w}^{} 
    (-1)^{(a+w) \cdot (v + b) + a \cdot b} e_w \\
    &= (-1)^{a \cdot b} \frac{1}{N} \sum_{w \in V}^{}  
    \left( 
      \sum_{v \in V}^{} 
      (-1)^{(a+w) \cdot (v+b)} 
    \right) e_w \\
    &= ? \\
    &= (-1)^{a \cdot b} e_a.
  \end{align}
  Therefore $H^{-1} X(b) H = H X(b) H = Y(b)$. Now define
  the unit vector
  \begin{equation}
    e_b^{*}
    := \frac{1}{\sqrt{N}} \sum_{v \in V}^{}
    (-1)^{b \cdot v} e_v,
  \end{equation}
  to be the $b$-th row of $H$.

  We also need a basis of $\overline{E}$. Let
  $v_1,\ldots,v_k$ denote the standard basis of $V =
  \Z_2^{k}$. This determines bases $x_1,\ldots,x_k$ of the
  subspace $\overline{X}(V)$ and $y_1,\ldots,y_k$ of
  $\overline{Y}(V)$, given by $x_j = \overline{X}(v_j)$  and
  $y_j = \overline{Y}(v_j)$. These two bases are ``dual''
  bases of $\overline{X}(V)$ and $\overline{Y}(V)$ with
  respect to the bilinera alternating form on
  $\overline{E}$, i.e.,
  \[
    (x_i, y_j) = \delta_{ij},
    \quad
    \text{for all } i, j.
  \] 
  To prove this, let $x_i = \overline{X}(v_j)$ and $y_j =
  \overline{Y}(v_j)$. Then
  \begin{align}
    \left( x_i, y_j \right) 
    &= \left( 
      \overline{X}(v_i)\overline{Y}(0),
      \overline{X}(0)\overline{Y}(v_j)
    \right)  \\
    &= v_i \cdot v_j - 0 \cdot 0 \\
    &= \delta_{ij},
  \end{align}
  since the standard basis is orthogonal with respect to the
  inner product. The matrices of linear transformations on
  $\overline{E}$ will be written with respect to the ordered
  basis, $x_1,\ldots,x_k,y_1,\ldots,y_k$. Since $HX(b)H =
  Y(b)$, it follows that $H$ induces the map $x_j
  \leftrightarrow y_j$ on $\overline{E}$.

  The authors then state that every matrix $A$ in the
  general linear group $\GL(V)$ induces an orthogonal
  transformation $\tilde A$ of $\R^{N}$ by permuting
  coordinates: $e_v \mapsto e_{vA}$. Recall the $\GL(V)$ is
  group of invertible maps of $V$, but $V$ a $k$-vector
  space and so these maps are just permutation matrices. The
  group $\GL(V)$ is to be identified with $\{\tilde A | A
  \in \GL(V)\}$.

  \begin{lemma}
    \begin{enumerate}
      \item The orthogonal transformation $\tilde A$ 
        normalizes $E$ and induces $\begin{pmatrix} A & O \\
        O & A^{-T}\end{pmatrix}$ on $\overline{E}$ by
        conjugation:
        \[
          \tilde A^{-1}\overline{X}(a)\overline{Y}(b)\tilde
          A
          = \overline{X}(aA)\overline{Y}(bA^{-T})
          = \overline{X}(a)\overline{Y}(b) 
          \begin{pmatrix} A & O \\ O & A^{-T} \end{pmatrix}.
        \] 
      \item The group $\GL(V)$ acts transitively by
        conjugation on both
         \begin{equation}
          \{\overline{X}(a)\overline{Y}(b) | a \cdot b = 0,
          a, b \neq 0\}
          \quad \text{and} \quad
          \{\overline{X}(a)\overline{Y}(b) | a \cdot b =
          1\}.
        \end{equation}
    \end{enumerate}
  \end{lemma}
  \begin{proof}
    Let $A \in \GL(V)$ and observe that
    \[
      e_v 
      = e_v \left( \widetilde{AA^{-1}} \right) 
      = e_{v AA^{-1}}
      = e_{(vA) A^{-1}}
      = e_{vA} \widetilde{A^{-1}} 
      = \left( e_v \tilde A \right) \widetilde{A^{-1}}.
    \] 
    Therefore $\widetilde{A^{-1}} = \tilde A^{-1}$.  Now,
    for all $b,v \in V$ we have
    \begin{equation*}
      e_v(\tilde A^{-1}X(b)\tilde A)
      = e_{vA^{-1}+b}\tilde A \\
      = e_{v+bA}  \\
      = e_v X(bA),
    \end{equation*}
    so that $\tilde A^{-1}X(b) \tilde A = X(bA)$.
    Furthermore, observing that
    \[
      vA^{-1} \cdot b
      = v \cdot b A^{-T},
    \] 
    we obtain
    \begin{align*}
      e_v \left( \tilde A^{-1} Y(b) \tilde A \right) 
      &= (-1)^{vA^{-1} \cdot b} e_{vA^{-1}} \tilde A \\
      &= (-1)^{vA^{-1} \cdot b} e_{v} \\
      &= (-1)^{v \cdot bA^{-T}} e_v \\
      &= e_v Y(bA^{-T}).
    \end{align*}
    In other words, $\tilde A^{-1}Y(b)\tilde A =
    Y(bA^{-T})$. Then (1) follows:
    \begin{align*}
       A^{-1} \overline{X}(a)\overline{Y}(b) A
      &= \tilde A^{-1} X(a)\tilde A \tilde A^{-1}Y(b) \tilde
      A \\
      &= X(aA) Y(bA^{-T}) \\
      &= \overline{X}(aA)\overline{Y}(bA^{-T}) \\
      &= \overline{X}(a) \overline{Y}(b) \begin{pmatrix} A &
      O \\ 0 & A^{-T}\end{pmatrix}.
    \end{align*}
    For the second part, the authors give an argument about
    the sets of incident and non-incident point-hyperplane
    pairs from $V$. We do not understand this, but an action
    on a set $X$ is transitive if there is only one orbit,
    that is, if there exists $x$ in $X$ such that $Gx = X$.
  \end{proof}

  A part from the isometries mentioned before, the authors
  state that more isometries of $\overline{E}$ and $\R^{N}$
  will be needed. It is mentioned quickly that the
  isometries of $\overline{E}$ that induce the identity on
  $\overline{Y}(V)$ are precisely those described by
  matrices $\begin{pmatrix} I & M \\ O & I \end{pmatrix}$,
  where $M$ is a skew-symmetric $k \times k$ matrix. I
  assume what is meant by an isometry inducing the identity
  on $\overline{Y}(V)$ is that the isometry restricted to
  the subspace is the identity. A direct proof: let $b \in
  V$, then $\overline{X}(0)\overline{Y}(b) \in
  \overline{Y}(V)$ and
  \begin{align*}
    \overline{Y}(b) 
    \begin{pmatrix} I & M \\ O & I \end{pmatrix}
    &= \overline{X}(0)\overline{Y}(b) 
    \begin{pmatrix} I & M \\ O & I \end{pmatrix} \\
    &= 
    \begin{pmatrix} O & O  \end{pmatrix} +
    \begin{pmatrix} O & \overline{Y}(b) \end{pmatrix} \\
    &= \begin{pmatrix} O & \overline{Y}(b) \end{pmatrix}  \\
    &= \overline{Y}(b).
  \end{align*}
  I suppose the necesity of $M$ being a skew-symmetric
  matrix is to preserve the inner product. What the authors
  want is an isometry $d_M$ of $\R^{N}$ that induces
  precisely $\begin{pmatrix} I & M \\ O & I \end{pmatrix}$ 
  on $\overline{E}$ by conjugation.

  To do this consider any quadratic form $Q_M$ on $V$ such
  that the associated bilinear form is $u M v^{T}$, so that
  $Q_M(u+v) = Q_M(u) + Q_M(v) + uMv^{T}$ for all $u, v \in
  V$. Now define
  \begin{equation}
    d_M := \diag\left[(-1)^{Q_M(v)}\right]_{v \in V}.
  \end{equation}
  Then 
  \begin{align*}
    e_v d_M^{-1} X(a) d_M X(a)
    &= (-1)^{Q_M(v)} e_v X(a) d_M X(a) \\
    &= (-1)^{Q_M(v)} e_{v+a} d_M X(a) \\
    &= (-1)^{Q_M(v)} (-1)^{Q_M(v+a)} e_{v+a} X(a) \\
    &= (-1)^{Q_M(v)} (-1)^{Q_M(v+a)} e_v \\
    &= (-1)^{Q_M(v)} (-1)^{Q_M(v) + Q_M(a) + vMa^{T}} e_v \\
    &= (-1)^{Q_M(a)} (-1)^{vMa^{T}} e_v \\
    &= (-1)^{Q_M(a)} (-1)^{aM^{T}v^{T}} e_v \\
    &= (-1)^{Q_M(a)} (-1)^{aM \cdot v} e_v,
  \end{align*}
  since $M$ is symmetric. And so $d_M^{-1} X(a) d_M X(a) =
  (-1)^{Q_M(a)} Y(aM)$. Furthermore
  \begin{align*}
    e_v d_M^{-1} Y(b) d_M
    &= (-1)^{Q_M(v)} e_v Y(b) d_M \\
    &= (-1)^{Q_M(v)} (-1)^{b \cdot v} e_v d_M \\
    &= (-1)^{Q_M(v)} (-1)^{b \cdot v} (-1)^{Q_M(v)} e_v \\
    &= (-1)^{b \cdot v} e_v \\
    &= e_v Y(b).
  \end{align*}
  Hence
  \begin{align*}
    d_M^{-1} X(a)Y(b)d_M
    &= d_M^{-1} X(a) d_M X(a) X(a) d_M^{-1} Y(b) d_M \\
    &= (-1)^{Q_M(a)} Y(aM) X(a) Y(b) \\
    &= (-1)^{Q_M(a)} X(a) X(a) Y(aM) X(a) Y(b) \\
    &= (-1)^{Q_M(a)} X(a) X(a+a) Y(aM+b) (-I)^{a \cdot aM}
    \\
    &= (-I)^{\gamma} X(a) Y(aM+b) \\
    &= (-I)^{\gamma} X(a) Y(aM) Y(b).
  \end{align*}
  The ``phase'' doesn't matter under the bar map, so we have
  \begin{equation}
    d_M^{-1} \overline{X}(a)\overline{Y}(b)d_M
    = \overline{X}(a) \overline{Y}(aM) \overline{Y}(b)
    = \overline{X}(a)\overline{Y}(b) 
    \begin{pmatrix} I & M \\ O & I \end{pmatrix}.
  \end{equation}
  In this manner $d_M$ induces (by conjugation)
  $\begin{pmatrix} I & M \\ O & I\end{pmatrix}$ on
  $\overline{E}$.

  This construction is used in proof of the following lemma,
  which the authors cite an earlier work of Kantor.

  \begin{lemma}
    \begin{enumerate}
      \item Every totally singular $k$-space $W$ of
        $\overline{E}$ such that $\overline{Y}(V) \cap W =
        0$ has the form
        \[
          W = d_M^{-1} \overline{X}(V) d_M
          = \overline{X}(V)
          \begin{pmatrix} I & M \\ O & I \end{pmatrix} 
          = \{\overline{X}(a)\overline{Y}(aM) | a \in V\}
        \] 
        for a unique binary skew-symmetric $k \times k$ 
        matrix $M$. The linear transformation of
        $\overline{E}$ produced by $\begin{pmatrix} I & M \\
        O & I\end{pmatrix}$ preserves the quadratic form
        $Q$.
      \item Let $M_1$ and $M_2$ be binary skew-symmetric $k
        \times k$ matrices for which the corresponding
        totally singular $k$-spaces $W_1$ and $W_2$ satisfy
        $\overline{Y} \cap W_1 = \overline{Y}(V) \cap W_2 =
        0$. Then $W_1 \cap W_2 = 0$ if and only if $M_1 -
        M_2$ is non-singular.
    \end{enumerate}
  \end{lemma}
  The first part of the lemma uses the fact that the
  subspace $\overline{X}(V)$ is a totally singular $k$-space
  of $\overline{E}$, and since the transformation induced by
  $d_M$ preserves the quadratic form, any of the $k$-spaces
  $W$ obtained in this manner will also be totally singular. 

  The authors add one more element of $O(\R^{N})$ that
  normalizes $E$, one that induces a transvection on
  $\overline{E}$, denoted by $\tilde H_2$. Not sure what it
  is for, but they assure that with it, we have enough
  isometries of $ \overline{E}$ to generate the orthogonal
  group $O^{+}(2k,2)$ using conjugation by isometries in
  $O(\R^{N})$ that normalize $E$. The defined the generated
  group
  \begin{equation}
    L := \langle E, \GL(V), d_M, H, \tilde H_2 : M \text{ is
      skew-symmetric}\rangle,
  \end{equation}
  which contains $E$ as a normal subgroup and $L / E$ acts
  on $\overline{E}$ by conjugation. $L$ contains all
  isometries of $\overline{E}$ that fix $\overline{Y}(V)$,
  and hence all that fix $H^{-1}\overline{Y}(V)H =
  \overline{X}(V)$ ; these two groups generate
  $\Omega^{+}(2k,2)$.

  \section{Binary Kerdock codes, orthogonal spreads, and
  real line-set with prescribed angles}

  The authors start by fixing an odd integer $m$ and setting
  $k = m + 1$ for the preceding section. So $E = E_{m+1}$ is
  an extraspecial 2-group of order $2^{1+2(m+1)}$, and
  $\overline{E}$ is an $\Omega^{+}(2m+2,2)$-space. Now
  define our main character.

  \begin{definition}
    An orthogonal spread of $\overline{E}$ is a family
    $\Sigma$ of $2^{m}+1$ totally singular $(m+1)$-spaces
    such that every singular point of $\overline{E}$ belongs
    to exactly one member of $\Sigma$.
  \end{definition}

  The authors defer the construction of orthogonal spreads
  for later. In what follows they establish the connection
  with Kerdock codes. Using one of the previous lemmas, we
  know that $L$ acts transitively on the ordered pairs of
  disjoint (non-trivially) totally singular $(m+1)$-spaces
  of $\overline{E}$. By replacing $\Sigma$ by some $\Sigma
  \ell$ where $\ell \in L$, it can be assumed that
  $\overline{X}(V), \overline{Y}(V) \in \SIgma$. By the
  latest lemma, any $\overline{A} \in \Sigma \setminus
  \{\overline{Y}(V)\}$ can be written as $\overline{X}(V)
  \begin{pmatrix} I & M_A \\ O & I \end{pmatrix}$ for some
  unique skew-symmetric $(m+1) \times (m+1)$ matrix $M_A$.
  The linear transformation induced by such a matrix
  preserves the quadratic form $Q$ defined on $\overline{E}$.

  Now consider a set of binary skew-symmetric $(m+1) \times
  (m+1)$ matrices such that the difference between any two
  of them is non-singular. The authors mention that such a
  set has size at most $2^{m}$. With this they define
  maximal sets of such matrices which they identify with
  \texit{Kerdock sets}.

  \begin{definition}
    A Kerdock set is a set of $2^{m}$ binary skew-symmetric
    $(m+1) \times (m+1)$ matrices such that the difference
    of any two is non-singular.
  \end{definition}

  It happens that $\{M_A : \overline{A} \in \Sigma \setminus
  \{\overline{Y}(V)\}\}$ is a Kerdock set for any orthogonal
  spread $\Sigma$ containing $\overline{Y}(V)$. The
  corresponding \textit{Kerdock code} $\mathcal K(\Sigma)$ 
  is a binary code of length $2^{m+1}$ ; its $2^{m+1}$ 
  coordinate positions are labelled by vectores in $V =
  \Z_2^{m+1}$, and
  \begin{equation}
    \mathcal K(\Sigma)
    := \left\{
      \left( Q_{M_A}(v) + s \cdot v + \varepsilon \right)_{v
      \in V} : \overline{A} \in \Sigma \setminus
      \{\overline{Y}(V)\}, s \in V, \varepsilon \in \Z_2
    \right\},
  \end{equation}
  where $Q_{M_A}$ denotes any quadratic form asociated with
  the alternating bilinear form $u M_A v^{T}$ for each
  $\overline{A} \in \Sigma \setminus \{\overline{Y}(V)\}$.
  There is a lot arbitrariness that goes on in this
  definition, but up to equivalence of codes, the Kerdock
  code $\mathcal K(\Sigma)$ depends only on $\Sigma$ and on
  the distinguished member $\overline{Y}(V)$ of $\Sigma$ 
  that was discarded in the construction of the Kerdock set.
  It is noted that there is not a one-to-one correspondence
  between Kerdock sets and orthogonal spreads.

  Now they consider the Euclidean geometry of the
  extraspecial 2-group $E$. This is where the relationship
  with the rest of the works comes into view, as they
  authors now start studying the invariant subspaces of
  matrices $Y(V)$ and $X(V)$. Let us first recall that a
  module is basically a vector space over rings instead of
  fields. A submodule is a subset of a module that is
  itself a module. An irreducible submodule $N$ of a module
  $M$ is a submodule that is not the intersection of two
  submodules of $M$ in which it is properly contained. This
  is to say, that for all submodules $N_1$ and $N_2$ of $M$,
  \[
    N = N_1 \cap N_2 \implies N_1 = N \quad \text{ or }
    \quad N_2 = N.
  \] 

  \begin{lemma}
    If $e_b^{*}$ is as defined previously, then
    \begin{enumerate}
      \item $\{\langle e_v \rangle : v \in V\}$ is the set
        of irreducible submodules for $Y(V)$.
      \item $\{\langle e_b^{*} \rangle : b \in V\}$ is the
        set of irreducible submodules for $X(V)$.
    \end{enumerate}
  \end{lemma}
  \begin{proof}
    For the first part of the lemma, it is easily seen the
    $Y(V)$, being a group of a diagonal matrices, leaves
    invariant each of the $1$-spaces $\langle e_v \rangle$.
    The authors then make an argument that I don't exactly
    follow in which they prove that they have $2^{m+1}$ 
    inequivalent $1$-dimensional submodules of an
    $2^{m+1}$-dimensional module, so these comprise all
    irreducible submodules.

    For the second part, the $1$-spaces $\langle e_b H
    \rangle = \langle e_b^{*} \rangle$ are the irreducible
    submodules of $H^{-1}Y(V)H = X(V)$. To see this note the
    following:
    \begin{align*}
      e_b^{*} X(v) 
      &= e_b^{*} X(v) H^2 \\
      &= \left(e_b H\right) X(v) H^2 \\
      &= e_b H X(v) H H \\
      &= e_b Y(v) H \\
      &= e_c H \\
      &= e_c^{*} \in \langle e_b H \rangle.
    \end{align*} 
  \end{proof}

  And now for the main lemma Kantor used in his MUBs
  inequivalence proof.
  \begin{lemma}
    Let $A$ and $B$ be subgroups of $E$ such that
    $\overline{A}$ and $\overline{B}$ are totally singular
    $(m+1)$-spaces of $\overline{E}$.
    \begin{enumerate}
      \item The set $\mathcal F(A)$ of $A$-irreducible
        subspaces of $\R^{2^{m+1}}$ is an orthogonal frame:
        a set of $2^{m+1}$ pairwise orthogonal lines through
        the origin.
      \item Assume that $\overline{A} \cap \overline{B} =
        0$, and let $u_1$ and $u_2$ be unit vectors in
        different members of $\mathcal F(A) \cup \mathcal
        F(B)$. If $u_1$ and $u_2$ are both in different
        members of $\mathcal F(A)$, or both are in different
        members of $\mathcal F(B)$, then $(u_1,u_2) = 0$ ;
        otherwise $|(u_1,u_2)| = 2^{-(m+1) / 2}$.
      \item The set $\mathcal F(A)$ is left invariant by
        $E$.
    \end{enumerate}
  \end{lemma}
 
\end{document}

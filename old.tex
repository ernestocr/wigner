La idea de la cuantización de Weyl, es la siguiente:
  consideramos un observable clásico $A : \R^{2n} \to \R$,
  comunmente llamado \textit{símbolo} en el análisis
  armónico y en la óptica cuántica, y lo expresamos mediante
  la transformación de Fourier inversa:
  \begin{equation}
    A(x,p)
    = \frac{1}{(2\pi\hbar)^{n}} \int_{\R^{2n}} \Fr A(\xi,
    \eta) e^{\frac{i}{\hbar} \left( \xi x + \eta p\right) }
    \, d\xi \, d\eta,
  \end{equation}
  donde $\Fr A$ es la transformada de Fourier de $A$.
  Enseguida reemplazamos de manera formal a las variables
  $x$ y $p$ por los operadores $\hat{X}$ y $\hat{P}$, (ésto
  corresponde con el orden simétrico de Weyl) para obtener
  al operador $\hat{A}$ correspondiente:
  \begin{equation}
    \label{eqn:weyl_quant_1}
    \hat{A}(\hat{X},\hat{P})
    = \frac{1}{(2\pi\hbar)^{n}} \int_{\R^{2n}} \Fr
    A(\xi,\eta) e^{\frac{i}{\hbar} \left( \xi \hat{X} + \eta
    \hat{P}\right) } \, d\xi \, d\eta.
  \end{equation}
  Para darle un sentido riguroso a la integral es necesario
  pedir ciertas condiciones de regularidad a la función $A$.
  En la literatura matemática generalmente se comienza por
  definir la transformación de Weyl sobre el espacio de las
  funciones rápidamente decrecientes $\Sz(\R^{2n}) \subset
  L^2(\R^{2n})$ y luego se extiende a las funciones
  cuadraticamente integrables y por dualidad a las
  distribuciones templadas $\Sz'(\R^{2n})$. Omitiremos las
  pruebas de extensión, pero para darle sentido riguroso
  comenzamos por definir al operador exponencial que aparece
  en el integrando de (\ref{eqn:weyl_quant_1}).
  \begin{definition}
    El operador $\hat{M}(\xi, \eta) : L^2(\R^{n}) \to
    L^2(\R^{n})$ definido como
    \begin{equation*}
      \hat{M}(\xi,\eta)
      = e^{\frac{i}{\hbar} \left( \xi \hat{X} + \eta \hat{P}
      \right) }
      = e^{-\frac{i}{2\hbar} \xi \eta} e^{\frac{i}{\hbar}
      \eta \hat{P}} e^{\frac{i}{\hbar} \xi \hat{X}}
      = e^{\frac{i}{2\hbar} \xi \eta} e^{\frac{i}{\hbar}
      \xi \hat{X}} e^{\frac{i}{\hbar} \eta \hat{P}},
    \end{equation*} 
    se conoce como el operador característico de Moyal
    (entre otros nombres). Actúa sobre alguna función $\psi
    \in L^2(\R^{n})$ de la siguiente manera:
    \begin{equation}
      \hat{M}(\xi,\eta)\psi(x)
      = e^{\frac{i}{2\hbar} \xi\eta} e^{\frac{i}{\hbar} \xi
      x} \psi(x + \eta).
    \end{equation}
  \end{definition}
  Utilizando la definición del operador $\hat{M}(\xi,\eta)$,
  se puede dar una definición rigurosa de la transformación
  de Weyl sobre el espacio de Schwartz:
  \begin{definition}
    Sea $A \in \Sz(\R^{2n})$. El operador de Weyl, $\hat{A}
    = \Op_W(A)$ del símbolo $A$ se define para $\psi \in
    \Sz(\R^{n})$ como
    \begin{equation}
      \label{eqn:weyl_quant_2}
      \hat{A}\psi(x)
      = \frac{1}{(2\pi\hbar)^{n}}
      \int_{\R^{2n}} \Fr A(\xi,\eta) \hat{M}(\xi,\eta)
      \psi(x) \, d\xi \, d\eta.
    \end{equation}
  \end{definition}
  Es más común expresar a la transformación de Weyl en términos
  del símbolo $A$ directamente, sin recurrir a la
  transformada de Fourier. Utilizando la definición de $\Fr$,
  de manera formal tenemos:
  \begin{align*}
    \Op_W(A)
    &= \frac{1}{(2\pi\hbar)^{n}} \int_{\R^{2n}} \Fr
    A(\xi,\eta)\hat{M}(\xi,\eta) \, d\xi \, d\eta \\
    &= \frac{1}{(2\pi\hbar)^{n}} \int_{\R^{2n}} \left(
    \int_{\R^{2n}} A(x,p)e^{-i(\xi x + \eta p)} \, dx \, dp
    \right) \hat{M}(\xi,\eta) d\xi \, d\eta \\
    &= \frac{1}{(2\pi\hbar)^{n}} \int_{\R^{2n}} A(x,p) \left(
    \int_{\R^{2n}} e^{-i(\xi x + \eta p)} \hat{M}(\xi,\eta)
    \, d\xi \, d\eta \right) \, dx \, dp \\
    &= \frac{1}{(2\pi\hbar)^{n}} \int_{\R^{2n}} A(x,p)
    \Delta(x,p) \, dx \, dp.
  \end{align*} 
  El la literatura matemática, el operador $\Delta(x,p)$ se
  conoce como el operador de Grossmann-Royer y generalmente
  se denota $\hat{R}(\xi,\eta)$. En la literatura física
  comunmente se llama el operador puntual de cuantización,
  en donde formalmente se puede ver como la transformada de
  Fourier del operador de Weyl $\hat{M}(\xi,\eta)$,
  \begin{equation}
    \label{eqn:phase_point_operator}
    \Delta(x,p)
    = \int_{\R^{2n}} e^{-i(\xi x + \eta p)}
    \hat{M}(\xi,\eta) \, d\xi \, d\eta.
  \end{equation}
  Extendiendo las definiciones anteriores a las
  distribuciones templadas, resulta que el operador puntual
  es la cuantización de Weyl de la medida de Dirac $\delta_x
  \otimes \delta_p \in \Sz'(\R^{2n})$ concentrada en el
  punto $(x,p)$. La acción del operador puntual sobre un
  elemento de $\Sz(\R^{2n})$ está dado de la siguiente
  manera.


  \begin{definition}
    El operador de Heisenberg-Weyl $\hat{T}(\xi,\eta)$ se
    define como
    \[
      \hat{T}(\xi,\eta)\psi(x)
      = e^{\frac{i}{\hbar} (\eta x - \frac{1}{2} \eta
      \xi)}\psi(x - \xi).
    \] 
    Equivalentemente
    \[
      \hat{T}(\xi,\eta)\psi(x)
      = e^{\frac{i}{\hbar} \sigma((\xi,\eta),
      (\hat{x},\hat{p}))}\psi(x),
    \] 
    donde $\sigma((x,p),(x',p')) = p \cdot x' - p' \cdot x$.
    El operador de Heisenberg-Weyl satisface las relaciones
    \[
      \hat{T}(z_0)\hat{T}(z_1)
      = e^{\frac{i}{\hbar} \sigma(z_0,z_1)}
      \hat{T}(z_1)\hat{T}(z_0),
    \] 
    y
    \[
      \hat{T}(z_0+z_1)
      = e^{-\frac{i}{2\hbar} \sigma(z_0,z_1)}
      \hat{T}(z_0)\hat{T}(z_1).
    \] 
  \end{definition}
  \begin{definition}
    El operador de Grossmann-Royer $\hat{R}(\xi,\eta)$ es el
    operador
    \[
      \hat{R}(\xi,\eta) : \Sz(\R^{n}) \to \Sz(\R^{n})
    \] 
    definido por las formulas 
    \[
      \hat{R}(0,0)\psi(x)
      = \psi(-x),
    \]
    y
    \[
      \hat{R}(\xi,\eta)
      = \hat{T}(\xi,\eta) \hat{R}(0,0)
      \hat{T}(\xi,\eta)^{-1},
    \] 
    donde $T$ es el operador de Heisenberg-Weyl. El operador
    de Grossmann-Royer es unitario y es una involución en el
    espacio de Schwartz. Su acción sobre cualquier función
    $\psi : \R^{n} \to \C$ está dado por
    \[
      \hat{R}(\xi,\eta)\psi(x)
      = e^{\frac{2i}{\hbar} \eta (x - \xi)}\psi(2\xi - x).
    \] 
  \end{definition}
  El operador de Heisenberg-Weyl y de Grossmann-Royer están
  relacionados por medio de la transformada de Fourier
  simpléctica, i.e.,
  \[
    \hat{R}(\xi,\eta)\psi(x)
    = \frac{1}{2^{n}} F_\sigma(\hat{T}(\cdot)\psi(x)](-z_0).
  \] 
  \begin{definition}
    El operador puntual de cuantización $\Delta(x,p)$
    (operador de Grossmann-Royer) es un operador acotado de
    la forma
    \begin{equation}
      \label{eqn:point_operator}
      \Delta(x,p)
      = 2e^{-2i x p} V(2x)U(-2x) P
      %= \hat{R}(\xi,\eta)\psi(x)
      %= e^{\frac{2i}{\hbar} \eta \cdot (x - \xi)} \psi(2\xi
      %- x).
    \end{equation} 
    donde $V = e^{ix \hat{x}}$ y $U = e^{ix \hat{p}}$ son
    subgrupos del grupo de Weyl y $P$ es el operador de
    paridad definido como $(P\psi)(x) = \psi(-x)$.  Su
    acción sobre un elemento $\psi \in \Sz(\R^{2n})$ está
    dado por
    \begin{equation}
      \label{eqn:grossmann_royer_op}
      \left(\Delta(x,p)\psi\right)(y)
      = 2e^{2i p (y-x)}\psi(2x - y)
    \end{equation}
  \end{definition}
  El operador puntual de cuantización es unitario sobre
  $L^2(\R^{n})$, pero no es de clase traza. Utilizando al
  operador de Grossmann-Royer podemos definir el operador de
  Weyl en términos del símbolo $A$ directamente.
  
  \begin{definition}
    Sea $A \in \Sz(\R^{2n})$. El operador de Weyl $\hat{A} =
    \Op_W(A)$ está dado para todo $\psi \in \Sz(\R^{n})$ 
    como
    \begin{equation}
      \left( \hat{A}\psi \right)(x)
      = \frac{1}{(\pi\hbar)^{n}} \int_{\R^{2n}}
      A(\xi,\eta)\hat{R}(\xi,\eta)\psi(x) \, d\xi \, d\eta,
    \end{equation}
    donde $\hat{R}(\xi,\eta)$ se conoce como el operador de
    Grossmann-Royer.
  \end{definition}

  Expresando en términos del observable $A$ nos permite
  representar al operador de Weyl de forma integral.
  Utilizando la definición de $\hat{R}(\xi,\eta)$ tenemos
  \[
    \hat{A}\psi(x)
    = (\pi\hbar)^{-n} \int_{\R^{2n}} A(\xi,\eta)
    e^{\frac{2i}{\hbar} \eta \cdot (x - \xi)} \psi(2\xi - x)
    \, d\xi \, d\eta,
  \] 
  y luego aciendo el cambio de variable $y = 2\xi - x$,
  obtenemos
  \begin{equation}
    \label{eqn:weyl_quant_k}
    \hat{A}\psi(x)
    = (2\pi\hbar)^{-n} \int_{\R^{2n}} e^{\frac{i}{\hbar} p
    \cdot (x - y)} A\left( \frac{x+y}{2}, p \right) \psi(y)
    \, dy \, dp.
  \end{equation}
  No todos los operadores se pueden expresar de forma
  integral, los que sí, se llaman operadores integrales y
  son de la forma
  \[
    \hat{A}\psi(x) = \int_{\R^{n}} K(x,y) \psi(y) \, dy
  \] 
  donde $K$ se conoce como el \textit{núcleo
  (distribucional)} de $\hat{A}$. Observando el integrando
  de la ecuación (\ref{eqn:weyl_quant_k}), podemos ver que
  para un operador de Weyl, el núcleo está dado por
  \begin{equation}
    K(x,y)
    = (2\pi\hbar)^{-n} \int_{\R^{n}} e^{\frac{i}{\hbar} p
    \cdot (x - y)}A\left( \frac{x+y}{2}, p \right) \, dp.
  \end{equation}
  Podemos invertir ésta relación por medio de la fórmula de
  inversión de Fourier y otro cambio de variable, para
  obtener una expresión para el símbolo $A$ en términos del
  núcleo $K$ del operador,
  \begin{equation}
    A(x,p)
    = \int_{\R^{n}} e^{-\frac{i}{\hbar} p \cdot y} K\left( x
    + \frac{1}{2}y, x - \frac{1}{2}y\right) \, dy.
  \end{equation}
  Ésto nos permite transformar un operador que admite una
  representación integral a un símbolo que es integrable en
  el espacio de fase. Al símbolo obtenido a partir de
  operador de Weyl se conoce como símbolo de Weyl. Lo
  anterior nos dice que existe una correspondencia entre el
  símbolo de Weyl y los núcleos distribucionales de los
  operadores de Weyl. Más aún, existe una biyección entre
  los símbolos cuadracticamente integrables sobre el esapcio
  de fase y una clase de operadores acotados de
  $L^2(\R^{n})$ conocidos como los operadores
  Hilbert-Schmidt.

  Gosson demuestra que la definición que hemos dado de la
  transformación de Weyl se puede extender a las
  distribuciones templadas del espacio de fase,
  específicamente, 
  \[
    \Op_W : \Sz'(\R^{2n}) \to \mathcal
    L(\Sz(\R^{n}),\Sz'(\R^{n})),
  \]
  donde $\mathcal L(\Sz(\R^{n}),\Sz'(\R^{n}))$ es el espacio
  de los operadores continuos del espacio de Schwartz a su
  dual. Además, utilizando el triplete de Gel'fand
  \[
    \Sz(\R^{2n})
    \subset L^2(\R^{2n})
    \subset \Sz'(\R^{2n}),
  \]
  Gosson demuestra en partícular que la transformación de
  Weyl de una función cuadraticamente integrable corresponde
  a un operador Hilbert-Schmidt. Ésto es particularmente
  útil para nosotros porque los estados cuánticos
  representados por operadores de densidad son de
  Hilbert-Schmidt.

  En resumen tenemos una manera de cuantizar observables
  clásicos, que brinda operadores cuánticos que satisfacen
  varias propiedades útiles, por ejemplo nos brinda los
  operadores correctos correspondientes a las funciones
  coordenadas
  $x_j$ y $p_j$:
  \begin{equation}
    \Op_W(x_j)\psi = x_j\psi,
    \quad
    \Op_W(p_j)\psi = -i\hbar \partial_{x_j}\psi.
  \end{equation}
  Con la transformación de Weyl definida, pasamos a definir
  la transformación de Wigner para el caso continuo y luego
  mostramos la relación entre las dos transformaciones.
  Dicha relación nos permite extender la transformación de
  Wigner a las distribuciones templadas de manera rigurosa.

  La cuestión de la \textit{decuantización} no es trivial.
  Una fuente de los posibles problemas es que para muchos
  operadores de Weyl de observables clásicos es dificil
  identificar si son de clase traza. Ésto es importante
  porque generalemente se denota la decuantización de
  operadores de Weyl mediante
  \[
    A(x,p)
    = \Tr\left( \Delta(x,p) \Op_W(A) \right). 
  \] 
  Pero dicha expresión es problemática por no tiene sentido
  para todo observable $A$. De hecho la traza solo existe de
  manera obvia cuando $\Op_W(A)$ es de clase traza.

  

  

  \subsection{La transformación de Wigner}

  Lo que sigue está principalmente basado en los libros de
  Gosson \cite{gossonWignerTransform2017} y de Folland
  \cite{follandHarmonicAnalysisPhase1989}, quienes
  introducen la función de Wigner de manera general, como
  una transformación integral entre ciertos espacios de
  funciones. Iniciamos definiendo una versión más general de
  la transformación de Wigner, la cual se conoce como la
  \textit{transformación de Wigner cruzada}. 

  \begin{definition}
    La transformación de Wigner cruzada es una
    transformación integral
    \[
      W : L^2(\R^{n}) \times L^2(\R^{n}) \to L^2(\R^{2n}),
    \]
    definida como
    \begin{equation}
      \label{eqn:cross_wigner_transform}
      W(\psi,\phi)(x,p)
      = (2\pi\hbar)^{-n} \int_{\R^{n}} e^{-\frac{i}{\hbar} p
      \cdot y} \psi(x + \tfrac{1}{2}y) \overline{\phi(x -
      \tfrac{1}{2}y)} \, dy,
    \end{equation}
    para $\psi, \phi \in L^2(\R^{n})$ y $x,p \in \R^{n}$.  
  \end{definition}
  En partícular, denotaremos la transformación de Wigner de
  un elemento $\psi$ de $L^2(\R^{n})$ como $W\psi =
  W(\psi,\psi)$. Gosson demuestra que la transformación de
  Wigner cruzada puede ser extendida a un mapeo sesquilineal
  sobre las distribuciones templadas, $W : \Sz'(\R^{n})
  \times \Sz'(\R^{n}) \to \Sz'(\R^{2n})$. Por ejemplo, la
  transformación de Wigner de la delta de Dirac es
  \[
    W\delta = (2\pi\hbar)^{-n} (\delta \otimes 1).
  \] 
  Utilizando los operadores puntuales (operador de
  Grossmann-Royer) nos permite dar una definición
  particularmente sencilla de la transformación de Wigner:
  \begin{definition}
    Sean $\psi,\phi \in \Sz(\R^{n})$. La función
    $(\psi,\phi) \mapsto W(\psi,\phi)$ definida como
    \begin{equation}
      W(\psi,\phi)(x,p)
      = \frac{1}{(\pi\hbar)^{n}} \langle \hat{R}(x,p)\psi,
      \overline{\phi} \rangle_{gosson}
      = \frac{1}{(\pi\hbar)^{n}} \langle \phi,
      \hat{R}(x,p)\psi \rangle.
    \end{equation}
  \end{definition}
  La transformación de Wigner cruzada satisface varias
  propiedades interesantes, enunciamos algunas de ellas.
  \begin{itemize}
    \item Es real valuada.
    \item Es acotada y continua para $\psi,\phi \in
      L^2(\R^{n})$.
    \item Tiene ciertas propiedades de traslación.
    \item \ldots
  \end{itemize}
  \begin{proposition}
    Si $\psi, \phi \in L^2(\R^{n})$, entonces $W(\psi,\phi)
    = \overline{W}(\phi,\psi)$. En particular
    \[
      W\psi
      = W(\psi,\psi)
      = \overline{W}(\psi,\psi)
      = \overline{W\psi},
    \] 
    por lo tanto $W\psi$ es real-valuada. 
  \end{proposition}

  Tiene varias propiedades de traslación interesantes. 
  \begin{proposition}
    Para todo $\psi \in L^2(\R^{n})$ y $z_0 = (x_0,p_0) \in
    \R^{2n}$ tenemos
    \[
      W(\hat{T}(z_0)\psi, \hat{T}(z_0)\phi)(z)
      = T(z_0)W(\psi,\phi)(z),
    \] 
    donde $T(z_0)$ es una traslación, $z \mapsto z + z_0$, y
    sobre funciones definidas en $\R^{2n}$ actúa como
    $T(z_0)f(z) = f(z-z_0)$.
  \end{proposition}
  
  \begin{proposition}
    Sean $\psi \in L^2(\R^{n})$ y supongamos que $W\psi \in
    L^{1}(\R^{2n})$. Entonces $W\psi \in L^{1}(\R^{2n})$ y
    \begin{equation}
      \int_{\R^{2n}} (W\psi)(x,p) \, dx \, dp 
      = \|\psi\|^2_{L^2(\R^{n})}.
    \end{equation}
  \end{proposition}

  \begin{proposition}
    Supongamos que $\psi, \phi \in L^2(\R^{n})$. La función
    $z \mapsto W(\psi,\phi)(z)$ es acotada y continua en
    $\R^{2n}$.
  \end{proposition}

  \begin{proposition}
    Sean $\psi,\phi \in \Sz'(\R^{n})$. La función $z
    \mapsto W(\psi,\phi)(z)$ es continua en $\R^{2n}$.
  \end{proposition}

  \begin{proposition}
    La transformación de Wigner cruzada satisface la
    identidad de Moyal, en particular
    \begin{equation}
      \|W\psi\|_{L^2(\R^{2n})}
      = (2\pi\hbar)^{-n / 2} \|\psi\|_{L^2(\R^{n})}.
    \end{equation}
  \end{proposition}

  \begin{proposition}
    Supongamos que $\psi, \phi \in L^{1}(\R^{n}) \cap
    L^2(\R^{n})$. Entonces
    \begin{equation}
      \int_{\R^{n}} W(\psi,\phi)(x,p) \, dp
      = \psi(x)\overline{\phi}(x),
    \end{equation}
    \begin{equation}
      \int_{\R^{n}} W(\psi,\phi)(x,p) \, dx
      = \Fr\psi(p) \overline{\Fr\phi}(p).
    \end{equation}
    En particular tenemos
    \begin{equation}
      \int_{\R^{n}} W\psi(x,p) \, dp
      = |\psi(x)|^2,
      \quad
      \int_{\R^{n}} W\psi(x,p) \, dx
      = |\Fr\psi(p)|^2.
    \end{equation}
  \end{proposition}

  Gosson menciona que las dos fórmulas anteriores solo son
  casos particulares de una transformación de Radon
  (generalizada) de la distribución de Wigner $W\psi$,
  correspondiente a una integración sobre planos
  Lagrangianos particulares $\ell_P = 0 \times \R^{n}$ y
  $\ell_X = \R^{n} \times 0$, respectivamente. En general
  uno puede definir dicha transformación por medio de
  \[
    R_{\ell}(u)
    = \int_{\ell} W\psi(z) d\mu_{\ell}(z)
  \] 
  donde $d\mu_{\ell}(z)$ es la medida euclideana sobre el
  plano Lagrangiano $\ell$. Un tratado riguroso se puede
  encontrar bajo el área de la geometría integral.

  De las proposiciones anteriores ya podemos ver la utilidad
  que nos puede brindar la transformación de Wigner, ya que
  por medio de integrales sobre algún eje podemos recuperar
  las supuestas densidades de la posición y momentum de las
  funciones de onda. Aún con éste primer acercamiento, no
  hemos hablado sobre la transformación de Wigner de estados
  cuánticos ya sean puros o mixtos, ni de operadores en
  general. Para ésto nos conviene definir la transformación
  de Weyl, la cual nos permitirá asociar observables
  clásicos en el espacio de fase con operadores lineales de
  un espacio de Hilbert.

  Consideremos el operador de Weyl de un observable adecuado
  sobre el espacio de fase. Entonces podemos utilizar a la
  transformación cruzada de Wigner para calcular el producto
  interior $\langle \psi, \hat{A}\phi \rangle$, para
  funciones de Schwartz.
  \begin{proposition}
    \label{prop:wigner-weyl}
    Sea $A \in \Sz(\R^{2n})$. Tenemos que
    \begin{equation}
      \langle \psi, \hat{A}\phi \rangle
      = \int_{\R^{2n}} A(x,p)W(\psi,\phi)(z) \, dx \, dp.
    \end{equation}
    para todo $\psi, \phi \in \Sz(\R^{n})$, donde $\hat{A} =
    \Op_W(A)$.
  \end{proposition}
  De hecho, Gosson menciona que muchas veces se define la
  transformación de Wigner $W$ mediante el producto interno
  anterior. Utilizando ésto, finalmente podemos expresar el
  valor esperado de un operador de Weyl en un estado $\psi$
  mediante una integral en el espacio de fase.
  \begin{proposition}
    El valor esperado de un operador de Weyl $\hat{A} =
    \Op_W(A)$ en un estado $\psi$ está dado por
    \begin{equation}
      \langle \hat{A} \rangle_\psi
      = \frac{1}{\langle \psi, \psi \rangle} 
      \int_{\R^{2n}} A(z) W\psi(z) \, dz.
    \end{equation}
  \end{proposition}  

  Con las propiedades que se probaron en la sección de la
  transformación de Wigner respecto a la integración en los
  ejes del espacio de fase, y con el resultado anterior que
  nos permite calcular el valor esperado de un operador de
  Weyl con la transformación de Wigner, podemos formular
  parte de la mecánica cuántica en el espacio de fase.

  \subsection{Análisis armónico cuántico}

  Recordemos que el operador de densidad describe el
  conjunto estadístico de estados cuánticos puros. En
  general un estado cuántico es un operador que nos brinda
  una distribución de las posibles mediciones de algún
  observable en un experimento. El estado cuántico puro
  coincide con la noción de información completa del
  sistema. Como hemos mencionado antes, generalmente se
  identifica a los estados puros con los elementos del
  espacio de Hilbert $\H$ en cuestión, matemáticamente los
  estados puros se identifican con proyección ortogonales
  $\Pi_\psi$ sobre el rayo $\C \psi = \{\lambda \psi :
  \lambda \in \C\}$. Generalmente se denota la proyección
  utilizando la notación Dirac y en ocasiones la
  adoptaremos:
  \[
    \ket{\psi}\bra{\psi} := \Pi_\psi. 
  \] 
  En la práctica no conocemos la información completa del
  sistema y en éste caso decimos que el sistema está en un
  \textit{estados mixtos}. Podemos definir un estado mixto
  como un conjunto de pares $\{(\lambda_j,\psi_j)\}_{j \in
  \N}$ donde $\psi_j \in L^2(\R^{n})$ es un estado puro y
  $0 \leq \lambda_j \leq 1$ es un probabilidad clásica, es
  decir, $\sum_j \lambda_j = 1$. Con ésto identificamos al
  estado mixto con la suma de proyectores
  \[
    \hat{\rho} = \sum_{j}^{} \lambda_j \Pi_j.
  \] 
  Al operador $\rho$ se le conoce como el operador de
  densidad. Es de clase de traza con $\Tr(\rho) = 1$. La
  \textit{distribución de Wigner} del operador de densidad
  es la función
  \[
    \rho = \sum_{j}^{} \lambda_j W\psi_j.
  \] 

  \begin{definition}
    Un estado cuántico es el conjunto de pares $(\lambda_j,
    \psi_j) \in [0,1] \times L^2(\R^{n})$, indexado por un
    conjunto discreto $F$ donde $\|\psi_j\|_{L^2(\R^{n})} =
    1$ para todo $j \in F$ y $\sum_{j \in F} \lambda_j = 1$.
    El operador lineal $\hat{\rho}$ sobre $L^2(\R^{n})$
    definido por
    \begin{equation}
      \hat{\rho} \psi
      = \sum_{j \in F}^{} \lambda_j \Pi_{\psi_j} \psi
      = \sum_{j \in F}^{} \lambda_j \langle \psi, \psi_j
      \rangle \psi_j,
    \end{equation}
    es el operador de densidad asociado al estado. La
    distribución de Wigner del estado es
    \begin{equation}
      \rho = \sum_{j \in F}^{} \lambda_j W\psi_j.
    \end{equation}
  \end{definition}
  El operador de densidad es auto-adjunto, semi-definido
  positivo, es de clase de traza con $\Tr(\hat{\rho}) = 1$.
  La serie $\rho = \sum_{j}^{} \lambda_j W\psi_j$, converge
  en $L^2(\R^{2n})$ y 
  \[
    \rho \in L^2(\R^{2n}) \cap L^{\infty}(\R^{2n}).
  \] 
  El espectro de un operador de densidad $\hat{\rho}$ sobre
  $L^2(\R^{n})$ es discreto y consiste de números no
  negativos tales que $\lambda_1 \geq \lambda_2 \geq \ldots
  \geq 0$, y $\lim_{j \to \infty} \lambda_j = 0$ (en el
  caso en que $F$ es infinito). Junto con los eigenvectores
  $\psi_j$ podemos expresar al operador como
  \[
    \rho \psi
    = \sum_{j}^{} \lambda_j \langle \psi, \psi_j \rangle
    \psi_j,
  \] 
  y su distribución de Wigner es
  \[
    \rho 
    = \sum_{j}^{} \lambda_j W\psi_j,
  \] 
  donde los $\psi_j$ son ortogonales. En particular notamos
  que distintos estados mixtos nos pueden brindar el mismo
  operador, Gosson menciona que el significado físico de
  ésta ambigüedad no está bien comprendida. Podemos decir
  que la información estadística del estado mixto está
  codificado en el operador de densidad de tal manera que
  podemos hacer predicciones sobre el estado que no dependen
  de la manera en que está expresado. 

  Existe una relación directa entre la distribución de
  Wigner de un operador de densidad por medio de la
  transformación de Weyl. El operador de densidad
  $\hat{\rho}$ puede ser expresado en términos de su
  distribución de Wigner:
  \begin{equation}
    \hat{\rho} 
    = (2\pi\hbar)^{n} \Op_W(\rho).
  \end{equation}

  Se puede demostrar que todo operador de clase traza es el
  producto de dos operadores Hilbert-Schmidt. Ahora
  consideremos en particular el caso en que los operadores
  se define por medio de su símbolo $\hat{A} = \Op_W(A)$.

  \begin{proposition}
    Sean $\hat{A} = \Op_W(a)$ y $\hat{B} = \Op_W(b)$
    operadores de Hilbert-Schmidt. Entonces
    \begin{equation}
      \Tr\left( \hat{A}\hat{B} \right) 
      = \Tr\left( \hat{B}\hat{A} \right) 
      = \frac{1}{(2\pi\hbar)^{n}} \int_{\R^{2n}} a(z)b(z) \,
      dz.
    \end{equation}
  \end{proposition}

  \begin{proposition}
    Sea $\hat{A} = \Op_W(a)$ un operador de clase de traza.
    Si adicionalmente $a \in L^{1}(\R^{n})$, entonces
    \begin{equation}
      \Tr\left( \hat{A} \right) 
      = \frac{1}{(2\pi\hbar)^{n}} \int_{\R^{2n}} a(z) \, dz.
    \end{equation}
  \end{proposition}

  Si $\psi \in L^2(\R^{n})$ es una función con norma
  unitaria, la información de $\psi$ es equivalente a la
  información de su transformación de Wigner [GOSSON]. Se
  puede probar que la proyección ortogonal $\hat{\Pi}_\psi$ 
  sobre el rayo $\{\lambda \psi : \lambda \C\}$ es el
  operador de Weyl con símbolo
  \[
    \pi_\psi = (2\pi\hbar)^{n}W\psi.
  \] 
  Con ésto podemos identificar un estado cuántico $\psi$ con
  su función de Wigner.

  \begin{definition}
    Consideremos el estado mixto $\{(\psi_j,\alpha_j)\}$,
    con $\|\psi_j\| = 1$, $\alpha_j \geq 0$ y $\sum_{j}^{}
    \alpha_j = 1$. El operador de densidad $\rho$ de
    éste estado es el operador de Weyl
    \begin{equation}
      \rho
      = (2\pi\hbar)^{n} \sum_{j}^{} \alpha_j \Op_W(W\psi_j).
    \end{equation}
    La transformación de Wigner de éste estado es la función
    \begin{equation}
      \rho = \sum_{j}^{} \alpha_j W\psi_j.
    \end{equation}
  \end{definition}

  Consideremos un operador de densidad $\hat{\rho}$,
  utilizando la fórmula () verificaremos que obtenemos, una
  expresión que coincide con la definición que hemos dado de
  la distribución de Wigner. Dado que la traza es
  independiente de la base ortonormal, utilizaremos la base
  de eigenvectores del operador de densidad:
  \begin{align*}
    \rho(x,p) &= \Tr\left( \hat{\rho} \Delta(x,p) \right) \\
              &= \sum_{j}^{} \langle \psi_j, 
              \Delta(x,p) \hat{\rho} \psi_j \rangle \\
              &= \sum_{j}^{} \langle \psi_j, \Delta(x,p)
              \sum_{k}^{} \lambda_k \Pi_{\psi_k} \psi_j
              \rangle \\
              &= \sum_{j}^{} \langle \psi_j, \Delta(x,p)
              \lambda_j \psi_j \rangle \\
              &= \sum_{j}^{} \lambda_j W\psi_j.
  \end{align*}

  \subsection{Gosson again}

  \begin{definition}
    Un operador de densidad sobre un espacio de Hilbert
    separable es un operador acotado $\hat{\rho} : \H \to
    \H$ tal que
    \begin{itemize}
      \item $\hat{\rho}$ es auto-adjunto y semi-definido
        positivo: $\hat{\rho} = \hat{\rho}^{*}, \hat{\rho}
        \geq 0$.
      \item $\hat{\rho}$ es de clase de traza y
        $\Tr(\hat{\rho}) = 1$.
    \end{itemize}
  \end{definition}

  El ejemplo más sencillo es el de un estado puro, $\psi \in
  \H$ tal que $\|\psi\| = 1$. Consideremos ahora el operador
  de proyección
  \[
    \hat{\rho}_\psi : \H \to \{\alpha \psi : \alpha \in
    \C\}.
  \] 
  Para todo $\phi \in \H$ tenemos
  \[
    \hat{\rho}_\psi \phi
    = \alpha \psi,
    \quad \alpha = \langle \psi, \phi \rangle.
  \] 
  El operador $\hat{\rho}_\psi$ es un operador de densidad
  puro asociado a $\psi$. Si $\H = L^2(\R^{n})$, entonces
  \[
    \hat{\rho}_\psi \phi(x)
    = \int_{\R^{n}} \overline{\psi}(y)\phi(y) \psi(x) \, dy,
  \] 
  por lo tanto el núcleo asociado a $\hat{\rho}_\psi$ es el
  producto tensorial
  \[
    K_{\hat{\rho}_\psi} = \psi \otimes \overline{\psi}.
  \] 
  Resulta que el operador de densidad puro $\hat{\rho}_\psi$ 
  es simplemente el operador de Weyl cuyo símbolo (salvo un
  factor) es simplemente la transformación de Wigner de
  $\psi$.

  \begin{proposition}
    Sea $\hat{\rho}_\psi$ un operador de densidad asociado a
    un estado puro $\psi$ por medio de la proyección.
    Entonces
    \begin{itemize}
      \item el símbolo de Weyl $\rho_\psi$ de
        $\hat{\rho}_\psi$ y la transformación de Wigner
        $W\psi$ de $\psi$ están relacionados por la fórmula
        \begin{equation}
          \rho_\psi(z)
          = (2\pi\hbar)^{n} W\psi(z).
        \end{equation}
      \item Si $\|\psi\|_{L^2(\R^{n})} = 1$ entonces el
        valor esperado $\langle \hat{A} \rangle_\psi$ de un
        operador de Weyl $\hat{A}$ está dado por
        \[
          \langle \hat{A} \rangle_\psi
          = (\frac{1}{2\pi\hbar})^{n} \Tr(\hat{\rho}_\psi
          \hat{A}).
        \] 
    \end{itemize}
  \end{proposition}

  En el caso en que no tenemos toda la información del
  estado del sistema, podemos formar una mezcla de estados
  $\psi_1,\psi_2,\ldots$ donde cada $\psi_j$ tiene una
  probabilidad $\alpha_j$ de ser el estado ``verdadero''.
  Entonces el operador de densidad del estado mixto es el
  operador auto-adjunto
  \[
    \hat{\rho}
    = \sum_{j}^{} \alpha_j \hat{\rho}_{\psi_j}j
  \] 
  donde $\sum_{j}^{} \alpha_j = 1$ y $\alpha_j \geq 0$.

  \begin{proposition}
    Sea $\hat{\rho} : L^2(\R^{n}) \to L^2(\R^{n})$ un
    operador de densidad. Entonces existe una familia
    $\{\psi_j\}_{j \in F}$ en $L^2(\R^{n})$ y una sucesión
    $\{\lambda_j\}_{j \in F}$ de números con $\lambda_j \geq
    0$ y $\sum_{j}^{} \lambda_j = 1$ tales que el símbolo de
    Weyl $\rho$ de $\hat{\rho}$ está dado por
    \[
      \rho 
      = \sum_{j}^{} \lambda_j W\psi_j.
    \] 
  \end{proposition}
  \begin{proof}
    Existen subespacio ortogonales entre si $\H_1,
    \H_2,\ldots$ de $L^2(\R^{n})$ tales que
    \[
      \hat{\rho}
      = \sum_{j}^{} \alpha_j \hat{\rho}_{\H_j},
      \quad 
      \sum_{j}^{} m_j \alpha_j = 1,
    \] 
    donde $\hat{\rho}_j$ es la proyección ortogonal de $\H$ 
    a $\H_j$ y $m_j = \dim \H_j$. Elige una base ortonormal
    $\psi_1,\ldots,\psi_{m_1}$ de $\H_1$, una base
    ortonormal $\psi_{m_1+1},\ldots,\psi_{m_1+m_2+1}$ de
    $\H_2$, etc. El símbolo de Weyl de $\hat{\rho}$ es
    \[
      \rho
      = \alpha_1 \sum_{j=1}^{m_1} W\psi_j
      + \alpha_2 \sum_{j=m_1}^{m_1+m_2+1} W\psi_j + \ldots.
    \] 
  \end{proof}

  \subsection{Integrando la función de Wigner sobre rectas}

  Una caracterización de la distribución de Wigner de un
  estado cuántico es que integrando sobre ciertos conjuntos
  del espacio de fase nos brinda las probabilidades de
  observar ciertos valores de ciertos observables. Ésto es
  una generalización de las densidades de posición y de
  momentum y se ha demostrado que la función de Wigner es la
  única que tiene ésta propiedad y por lo tanto es una que
  es deseable preservar.

  No es suficiente con obtener las marginales de la posición
  y de momentum correctas para poder determinar la
  distribución conjunta, pues hay más de una. En el espacio
  de fase, podemos integrar sobre franjas formadas por
  rectas paralelas y se puede demostrar que si tenemos un
  conjunto completo de éstas ``marginales'', podemos
  recuperar la distribución conjunta sobre todo el espacio.
  [Braasch] le llama a éste conjunto de marginales un
  conjunto tomograficamente completo.

  Como se ha mencionado anteriormente, integrando sobre $x$
  ó $p$ brinda las densidades de momentum ó de posición
  respectivamente. Un resultado más general nos dice que
  podemos integrar la función de Wigner sobre una franja
  definida por dos rectas paralelas $ax + bp = c_1$ y $ax +
  bp = c_2$ en el espacio de fase y obtener la probabilidad
  de observar un valor entre $c_1$ y $c_2$ del el operador
  de Weyl
  \[
    \Op_W(ax+bp) = a \hat{x} + b \hat{p}.
  \] 
  En lugar de usar la teoría espectral para operadores
  auto-adjuntos, utilizaremos la maquinaria de Dirac.
  [2201.05911] nos dice que $a \hat{x} + b \hat{p}$ es
  esencialmente auto-adjunto y su espectro es $\R$. Entonces
  por Dirac tenemos que $\ket r$ es una eigenvector
  generalizado de $a \hat{x} + b \hat{p}$ para cada $r \in
  \sigma\left( a \hat{x} + b \hat{p} \right) = \R$. 

  Ahora consideremos el mapeo unitario sobre las funciones
  de Schwartz
  \[
    U\psi(x) = e^{icx^2 / 2} \psi(x).
  \] 
  Notemos que $UPU^{-1} = Z = -c \hat{x} + \hat{p}$ para
  funciones de prueba. Tiene sentido pensar que $U$ 
  transforma los eigenvectores generalizados de $P$ a los de
  $Z$. Para cada $r \in \R$ la distribución
  \[
    e^{irx + icx^2 / 2}
  \] 
  es un eigenvector generalizado de $Z$. Los eigenvectores
  normalizados
  \[
    \frac{1}{\sqrt{2\pi}} e^{irx + icx^2 / 2}
  \] 
  forman un sistema ``$\delta$-completo''. Entonces la
  densidad de probabilidad de las amplitudes para $Z$ está
  dado por
  \[
    \braket{r|\psi}
    = \frac{1}{\sqrt{2\pi}} \langle e^{irx + i cx^2 / 2},
    \psi \rangle
    = \frac{1}{\sqrt{2\pi}} \int_{\R} e^{-irx-icx^2 /
    2}\psi(x) \, dx.
  \] 
  Y la densidad de probabilidad es
  \[
    D_Z(r)
    = \frac{1}{2\pi} \left|
    \braket{e^{irx+icx^2 / 2}|\psi}
    \right|^2.
  \] 
  Utilizando la teoría de medida [?] prueba que para todo
  estado $\ket \psi$, el push-forward $(x)_* \mathcal P_W =
  \mathcal P_{\hat{x}}$ donde $\mathcal P_W$ es la
  distribución dada por la densidad de Wigner del estado
  $\psi$ y $\mathcal P_{\hat{x}}$ es la distribución dada la
  densidad del operador de posición. En otras palabras, es
  simplemente el cálculo de las marginales.

  \begin{proposition}
    Para todo estado $\psi \in L^2(\R)$, la distribución de
    Wigner $\mathcal P_W$ es la única distribución sobre
    $\R^{2}$ tal que para todo número real $a$ y $b$, la
    ``distribución marginal'' $(ax+bp)_{*} \mathcal P_W$ es
    correcta en el sentido de que coincide con la
    distribución de probabilidad $\mathcal P_{a \hat{x} + b
    \hat{p}}$.
  \end{proposition}

\subsection{Gosson again}

  \begin{definition}
    Un operador de densidad sobre un espacio de Hilbert
    separable es un operador acotado $\hat{\rho} : \H \to
    \H$ tal que
    \begin{itemize}
      \item $\hat{\rho}$ es auto-adjunto y semi-definido
        positivo: $\hat{\rho} = \hat{\rho}^{*}, \hat{\rho}
        \geq 0$.
      \item $\hat{\rho}$ es de clase de traza y
        $\Tr(\hat{\rho}) = 1$.
    \end{itemize}
  \end{definition}

  El ejemplo más sencillo es el de un estado puro, $\psi \in
  \H$ tal que $\|\psi\| = 1$. Consideremos ahora el operador
  de proyección
  \[
    \hat{\rho}_\psi : \H \to \{\alpha \psi : \alpha \in
    \C\}.
  \] 
  Para todo $\phi \in \H$ tenemos
  \[
    \hat{\rho}_\psi \phi
    = \alpha \psi,
    \quad \alpha = \langle \psi, \phi \rangle.
  \] 
  El operador $\hat{\rho}_\psi$ es un operador de densidad
  puro asociado a $\psi$. Si $\H = L^2(\R^{n})$, entonces
  \[
    \hat{\rho}_\psi \phi(x)
    = \int_{\R^{n}} \overline{\psi}(y)\phi(y) \psi(x) \, dy,
  \] 
  por lo tanto el núcleo asociado a $\hat{\rho}_\psi$ es el
  producto tensorial
  \[
    K_{\hat{\rho}_\psi} = \psi \otimes \overline{\psi}.
  \] 
  Resulta que el operador de densidad puro $\hat{\rho}_\psi$ 
  es simplemente el operador de Weyl cuyo símbolo (salvo un
  factor) es simplemente la transformación de Wigner de
  $\psi$.

  \begin{proposition}
    Sea $\hat{\rho}_\psi$ un operador de densidad asociado a
    un estado puro $\psi$ por medio de la proyección.
    Entonces
    \begin{itemize}
      \item el símbolo de Weyl $\rho_\psi$ de
        $\hat{\rho}_\psi$ y la transformación de Wigner
        $W\psi$ de $\psi$ están relacionados por la fórmula
        \begin{equation}
          \rho_\psi(z)
          = (2\pi\hbar)^{n} W\psi(z).
        \end{equation}
      \item Si $\|\psi\|_{L^2(\R^{n})} = 1$ entonces el
        valor esperado $\langle \hat{A} \rangle_\psi$ de un
        operador de Weyl $\hat{A}$ está dado por
        \[
          \langle \hat{A} \rangle_\psi
          = (\frac{1}{2\pi\hbar})^{n} \Tr(\hat{\rho}_\psi
          \hat{A}).
        \] 
    \end{itemize}
  \end{proposition}

  En el caso en que no tenemos toda la información del
  estado del sistema, podemos formar una mezcla de estados
  $\psi_1,\psi_2,\ldots$ donde cada $\psi_j$ tiene una
  probabilidad $\alpha_j$ de ser el estado ``verdadero''.
  Entonces el operador de densidad del estado mixto es el
  operador auto-adjunto
  \[
    \hat{\rho}
    = \sum_{j}^{} \alpha_j \hat{\rho}_{\psi_j}j
  \] 
  donde $\sum_{j}^{} \alpha_j = 1$ y $\alpha_j \geq 0$.

  \begin{proposition}
    Sea $\hat{\rho} : L^2(\R^{n}) \to L^2(\R^{n})$ un
    operador de densidad. Entonces existe una familia
    $\{\psi_j\}_{j \in F}$ en $L^2(\R^{n})$ y una sucesión
    $\{\lambda_j\}_{j \in F}$ de números con $\lambda_j \geq
    0$ y $\sum_{j}^{} \lambda_j = 1$ tales que el símbolo de
    Weyl $\rho$ de $\hat{\rho}$ está dado por
    \[
      \rho 
      = \sum_{j}^{} \lambda_j W\psi_j.
    \] 
  \end{proposition}
  \begin{proof}
    Existen subespacio ortogonales entre si $\H_1,
    \H_2,\ldots$ de $L^2(\R^{n})$ tales que
    \[
      \hat{\rho}
      = \sum_{j}^{} \alpha_j \hat{\rho}_{\H_j},
      \quad 
      \sum_{j}^{} m_j \alpha_j = 1,
    \] 
    donde $\hat{\rho}_j$ es la proyección ortogonal de $\H$ 
    a $\H_j$ y $m_j = \dim \H_j$. Elige una base ortonormal
    $\psi_1,\ldots,\psi_{m_1}$ de $\H_1$, una base
    ortonormal $\psi_{m_1+1},\ldots,\psi_{m_1+m_2+1}$ de
    $\H_2$, etc. El símbolo de Weyl de $\hat{\rho}$ es
    \[
      \rho
      = \alpha_1 \sum_{j=1}^{m_1} W\psi_j
      + \alpha_2 \sum_{j=m_1}^{m_1+m_2+1} W\psi_j + \ldots.
    \] 
  \end{proof}

Consideremos un operador de densidad $\hat{\rho}$,
  utilizando la fórmula () verificaremos que obtenemos, una
  expresión que coincide con la definición que hemos dado de
  la distribución de Wigner. Dado que la traza es
  independiente de la base ortonormal, utilizaremos la base
  de eigenvectores del operador de densidad:
  \begin{align*}
    \rho(x,p) &= \Tr\left( \hat{\rho} \Delta(x,p) \right) \\
              &= \sum_{j}^{} \langle \psi_j, 
              \Delta(x,p) \hat{\rho} \psi_j \rangle \\
              &= \sum_{j}^{} \langle \psi_j, \Delta(x,p)
              \sum_{k}^{} \lambda_k \Pi_{\psi_k} \psi_j
              \rangle \\
              &= \sum_{j}^{} \langle \psi_j, \Delta(x,p)
              \lambda_j \psi_j \rangle \\
              &= \sum_{j}^{} \lambda_j W\psi_j.
  \end{align*}

  Brasch

Ahora consideremos el mapeo unitario sobre las funciones
  de Schwartz
  \[
    U\psi(x) = e^{icx^2 / 2} \psi(x).
  \] 
  Notemos que $UPU^{-1} = Z = -c \hat{x} + \hat{p}$ para
  funciones de prueba. Tiene sentido pensar que $U$ 
  transforma los eigenvectores generalizados de $P$ a los de
  $Z$. Para cada $r \in \R$ la distribución
  \[
    e^{irx + icx^2 / 2}
  \] 
  es un eigenvector generalizado de $Z$. Los eigenvectores
  normalizados
  \[
    \frac{1}{\sqrt{2\pi}} e^{irx + icx^2 / 2}
  \] 
  forman un sistema ``$\delta$-completo''. Entonces la
  densidad de probabilidad de las amplitudes para $Z$ está
  dado por
  \[
    \braket{r|\psi}
    = \frac{1}{\sqrt{2\pi}} \langle e^{irx + i cx^2 / 2},
    \psi \rangle
    = \frac{1}{\sqrt{2\pi}} \int_{\R} e^{-irx-icx^2 /
    2}\psi(x) \, dx.
  \] 
  Y la densidad de probabilidad es
  \[
    D_Z(r)
    = \frac{1}{2\pi} \left|
    \braket{e^{irx+icx^2 / 2}|\psi}
    \right|^2.
  \] 
  Utilizando la teoría de medida [?] prueba que para todo
  estado $\ket \psi$, el push-forward $(x)_* \mathcal P_W =
  \mathcal P_{\hat{x}}$ donde $\mathcal P_W$ es la
  distribución dada por la densidad de Wigner del estado
  $\psi$ y $\mathcal P_{\hat{x}}$ es la distribución dada la
  densidad del operador de posición. En otras palabras, es
  simplemente el cálculo de las marginales.

  \begin{proposition}
    Para todo estado $\psi \in L^2(\R)$, la distribución de
    Wigner $\mathcal P_W$ es la única distribución sobre
    $\R^{2}$ tal que para todo número real $a$ y $b$, la
    ``distribución marginal'' $(ax+bp)_{*} \mathcal P_W$ es
    correcta en el sentido de que coincide con la
    distribución de probabilidad $\mathcal P_{a \hat{x} + b
    \hat{p}}$.
  \end{proposition}


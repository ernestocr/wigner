\documentclass[a4paper, 11pt]{article}
\usepackage[margin=1.3in]{geometry}
\usepackage[utf8]{inputenc}
\usepackage[T1]{fontenc}
\usepackage{textcomp}
%\usepackage[spanish]{babel}
\usepackage{amsmath, amssymb}
\usepackage{amsthm}
\usepackage{braket}
%\decimalpoint

\usepackage{biblatex}
\addbibresource{refs.biblatex}

\DeclareMathOperator{\R}{\mathbb{R}}
\DeclareMathOperator{\C}{\mathbb{C}}
\DeclareMathOperator{\N}{\mathbb{N}}
\DeclareMathOperator{\Z}{\mathbb{Z}}
\DeclareMathOperator{\F}{\mathbb{F}}
\DeclareMathOperator{\GF}{GF}
\DeclareMathOperator{\GR}{GR}
\DeclareMathOperator{\tr}{tr}
\DeclareMathOperator{\Tr}{Tr}
\DeclareMathOperator{\im}{img}

\newtheorem{definition}{Definition}
\newtheorem{theorem}{Theorem}
\newtheorem{proposition}{Proposition}
\newtheorem{lemma}{Lemma}
\newtheorem{corollary}{Corrollary}
\newtheorem{example}{Example}

\title{Wigner point operators from Albert's twisted fields}
\author{Ernesto Camacho Ramírez}

\begin{document}

  \maketitle

  \section{Introduction}

  One of the simplest examples of inequivalent symplectic
  spreads for odd characteristic, with respect to the
  Desarguesian spread, is given by a particular case of
  Albert's twisted fields.  The lowest dimension
  $\C^{p^{n}}$ space for which we can apply the construction
  is of dimension $d = 3^{3}$. These families of symplectic
  spreads are actually unbounded in the number of pairwise
  inequivalent examples as a function of $n$.

  We had previously defined the discrete Wigner function
  $W_\rho$ of some state $\rho$ in terms of a set of
  covariant, Hilbert-Schmidt orthogonal \textit{point
  operators} $A(\alpha)$ for $\alpha \in V \oplus V$. These
  point operators are defined by summing the mutually
  unbiased projection operators designated by some
  \textit{quantum net} or \textit{quadrature system}, for
  each line in phase space that passes through the point
  $\alpha$. The quadrature system assigns eigenstates of
  maximal commuting classes of generalized Pauli operators
  to lines in phase space in a manner given by some
  symplectic spread of the space $V \oplus V$, thus
  guaranteeing the translation covariance property. The
  operation that produces Albert's twisted fields is a
  symplectic presemifield operation, from which we can use
  Kanat's (or Kantor's) explicit construction for the
  eigenstates.

  Given the unitary inequivalence of the MUBs constructed
  from the Desarguesian spread and from Albert's spread, we
  had hoped to construct Wigner functions which were
  inequivalent in terms of unitary transformations.
  Suprisingly, numerical comparisons immediately showed no
  difference between both Wigner functions for the simplest
  case of three qutrits. This logically motivated us to
  study the equivalence of point operators of each
  construction in order to determine if and how the
  quadrature system managed to nullify the differences
  between the inequivalent MUBs. Using known results for
  certain Weil sums, we try to prove the equivalence of the
  point operators, and consequently, the equivalence of the
  Wigner functions when using Albert's twisted fields.

  \section{Albert's twisted fields}


  %The usual way of defining the twisted fileds is to take a
  %finite field $E$ of odd order, and a non-trivial
  %automorphism $\rho$ that $-1 \neq E^{\rho-1}$. Kantor
  %notes that this $\rho$ behaves in this manner only when
  %$E$ has odd digree over the fixed field $E_\rho$ of
  %$\rho$.

  %Now consider $F = \GF(q)$ as a subfield of $E_\rho$ and
  %set $n = [E : F]$, i.e., the degree of the extension $E$ 
  %over $F$. Then the subspaces 
  %\begin{equation}
  %  \{(0,y) : y \in E\},
  %  \quad
  %  \{(x,mx^{\rho^{-1}} + m^{\rho}x^{\rho} : x \in E\},
  %  \quad m \in E
  %\end{equation}
  %of the $F$-space $E \oplus E$ form a spread $\Sigma$ of
  %the corresponding projective space $PG(2n-1,q)$. The
  %spread is symplectic with respect to the bilinear
  %alternating form given by
  %\begin{equation}
  %  \langle (x_1,y_1), (x_2,y_2) \rangle
  %  = \tr(x_1y_2 - x_2y_1).
  %\end{equation}
  %\begin{proof}
  %  To prove this we must show that every subspace of the
  %  spread is totally isotropic.
  %  \begin{align}
  %    \langle (x_1,y_1), (x_2,y_2) \rangle
  %    &= \tr\left[
  %      x \left( my^{\rho^{-1}}+m^{\rho}y^{\rho} \right)
  %      - \left( mx^{\rho^{-1}}+m^{\rho}x^{\rho} \right) y
  %    \right] \\
  %    &= \tr\left(xmy^{\rho^{-1}}\right)
  %    + \tr\left( xm^{\rho}y^{\rho} \right) 
  %    - \tr\left( ymx^{\rho^{-1}} \right) 
  %    - \tr\left( ym^{\rho}x^{\rho} \right) \\
  %    &= \tr\left( x^{\rho}m^{\rho}y \right)
  %    + \tr\left( xm^{\rho}y^{\rho} \right) 
  %    - \tr\left( y^{\rho}m^{\rho}x \right) 
  %    - \tr\left( ym^{\rho}x^{\rho} \right) \\
  %    &= 0.
  %  \end{align}
  %\end{proof}
  %This spread arises from a semifield. The presemifield
  %operation is given by
  %\begin{equation}
  %  m * x = mx^{\rho^{-1}} + m^{\rho}x^{\rho},
  %  \quad
  %  m, x \in E,
  %\end{equation}
  %and is isotopic to the presemifield defined by $m \circ x
  %= mx^{\rho^{-1}}+m^{\rho}x^{\rho^2}$, which produces
  %Albert's twisted fields.

  In \cite{bader}, Kantor presents a family of spreads based
  on Albert's twisted fields.  Consider the field $F =
  \GF(p^{n})$ with $n$ odd, choose an integer $s$ relatively
  prime to $n$ such that $1 \leq s < n / 2$, and let $\tr(x)
  := \sum_{j=0}^{n-1} x^{p^{j}}$ for $x \in F$, be the trace
  map. Then the operation defined as
  \begin{equation}
    x * m := mx^{p^{n-s}} + m^{p^{s}} x^{p^{s}},
  \end{equation}
  defines a presemifield $(K,+,*)$. This presemifield
  produces a symplectic spread of $F \oplus F$ given by the
  $p^{n}+1$ subspaces:
  \begin{equation}
    \{(0,y) : y \in K\}
    \quad \text{and} \quad
    \{(x,x * m) : x \in K\},
    \quad \text{for all } m \in K,
  \end{equation}
  with respect to the alternating bilinear form
  \begin{equation}
    \langle (x_1,y_1), (x_2,y_2) \rangle
    := \tr\left( x_1y_2 - y_1x_2 \right).
  \end{equation}
  \begin{proof}
    By definition each subspace is totally isotropic if the
    bilinear form vanishes for all elements of the subspace.
    Let $n$ and $s$ be suitable parameters. Since
    $\tr\left(x^{p^{s}}\right) = \tr(x)$ and $x^{p^{n}} = x$
    for all $x \in K$, it follows that
    \begin{align}
      \langle (x, &x * m), (y, y * m) \rangle \\
      &= \tr\left( x(m*y) - (m*x)y \right) \\
      &= \tr\left[x\left(
          my^{p^{n-s}} + m^{p^{s}}y^{p^{s}}
        \right) 
        - \left(
          mx^{p^{n-s}} + m^{p^{s}}x^{p^{s}}
        \right) y
      \right] \\
      %&= \tr\left(
      %  xmy^{p^{n-s}} + xm^{p^{s}}y^{p^{s}} - ymx^{p^{n-s}}
      %  - ym^{p^{s}}x^{p^{s}}
      %\right) \\
      &= \tr\left( xmy^{p^{n-s}} \right)
      + \tr\left( xm^{p^{s}}y^{p^{s}} \right) 
      - \tr\left( ymx^{p^{n-s}}  \right)
      - \tr\left( ym^{p^{s}}x^{p^{s}} \right) \\
      &= \tr\left( x^{p^{s}}m^{p^{s}}y^{p^{n-s+s}} \right) 
      + \tr\left( xm^{p^{s}}y^{p^{s}} \right) 
      - \tr\left( y^{p^{s}}m^{p^{s}}x^{p^{n-s+s}} \right) 
      - \tr\left( ym^{p^{s}}x^{p^{s}} \right) \\
      &= \tr\left( x^{p^{s}}m^{p^{s}}y^{p^{n}} \right) 
      + \tr\left( xm^{p^{s}}y^{p^{s}} \right) 
      - \tr\left( y^{p^{s}}m^{p^{s}}x^{p^{n}} \right) 
      - \tr\left( ym^{p^{s}}x^{p^{s}} \right) \\
      &= \tr\left( x^{p^{s}}m^{p^{s}}y \right) +
      \tr\left( xm^{p^{s}}y^{p^{s}} \right) 
      - \tr\left( y^{p^{s}}m^{p^{s}}x \right) 
      - \tr\left( ym^{p^{s}}x^{p^{s}} \right) \\
      &= 0.
    \end{align}
  \end{proof}
  The proof of inequivalence with respect to the
  Desarguesian spread can be found in \cite{bader}. By
  appropriately defining the generalized Pauli operators
  of the Hilbert space $\mathcal H = \C^{d}$ using the
  finite field $V = \GF(d)$ where $d = p^{n}$, we can use
  the alternating bilinear form to identify maximal
  commuting subsets of such operators. The Pauli operators
  are each diagonalizable and so commuting Pauli operators
  are \textit{simultaneously diagonalizable}. The unique
  eigenbasis they define is mutually unbiased with respect
  to the eigenbasis of another subset of commuting operators
  (c.f. to Bandyophyay's proof \cite{bandyopadhyay2001}). In
  this manner we can identify sets of MUBs with the
  subspaces of the symplectic spread. By assigning an
  eigenbasis element to each ray  of the corresponding
  affine plane, we can cover the \textit{phase space}
  $\Gamma = V \oplus V$ by lines which only intersect at the
  origin point $(0,0) \in \Gamma$. Parallel classes of lines
  are then assigned the rest of the elements of the
  corresponding eigenbasis. This is enough to construct the
  covariant point operators which are then used to define
  the discrete Wigner function, following Wootters and
  Gibbons \cite{gibbons2004}.

  \section{Wigner point operators}

  \subsection{The discrete Wigner function}

  Wootters and Gibbons define a discrete Wigner function for
  the finite phase space $\Gamma = V \oplus V$ where $V$ is
  a finite field of order $d = p^{n}$ for some prime $p$.
  Their method is based on the \textit{tomographic} property
  of the continuous Wigner function, in which a complete set
  of Radon transforms, which can be seen as generalized
  marginal densities, is enough to recover the Wigner
  function of some quantum state $\rho$  by inversion
  \cite{gibbons2004}. The marginal densities correspond to
  straight lines in the phase space $\Gamma = \R^2$, the
  analogy of course is to identify some sort of
  probabilistic mass function with a ``straight'' line of
  the discrete phase space. The measurements used to produce
  these mass functions are given by the sets of MUBs.

  By using the properties of the finite geometry of
  $\Gamma$, Wootters and Gibbons obtain an expression for
  the Wigner function in terms of point operators akin to
  the point kernels of the continous Wigner function. A
  point operator $A(\alpha)$ is constructed by summing the
  projection operators of the eigenstates assigned to all
  the lines that intersect at the point $\alpha$:
  \begin{equation}
    A(\alpha)
    = \sum_{\lambda \ni \alpha}^{} Q(\lambda) - I,
  \end{equation}
  where $Q(\lambda)$ is the projection corresponding to the
  line $\lambda \subset \Gamma$. The map $Q : \mathcal
  P(\Gamma) \to \mathcal L(\mathcal H)$ is called the
  \textit{quadrature system} or \textit{quantum net}. By
  construction $Q$ is covariant with respect to phase space
  translations, this is because the union of the subspaces
  generated by the MUBs is invariant under the action of
  Pauli operators \cite{kantor2012}.

  The Wigner function of a state $\rho$ at a point $\alpha$,
  is defined as the exptectation value of the point operator
  $A(\alpha)$ in the state $\rho$:
  \begin{equation}
    W_\rho(\alpha)
    = \Tr\left( \rho A(\alpha) \right).
  \end{equation}
  This definition satisfies a discrete version of the
  Stratonovich-Weyl criteria, as any sensible Wigner
  function should.

  By construction the quadrature system $Q$ is covariant
  with respect to phase space translations. This means that
  for an arbitrary translation $T(a,b)$ we have
  \begin{equation}
    D(a,b) Q(\lambda) D(a,b)^{*}
    = Q(T(a,b)\lambda).
  \end{equation}
  Now consider two points $(a,b)$ and $(c,d)$ of $\Gamma$.
  The translation that takes $(a,b)$ to $(c,d)$ is given by
  $T(c-a, d-b)$. So if $(a,b) \in \lambda$ then
  \begin{equation*}
    Q\left( T(c-a, d-b)\lambda \right) 
    = Q(\lambda'),
  \end{equation*}
  where $\lambda'$ contains the point $(c,d)$. By
  definition of the point operators in terms of the
  quadrature system we have that
  \begin{align*}
    D(c-a, d-b) &A(a,b) D(c-a, d-b)* \\
    &= D(c-a, d-b) \left( 
      \sum_{\lambda \ni (a,b)}^{} Q(\lambda) - I
    \right)  D(c-a, d-b)* \\
    &= \sum_{\lambda \ni (a,b)}^{}
    D(c-a,d-b) Q(\lambda) D(c-a,d-b)^{*} - I \\
    &= \sum_{\lambda \ni (a,b)}^{} Q(T(c-a,d-b)\lambda) - I
    \\
    &= \sum_{\lambda' \ni (c,d)}^{} Q(\lambda') - I \\
    &= A(c,d),
  \end{align*}
  where the $D(a,b)$ operators are known as the
  \textit{displacement operators}, which are formed by
  products of the generalized Pauli matrices. In particular
  we have 
  \[
    A(a,b) = D(a,b) A(0,0) D(a,b)^{*},
  \]
  and so in order to analyze the possible unitary
  equivalence of the point operators it is enough to study
  the origin point operator $A(0,0)$, because the
  displacement operators are unitary \cite{gibbons2004}.

  \subsection{Construction of MUBs}

  The task of simultaneously diagonalizing sets of commuting
  Pauli matrices is greatly simplified by explicit
  expressions obtained \textit{directly} from the
  presemifield operations. We use the constructions of Kanat
  \cite{abdukhalikov2015} which are based on those initially
  found by Calderbank, Cameron, Kantor and Seidel
  \cite{calderbank}.

  \begin{theorem}[Kanat (3.3) \cite{abdukhalikov2015}]
    Let $(F,+,\circ)$ be a finite symplectic presemifield of
    odd characteristic and let $\ket{e_w}$ for $w \in F$ be
    the standard basis in $\C^{d}$. Then the following set
    forms a complete set of MUBs:
    \begin{equation}
      B_\infty
      = \{\ket{e_w} : w \in F\},
      \quad
      B_m
      = \{\ket{b_{m,v}} : v \in F\},
      \quad m \in F,
    \end{equation}
    where
    \begin{equation}
      \label{eqn:kanat_mubs}
      \ket{b_{m,v}}
      = \frac{1}{\sqrt{d}} \sum_{w \in F}^{}
      \omega^{\tr\left( \frac{1}{2} w (w \circ m) + vw
      \right) } \ket{e_w},
    \end{equation}
    and $\omega$ is a $p$-th root of unity.
  \end{theorem}
  By iteratively evaluating equation (\ref{eqn:kanat_mubs})
  we obtain an eigenbasis corresponding to each $m \in F$.
  Although we are allowed a choice of which eigenstate we
  assign to the ray of the affine plane corresponding to the
  ``slope'' $m$, we always choose the state $\ket{b_{m,0}}$
  in order to minimize the arbitrariness. The rest of the
  states corresponding to the parallel lines are obtained by
  using the different values of $v \in F$, and these are
  fixed by the covariance property of $Q$.

  Let us now express the point operators in terms of the
  projections $Q(\lambda)$ and the eigenstates obtained
  using Kanat's theorem. For any point $(u,v) \in \Gamma$ we
  can obtain the expressions for the lines that are incident
  to it by solving for $b$ in the equation $v = u \circ m +
  b$ for $m \in F$. To simplify the calculations we will use
  the fact that we can obtain $A(u,v)$ from $A(0,0)$ by
  using the appropiate displacement operators, and so the
  rest of this analysis is made using only the origin
  operator. By definition we have
  \begin{equation}
    \label{eqn:point_op}
    A(u,v)
    = \sum_{\lambda \ni (u,v)}^{} Q(\lambda) - I
    = \ket{e_u}\bra{e_u}
    + \sum_{m \in F}^{}
    \ket{\lambda_{m,b}}
    \bra{\lambda_{m,b}} - I,
  \end{equation}
  where $Q(\lambda) =
  \ket{\lambda_{m,b}}\bra{\lambda_{m,b}}$.  For the
  particular case of $A(0,0)$ we obtain
  \begin{equation}
    \label{eqn:origin_op}
    A(0,0)
    = \ket{e_0}\bra{e_0}
    + \sum_{m \in F}^{} \ket{\lambda_{m,0}}
    \bra{\lambda_{m,0}} - I,
  \end{equation}
  where
  \begin{align}
    \ket{\lambda_{m,b}} \bra{\lambda_{m,b}}
    &= \frac{1}{d} \left( 
      \sum_{k \in F}^{} \omega^{\tr\left(
          \frac{1}{2} k (k \circ m)
      \right)} \ket{e_k}
    \right) \left( 
      \sum_{j \in F}^{} \omega^{-\tr\left(
          \frac{1}{2} j (j \circ m)
      \right)} \bra{e_j}
    \right) \\
    &= \frac{1}{d} \sum_{k, j \in F}^{} 
    \omega^{\tr\left( 
        \frac{1}{2} k (k\circ m) - \frac{1}{2} j (j
        \circ m) 
    \right) } \ket{e_k} \bra{e_j} \\
    &= \frac{1}{d} \sum_{k, j \in F}^{} 
    \omega^{\tr\left( 
        \frac{1}{2} \left( k(k \circ m) - j (j\circ m) \right)
    \right) } \ket{e_k} \bra{e_j}.
  \end{align}

  \subsection{Equivalence of the origin point operator}

  From the expression (\ref{eqn:point_op}) we can see that
  the point operator at $(u,v)$ for two presemifields
  $(F,+,\circ)$ and $(F,+,*)$, is equal if the middle term
  consisting of the sum over $m \in F$ is equal. Since
  $\left(\ket{e_k}\bra{e_j}\right)_{kj} = \delta_{kj}$ with
  respect to the standard basis, equality follows if and
  only if the sums
  \begin{equation}
    \label{eqn:weil_left}
    S_1 
    = \sum_{m \in F}^{} 
    \omega^{\tr\left\{ 
        \frac{1}{2}\left[ k(k\circ m) - j(j\circ m) \right]
    \right\} }
  \end{equation}
  and
  \begin{equation}
    \label{eqn:weil_right}
    S_2
    = \sum_{m \in F}^{} 
    \omega^{\tr\left\{ 
        \frac{1}{2} \left[ k(k * m) - j(j * m)\right]
    \right\} },
  \end{equation}
  are equal for all $k,j \in F$. These types of exponential
  sums are known as \textit{Weil sums} in the theory of
  characters over finite fields. It is usually very
  difficult to find closed form expressions for these types
  of character sums, but known formulas for certain cases do
  exist. We now present two results found in Lidl's book on
  finite fields \cite{lidl1997} that we will use shortly.
  Let $\chi$ be a non-trivial additive character of the
  finite field $F$, and define the character $\chi_c(a) =
  \chi(ca)$ for all $c \in F$.

  \begin{theorem}[Lidle (5.31) \cite{lidl1997}]
    \label{thm:lidl_1}
    If $\chi$ is a nontrivial additive character of a finite
    field $F$ of order $d$ and $\gcd(n,d-1) = 1$, then
    \begin{equation}
      \sum_{c \in F}^{} \chi\left( ac^{n}+b \right) 
      = 0,
    \end{equation}
    for any $a,b \in F$ with $a \neq 0$.
  \end{theorem}

  Suppose the field $F$ of $d$ elements is of characteristic
  $p$. A polynomial $f(x) \in F[x]$ of the form
  \begin{equation}
    \label{eqn:affine_poly}
    f(x)
    = a_r x^{p^{r}} + a_{r-1} x^{p^{r-1}} + \cdots + a_1
    x^{p} + a_0 x + a,
  \end{equation}
  for some $r \in \N$, is known as an affine $p$-polynomial
  over $F$. Just as in the linear case, we can evaluate
  Weils sums with affine $p$-polynomials as arguments.
  \begin{theorem}[Lidle (5.34) \cite{lidl1997}]
    \label{thm:lidl_2}
    Let $f(x) \in F[x]$ be an affine $p$-polynomial of the
    form (\ref{eqn:affine_poly}) and let $\chi_b$, with $b
    \in F^{*}$, be a nontrivial additive character of $F$.
    Then
    \begin{equation}
      \sum_{c \in F}^{} 
      \chi_b\left( f(c) \right) 
      = \begin{cases}
        \chi_b(a) d
        & \text{if } ba_r + b^{p}a^{p}_{r-1} + \cdots +
        b^{p^{r-1}}a^{p^{r-1}}_1 + b^{p^{r}}a^{p^{r}}_0 = 0,
        \\
        0 & \text{otherwise.}
      \end{cases}
    \end{equation}
  \end{theorem}

  With these theorems on hand, we now we consider the sum
  $S_1$ (\ref{eqn:weil_left}) where the symplectic spread is
  the Desarguesian spread. Recall that the presemifield
  operation that generates this spread is simply the field
  multiplication, $a \circ b = ab$ for $a,b \in F$. The
  argument of the trace is then simplified to
  \begin{equation}
    \frac{1}{2} \left[
      k(k\circ m) - j(j\circ m)
    \right]
    = \frac{1}{2} \left( k^2m - j^2m \right) 
    = \frac{1}{2} \left( k^2-j^2 \right) m.
  \end{equation}
  In this form, and noting that $\chi(c) = \omega^{\tr(c)}$
  is the canonical additive character, we clearly see from
  theorem (\ref{thm:lidl_1}) that for the Desarguesian
  spread the Weil sum (\ref{eqn:weil_left}) will evaluate to
  0 as long as $k^2-j^2 \neq 0$. If this is not the case,
  then the argument of $\chi$ becomes 0 and so the sum
  evaluates to $d$ since $\chi(0) = 1$. And so we have:
  \begin{equation}
    \label{eqn:weil_left_sum}
    S_1
    = \begin{cases}
      d & \text{if } k^2-j^2 = 0 \\
      0 & \text{otherwise,}
    \end{cases}
  \end{equation}
  for all entries indexed by $j, k \in F$. We now evaluate
  the second Weil sum given by (\ref{eqn:weil_right}) using
  Kantor's presemifield version of Albert's twisted fields.
  We recall that the operation is given by $x * m =
  mx^{p^{n-s}} + m^{p^{s}}x^{p^{s}}$ for $s$ relatively
  prime to $n$ and $1 \leq s \leq n / 2$.  In order to use
  theorem (\ref{thm:lidl_2}) we must express the argument of
  the trace as a scalar multiple of an affine
  $p$-polynomial, but for the moment we ignore the $1 / 2$
  factor:
  \begin{align}
    k(k*m)-j(j*m)
    &= k\left( mk^{p^{n-s}}+m^{p^{s}}k^{p^{s}} \right)
    - j\left( mj^{p^{n-s}}+m^{p^{s}}j^{p^{s}} \right) \\
    &= m k^{p^{n-s}+1} + m^{p^{s}} k^{p^{s}+1} 
    - m j^{p^{n-s} + 1} - m^{p^{s}} j^{p^{s}+1}\\
    &= \left( k^{p^{s}+1} - j^{p^{s}+1} \right) m^{p^{s}}
    + \left( k^{p^{n-s}+1} - j^{p^{n-s}+1} \right)
    m.\label{eqn:albert_affine_poly}
  \end{align}
  Observe that the expression (\ref{eqn:albert_affine_poly})
  is an affine $p$-polynomial where
  \[
    a_s = k^{p^{s}+1} - j^{p^{s}+1},
    \quad
    a_0 = k^{p^{n-s}+1} - j^{p^{n-s}+1},
    \quad
    \text{and}
    \quad
    a = 0.
  \] 
  The rest of the coeficients $a_{r-i}$ are null. Since
  $\chi_b(c) = \chi(bc) = \omega^{\tr(bc)}$, it is easy to
  see that $S_2$ (\ref{eqn:weil_right}) has the required
  form to use theorem (\ref{thm:lidl_2}) for $b = 1 / 2$ and
  $a = 0$. And so
  \begin{equation}
    \label{eqn:s2_sum}
    S_2
    = \begin{cases}
      d & \text{if } 2a_s + 2^{p^{s}}a_0^{p^{s}} = 0, \\
      0 & \text{otherwise.}
    \end{cases}
  \end{equation}
  We can simplify the equation in the first case a bit,
  using properties of odd characteristic finite fields:
  \begin{align*}
    2a_s + 2^{p^{s}} a_0^{p^{s}}
    &= 2 \left( k^{p^{s}+1} - j^{p^{s}+1} \right) 
    + 2^{p^{s}} \left( k^{p^{n-s}+1} - j^{p^{n-s}+1}
    \right)^{p^{s}} \\
    &= 2 \left( k^{p^{s}+1} - j^{p^{s}+1} \right)
    + 2^{p^{s}} \left( k^{p^{s}+1} - j^{p^{s}+1} \right) \\
    &= (2 + 2) \left( k^{p^{s}+1} - j^{p^{s}+1} \right) \\
    &= k^{p^{s}+1} - j^{p^{s}+1}. 
  \end{align*}
  since the following is valid for odd prime characteristic
  (Lidl 1.46):
  \begin{align*}
    \left( k^{p^{n-s}+1} - j^{p^{n-s}+1} \right)^{p^{s}}
    &= k^{p^{s}(p^{n-s}+1)} - j^{p^{s}(p^{n-s}+1)} \\
    &= k^{p^{n} + p^{s}} - j^{p^{n}+p^{s}} \\
    &= k^{p^{n}}k^{p^{s}} - j^{p^{n}}j^{p^{s}} \\
    &= k^{p^{s}+1} - j^{p^{s}+1}.
  \end{align*}
  And so equation (\ref{eqn:s2_sum}) becomes 
   \begin{equation}
    \label{eqn:s2_value}
    S_2
    = \begin{cases}
      d & \text{if } k^{p^{s}+1} - j^{p^{s}+1} = 0, \\
      0 & \text{otherwise}.
    \end{cases}
  \end{equation}
  Comparing the values of $S_1$ and $S_2$ we see that the
  point operators at the origin for both symplectic spreads
  will be equal if and only if the equations
  \begin{equation}
    \label{eqn:system}
    k^2 - j^2 = 0
    \quad
    \text{and}
    \quad
    k^{p^{s}+1} - j^{p^{s}+1} = 0
  \end{equation}
  share the \textit{same} set of solutions in $\GF(p^{n})$.
  The form of both equations allows for a somewhat simple
  verification that this indeed happens for the appropriate
  $p, n$ and $s$.

  Ignoring the trivial solution, by the change of variables
  $r = k / j$ the equations in (\ref{eqn:system}) can be
  expressed as
  \begin{equation}
    r^2 = 1
    \quad
    \text{and}
    \quad
    r^{p^{s}+1} = 1.
  \end{equation} 
  These two equations share the same solution set if and
  only if the original equations do too. In odd
  characteristic the equation $r^2 = 1$ has two solutions,
  namely $r = 1$ and $r = -1$, and so all we have to do is
  prove that $r^{p^{s}+1} = 1$ has \textit{exactly} the same
  solutions.  First note that $r = 1$ and $r = -1$ are
  indeed solutions of the equation $r^{p^{s}+1} = 1$, so we
  actually only need to prove that the equation $r^{p^{s}+1}
  = 1$ doesn't have any other solutions in $\GF(p^{n})$. 

  Consider $g$ a generator of the cyclic group $F^{*}$ of
  $p^{n}-1$ elements, which is the multiplicative group of
  the finite field $F$. It is well known [??] in the theory
  of cyclic groups, that the order of the element
  $g^{p^{s}+1}$ is
  \begin{equation}
    \label{eqn:order_phi}
    \left|g^{p^{s}+1}\right|
    = \frac{p^{n}-1}{\gcd\left( p^{n}-1, p^{s}+1 \right) }.
  \end{equation} 
  Now define $\phi : F^{*} \to F^{*}$ to be the map $x
  \mapsto x^{p^{s}+1}$.  The map $\phi$ is obviously a group
  homomorphism and since $g$ is a generator for $F^{*}$, the
  image of $g$ under $\phi$ is a generator for $\im(\phi)$,
  which implies that $|\im(\phi)| =
  \left|g^{p^{s}+1}\right|$. So by the first isomorphism
  theorem and by equation (\ref{eqn:order_phi}) we obtain
  %\[
  %  F^{*} / \ker(\phi) \cong \im(\phi),
  %\] 
  \begin{equation}
    |\ker(\phi)|
    = \frac{|F^{*}|}{|\im(\phi)|}
    = \frac{p^{n}-1}{\left|g^{p^{s}+1}\right|}
    = \gcd(p^{n}-1, p^{s}+1).
  \end{equation}
  So proving that $r^{p^{s}+1}=1$ has at most two solutions
  is equivalent to proving $\gcd(p^{n}-1, p^{s}+1) \leq 2$.
  Now assume $d$ is a divisor of $p^{n}-1$ and $p^{s}+1$.
  Since $d$ divides $p^{s}+1$, it also divides $p^{2s}-1 =
  \left( p^{s}+1\right) \left( p^{s}-1 \right)$. It can be
  proved [??] using the Euclidean algorithm on the exponents
  that 
  \[
    d \mid p^{2s}-1, p^{n}-1
    \implies
    d \mid p^{\gcd(2s, n)} - 1.
  \] 
  But by assumption $n$ is odd and $\gcd(s,n) = 1$, so
  $\gcd(2s, n) = \gcd(s, n) = 1$. Therefore $d \mid p - 1$,
  which means $p \equiv 1 \mod d$. This in turn implies that
  $p^{s} \equiv 1 \mod d$ for any $s$, and so $d \mid p^{s}
  - 1$. It follows from $d \mid p^{s} + 1, p^{s} - 1$ that
  $d \mid 2$. So $\gcd(p^{n}-1, p^{s}+1) \leq 2$ and so
  $|\ker(\phi)| \leq 2$. This concludes the proof. 

  We have proven that the equations in (\ref{eqn:system})
  share the same solution set. As we have mentioned, this
  means that the Weil sums $S_1$ and $S_2$ are equal for all
  $k,j \in F$.  This equality in turn implies that the
  origin operator for the Desarguesian and the whole family
  of Albert's spreads are the same.  Translational
  covariance then implies that \textit{all} of the point
  operators obtained from these MUBs are the same and so
  produce identical discrete Wigner functions, even though
  the sets of MUBs used to construct the functions are not
  equivalent under any unitary transformation.

  \section{Discussion}

  The whole family of inequivalent symplectic spreads
  given by Albert's twisted fields produce unitarily
  inequivalent MUBs in $\C^{p^{n}}$, but as we have shown,
  this is not enough to obtain unitarily inequivalent Wigner
  functions. In fact the discrete Wigner functions turn out
  to be identical for all $p$, $n$ and $s$ satisfying the
  necessary conditions. An obvious next step would be to
  study different odd characteristic symplectic spreads
  which are also inequivalent to the Desarguesian spread,
  and see if something similar happens. The main problem
  with this approach would be the difficulty of finding
  closed form evaluations of the corresponding Weil sums.
  Furthermore, a general argument using arbitrary
  inequivalent symplectic spreads seems unreasonable with
  this approach, as we explicitely require the definition of
  the presemifield operation. It is still interesting to
  investigate the differences between even and odd
  characteristic, as we have found (in another work)
  empirical evidence that this unitary equivalence of
  discrete Wigner functions using unitary inequivelent MUBs
  does not always occur for even characteristic.

  \printbibliography

  % stackexchange references
  % https://math.stackexchange.com/questions/32737/prove-gcdab-a-b-1-or-2-if-gcda-b-1
  % https://math.stackexchange.com/questions/875940/show-m-odd-rightarrow-gcdam-1-an1-1-or-2
  % https://math.stackexchange.com/questions/2588518/find-gcdpn-1-pm1
  % https://proofwiki.org/wiki/Homomorphic_Image_of_Cyclic_Group_is_Cyclic_Group
  % https://math.stackexchange.com/questions/1906718/how-to-prove-ak-n-gcdn-k-whenever-a-n
  % https://proofwiki.org/wiki/Congruence_of_Powers
\end{document}

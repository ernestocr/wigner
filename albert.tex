\documentclass[a4paper, 11pt]{article}
\usepackage[margin=1.3in]{geometry}
\usepackage[utf8]{inputenc}
\usepackage[T1]{fontenc}
\usepackage{textcomp}
\usepackage[spanish]{babel}
\usepackage{amsmath, amssymb}
\usepackage{amsthm}
\usepackage{braket}
\decimalpoint

\DeclareMathOperator{\R}{\mathbb{R}}
\DeclareMathOperator{\C}{\mathbb{C}}
\DeclareMathOperator{\N}{\mathbb{N}}
\DeclareMathOperator{\Z}{\mathbb{Z}}
\DeclareMathOperator{\F}{\mathbb{F}}
\DeclareMathOperator{\GF}{GF}
\DeclareMathOperator{\GR}{GR}
\DeclareMathOperator{\tr}{tr}

\newtheorem{definition}{Definición}
\newtheorem{theorem}{Teorema}
\newtheorem{proposition}{Proposición}
\newtheorem{lemma}{Lema}
\newtheorem{corollary}{Corolario}
\newtheorem{example}{Example}

\title{Wigner point operators from Albert's twisted fields}
\author{Ernesto Camacho Ramírez}

\begin{document}

  \maketitle

  \section{Introduction}

  One of the simplest examples of inequivalent symplectic
  spreads for odd characteristic, with respect to the
  Desarguesian spread, in terms of lowest Hilbert space
  dimension, is given by a particular case of Albert's
  twisted fields.  The lowest dimension $\C^{p^{n}}$ space
  for which we can apply the construction is of dimension $d
  = 3^{3}$. These families of symplectic spreads are
  actually unbounded in the number of pairwise inequivalent
  examples as a function of $n$.

  We had previously defined the discrete Wigner function
  $W_\rho$ of some state $\rho$ in terms of a set of
  covariant, Hilbert-Schmidt orthogonal \textit{point
  operators} $A(\alpha)$ for $\alpha \in V \oplus V$. These
  point operators are defined by summing the mutually
  unbiased projection operators designated by some
  \textit{quantum net} or \textit{quadrature system}, for
  each line in phase space that passes through some point
  $\alpha$. The quadrature system assigns eigenstates of
  maximal commuting classes of generalized Pauli operators
  to lines in phase space in a manner given by some
  symplectic spread of the space $V \oplus V$, thus
  guaranteeing the translation covariance property. The
  operation that produces Albert's twisted fields is a
  symplectic presemifield operation, from which we can use
  Kanat's (or Kantor's) explicit construction for the
  eigenstates.

  Given the unitary inequivalence of the MUBs constructed
  from the Desarguesian spread and from Albert's spread, we
  had hoped to construct Wigner functions which were
  inequivalent in terms of unitary transformations.
  Suprisingly, numerical comparisons immediately showed no
  difference between both Wigner functions for the simplest
  case of three qutrits. This logically motivated us to
  study the equivalence of point operators of each
  construction in order to determine if and how the
  quadrature system managed to nullify the differences
  between the inequivalent MUBs. Using known results for
  certain Weil sums, we were indeed able to prove the
  equivalence of the point operators, and consequently, the
  equivalence of the Wigner functions when using Albert's
  twisted fields.

  \section{Albert's twisted fields}

  Let $V$ be $K = \GF(p^{n})$ with $n$ odd, choose an
  integer $s$ relatively prime to $n$ such that $1 \leq s <
  n / 2$, and let $\tr(x) := \sum_{j=0}^{n-1} x^{p^{j}}$ for
  $x \in K$, be the trace map. Then the operation defined as
  \begin{equation}
    b * x := bx^{p^{n-s}} + b^{p^{s}} x^{p^{s}+1},
  \end{equation}
  defines a symplectic spread of $V \oplus V$.

  \section{Wigner point operators}

  \section{Equivalence of standard and non standard
  construction}

  Una comparación numérica de las funciones de Wigner nos
  dirigió hacia la igualdad de los operadores puntuales para
  ambas coberturas simplécticas. Utilizando algunos
  resultados sobre sumas de caracteres de campos finitos
  logramos demostrar que, \textit{utilizando la cobertura de
  Albert obtenemos el mismo operador puntual del origen que
  se obtiene en la construcción estándar.}
  Un argumento similar puede ser utilizado para demostrar la
  igualdad del resto de los operadores. Ésto tiene
  implicaciones obvias a la hora de estudiar la equivalencia
  entre ambas construcciones ya que la función de Wigner
  queda totalmente determinada por los operadores puntuales.
  Para simplificar la demostración solo considaremos el
  operador puntual en el origen $A(0,0)$. De la definición
  tenemos que
  \begin{align}
    A(0,0)
    &= \sum_{\lambda \ni (0,0)}^{} Q(\lambda) - I \\
    &= \sum_{m \in \F \cup \{\infty\}}^{} \ket{\lambda_{m,0}}
    \bra{\lambda_{m,0}} - I \\
    &= \ket{e_0}\bra{e_0}
    + \sum_{m \in \F}^{}
    \ket{\lambda_{m,0}}\bra{\lambda_{m,0}} - I,
  \end{align}
  donde $\lambda_{m,0}$ corresponde al rayo con pendiente
  $m$. El caso $m = \infty$ corresponde con la estría
  vertical, y en particular $\ket{\lambda_{\infty,0}}$ es el
  vector $\ket{e_0}$ de la base estándar. Utilizando la
  expresión (\ref{eqn:kanat_presemi_mubs}) tenemos que
  \begin{equation}
    \ket{\lambda_{m,0}}
    = \frac{1}{\sqrt{d}} \sum_{w \in \F}^{}
    \omega^{\tr\left( \frac{1}{2} w(w\circ m) \right) }
    \ket{e_w},
  \end{equation}
  donde $\circ$ es la operación del presemicampo en
  cuestión. Notemos que la equivalencia del operador puntual
  se da solo si la suma de las proyecciones respecto a la
  pendiente $m$ es igual, ya que el término
  $\ket{e_0}\bra{e_0}$ y el operador de identidad $I$ se
  comparten en ambas sumas.  Para la cobertura Desarguesiana
  la operación del presemicampo es $w \circ m = wm$ y para
  la cobertura de Albert tenemos que $w * m = mw^{9} +
  m^3 w^3$.  Utilizando la definición de
  $\ket{\lambda_{m,0}}$, cada proyección se puede expresar
  como:
  \begin{align}
    \ket{\lambda_{m,0}} \bra{\lambda_{m,0}}
    &= \frac{1}{d} 
    \left(
      \sum_{w \in \F}^{} \omega^{\tr\left( \frac{1}{2}
      w(w\circ m) \right) }
      \ket{e_w}
    \right) 
    \left(
      \sum_{u \in \F}^{} \omega^{-\tr\left( \frac{1}{2}
      u(u \circ m) \right) }
      \bra{e_u}
    \right) \\
    &= \frac{1}{d}
    \sum_{w \in \F}^{} \sum_{u \in \F}^{} 
    \omega^{\tr( \frac{1}{2} w(w\circ m)) - \tr( \frac{1}{2}
    u(u \circ m))} 
    \ket{e_w} \bra{e_u}.
  \end{align}
  Enseguida expresamos el argumento de la traza en términos
  de la definición de las operaciones del presemicampo.
  Primero, para la cobertura Desarguesiana obtenemos la
  expresión:
  \begin{align}
    \tr\left(\frac{1}{2} w(w\circ m)\right)
    - \tr\left(\frac{1}{2} u(u\circ m)\right)
    %&= \tr\left(\frac{1}{2} m w^2\right)
    %- \tr\left(\frac{1}{2} m u^2\right) \\
    &= \tr\left(\frac{1}{2} m \left( w^2-u^2 \right)\right).
  \end{align}
  Similarmente para la cobertura de Albert obtenemos:
  \begin{align}
    \tr\left(\frac{1}{2} w(w\circ m)\right)
    - \tr\left(\frac{1}{2} u(u\circ m)\right)
    %&= \tr\left(\frac{1}{2} w (mw^{9}+m^3w^3)\right)
    %- \tr\left(\frac{1}{2} u (mu^{9}+m^3u^3)\right) \\
    %&= \tr\left( \frac{1}{2} ( 
    %  m w^{10} + m^3 w^{4} - mu^{10} - m^3u^{4}
    %) \right) \\
    &= \tr\left( 
      \frac{1}{2} m \left( w^{10} - u^{10}  
      + m^2 \left( w^{4} - u^{4}\right) \right)
    \right). 
  \end{align}
  Entonces la igualdad de los operadores puntuales es
  equivalente a la igualdad de la siguiente triple suma:
  \begin{equation}
    \sum_{m,w,u \in \F}^{}
    \omega^{\tr(\frac{1}{2} m(w^2-u^2))}
    \ket{e_w}\bra{e_u}
    = 
    \sum_{m,w,u \in \F}^{}
    \omega^{\tr\left(\frac{1}{2} m
    \left(w^{10}-u^{10} +  m^2(w^{4}-u^{4})\right) \right)
    }
    \ket{e_w}\bra{e_u}.
  \end{equation}
  Las proyecciones $\ket{e_w}\bra{e_u}$ en la base
  estándar se representan por matrices con todas sus
  entradas iguales a cero excepto en la entrada indexada por
  los elementos $w$ y $u$. Por lo tanto la igualdad de los
  operadores puntuales en el origen es equivalente a la
  siguiente igualdad de sumas sobre una sola variable:
  \begin{equation}
    \label{eqn:qutrit_weil_sum}
    \sum_{m \in \F}^{}
    \omega^{\tr\left( 
        \frac{1}{2} m \left( w^2 - u^2 \right) 
    \right) }
    =
    \sum_{m \in \F}^{} 
    \omega^{\tr\left( 
        \frac{1}{2} m\left( w^{10}-u^{10} + m^2(w^{4}-u^{4})
        \right) 
    \right) },
  \end{equation}
  para todo par $u,w \in \F$. A éste tipo de sumas se les
  conoce como sumas de Weil, generalmente se expresan en
  términos de los caracteres $\chi$ del grupo aditivo:
  \begin{equation}
    \sum_{c \in \F}^{} \chi(f(x)),
  \end{equation}
  donde $f \in \F[x]$, (ver la sección de \textit{sumas
  exponenciales} (\ref{subsec:exp_sums}) del apéndice para
  un resumen rápido de los caracteres aditivos de un campo
  finito). Las sumas de Weil generalmente no son fáciles de
  evaluar \cite{lidl1994}.  Usualemente solo podemos obtener
  estimaciones de las cotas superiores e inferiores del
  valor absoluto de la suma. En éste caso particular podemos
  utilizar algunos resultados conocidos para demostrar que
  la igualdad (\ref{eqn:qutrit_weil_sum}) sí se cumple.

  Primero analizamos el lado izquierdo de la ecuación
  (\ref{eqn:qutrit_weil_sum}). Expresamos en términos del
  caracter aditivo no trivial $\chi$:
  \begin{equation}
    \sum_{m \in \F}^{} \omega^{\tr\left( \frac{1}{2} \left(
    w^2-u^2\right) m \right) }
    = \sum_{m \in \F}^{} \chi\left( \frac{1}{2} \left(
    w^2-u^2 \right) m \right).
  \end{equation}
  Fijando a $w, u \in \F$, observamos que el argumento
  $\chi$ es lineal siempre y cuando $w^2-u^2 \neq 0$. Si
  $w^2-u^2 = 0$ entonces
  \begin{equation}
    \chi\left( \frac{1}{2} \left(w^2-u^2\right) m \right) 
    = \chi(0)
    = 1.
  \end{equation}
  Por lo tanto la suma de Weil es igual a la cardinalidad de
  la extensión de Galois, $d = |\F| = 27$. Si $w^2-u^2 \neq
  0$, de la linealidad del argumento obtenemos que la suma
  de Weil es igual a 0, por el teorema
  (\ref{thm:lidl_linear_sum}):
  \begin{equation}
    \sum_{m \in \F}^{} \chi(m) = 0.
  \end{equation}
  Se sigue que la suma de Weil tiene dos posibles valores:
  \begin{equation}
    \label{eqn:left_weil_sum}
    \sum_{m \in \F}^{} \omega^{\tr\left(
    \frac{1}{2}(w^2-u^2)m \right) }
    = 
    \begin{cases}
      27 & \text{cuando } w^2-u^2 = 0, \\
      0 & \text{en caso contrario}.
    \end{cases}
  \end{equation}

  Ahora pasemos al lado derecho de la ecuación
  (\ref{eqn:qutrit_weil_sum}). Reescribiendo el argumento de
  la traza y expresando con la notación del caracter aditivo
  obtenemos:
  \begin{equation}
    \sum_{m \in \F}^{} \omega^{\tr\left( \frac{1}{2} m\left(
    w^{10}-u^{10} + m^2(w^{4}-u^{4})\right)  \right) }
    = 
    \sum_{m \in F}^{} \chi\left(
      \frac{1}{2}
      \left( w^{4} - u^{4} \right) m^3
      + \frac{1}{2}
      \left( w^{10} - u^{10} \right) m 
    \right).
  \end{equation}
  El argumento de $\chi$ (sin el factor $1 / 2$) es un
  $p$-polinomio afín (ver definición \ref{eqn:affine_poly})
  de la forma
  \begin{equation}
    f(x) = a_1 x^{p} + a_0 x + a,
  \end{equation}
  donde $p = 3$, $a_1 = w^{4} - u^{4}$,  $a_0 = w^{10} -
  u^{10}$, $r = 1$ y $a = 0$. Resulta que podemos evaluar
  directamente a éste tipo de suma de Weil utilizando el
  teorema \ref{thm:lidl_weil_poly}. Sean $b = \frac{1}{2}$ y
  $d = 27$, entonces
  \begin{equation}
    \sum_{m \in \F}^{} \chi\left(
      b \left(
        a_1 m^{3} + a_0 m
      \right)
    \right)
    = 
    \begin{cases}
      \chi_{b}(a) d & \text{cuando } b a_1 + b^{3} a_0^{3} =
      0 \\
      0 & \text{en otro caso}.
    \end{cases}
  \end{equation}
  Primero notemos que $a = 0$, por lo que $\chi_b(0) = 1$.
  Además el inverso multiplicativo del dos en característica
  tres es el dos, i.e., $1 / 2 = 2$, por lo tanto la suma de
  Weil es igual a
  \begin{equation}
    \label{eqn:right_weil_sum}
    \sum_{m \in \F}^{} \omega^{\tr\left( \frac{1}{2} m\left(
    w^{10}-u^{10} + m^2(w^{4}-u^{4})\right)  \right) }
    =
    \begin{cases}
      27 & \text{para } 2\left( w^{4}-u^{4} \right) +
      2^{3} \left( w^{10} - u^{10} \right)^{3} = 0, \\
      0 & \text{en otro caso}.
    \end{cases}
  \end{equation}
  Podemos simplificar la ecuación que aparece en el primer
  caso de la suma (\ref{eqn:right_weil_sum}), ya que en
  característica tres se satisface la identidad $(a+b)^{p} =
  a^{p} + b^{p}$:
  \begin{align}
    2\left( w^{4}-u^{4} \right) + 2^{3} \left( w^{10} -
    u^{10} \right) ^{3}
    &= 2 \left(w^{4} - u^{4}\right) + 2^{3} \left( w^{30}
    - u^{30} \right) \\
    &= 2\left(w^{4} - u^{4}\right) + 2\left(w^{26 + 4} -
      u^{26 + 4}\right) \\
    &= 2\left(w^{4} - u^{4}\right) + 2\left(w^{4} -
    u^{4}\right) \\
    &= w^{4} - u^{4}.
  \end{align}
  ya que $a^{26} = 1$ para todo $a \in \F$ y $2^{3} = 2$.
  De las ecuaciones (\ref{eqn:left_weil_sum}) y
  (\ref{eqn:right_weil_sum}) podemos concluír que ambas
  sumas serán iguales para todo $w, u \in \F$ siempre y
  cuando las ecuaciones
  \begin{equation}
    w^2 - u^2 = 0,
    \quad
    \text{ y }
    \quad
    w^{4} - u^{4} = 0,
  \end{equation}
  compartan el \textit{mismo} conjunto de soluciones. Dado
  que estamos trabajando con campos finitos, simplemente
  hemos buscado todas las soluciones de ambas ecuaciones
  para poder compararlas. Para ésto utilizamos el sistema de
  álgebra computacional de \texttt{SageMath} y en particular
  su módulo de campos finitos para encontrarlas.
  \textit{Efectivamente comparten el mismo conjunto de
  soluciones!\footnote{La cardinalidad de éste conjunto de
  soluciones es de 53 elementos $(w,u)$.}} Ésto significa
  que para todo $w, u \in \F$, ambas sumas de la ecuación
  (\ref{eqn:qutrit_weil_sum}) nos dan el mismo valor, ya sea
  $0$ ó $27$. Ésto a su vez significa que los operadores
  puntuales en el origen para ambas coberturas simplécticas
  son iguales. Un cálculo computacional similar nos muestra
  que el resto de los operadores puntuales también lo son.
  Como consecuencia, obtenemos la misma función de Wigner
  para la construcción estándar y no estándar en el caso de
  los tres qutrits, respecto a la cobertura de Albert. Éste
  ejemplo nos indica que la no equivalencia unitaria de las
  MUBs \textit{no} es suficiente para garantizar la
  desigualdad de las funciones de Wigner estándar y no
  estándar.
  
\end{document}

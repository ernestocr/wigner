\documentclass{beamer}
\usetheme{Boadilla}

\usepackage[spanish]{babel}
\decimalpoint

\usepackage{amsmath, amssymb}
\usepackage{amsthm}
\usepackage{bm}
\usepackage{braket}

\newtheorem{axiom}{Axioma}

\DeclareMathOperator{\R}{\mathbb{R}}
\DeclareMathOperator{\C}{\mathbb{C}}
\DeclareMathOperator{\N}{\mathbb{N}}
\DeclareMathOperator{\Z}{\mathbb{Z}}

\let\H\relax
\DeclareMathOperator{\H}{\mathcal H}
\DeclareMathOperator{\Sz}{\mathcal S}

\DeclareMathOperator{\dom}{Dom}
\DeclareMathOperator{\prob}{Prob}
\DeclareMathOperator{\id}{id}
\DeclareMathOperator{\Tr}{Tr}
\DeclareMathOperator{\Op}{Op}
\DeclareMathOperator{\W}{W}
\DeclareMathOperator{\F}{F}

\title{La transformación de Wigner-Weyl}
\author{Ernesto}
\institute{U de G}

\begin{document}

\frame{\titlepage}

\frame{\tableofcontents}

\section{Mecánica clásica}

\begin{frame}
  \frametitle{Mecánica clásica vs mecánica cuántica}

  \begin{itemize}
    \item La mecánica cuántica surge de la incapacidad de
      describir varios fenómenos físicos utilizando la
      mecánica clásica.
    \item La mecánica cuántica es una teoría probabilística,
      en el sentido de que no nos permite predecir con
      exactitud los valores de los observables y la
      evolución temporal de un sistema arbitrario.
  \end{itemize}
\end{frame}

\begin{frame}
  \frametitle{Mecánica clásica}

  \begin{itemize}
    \item La mecánica clásica estudia partículas y sus
      trayectorias, las cuales se rigen por la las leyes de
      Newton.
    \item El sistema físico básico que estudiamos es la
      \textit{posición} $x$ de una partícula en un espacio
      euclideano $\R^{n}$ y su \textit{momentum} $p = m
      \dot{x}$. 
    \item Nos interesa medir ciertas cantidades del sistema,
      los cuales se llaman \textit{observables}, por ejemplo
      la energía o el momentum de la partícula.
      Matemáticamente, los observables son funciones de las
      variables $x$ y $p$.
    \item El observable correspondiente a la energía total
      de un sistema se conoce como el \textit{Hamiltoniano}:
      \[
        H(x,p)
        = \frac{1}{2m} \sum_{n=1}^{n} p_j^2 + V(x).
      \] 
  \end{itemize}
\end{frame}

\begin{frame}
  \frametitle{Ecuaciones de Hamilton}

  Utilizando el Hamiltoniano de un sistema, podemos expresar
  a la ley de movimiento de Newton mediante el siguiente
  sistema de ecuaciones diferenciales:
  \begin{equation}
    \frac{dx_j}{dt} = \frac{\partial H}{\partial p_j},
    \quad
    \frac{dp_j}{dt} = -\frac{\partial H}{\partial x_j}.
  \end{equation}

  Las soluciones $(x,p)$ de las ecuaciones de Hamilton se
  conoce como las trayectorias y nos describe la evolución
  temporal del sistema.
\end{frame}

\subsection{El espacio de fase}

\begin{frame}
  \frametitle{El espacio de fase}

  Las trayectorias son curvas en el espacio $\R^{2n}$ y la
  herramienta básica para estudiar la dinámica de un sistema
  físico es el espacio de fase.

  \begin{definition}
    El espacio de fase de una partícula que se mueve en
    $\R^{n}$ es el espacio $\R^{2n}$, donde sus elementos
    son las tuplas $(x,p)$ tales que $x, p \in \R^{n}$.
  \end{definition}
\end{frame}

\section{Mecánica cuántica}

\begin{frame}
  \frametitle{La mecánica cuántica}

  Existen diversas formulaciones matemáticas equivalentes de
  la mecánica cuántica. 
  \begin{itemize}
    \item Originales
      \begin{itemize}
        \item Formulación de Schrödinger
        \item Formulación de Heisenberg
      \end{itemize}
    \item Formulación en el espacio de Hilbert (Dirac, von
      Neumann).
    \item Formulación de la integral de trayectoria
      (Feynman)
    \item Formulación en el espacio de fase (Wigner, Weyl,
      Moyal, Groenwold).
  \end{itemize}
\end{frame}

\subsection{Formulación en el espacio de Hilbert}

\begin{frame}
  \frametitle{Formulación en el espacio de Hilbert}

  \begin{axiom}
    A cada sistema cúantico le corresponde un espacio de
    Hilbert $\H$. Los estados del sistema son los operadores
    lineales $\rho$, definidos-positivos y de traza finita,
    tales que $\Tr(\rho) = 1$.
  \end{axiom}

  Para estados \textit{puros}, podemos asociar un elemento
  $\psi$ del espacio de Hilbert al estado, y generalmente se
  considera a $\psi$ como el estado.
\end{frame}

\begin{frame}
  \frametitle{Formulación en el espacio de Hilbert}

  \begin{axiom}
    A cada observable $A$ clásico, le corresponde un
    observable cuántico representado por un operador
    auto-adjunto $\hat{A} : \mathcal D_{\hat{A}} \subset \H
    \to \H$.
  \end{axiom}
\end{frame}

\begin{frame}
  \frametitle{Formulación en el espacio de Hilbert}

  \begin{axiom}
    Si un sistema cuántico está en un estado $\psi \in \H$,
    la distribución de probabilidad de las mediciones de un
    observable $A$ satisface
    \[
      \mathbb E(A)
      = \langle \psi, \hat{A}\psi \rangle.
    \] 
  \end{axiom}

  De manera general, el valor esperado de $A$ de un sistema
  en un estado $\rho$ se puede calcular como
  \[
    \mathbb E(A)
    = \Tr\left( \rho \hat{A} \right)
    = \Tr\left( \hat{A} \rho \right).
  \] 
\end{frame}

\begin{frame}
  \frametitle{Espacio de Hilbert para una partícula}

  Para una partícula que se mueve en un espacio $\R^{n}$,
  generalmente se toma como el espacio de Hilbert 
  \[
    \H = L^2(\R^{n}).
  \] 
  Los estados puros de éste sistema son las
  \textit{funciones de onda}.
\end{frame}

\begin{frame}
  \frametitle{Operador de posición}

  \begin{definition}
    El operador correspondiente a la posición de la
    partícula $\hat{x}$, se define como
    \[
      (\hat{x}\psi)(x)
      = x\psi(x).
    \] 
    El dominio adecuado de $\hat{x}$ es un subespacio denso
    en $\mathcal D_{\hat{x}} \subset L^2(\R^{n})$.
  \end{definition}
\end{frame}

\begin{frame}
  \frametitle{Operador de momentum}

  \begin{definition}
    El operador correspondiente al momentum de la partícula
    $\hat{p}$, se define como
    \[
      (\hat{p}\psi)(x)
      = -i\hbar \frac{d\psi(x)}{dx}.
    \] 
    Su dominio también es un subespacio de $L^2(\R^{n})$.
  \end{definition}
\end{frame}

\begin{frame}
  \frametitle{Posición y momentum}

  \begin{itemize}
    \item Los operadores $\hat{x}$ y $\hat{p}$ son
      operadores auto-adjuntos no acotados y no son
      conmutativos:
      \[
        \hat{x}\hat{p} - \hat{p}\hat{x} = i\hbar.
      \] 
    \item Generalemente se trabaja con estados puros en su
      \textit{representación de posición o de momentum}:
      \[
        \psi(x) = \langle x, \psi \rangle.
      \] 
      Los $x$ son los ``eigenestados generalizados'' del
      operador de posición.
    \item La representación de momentum se pueden obtener
      mediante la transformación de Fourier de la función de
      onda $\psi(x)$.
  \end{itemize}
\end{frame}

\section{Motivación}

\begin{frame}
  \frametitle{Estadística en el espacio de fase} Dada una
  densidad de probabilidad de Liouville $F(x,p)$, y un
  observable $a : \R^{2n} \to \R$ que depende de la posición
  y momentum, se puede calcular el valor esperado de $a$
  sobre un conjunto de partículas como
  \begin{equation}
    \mathbb E[a] = \iint a(x,p) F(x,p) \, dx \, dp.
  \end{equation}
  \textbf{Idea}. Nos gustaría poder
    calcular valores esperados de operadores que dependen de
    $\hat{x}$ y $\hat{p}$ de una manera análoga.
\end{frame}

\begin{frame}
  \frametitle{Estadística en el
    espacio de fase} En otras palabras, nos gustaría poder
    hacer algo como
  \begin{equation}
    \Tr\left( \hat{\rho}\hat{A} \right) 
    = \iint a(x,p) F(x,p) \, dx \, dp,
  \end{equation}
  donde $a(x,p)$ es un observable en el espacio de fase
  correspondiente al operador $\hat{A}$ y $F(x,p)$ es una
  `densidad' correspondiente al estado $\hat{\rho}$.
\end{frame}

\begin{frame}
  \frametitle{Mapeo entre operadores en $\H$ y funciones en
  $\R^{2n}$}

  Dada la no conmutatividad de los operadores $\hat{x}$ y
  $\hat{p}$, es existe una manera única de mapear $\hat{A}
  \mapsto a$. Ésto es evidente si por ejemplo consideramos
  el operador
  \[
    e^{i(\xi \hat{x} + \eta \hat{p})},
    \quad \xi, \eta \in \R.
  \]

  Si ingenuamente sustituímos los operadores en la
  exponencial por las variables correspondientes tenemos que
  \[
    e^{i(\xi \hat{x} + \eta \hat{p})} \mapsto e^{i(\xi x +
    \eta p)},
    \quad
    e^{i\xi \hat{x}}e^{i\eta \hat{p}} \mapsto e^{i(\xi x +
    \eta p)}.
  \] 
\end{frame}

\begin{frame}
  \frametitle{Mapeo entre operadores en $\H$ y funciones en
  $\R^{2n}$}

  Debemos elegir una asignación, la más común es el orden
  simétrico de Weyl. En éste caso
  \begin{equation}
    e^{i(\xi \hat{x} + \eta \hat{p})}
    \mapsto
    e^{i(\xi x + \eta p)}.
  \end{equation}
  Para éste operador podemos expresar la traza como
  \begin{equation}
    \Tr\left( 
      \hat{\rho} e^{i(\xi \hat{x} + \eta \hat{p})}
    \right) 
    = \iint e^{i(\xi x + \eta p)}F(x,p) \, dx \, dp.
  \end{equation}
\end{frame}

\begin{frame}
  \frametitle{La densidad $F$}

  Aplicando la  transformación de Fourier de manera formal
  respecto a las variables $\xi$ y $\eta$ para obtener una
  expresión de $F$:
  \begin{equation}
    F(x,p)
    = \frac{1}{(2\pi\hbar)^2} 
    \iint \Tr\left( \hat{\rho}e^{i(\xi \hat{x} +
    \eta \hat{p})} \right) e^{-i(\xi x + \eta p)} \, d\xi \,
    d\eta.
  \end{equation}
\end{frame}

\begin{frame}
  \frametitle{La función de Wigner}

  Haciendo unos malabares con la notación de Dirac, podemos
  obtener la densidad para un estado $\hat{\rho}$ (para una
  dimensión),
  \begin{equation}
    W(x,p)
    := \frac{1}{2\pi\hbar} 
    \int \braket{x - \tfrac{1}{2}y|\hat{\rho}|x +
    \tfrac{1}{2}y} e^{-\frac{i}{\hbar} p y} \, dy.
  \end{equation}
  Similarmente para un observable cuántico $\hat{A}$ tenemos
  \begin{equation}
    A(x,p)
    = \frac{1}{(2\pi\hbar)} 
    \int \braket{x - \tfrac{1}{2}y|\hat{A}|x +
    \tfrac{1}{2}y} e^{-\frac{i}{\hbar} p y} \, dy.
  \end{equation}
\end{frame}

\begin{frame}
  \frametitle{La función de Wigner}

  Con la transformación de Wigner, podemos finalmente
  expresar los valores esperados de un observable $\hat{A}$
  en un estado $\hat{\rho}$ por medio de la integración en
  el espacio de fase.

  \begin{equation}
    \Tr\left( \hat{\rho}\hat{A} \right) 
    = \iint A(x,p) W(x,p) \, dx \, dp.
  \end{equation}

  Las integraciones sobre el espacio de posición o el de
  momentum nos brindan las densidades correspondientes la
  función de onda.
\end{frame}

\section{Transformación de Weyl}

\subsection{Cuantización}

\begin{frame}
  \frametitle{Cuantización}

  La idea básica de la cuantización es traducir observables
  clásicos a observables cuánticos. Por ejemplo, pasar de
  las variables del espacio de fase $x$ y $p$, a los
  operadores de posición y momentum en el espacio de
  Hilbert. Pero no basta con simplemente reemplazar
  formalmente a los símbolos:
  \[
    A \mapsto \hat{A} = A(\hat{x},\hat{p}).
  \] 
\end{frame}

\begin{frame}
  \frametitle{Cuantización}

  Es necesario elegir un orden de los operadores debido a la
  no conmutatividad, por ejemplo el orden de Weyl (que
  corresponde a la transformación de Wigner), es 
  \[
    xp \mapsto \frac{1}{2} \left( \hat{x}\hat{p} +
    \hat{p}\hat{x} \right). 
  \] 
\end{frame}

\begin{frame}
  \frametitle{Cuantización de Weyl}

  Consideramos la representación de Fourier de un observable
  $A(x,p)$:
  \begin{equation}
    A(x,p) = (2\pi\hbar)^{-n} \int_{\R^{2n}}
    \mathcal{F}A(\xi,\eta) e^{\frac{i}{\hbar} (\xi x + \eta
    p} \, d\xi \, d\eta.
  \end{equation}
\end{frame}

\begin{frame}
  \frametitle{Cuantización de Weyl}

  La idea es usar la asignación de Weyl y expresar el
  operador correspondiente al observable $A$ mediante una
  integral de Fourier:
  \begin{equation}
    A(x,p) = (2\pi\hbar)^{-n} \int_{\R^{2n}}
    \mathcal{F}A(\xi,\eta) e^{\frac{i}{\hbar} (\xi \hat{} + \eta
    \hat{p}} \, d\xi \, d\eta.
  \end{equation}
\end{frame}

\begin{frame}
  \frametitle{Cuantización de Weyl}

  Incluyendo la exprsión de $\mathcal{F}A$ en la integral
  anterior obtenemos la cuantización de Weyl:
  \begin{equation}
    \label{eqn:weyl}
    \Op(A)
    = \frac{1}{(2\pi\hbar)^{n}} 
    \iiiint A(x,p) e^{\frac{i}{\hbar}[\xi(x-\hat{x}) +
    \eta(p-\hat{p})]} \, d\xi \, d\eta \, dp \, dx.
  \end{equation}

  Ésta cuantización se conoce como la \textit{cuantización
  de Weyl}.
\end{frame}

\begin{frame}
  \frametitle{Cuantización de Weyl}

  Es más común expresar el operador de Weyl de un observable
  operacionalmente. Para $\psi \in L^2(\R^{n})$ tenemos
  \begin{equation}
    (\Op(A)\psi)(x)
    = \frac{1}{(2\pi\hbar)^{n}} \iint_{\R^{2n}} A\left(
    \frac{x + z}{2}, p \right) e^{\frac{i}{\hbar} p (x -
  z)} \psi(z) \, dp \, dz.
  \end{equation}
\end{frame}

\begin{frame}
  \frametitle{Representación integral de un operador}

  Supongamos que $\hat{A}$ tiene una representación con
  núcleo integral $K_{\hat{A}}(x,y)$, es decir,
  \begin{equation}
    (\hat{A}\psi)(x)
    = \int_{\R^{n}} K_{\hat{A}}(x,y)\psi(y) \, dy.
  \end{equation}
  Comparando con la ecuación anterior, podemos ver que el
  operador de Weyl tiene el núcleo
  \begin{equation}
    K_{\Op(A)}(x,y)
    = \frac{1}{(2\pi\hbar)^{n}} 
    \int_{\R^{n}} A\left( \frac{x+y}{2}, p \right)
    e^{\frac{i}{\hbar} p (x-y)} \, dp.
  \end{equation}
\end{frame}

\begin{frame}
  \frametitle{Inversión}

  Para obtener un relación inversa, hacemos un cambio de
  variable y expresamos al núcleo como
  \begin{equation}
    K_{\Op(A)}\left(
      x + \frac{1}{2}y, x - \frac{1}{2}y
    \right)
    = \frac{1}{(2\pi\hbar)^{n}} \int_{\R^{n}}
    A(x,p) e^{\frac{i}{\hbar} p y} \, dy.
  \end{equation}
  Y finalmente aplicamos la transformación de Fourier
  inversa para obtener lo que se llama el \textit{símbolo}
  del operador $\Op(A)$:
  \begin{equation}
    A(x,p)
    = \int_{\R^{n}} K_{\Op(A)}\left( x + \frac{1}{2}y, x -
    \frac{1}{2}y \right) e^{-\frac{i}{\hbar} py} \, dy.
  \end{equation}
\end{frame}

\begin{frame}
  \frametitle{La función de Wigner}

  El mapa anterior nos permite transformar un operador del
  espacio de Hilbert a una función en el espacio de fase y
  vice versa. ¿Cómo representaríamos la función de onda
  $\psi$ en el espacio de fase?

  Primero consideremos la proyección $\hat{P}_\psi$:
  \begin{align}
    (\hat{P}_\psi \phi)(x)
    &= \langle \psi, \phi \rangle \psi \\
    &= \left(
      \int_{\R^{n}} \overline{\psi(y)}\phi(y) \, dy
      \right) \psi(x) \\
    &= \int_{\R^{n}} \psi(x)\overline{\psi(y)} \phi(y) \,
    dy.
  \end{align}
\end{frame}

\begin{frame}
  \frametitle{La función de Wigner}

  El núcleo de la representación integral de la proyección
  es de la forma
  \[
    \psi(x) \overline{\psi(y)}.
  \] 

  El símbolo de $\hat{P}_\psi$ se conoce como la
  \textit{función de Wigner} de un estado puro:
  \begin{equation}
    (W\psi)(x,p)
    = \int_{\R^{n}} \psi\left( x + \frac{1}{2} y \right)
    \overline{\psi\left( x - \frac{1}{2}y \right) }
    e^{-\frac{i}{\hbar} p y} \, dy.
  \end{equation}
\end{frame}

\begin{frame}
  \frametitle{La función de Wigner}

  Para un operador de densidad arbitrario
  \[
    \hat{\rho}
    = \sum_{n}^{} p_n \ket{\psi_n} \bra{\psi_n},
    = \sum_{n}^{} p_n \hat{P}_{\psi_n},
  \] 
  podemos definir su función de Wigner mediante la
  transformación de Wigner de los elementos de la suma (?).
\end{frame}

\begin{frame}
  \frametitle{Valor esperado de un operador}

  Podemos calcular el valor esperado de un operador
  $\hat{A}$ en un estado $\psi$ por medio de una integral
  sobre el espacio de fase, de manera análoga a las
  distribuciones de la mecánica clásica.
  \begin{equation}
    \langle \psi, \hat{A}\psi \rangle
    = \frac{1}{(2\pi\hbar)^{n}} 
    \iint_{\R^{2n}} (W\psi)(x,p) A(x,p) \, dx \, dp.
  \end{equation}
\end{frame}

\begin{frame}
  \frametitle{Marginales}

  Además, integrando sobre el `espacio de momentum',
  obtenemos la densidad de probabilidad de la posición, y
  vice-versa.

  \begin{equation}
    \frac{1}{(2\pi\hbar)^{n}} \int_{\R^{n}} (W\psi)(x,p) \,
    dp
    = |\psi(x)|^2.
  \end{equation}

  No es dificil verificar la función de Wigner puede tomar
  valores negativos, por lo tanto solo llega a ser una
  \textit{cuasi-distribución}.
\end{frame}

\begin{frame}
  \frametitle{Mecánica cuántica en el espacio de fase}

  Se puede desarrollar una teoría en el espacio de fase
  equivalente a la formulación de Schrödinger, ó a la de la
  integral de trayectoria de Feynman. Podemos desarrollar la
  dinámica de un sistema cuántico despues de introducir un
  producto de los símbolos, conocido como el producto de
  Moyal.
\end{frame}


\begin{frame}
  \frametitle{Rigor?}

  Las ideas anteriores se puede hacer rigurosas trabajando
  en los espacios correctos. En lugar de solo considerar a
  $\H = L^2(\R^{n})$, es necesario consider el triplete de
  Gel'fand
  \begin{equation}
    \Sz(\R^{n}) \subset L^2(\R^{n}) \subset \Sz'(\R^{n}),
  \end{equation}
  y los espacios de operadores lineales que actúan sobre
  ellos.

  \hspace{2mm} Luego definimos la correspondencia de
  Wigner-Weyl para observables y símbolos utilizando la
  función de Wigner y la cuantización de Weyl. Finalmente
  podemos acoplar todo para darle sentido en la mecánica
  cuántica.

  \vspace{2mm}

  \textit{Creo que no es necesario para las necesidades del
  trabajo.}
\end{frame}

\end{document}

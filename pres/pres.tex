\documentclass{beamer}

\usepackage[spanish]{babel}
\decimalpoint

\usepackage{amsmath, amssymb}
\usepackage{amsthm}
\usepackage{bm}
\usepackage{braket}

\DeclareMathOperator{\R}{\mathbb{R}}
\DeclareMathOperator{\C}{\mathbb{C}}
\DeclareMathOperator{\N}{\mathbb{N}}
\DeclareMathOperator{\Z}{\mathbb{Z}}

\let\H\relax
\DeclareMathOperator{\H}{\mathcal H}
\DeclareMathOperator{\Sz}{\mathcal S}

\DeclareMathOperator{\dom}{Dom}
\DeclareMathOperator{\prob}{Prob}
\DeclareMathOperator{\id}{id}
\DeclareMathOperator{\Tr}{Tr}
\DeclareMathOperator{\Op}{Op}
\DeclareMathOperator{\W}{W}
\DeclareMathOperator{\F}{F}

\title{La transformación de Wigner-Weyl}
\author{Ernesto}
\institute{U de G}

\begin{document}

\frame{\titlepage}

\frame{\tableofcontents}

\section{Primera vía}

\begin{frame}
\frametitle{Cuantización}

La idea básica de la cuantización es traducir observables
clásicos a observables cuánticos. Por ejemplo, pasar de las
variables del espacio de fase $x$ y $p$, a los operadores de
posición y momentum en el espacio de Hilbert. Pero no basta
con simplemente reemplazar formalmente a los símbolos:
\[
  A \mapsto \hat{A} = A(\hat{x},\hat{p}).
\] 
\end{frame}

\begin{frame}
  \frametitle{Cuantización}

  Es necesario elegir un orden de los operadores debido a la
  no conmutatividad, por ejemplo el orden de Weyl (que
  corresponde a la transformación de Wigner), es 
  \[
    xp \mapsto \frac{1}{2} \left( \hat{x}\hat{p} +
    \hat{p}\hat{x} \right). 
  \] 
\end{frame}

\begin{frame}
  \frametitle{Cuantización de Weyl}

  Informalmente podemos intentar escribir el operador
  correspondiente a un observable $A(x)$ mediante una
  `función delta de un operador',
  \begin{equation}
    \label{eqn:op_int_delta}
    \hat{A}
    = \int_{\R^{n}} A(x) \delta(x - \hat{x}) \, dx.
  \end{equation} 
  Dado que podemos expresar a la función delta mediante una
  transformación de Fourier,
  \[
    \delta(x)
    = \frac{1}{(2\pi\hbar)^{n}} 
    \int_{\R^{n}} e^{\frac{i}{\hbar} \xi x} \, d\xi,
  \] 
  podemos escribir a (\ref{eqn:op_int_delta}) como
  \[
    \hat{A}
    = \frac{1}{(2\pi\hbar)^{n}} \int_{\R^{2n}} A(x)
    e^{\frac{i}{\hbar}\xi (x - \hat{x})} \, d\xi \, dx.
  \] 
\end{frame}

\begin{frame}
  \frametitle{Cuantización de Weyl}

  Ahora consideremos un observable $A(x,p)$. Por analogía
  nos gustaría escribir
  \[
    \hat{A}
    = \int_{\R^{n}} \int_{\R^{n}}
    A(x,p) \delta(x - \hat{x}) \delta(p - \hat{p}) \, dp \,
    dx.
  \] 
  Digamos que
  \[
    \delta(x - \hat{x}) \delta(p - \hat{p})
    = \frac{1}{(2\pi\hbar)^{n}} \int_{\R^{n}} \int_{\R^{n}} 
    e^{\frac{i}{\hbar}[\xi(x-\hat{x}) + \eta(p-\hat{p})]} \,
    d\xi \, d\eta.
  \]
\end{frame}

\begin{frame}
  \frametitle{Cuantización de Weyl}

  Con lo anterior nuestra expresión para el operador
  correspondiente al observable $A$ es
  \begin{equation}
    \label{eqn:weyl}
    \Op(A)
    = \frac{1}{(2\pi\hbar)^{n}} 
    \iiiint A(x,p) e^{\frac{i}{\hbar}[\xi(x-\hat{x}) +
    \eta(p-\hat{p})]} \, d\xi \, d\eta \, dp \, dx.
  \end{equation}

  Ésta cuantización se conoce como la \textit{cuantización
  de Weyl}.
\end{frame}

\begin{frame}
  \frametitle{Cuantización de Weyl}

  Es más común expresar el operador de Weyl de un observable
  operacionalmente. Para $\psi \in L^2(\R^{n})$ tenemos
  \begin{equation}
    (\Op(A)\psi)(x)
    = \frac{1}{(2\pi\hbar)^{n}} \iint_{\R^{2n}} A\left(
    \frac{x + z}{2}, p \right) e^{\frac{i}{\hbar} p (x -
  z)} \psi(z) \, dp \, dz.
  \end{equation}
\end{frame}

\begin{frame}
  \frametitle{Representación integral de un operador}

  Supongamos que $\hat{A}$ tiene una representación con
  núcleo integral $K_{\hat{A}}(x,y)$, es decir,
  \begin{equation}
    (\hat{A}\psi)(x)
    = \int_{\R^{n}} K_{\hat{A}}(x,y)\psi(y) \, dy.
  \end{equation}
  Comparando con la ecuación anterior, podemos ver que el
  operador de Weyl tiene el núcleo
  \begin{equation}
    K_{\Op(A)}(x,y)
    = \frac{1}{(2\pi\hbar)^{n}} 
    \int_{\R^{n}} A\left( \frac{x+y}{2}, p \right)
    e^{\frac{i}{\hbar} p (x-y)} \, dp.
  \end{equation}
\end{frame}

\begin{frame}
  \frametitle{Inversión}

  Para obtener un relación inversa, hacemos un cambio de
  variable y expresamos al núcleo como
  \begin{equation}
    K_{\Op(A)}\left(
      x + \frac{1}{2}y, x - \frac{1}{2}y
    \right)
    = \frac{1}{(2\pi\hbar)^{n}} \int_{\R^{n}}
    A(x,p) e^{\frac{i}{\hbar} p y} \, dy.
  \end{equation}
  Y finalmente aplicamos la transformación de Fourier
  inversa para obtener lo que se llama el \textit{símbolo}
  del operador $\Op(A)$:
  \begin{equation}
    A(x,p)
    = \int_{\R^{n}} K_{\Op(A)}\left( x + \frac{1}{2}y, x -
    \frac{1}{2}y \right) e^{-\frac{i}{\hbar} py} \, dy.
  \end{equation}
\end{frame}

\begin{frame}
  \frametitle{La función de Wigner}

  El mapa anterior nos permite transformar un operador del
  espacio de Hilbert a una función en el espacio de fase y
  vice versa. ¿Cómo representaríamos la función de onda
  $\psi$ en el espacio de fase?

  Primero consideremos la proyección $\hat{P}_\psi$:
  \begin{align}
    (\hat{P}_\psi \phi)(x)
    &= \langle \psi, \phi \rangle \psi \\
    &= \left(
      \int_{\R^{n}} \overline{\psi(y)}\phi(y) \, dy
      \right) \psi(x) \\
    &= \int_{\R^{n}} \psi(x)\overline{\psi(y)} \phi(y) \,
    dy.
  \end{align}
\end{frame}

\begin{frame}
  \frametitle{La función de Wigner}

  El núcleo de la representación integral de la proyección
  es de la forma
  \[
    \psi(x) \overline{\psi(y)}.
  \] 

  El símbolo de $\hat{P}_\psi$ se conoce como la
  \textit{función de Wigner} de un estado puro:
  \begin{equation}
    (W\psi)(x,p)
    = \int_{\R^{n}} \psi\left( x + \frac{1}{2} y \right)
    \overline{\psi\left( x - \frac{1}{2}y \right) }
    e^{-\frac{i}{\hbar} p y} \, dy.
  \end{equation}
\end{frame}

\begin{frame}
  \frametitle{La función de Wigner}

  Para un operador de densidad arbitrario
  \[
    \hat{\rho}
    = \sum_{n}^{} p_n \ket{\psi_n} \bra{\psi_n},
    = \sum_{n}^{} p_n \hat{P}_{\psi_n},
  \] 
  podemos definir su función de Wigner mediante la
  transformación de Wigner de los elementos de la suma (?).
\end{frame}

\begin{frame}
  \frametitle{Valor esperado de un operador}

  Podemos calcular el valor esperado de un operador
  $\hat{A}$ en un estado $\psi$ por medio de una integral
  sobre el espacio de fase, de manera análoga a las
  distribuciones de la mecánica clásica.
  \begin{equation}
    \langle \psi, \hat{A}\psi \rangle
    = \frac{1}{(2\pi\hbar)^{n}} 
    \iint_{\R^{2n}} (W\psi)(x,p) A(x,p) \, dx \, dp.
  \end{equation}
\end{frame}

\begin{frame}
  \frametitle{Marginales}

  Además, integrando sobre el `espacio de momentum',
  obtenemos la densidad de probabilidad de la posición, y
  vice-versa.

  \begin{equation}
    \frac{1}{(2\pi\hbar)^{n}} \int_{\R^{n}} (W\psi)(x,p) \,
    dp
    = |\psi(x)|^2.
  \end{equation}

  No es dificil verificar la función de Wigner puede tomar
  valores negativos, por lo tanto solo llega a ser una
  \textit{cuasi-distribución}.
\end{frame}

\begin{frame}
  \frametitle{Mecánica cuántica en el espacio de fase}

  Se puede desarrollar una teoría en el espacio de fase
  equivalente a la formulación de Schrödinger, ó a la de la
  integral de trayectoria de Feynman. Podemos desarrollar la
  dinámica de un sistema cuántico despues de introducir un
  producto de los símbolos, conocido como el producto de
  Moyal.
\end{frame}

\section{Segunda vía}

\begin{frame}
  \frametitle{Estadística en el espacio de fase}

  Dada una densidad de probabilidad de Liouville $F(x,p)$, y
  un observable $a : \R^{2n} \to \R$ que depende de la
  posición y momentum, se puede calcular el valor esperado
  de $a$ sobre un conjunto de partículas como
  \begin{equation}
    \mathbb E[a]
    = \iint a(x,p) F(x,p) \, dx \, dp.
  \end{equation}

  \textbf{Idea}. Nos gustaría poder calcular valores
  esperados de operadores que dependen de $\hat{x}$ y
  $\hat{p}$ de una manera análoga.
\end{frame}

\begin{frame}
  \frametitle{Estadística en el espacio de fase}

  En otras palabras, nos gustaría poder hacer algo como
  \begin{equation}
    \Tr\left( \hat{\rho}\hat{A} \right) 
    = \iint a(x,p) F(x,p) \, dx \, dp,
  \end{equation}
  donde $a(x,p)$ es un observable en el espacio de fase
  correspondiente al operador $\hat{A}$ y $F(x,p)$ es una
  `densidad' correspondiente al estado $\hat{\rho}$.
\end{frame}

\begin{frame}
  \frametitle{Mapeo entre operadores en $\H$ y funciones en
  $\R^{2n}$}

  Dada la no conmutatividad de los operadores $\hat{x}$ y
  $\hat{p}$, es existe una manera única de mapear $\hat{A}
  \mapsto a$. Ésto es evidente si por ejemplo consideramos
  el operador
  \[
    e^{i(\xi \hat{x} + \eta \hat{p})},
    \quad \xi, \eta \in \R.
  \]

  Si ingenuamente sustituímos los operadores en la
  exponencial por las variables correspondientes tenemos que
  \[
    e^{i(\xi \hat{x} + \eta \hat{p})} \mapsto e^{i(\xi x +
    \eta p)},
    \quad
    e^{i\xi \hat{x}}e^{i\eta \hat{p}} \mapsto e^{i(\xi x +
    \eta p)}.
  \] 
\end{frame}

\begin{frame}
  \frametitle{Mapeo entre operadores en $\H$ y funciones en
  $\R^{2n}$}

  Una manera de evitar ésta ambigüedad (Cohen) es eligiendo
  un núcleo $g : \R^2 \to \C$ y haciendo la asignación
  \begin{equation}
    g(\xi,\eta) e^{i(\xi \hat{x} + \eta \hat{p})}
    \mapsto
    e^{i(\xi x + \eta p)}.
  \end{equation}
  Con ésto, podemos expresar la traza como
  \begin{equation}
    \Tr\left( 
      \hat{\rho} g(\xi,\eta) e^{i(\xi \hat{x} + \eta
      \hat{p})}
    \right) 
    = \iint e^{i(\xi x + \eta p)}F^{g}(x,p) \, dx \, dp,
  \end{equation}
  donde la distribución $F^{g}$ corresponde a la elección
  del núcleo.
\end{frame}

\begin{frame}
  \frametitle{La densidad $F^{g}$}

  Se puede aplicar una transformación de Fourier respecto a
  las variables $\xi$ y $\eta$ para obtener una expresión de
  $F^{g}$:
  \begin{equation}
    F^{g}(x,p)
    = \frac{1}{(2\pi\hbar)^2} 
    \iint \Tr\left( \hat{\rho}g(\xi,\eta)e^{i(\xi \hat{x} +
    \eta \hat{p})} \right) e^{-i(\xi x + \eta p)} \, d\xi \,
    d\eta.
  \end{equation}
\end{frame}

\begin{frame}
  \frametitle{La función de Wigner}

  Eligiendo el núcleo $g(\xi,\eta) = 1$ y haciendo unos
  malabares con la notación de Dirac, podemos obtener la
  densidad para un estado $\hat{\rho}$ (para una dimensión),
  \begin{equation}
    W(x,p)
    := F^{g}(x,p)
    = \frac{1}{2\pi\hbar} 
    \int \braket{x - \tfrac{1}{2}y|\hat{\rho}|x +
    \tfrac{1}{2}y} e^{-\frac{i}{\hbar} p y} \, dy.
  \end{equation}
  Similarmente para un observable cuántico $\hat{A}$ tenemos
  \begin{equation}
    a(x,p)
    = \frac{1}{(2\pi\hbar)} 
    \int \braket{x - \tfrac{1}{2}y|\hat{A}|x +
    \tfrac{1}{2}y} e^{-\frac{i}{\hbar} p y} \, dy.
  \end{equation}
\end{frame}

\begin{frame}
  \frametitle{La función de Wigner}

  Con la transformación de Wigner, podemos finalmente
  expresar los valores esperados de un observable $\hat{A}$
  en un estado $\hat{\rho}$ por medio de la integración en
  el espacio de fase.

  \begin{equation}
    \Tr\left( \hat{\rho}\hat{A} \right) 
    = \iint a(x,p) W(x,p) \, dx \, dp.
  \end{equation}

  Las integraciones sobre el espacio de posición o el de
  momentum nos brindan las densidades correspondientes la
  función de onda.
\end{frame}

\section{Tercera vía}

\begin{frame}
  \frametitle{Rigor?}

  Las ideas anteriores se puede hacer rigurosas trabajando
  en los espacios correctos. En lugar de solo considerar a
  $\H = L^2(\R^{n})$, es necesario consider el triplete de
  Gel'fand
  \begin{equation}
    \Sz(\R^{n}) \subset L^2(\R^{n}) \subset \Sz'(\R^{n}),
  \end{equation}
  y los espacios de operadores lineales que actúan sobre
  ellos.

  \hspace{2mm} Luego definimos la correspondencia de
  Wigner-Weyl para observables y símbolos utilizando la
  función de Wigner y la cuantización de Weyl. Finalmente
  podemos acoplar todo para darle sentido en la mecánica
  cuántica.

  \vspace{2mm}

  \textit{Creo que no es necesario para las necesidades del
  trabajo.}
\end{frame}

\end{document}

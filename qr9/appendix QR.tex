
\documentclass{article}
\usepackage{amsfonts}

%%%%%%%%%%%%%%%%%%%%%%%%%%%%%%%%%%%%%%%%%%%%%%%%%%%%%%%%%%%%%%%%%%%%%%%%%%%%%%%%%%%%%%%%%%%%%%%%%%%%%%%%%%%%%%%%%%%%%%%%%%%%%%%%%%%%%%%%%%%%%%%%%%%%%%%%%%%%%%%%%%%%%%%%%%%%%%%%%%%%%%%%%%%%%%%%%%%%%%%%%%%%%%%%%%%%%%%%%%%%%%%%%%%
%TCIDATA{OutputFilter=LATEX.DLL}
%TCIDATA{Version=5.50.0.2953}
%TCIDATA{<META NAME="SaveForMode" CONTENT="1">}
%TCIDATA{BibliographyScheme=Manual}
%TCIDATA{Created=Sunday, November 05, 2023 11:51:03}
%TCIDATA{LastRevised=Sunday, November 05, 2023 13:12:11}
%TCIDATA{<META NAME="GraphicsSave" CONTENT="32">}
%TCIDATA{<META NAME="DocumentShell" CONTENT="Standard LaTeX\Blank - Standard LaTeX Article">}
%TCIDATA{Language=American English}
%TCIDATA{CSTFile=40 LaTeX article.cst}

\newtheorem{theorem}{Theorem}
\newtheorem{acknowledgement}[theorem]{Acknowledgement}
\newtheorem{algorithm}[theorem]{Algorithm}
\newtheorem{axiom}[theorem]{Axiom}
\newtheorem{case}[theorem]{Case}
\newtheorem{claim}[theorem]{Claim}
\newtheorem{conclusion}[theorem]{Conclusion}
\newtheorem{condition}[theorem]{Condition}
\newtheorem{conjecture}[theorem]{Conjecture}
\newtheorem{corollary}[theorem]{Corollary}
\newtheorem{criterion}[theorem]{Criterion}
\newtheorem{definition}[theorem]{Definition}
\newtheorem{example}[theorem]{Example}
\newtheorem{exercise}[theorem]{Exercise}
\newtheorem{lemma}[theorem]{Lemma}
\newtheorem{notation}[theorem]{Notation}
\newtheorem{problem}[theorem]{Problem}
\newtheorem{proposition}[theorem]{Proposition}
\newtheorem{remark}[theorem]{Remark}
\newtheorem{solution}[theorem]{Solution}
\newtheorem{summary}[theorem]{Summary}
\newenvironment{proof}[1][Proof]{\noindent\textbf{#1.} }{\ \rule{0.5em}{0.5em}}
\input{tcilatex}

\begin{document}


In this Appendix we recall some basic properties of commutative curves in
even dimensions.

Commuting sets constituted by $2^{n}$ different monomials (stabilizers) $\{%
\hat{Z}_{\alpha _{\lambda }(\tau )}\hat{X}_{\beta _{\lambda }(\tau )},\tau
\in \mathbb{F}_{2^{n}}\}$ are labelled by points of a discrete grid
belonging to a non-singular curve  (i.e., with no self-intersection) $\Gamma
^{\lambda }$, such that $(\alpha (0),\beta (0))=(0,0)\in \Gamma ^{\lambda }$
and%
\[
tr\left( \alpha _{\lambda }(\tau ^{\prime })\beta _{\lambda }(\tau )\right)
=tr\left( \alpha _{\lambda }(\tau )\beta _{\lambda }(\tau ^{\prime })\right)
,\;\;\left( \alpha _{\lambda }(\tau ),\beta _{\lambda }(\tau )\right) \in
\Gamma ^{\lambda }.
\]

A general form of such curves is 
\begin{equation}
\alpha (\tau )=\sum_{i=0}^{n-1}\alpha _{i}\,\tau ^{2^{i}},\qquad \beta (\tau
)=\sum_{i=0}^{n-1}\beta _{i}\,\tau ^{2^{i}}\,,\;\alpha _{i},\beta _{i}\in 
\mathbb{F}_{2^{n}},  \label{curve1}
\end{equation}%
where $\sum_{i\neq j}tr(\alpha _{i}\beta _{j})=0\,$, such that  
\begin{equation}
\lbrack \hat{Z}_{\alpha (\tau )}\hat{X}_{\beta (\tau )},\hat{Z}_{\alpha
(\tau ^{\prime })}\hat{X}_{\beta (\tau ^{\prime })}]=0\,,\;\;\left( \alpha
_{\lambda }(\tau ),\beta _{\lambda }(\tau )\right) \in \Gamma ^{\lambda }.
\label{stab}
\end{equation}%
It is worth noting that $n$\ appropriately chosen points, e.g. $\tau
=\{\theta _{1},...,\theta _{n}\}$ generate the entire curve. 

The regular curves, non-degenerated at lest in one of directions  $\alpha $
or $\beta $, can be represented in the explicit form, 
\begin{equation}
\beta =f(\alpha )=\sum_{i=0}^{n-1}\phi _{i}\,\alpha ^{2^{i}}\quad \mathrm{or}%
\quad \alpha =g(\beta )=\sum_{i=0}^{n-1}\psi _{i}\,\beta ^{2^{i}}\,,
\label{RC}
\end{equation}%
where $\phi _{i},\psi _{i}\in \mathbb{F}_{2^{n}}.$ satisfy the conditions,  
\begin{eqnarray}
\phi _{k} &=&\phi _{n-k}^{2^{k}}\,,\qquad \psi _{k}=\psi
_{n-k}^{2^{k}}\,,\quad k=1,\ldots ,[(n-1)/2]\,,\text{ odd }n  \label{Acc} \\
\phi _{n/2} &=&\phi _{n/2}^{2^{n/2}},\qquad \psi _{n/2}=\psi
_{n/2}^{2^{n/2}},\text{ even }n\text{.}
\end{eqnarray}%
and $[\,]$ denotes the integer part. 

The degenerate (or exceptional) curves are characterized by multiple
appearance of the admissible points in both directions $\alpha $ and $\beta $%
. In other words, for every point $(\beta _{j},\alpha _{j})$  of such curve $%
\alpha _{j}$ and $\beta _{j}$ take only $2^{n-r_{\beta }}$ and  $%
2^{n-r_{\alpha }}$ values correspondingly, where  $r_{\alpha }$ and $%
r_{\beta }$ are the degrees of degenerations along the respective axes. The
admissible points of such curves are fixed by relations $tr(\sigma _{j}\beta
)=0,\quad tr(\sigma _{k}\alpha )=0$ where $\sigma _{j}$ are some given
elements of $\mathbb{F}_{2^{n}}$.

The partitions of DPS constituted by  $2^{n}+1$ commutative curves are
classified by their factorization structures (\ref{curve_part}), $\lambda
=\{m_{1},m_{2},\ldots ,m_{n}\}\,$, $\sum_{k}$ $m_{k}=2^{n}+1$, which
indicates the number and the lengths of the commuting sub-blocks of the
stabilizers labelled by the points of a curve. Locally equivalent
stabilziers can be labelled by points of different curves, but  have the
same factorization structure. On the other hand, curves with different
factorization structures are no locally equivalent.

In particular, in the partition (\ref{rays}) there are $3$ completely
factorized rays ($\beta =0,\alpha =0,\beta =\alpha $) , i.e. of the
structure $\{1,1,1\}$,and the other six rays have the structure $\{3\}$, so
that the whole partition is  $(3,0,6)$. 

The partitions $(0,9,0)$ include only curves with the factorization $\{1,2\}$%
, and always contain exceptional curves. One example of those partitions is:

a) Regular curves 
\begin{equation}
\begin{array}{ll}
\alpha =\sigma ^{2}\beta +\sigma ^{3}\beta ^{2}+\sigma ^{5}\beta ^{4},\qquad
\qquad  & \alpha =\sigma ^{6}\beta +\sigma ^{3}\beta ^{2}+\sigma ^{5}\beta
^{4}, \\ 
\beta =\sigma ^{2}\alpha +\sigma ^{3}\alpha ^{2}+\sigma ^{5}\alpha
^{4},\qquad \qquad  & \beta =\sigma ^{6}\alpha ^{2}+\sigma ^{3}\alpha ^{4},
\\ 
\alpha =\beta +\sigma ^{6}\beta ^{2}+\sigma ^{3}\beta ^{4},\qquad \qquad  & 
\beta =\alpha +\sigma ^{3}\alpha ^{2}+\sigma ^{5}\alpha ^{4}, \\ 
\alpha =\sigma ^{3}\beta ^{2}+\sigma ^{5}\beta ^{4},\qquad \qquad  & 
\end{array}%
\end{equation}%
b) Exceptional curves 
\begin{eqnarray}
\beta ^{2}+\sigma ^{5}\beta  &=&\sigma ^{2}\alpha ^{2}+\sigma ^{6}\alpha
,\qquad \qquad tr(\sigma ^{4}\beta )=0,\quad tr(\sigma ^{5}\alpha )=0; 
\nonumber \\
&& \\
\beta ^{2}+\sigma ^{2}\beta  &=&\sigma ^{6}\alpha ^{2}+\sigma ^{5}\alpha
,\qquad \qquad tr(\sigma ^{6}\beta )=0,\quad tr(\sigma ^{2}\alpha )=0. 
\nonumber
\end{eqnarray}

\end{document}

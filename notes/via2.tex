\subsubsection{Vía 2}

Para obtener una expresión para el valor esperado de un
observable en un estado en términos de una integral sobre
el espacio de fase, análogo al valor esperado de un
observable dada una densidad de Liouville, primero debemos
obtener una correspondencia entre los operadores del
espacio de Hilbert asociado al sistema cuántico, y las
funciones integrables en el espacio de fase, un proceso
que se conoce como \textit{cuantización}.  Ingenuamente
podríamos simplemente reemplazar formalmente a las
variables $x$ y $p$ por los operadores $\hat{x}$ y
$\hat{p}$,
\[
  A(x,p) \mapsto \hat{A}(\hat{x},\hat{p}).
\]
Ésto no funciona en general porque no hay una manera única
de asignar operadores a funciones con ésta asignación. En
1927, Herman Weyl introdujo una manera específica de mapear
observables clásicos a observables cuánticos, ésta
asignación se conoce como la \textit{cuantización de Weyl}
y mapea al producto de las variables de posición momentum
de manera simétrica:
\[
  xp \mapsto \frac{1}{2}\left( \hat{x}\hat{p} +
  \hat{p}\hat{x} \right).
\] 
Con ésta elección, supongamos que tenemos un operador
$\hat{A}$ correspondiente a una función que solamente
depende de la variable de posición, $A(x)$, entonces
informalmente, podríamos expresar a $\hat{A}$ por medio de
una función delta de Dirac para operadores:
\begin{equation}
  \hat{A}(\hat{x})
  = \int A(x) \delta(x-\hat{x}) \, dx.
\end{equation}
Utilizando la expresión de una delta de Dirac por medio de
una integral,
\[
  \delta(x)
  = \frac{1}{2\pi\hbar} \int e^{\frac{i}{\hbar} \xi x} \,
  d\xi,
\] 
podemos expresar al operador como
\[
  \hat{A}(\hat{x})
  = \frac{1}{2\pi\hbar} \iint A(x)e^{\frac{i}{\hbar} \xi
  (x - \hat{x})} \, d\xi \, dx.
\] 
Nos gustaría hacer lo mismo pero ahora para un observable
que depende de $x$ y de $p$. Por analogía podríamos
escribir 
\begin{equation}
  \hat{A}(\hat{x},\hat{p})
  = \frac{1}{(2\pi\hbar)^2} \iiiint A(x,p)
  e^{\frac{i}{\hbar} [\xi (x - \hat{x}) + \eta (p -
  \hat{p})]} \, d\xi \, d\eta \, dx \, dp.
\end{equation}
Tal expresión se conoce como la transformación de Weyl de
un observable. Para un estado puro $\psi \in L^2(\R)$,
podemos expresar la transformación de Weyl sin los
operadores en la exponencial. Para hacer ésto, primero
notemos que
\[
  \left( e^{-\frac{\hbar}{i} \xi \hat{x}} \psi \right)(x)
  = e^{-\frac{\hbar}{i} \xi x}\psi(x),
\] 
y 
\[
  \left( e^{-\frac{\hbar}{i} \eta \hat{p}} \psi \right)(x)
  = \psi(x-\eta).
\] 
Además haciendo uso de la identidad de
Baker-Campbell-Hausdorff, podemos separar el operador
exponencial como
\[
  e^{-\frac{i}{\hbar}(\xi \hat{x} + \eta \hat{p})}
  = e^{-\frac{i}{\hbar} \xi \hat{x}} e^{-\frac{i}{\hbar}
  \eta \hat{p}} e^{\frac{i}{2\hbar} \xi \eta}.
\] 
Con ésto, podemos expresar la transformación de Weyl de un
observable sobre un estado puro como
\begin{align}
  \left(\hat{A}\psi\right)(z)
  &= \frac{1}{(2\pi\hbar)2} 
  \iiiint A(x,p) e^{\frac{i}{\hbar}[\xi(x-\hat{x}) +
  \eta(p-\hat{p})]} \psi(z) \, d\xi \, d\eta \, dx \, dp
  \\
  &= \frac{1}{(2\pi\hbar)^2} \iiiint A(x,p)
  e^{\frac{i}{\hbar}(\xi x + \eta
  p)}e^{-\frac{i}{\hbar}(\xi \hat{x} + \eta \hat{p})}
  \psi(z) \, d\xi \, d\eta \, dx \, dp \\
  &= \frac{1}{(2\pi\hbar)^2} \iiiint A(x,p)
  e^{\frac{i}{\hbar}(\xi x + \eta p + \frac{1}{2} \xi
  \eta)} e^{-\frac{i}{\hbar} \xi z}\psi(z - \eta) \, d\xi
  \, d\eta \, dx \, dp.
\end{align}
Integrando sobre $\xi$ obtenemos
\begin{align*}
  \left( \hat{A} \psi \right)(z)
  &= \frac{1}{(2\pi\hbar)^2} \iiiint A(x,p)
  e^{\frac{i}{\hbar}\xi(x + \frac{1}{2}\eta - z)}
  e^{\frac{i}{\hbar} \eta p} \psi(z-\eta) \, d\xi \, d\eta
  \, dx \, dp \\
  &= \frac{1}{2\pi\hbar} \iiint A(x,p) \delta(x - z +
  \tfrac{1}{2}\eta) e^{\frac{i}{\hbar} \eta p} \psi(z -
  \eta) \, d\eta \, dx \, dp \\
  &= \frac{1}{2\pi\hbar} \iint A(z - \tfrac{1}{2} \eta, p)
  e^{\frac{i}{\hbar} \eta p} \psi(z- \eta) \, d\eta \, dp.
\end{align*}
Ahora hacemos el cambio de variable $x = z - \eta$ para
obtener 
\begin{equation}
  \label{eqn:weyl_map}
  \left( \hat{A}\psi \right)(z)
  = \frac{1}{2\pi\hbar} \iint A\left( \frac{x+z}{2},
  p\right) e^{\frac{i}{\hbar} p (z-x)} \psi(x) \, dx \,
  dp.
\end{equation}
Ésta última expresión es la más común cuando se habla de
la cuantización de Weyl de un observable $A(x,p)$. Ahora
que tenemos una manera de mapear un observable sobre el
espacio de fase a un operador de $L^2(\R)$, deseamos
obtener el mapa inverso. Para ésto, supongamos que el
operador $\hat{A}$ tiene una representación integral, es
decir, para todo $\psi \in L^2(\R)$ tenemos
\begin{equation}
  \left( \hat{A}\psi \right)(z)
  = \int K_{\hat{A}}(x,z) \psi(x) \, dx,
\end{equation}
donde $K_{\hat{A}}$ se conoce como el \textit{núcleo} del
operador $\hat{A}$. Si además suponemos que el operador
$\hat{A}$ es la transformada de Weyl de un observable $A$,
por medio de una comparación con la expresión
(\ref{eqn:weyl_map}) con la representación integral,
observamos que el núcleo correspondiente a $\hat{A}$ está
dado por
\begin{equation}
  K_{\hat{A}}(x,z)
  = \frac{1}{2\pi\hbar} \int A\left( \frac{x+z}{2}, p
  \right) e^{\frac{i}{\hbar} p (z-x)} \, dp.
\end{equation}
Haciendo el cambio de variable $u = \frac{x+z}{2}$ y $v =
x - z$ obtenemos
\begin{equation}
  K_{\hat{A}}(u - \tfrac{v}{2}, u + \tfrac{v}{2})
  = \frac{1}{2\pi\hbar} \int A\left( u, p \right)
  e^{-\frac{i}{\hbar} p v} \, dp.
\end{equation}
Para recuperar al observable $A(u,p)$, aplicamos la
transformación de Fourier para obtener
\begin{equation}
  A(x,p)
  = \int K_{\hat{A}}(x-\tfrac{z}{2}, u + \tfrac{z}{2})
  e^{\frac{i}{\hbar} p v} \, dp.
\end{equation}

. . . 

Los operadores de densidad que describen el estado de un
sistema cuántico pertenecen al espacio de operadores
Hilbert-Schmidt. De acuerdo a ésto, la función de Wigner
$\W_\rho$ de un estado puro $\rho$ con núcleo
$\rho_K(x,y) = \psi(x)\overline{\psi}(y)$ está dado por
\[
  \W_\rho(x,p)
  = \frac{1}{2\pi\hbar} \int \psi(x - \frac{1}{2}y)
  \overline{\psi}(x + \frac{1}{2}y) e^{ipy / \hbar} \,
  dy.
\] 

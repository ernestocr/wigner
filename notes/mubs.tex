\documentclass[a4paper]{article}
\usepackage[margin=1.2in]{geometry}
\usepackage[utf8]{inputenc}
\usepackage[T1]{fontenc}
\usepackage{textcomp}
\usepackage[spanish]{babel}
\usepackage{amsmath, amssymb}
\usepackage{amsthm}
\usepackage{braket}
\decimalpoint

\DeclareMathOperator{\R}{\mathbb{R}}
\DeclareMathOperator{\C}{\mathbb{C}}
\DeclareMathOperator{\N}{\mathbb{N}}
\DeclareMathOperator{\Z}{\mathbb{Z}}
\DeclareMathOperator{\Tr}{Tr}
\DeclareMathOperator{\ad}{ad}
\DeclareMathOperator{\diag}{diag}

\newtheorem{definition}{Definition}
\newtheorem{theorem}{Theorem}
\newtheorem{proposition}{Proposition}
\newtheorem{lemma}{Lemma}
%\newtheorem{corollary}{Corolario}
\newtheorem{example}{Example}
\newtheorem{conjecture}{Conjecture}

\hfuzz=6pt

\title{MUBs}
\author{Ernesto Camacho Ramírez}
\begin{document}
  \maketitle

  \section{Optimal State-Determination by Mutually Unbiased
  Measurements (1988)}

  In quantum formalism, the property of orthogonality
  appears naturally in the eigenstates of a non-degenerate
  measurement, which form an orthogonal basis of the vector
  space. There is a particular relation among different
  orthogonal bases, a relation called mutual unbiasedness.
  They play a special role in the problem of determining the
  state of a quantum ensemble. Measurments associated with
  mutually unbiased bases provide an optimal means of
  determining an ensemble's state. 

  The bases $\{v_i\}$ and $\{u_j\}$ of an $N$-dimensional
  Hilbert space are mutualy unbiased if inner products
  between all possible pairs of vectors with one vector from
  each basis have the same magnitud:
  \[
    |\langle v_i, u_j \rangle|
    = \frac{1}{\sqrt{N}}.
  \] 
  In terms of projection operators, let $P_i$ be the
  projection onto the vector $v_i$ and $Q_j$ onto to $u_j$.
  Then mutual unbiasedness requires that for all $P_i$ and
  $Q_j$,
  \[
    \Tr(P_iQ_j) = \frac{1}{N}.
  \] 
  For $N = 2$ we have an important and basic example:
  \[
    \left\{
      \begin{pmatrix} 1; 0 \end{pmatrix},
      \begin{pmatrix} 0; 1 \end{pmatrix} 
    \right\},
    \left\{
      \frac{1}{\sqrt{2}} \begin{pmatrix} 1; 1 \end{pmatrix},
      \frac{1}{\sqrt{2}} \begin{pmatrix} 1; -1 \end{pmatrix} 
    \right\},
    \left\{
      \frac{1}{\sqrt{2}} \begin{pmatrix} 1; i \end{pmatrix},
      \frac{1}{\sqrt{2}} \begin{pmatrix} 1; -i \end{pmatrix} 
    \right\},
  \] 
  which correspond to the eigenstates of the components of
  spin in the $z, x$ and $y$ directions (in standard
  representation).

  We say that two non-degenerate measurements are mutually
  unbiased if the bases comprising their eigenstates are
  mutually unbiased. Schewinger noted that a measurement
  over one basis leaves one completely uncertain as to the
  outcome of a measurement over a basis unbiased with
  respect to the first. A measurement of spin in the $x$ 
  direction for a spin-$1 / 2$ particle leaves one
  completely uncertain as to the component of spin in the
  $y$ direction. Given this maximal incompatibility,
  Schwinger called the operators corresponding to such bases
  as \textit{complementary}.

  \subsection{The problema of state determination}

  The problem is to ascertain the density matrix of an
  ensemble of identical $N$-state systems, which may be in a
  pure or a mixed state. Such a density matrix is specified
  by $N^2-1$ real parameters. Any given non-degenrate
  measurement applied to a subensemble will yield precisely
  $N-1$ real numbers, i.e., the probabilities of all but one
  of the $N$ possible outcomes. Thus the minimum number of
  different measurements one needs in order to determine the
  state uniquely is $(N^2-1) / (N-1) = N + 1$. If one can
  find $N+1$ bases that are all unbiased, then measurements
  corresponding to these bases are guaranteed to be
  sufficient for determingin the density matrix.

  \subsection{A construction of complete sets of mutually
  unbiased bases}

  For this method, the construction is different for powers
  of odd primes than for powers of two.

  \subsubsection{Unbiased bases for powers of odd primes}

  For $N = p$, an odd prime, one can find $N+1$ mutually
  unbiased bases. These bases are the standard basis
  \[
    (v_k^0)_l = \delta_{kl},
    \quad k,l = 0,1,\ldots,p-1,
  \] 
  and $N$ bases given by
  \[
    (v_k^0)_l = \frac{1}{\sqrt{N}} e^{(2\pi i /
    p)(rl^2+kl)},
    \quad r = 1,2,\ldots,N.
  \] 
  The key to generalizing this resulto (to powers of a prime
  number) is to note that the phases of the vectors are all
  $p$ th roots of unity, so that the integral indices in the
  exponent have induced upon them an lagebra modulo $p$. The
  primality of $p$ gives us the structure of a field,
  something that can be obtained for powers of primes.

  For $N = p^{n}$ with $p \neq 2$, we can find $N+1$ bases,
  the first is the standard basis
  \[
    (v_k^0)_l = \delta_{kl},
    \quad k,l \in \mathbb F_{p^n}.
  \] 
  The other $N$ bases are given by
  \[
    (v_k^r)_l
    = \frac{1}{\sqrt{N}} e^{(2\pi i / p)\Tr(rl^2+kl)},
    \quad r,k,l \in \mathbb F_{p^n}, \quad r \neq = 0.
  \] 
  The proof of the unbiasedness relies on the result from
  field theory that states
  \[
    \left|
    \sum_{j \in \mathbb F}^{} e^{(2\pi i / p)\Tr(mj^2+nj)}
    \right|
    = \sqrt{p^n},
    \quad (m \neq 0, p \text{ an odd prime}).
  \] 
  
  Wootters then goes on to give an expression the only
  involves the underlying prime field basis by using the
  expansion coefficients in some field basis.

  By anaology he uses this construction to obtain a versión
  for the case $N = 2^{n}$.

  Wootters concludes that having shown that if $N$ is a
  power a of a prime, then a complete set does exist. If $N$ 
  is not a power of a prime, then we don't know whether such
  a set of bases exists or not. He then mentions that it is
  clear that if such bases exist at all (for example if
  there are seven mutually unbiased bases for a six
  dimensional complex vector space), then any procedure for
  obtaining them must be very different from his procedure.

  \section{Mutually unbiased bases, spherical designs, and
  Frames}

  Two quantum mechanical observables $O$ and $O'$ are called
  complementary if and only if the precise knowledge of one
  of them implies that all possible outcomes are equally
  probable when measuring the other. Complementarity of $O$ 
  and $O'$ implies that if a quantum system is prepared in
  an eigenstate $b'$ of the observable $O'$, and $O$ is
  subsequently measure, then the probability of finding the
  system after the measurement in the state $b \in B$ is
  given by $|\langle b|b' \rangle|^2 = 1 / d$. This leads us
  to the definition of mutually unbiased bases: we call two
  orthonormal bases $B$ and $C$ of $\C^{d}$ mutually
  unbiased if
  \[
    |\langle b|c \rangle|^2 = \frac{1}{d},
  \] 
  holds for all $b \in B$ and $c \in C$. Ivanovic showed
  that complete measurements of at least $d + 1$ observables
  are needed to reconstruct the density matrix. The lower
  bound is obtained when $d + 1$ non-degenerate pairwise
  complementary observables are used.

  There are several constructions known to obtain MUBs. For
  prime power dimension the problem is completely solved.

  \begin{itemize}
    \item I (Alltop) Let $q$ be an odd prime power. For each
      pair $(a,b) \in \mathbb F^2_q$ we define the vector
      \[
        \ket{v_{a,b}}
        = q^{-1 / 2} \left( \omega_p^{\Tr((x+a)^3+b(x+a))}
        \right)_{x \in \mathbb F_q} \in \C^{q},
      \] 
      where $\omega_p = \exp(2\pi i / p)$. Then the standard
      basis together with the bases $B_a = \{\ket{v_{a,b}}
      | b \in \mathbb F_q\}$, $a\in \mathbb F_q$ form a set
      of $q + 1$ mutually unbiased bases of $\C^{q}$.
    \item II (Wootters and Fields) Let $q$ be an odd prime
      power. For each pair $(a,b) \in \mathbb F^2_q$ we
      define a vector
      \[
        \ket{v_{a,b}} = q^{-1 / 2} \left(
        \omega_p^{\Tr(ax^2+bx)} \right)_{x \in \mathbb F_q}
        \in \C^{q},
      \] 
      where $\omega_p = \exp(2\pi i/p)$. Then the standard
      basis together with the bases $B_a = \{\ket{v_{a,b}} |
      b \in \mathbb F_q\}$, $a \in \mathbb F_q$, form a set
      of $q + 1$ mutually unbiased bases of $\C^{q}$.
  \end{itemize}
  It can be shown that the bases from construction I and II
  are equivalent. But both constructions can only be used in
  case $q$ is a prime power of an \textit{odd} prime number.
  In the case $q$ is a power of two it is also possible to
  construct maximally sets of MUBs, however the finite
  fields used in constructions I and II have to be replaced
  by finite Galois rings.
  \begin{itemize}
    \item III (Galois rings) Let $GR(4,n)$ be a finite
      Galois ring of size $4^{n}$ with Teichmüller set
      $\mathcal T_n$. Define
      \[
        \ket{v_{a,b}}
        = 2^{-n / 2} \left( \exp\left( \frac{2\pi i}{4}
        \Tr(a+2b)x \right) \right)_{x \in \mathcal T_n}.
      \] 
      Then the standard basis together with the bases $B_a =
      \{\ket{v_{a,b}} | b \in \mathcal T_n\}$, $a \in
      \mathcal T_n$, form a set of $2^{n}+1$ mutually
      unbiased bases of $\C^{2^{n}}$.
  \end{itemize}

  Another method to construct sets of MUBs is based on the
  idea to partition a unitary error basis, which in
  dimensión $n$, is vector space basis of $\C^{n \times n}$ 
  which consists of unitary operations with the additional
  property that they are orthogonal with respect to the
  Hilbert-Schmidt producto $\braket{A|B} = \Tr(A^{*}B)$.

  \begin{itemize}
    \item IV (Bandyopadhyay) Suppose there exist subsets
      $\mathcal C_1,\ldots,\mathcal C_m$ of unitary error
      basis $\mathcal B$ such that $|\mathcal C_i| = d$,
      $\mathcal C_i \cap \mathcal C_j = \{1_d\}$ for $i \neq
      j$, and the elements of $\mathcal C_i$ pairwise
      commute. Let $B_i$ be a matrix which diagonalizes
      $\mathcal C_i$. Then $B_1,\ldots,B_m$ are MUBs.
  \end{itemize}

  \section{Constructions of mutually unbiased bases
  (Klappenecker)}

  The notion of mutually unbiased bases emerged in the
  literature of quantum mechanics in 1960 from the works of
  Schwinger. Two orthonormal bases $B$ and $B'$ of the
  vector space $\C^{d}$ are called mutually unbiased if and
  only if $|\braket{b|b'}|^2 = 1 / d$ holds for all $b \in
  B$ and $b' \in B'$. Schwinger realized that no information
  can be retrieved when a quantum system which is prepared
  in a basis state from $B'$ is measured with respect to the
  basis $B$.

  Any collection of pairwise mutually unbiased bases of
  $\C^{d}$ has cardinality $d+1$ or less. Let $N(d)$ denote
  the maximum cardinality of any set containing pairwise
  mutually unbiased bases of $\C^{d}$. It is known that
  $N(d) = d + 1$ holds when $d$ is a prime power. Here a
  simplified proof is derived using Weil-type exponential
  sums, for odd prime power dimensions. For even prime power
  dimensions, exponential sums over Galois \textit{rings}
  are used. 

  \subsection{Odd prime powers}

  Let $\mathbb F_q$ be a finte field with $q$ elements which
  has odd characteristic $p$. Each nonzero elemento $x \in
  \mathbb F_q$ defines a non-trivial additive character
  $\mathbb F_q \to \C^{\times}$ by
  \[
    y \mapsto \omega_p^{\Tr(xy)},
  \] 
  where $\omega_p = \exp(2\pi i / p)$ is a primitive $p$-th
  root of unity.

  \begin{lemma}
    Let $\mathbb F_q$ be a finite field of odd
    characteristic and $\chi$ a non-trivial additive
    character of $\mathbb F_q$. Let $p(X) \in \mathbb
    F_q[X]$ be a polynomial of degree 2. Then
    \[
      \left|
      \sum_{x \in \mathbb F_q}^{} \chi(p(x))
      \right|
      = \sqrt{q}.
    \] 
  \end{lemma}

  \begin{theorem}
    Let $\mathbb F_q$ be a finite fiel of charactersitic $p
    \geq 5$. Let $B_\alpha$ denote the set of vectors
    \[
      B_\alpha
      = \{b_{\lambda,\alpha} | \lambda \in \mathbb F_q\},
      \quad 
      b_{\lambda,\alpha}
      = \frac{1}{\sqrt{q}} \left(
      \omega_p^{\Tr((k+\alpha)^3+\lambda(k+\alpha))}
      \right)_{k \in \mathbb F_q}.
    \] 
    The standard basis and the sets $B_\alpha$, with $\alpha
    \in \mathbb F_q$, form an extremal set of $q + 1$ 
    mutually unbiased bases of the vector space $\C^{q}$.
  \end{theorem}

  A feature of the previous construction is that knowledge
  of one basis $B_\alpha$ is sufficient because shifting the
  indices by adding a field element yields the other bases.
  Although the construction does not work in characteristic
  2 and 3. Ivanovic gave explicit constructions of $p + 1$ 
  mutually unbiased bases of $\C^{p}$ for $p$ prime. His
  construction was later generalized in the influential
  paper by Wootters and Fields, who gave the first proof of
  the following theorem. The proof was rephrased by
  CHaturvedi and an alternate proof was given by
  Bandyopadhyay.

  \begin{theorem}
    Let $\mathbb F_q$ be a finite field with odd
    characteristic $p$. Denote by $B_a = \{v_{a,b} | b \in
    \mathbb F_q\}$ the set of vectors given by
    \[
      v_{a,b} 
      = q^{-1/2} \left( \omega_p^{\Tr(ax^2+bx)} \right)_{x
        \in \mathbb F_q}.
    \] 
    The standard basis and the sets $B_a$, with $a \in
    \mathbb F_q$, form an extremal set of $q + 1$ mutually
    unbiased bases of $\C^{q}$.
  \end{theorem}

  \subsection{Even prime powers}

  For the case when $q$ is a power of two we cannot use Weil
  sums because Lemma 1 does not apply in even
  characteristics. Howerver, it turns out the exponential
  sums over a finite Galois ring can serve as a substitute.
  Let $\Z_4$ denote the residue class ring of integers
  modulo 4. Denote by $\langle 2 \rangle$ the ideal
  generated by 2 in $\Z_4[x]$. A monic polynomial $h(x) \in
  \Z_4[x]$ is called basi primitive if and only if its image
  in $\Z_4[x] / \langle 2 \rangle \cong \Z_2[x]$ under the
  canonical map is a primitive polynomial in $\Z_2[x]$. Let
  $h(x)$ be a monic basic primitive polynomial of degree
  $n$. The ring $GR(4,n) = \Z_4[x] / \langle h(x) \rangle$ 
  is called the Galois ring of degree $n$ over $\Z_4$.

  The ring $GR(4,n)$ has $4^{n}$ elements. The element $\xi
  = x + \langle h(x) \rangle$ is of order $2^{n}-1$. Any
  element $r \in GR(4,n)$ can be uniquely written in the
  form $r = a + 2b$, where $a,b \in \mathcal T_n =
  \{0,1,\xi,\ldots,\xi^{2^{n}-2}\}$. This representation in
  terms of the Teichmüller set $\mathcal T_n$ is convenient,
  since it allows us to characterize the units of GR(4,n) as
  the elements of $a + 2b$ with $a \neq 0$.

  The automorphism $\sigma : GR(4,n) \to GR(4,n)$ defined by
  $\sigma(a+2b) = a^2 + 2b^2$ is called the Frobenius
  automorphism. This map leaves the elements of the prime
  ring $\Z_4$ fixed. All automorphisms of $GR(4,n)$ are of
  the form $\sigma^{k}$ for some integer $k \geq 0$. The
  trace map $\Tr : GR(4,n) \to \Z_4$ is defined by $\Tr(x) =
  \sum_{k=0}^{n-1} \sigma^{k}(x).$

  \begin{theorem}
    Let $GR(4,n)$ be a finite Galois ring with Teichmüller
    set $\mathcal T_n$. For $a \in \mathcal T_n$, denote by
    $M_a = \{v_{a,b} | b \in \mathcal T_n\}$ the set of
    vectors given by
    \[
      v_{a,b}
      = 2^{-n / 2} \left( \exp\left( \frac{2\pi i}{4}
      \Tr(a+2b)x \right) \right)_{x \in \mathcal T_n}.
    \] 
    The standard basis and the set $M_a$, with $a \in
    \mathcal T_n$, form an extremal set of $2^{n}+1$ mutally
    unbiased bases of $\C^{2^{n}}$.
  \end{theorem}

  \section{On the monomiality of nice error bases
  (Klappenecker}

  Unitary error bases generalize the Pauli matrices to
  higher dimensional systems. A unitary error basis is, by
  definition, an orthonormal basis of the vector space of
  complex $d \times d$ matrices with respect to the inner
  product $\langle A,B \rangle = 1 / d \Tr(A^*B)$. Such
  matrices form a natural generalization of the $2 \times 2$ 
  Pauli matrices to higher dimensional systems.

  \subsection{Construction of unitary error bases}

  A unitary error basis is a set $\mathcal E$ of $d^2$ 
  unitary $d \times d$ matrices such that $\Tr(E^*F) = 0$ 
  for all distinct $E,F \in \mathcal E$. The set $\mathcal
  P$ of Pauli matrices provides the most well-known example
  \[
    \mathcal P
    = \left\{
      \begin{pmatrix} 1 & 0\\ 0 & 1 \end{pmatrix} 
      \begin{pmatrix} 0 & 1\\ 1 & 0 \end{pmatrix} 
      \begin{pmatrix} 1 & 0\\ 0 & -1 \end{pmatrix} 
      \begin{pmatrix} 0 & -i\\ i & 0 \end{pmatrix} 
    \right\}.
  \] 
  Unitary error bases generalize this example to arbitrary
  dimensions. The nonbinary case is more interesting, since
  there exist different, nonequivalent, error bases. 

  \subsubsection{Equivalence}

  Let $\mathcal E$ and $\mathcal E'$ be two unitary error
  bases in $d$ dimensions. We say that $\mathcal E$ and
  $\mathcal E'$ are equivalent, if and only if there exist
  unitary matrices $A, B \in \mathcal U(d)$ and constants
  $c_E \in \mathcal U(1)$, $E \in \mathcal E$, such that
  \[
    \mathcal E'
    = \{c_E AEB : E \in \mathcal E\}.
  \] 
  It is not hard to prove that any unitary error basis in
  dimension 2 is equivalent to the Pauli basis.

  \subsubsection{Nice error bases}

  Let $G$ be a group of order $d^2$ with identity element 1.
  A nice error basis in $d$ dimensions is given by a set
  $\mathcal E = \{\rho(g) \in \mathcal U(n) : g \in G\}$ of
  unitary matrices such that
  \begin{itemize}
    \item $\rho(1)$ is the identity matrix.
    \item $\Tr(\rho(g)) = 0$ for all $g \in G \setminus
      \{1\}$.
    \item $\rho(g)\rho(h) = \omega(g,h)\rho(gh)$ for all
      $g,h \in G$,
  \end{itemize}
  where $\omega(g,h)$ is a phase factor. A nice error basis
  is a unitary error basis. As an example, let $d \geq 2$ be
  an integer $\omega = \exp(2\pi i / d)$. Let $Z_d$ denote
  the diagonal matrix
  $\diag(1,\omega,\omega^2,\ldots,\omega^{d-1})$. Then
  \[
    \mathcal E_d
    := \left\{
      X^i_d Z^j_d : (i,j) \in \Z_d \times \Z_d
    \right\}
  \] 
  is a nice error basis.

  There are an abundant amount of unitary error basis which
  are not equivalent to a nice error basis. Klappenacker
  calls these \textit{wicked} basis.

  \section{On mutually unbiased bases (Thomas Durt)}

  Mutually unbiased bases for quantum degrees of freedom are
  central to all theoretical investigations and practical
  exploitations of complementary properties. One can find
  maximal sets of $N+1$ mutually unbiased bases in Hilbert
  spaces of prime-power dimension $N = p^{M}$, and there is
  a continuum of mutually unbiased bases for a continuous
  degree of freedom. But not a single example of maximal set
  is known if the dimension is another composite number.a

  Two orthonormal bases of a Hilbert space are said to be
  mutually unbiased if the transition probabilities from
  each state in one basis to all states of the other basis
  are the same irrespective of which pair of states is
  choesn. That is to say, if the physical system is prepared
  in a state of the first basis, then all outcomes are
  equally probable when we conduct a measurement that probes
  for the states of the second basis. When the Hilbert space
  dimension $N$ is a prime power, $N = p^{M}$, there exist
  sets of $N+1$ mutually unbiased bases. We can reconstruct
  the statistical operator that characterizes a quantum
  state, i.e., we can perform full tomography or complete
  quantum state determination.

  Various methods have been used for the construction of
  maximal sets of MUB, including the Galois-Fourier method,
  methods based on the generalized Pauli matrices, discrete
  Wigner functions, abelian subgroups, mutually orthogonal
  Latin squares and finite-geometry methods.

  As in all constructions of MUB in prime power dimension, a
  crucial element is a finite commutative division ring -- a
  Galois field of $N$ elements. Finite fields with $N$ 
  elements exist if and only if $N$ is a power of a prime
  and the mathematical properties of Galois fields are
  exploited in all constructions of maximal sets of MUB. The
  discrete analog of Wigner's continuous phase space
  function, which -- jointly with its Fourier partner, the
  analog of Weyl's characteristic function -- is treated
  later. Durt comments on the covariance of the Wigner-type
  operator basis and disucss the $N \to \infty$ limit of
  continuous degrees of freedom. The relation to finite
  affine planes provides insight to the underlying geometry.

  \subsection{Elements of quantum kinematics}

  As emphasized by Bohr in his 1927 Como lecture, quantum
  systems have properties that are complementary: equally
  real but mutually exclusive. If one such property is known
  accurately (the outcome of a measurement can be predited
  with certainty), then the complementary property is completely
  unknown (all outcomes are equally liekly).  Familiar
  examples are the position and momentum of a particle
  moving along a line, and the $x$ and $z$ spin components
  of a spin-$\frac{1}{2}$ object. A pair of observables, $A$ 
  and $B$, complementary if their eigenvalues are not
  degenerate and the sets of normalized kets $\ket{a_j}$ and
  $\ket{b_k}$ that describe states with predictable
  measurement outcomes for $A$ and $B$ respectively, are MU,
  \[
    |\braket{a_j|b_k}|^2 = \frac{1}{N}.
  \] 
  The particular eigenvalues are irrelevant, we just need to
  know that they are not degenerate. It follows in
  particular that, if $A$ and $B$ are complementary, then
  $\alpha A$ and $\beta B$ with $\alpha \beta \neq 0$ are
  complementary as well. And if a unitary transform turns
  $A$ into $A'$ and $B$ into $B'$, then the pair $A',B'$ is
  complementary if the pair $A,B$ is.

  Whenever we need to be specific about the observables
  associated with a basis, we will follow the guidance of
  Weyl and Schwinger and choose unitary operators to
  represent physical quantities. In this context, these will
  be nondegenerate cyclic operators with period $N$,
  \[
    A^{N} = 1,
    \quad
    B^{N} = 1.
  \] 
  The eigenvalues of $A$ and $B$ are then the $N$ different
  $Nth$ roots of unity
  \[
    A\ket{a_j} = \gamma_N^j \ket{a_j},
    \quad
    B\ket{b_k} = \gamma_N^k \ket{b_k},
    \quad
    \gamma_N = e^{2\pi i / N}.
  \] 
  That these cyclic operators are a pair of complementary
  operators can be stated as
  \[
    \frac{1}{N} \Tr(A^{m}B^{n}) = \delta_{m,0}\delta_{n,0}.
  \] 
  
  There always exists a pair of complementary observables
  for each quantum degree of freedom. Select an orthonormal
  reference basis $\ket 0, \ket 1,\ldots,\ket{N-1}$ 
  referred to as the computational basis. Then define a
  second orthonormal basis
  $\ket{\hat{0}},\ket{\hat{1}},\ldots,\ket{\hat{N-1}}$ by
  means of the discrete quantum Fourier transformation
  \[
    \ket{\hat{j}}
    = \frac{1}{\sqrt{N}} \sum_{k=0}^{N-1} \ket k
    \gamma_N^{-jk}.
  \] 
  In analogy with the Pauli operators $\sigma_x$ and
  $\sigma_z$, we introduce the cyclic operators $X$ and $Z$ 
  in accordance with
  \[
    X\ket{\hat{j}} = \gamma_N^{j} \ket{\hat{j}},
    \quad X^{N} = 1,
    \quad \text{ and } \quad
    Z\ket{k} = \gamma_N^{k} \ket k,
    \quad Z^{N} = 1.
  \] 
  By definition, we note the $X$ and $Z$ are unitary shift
  operators that permute the kets or bras of the respective
  other basis cyclically,
  \[
    X \ket k = \ket{k+1},
    \quad k = 0,1,\ldots,N-2,
    \quad X \ket{N-1} = \ket 0,
  \] 
  and
  \[
    \bra{\hat{j}} Z = \bra{\hat{j+1}}, 
    \quad j = 0,1,\ldots,N-2,
    \quad \bra{\hat{N-1}} Z = \bra{\hat{0}}.
  \] 
  The fundamental Weyl commutation rule $ZX = \gamma_N XZ$ 
  holds. It is the analog of the familiar $N = 2$ identity
  $\sigma_z \sigma_x = -\sigma_x \sigma_z$ and is more
  generally, stated as
  \[
    X^{m}Z^{n} = \gamma_N^{-mn} Z^{n}X^{m}.
  \] 
  The pair $X,Z$ is algebraically complete, in the sense
  that there are no operators that are not lienar
  combinations of products of powers of $X$ and $Z$.

  \subsubsection{The Heisenberg-Weyl group; the Clifford
  group}

  Supplemented with powers of $\gamma_N$, the $XZ$-ordered
  products 
  \[
    Y_{l,m,n} = \gamma_N^{l} X^{m}Z^{n},
    \quad l,m,n = 0,1,\ldots,N-1,
  \] 
  make up the Heisenberg-Weyl group of unitary operators,
  also called the generalized Pauli group, with operator
  multiplciation as the composition,
  \[
    Y_{l_1,m_1,n_1} Y_{l_2,m_2,n_2}
    = Y_{l_1+l_2+n_1m_2,m_1+m_2,n_1+n_2},
  \] 
  where all subscripts are understood as integers modulo
  $N$.

  If $N$ is an odd prime, $N = 3,5,7,11,13,\ldots$ then all
  unitary Heisenberg-Weyl oeprators $Y_{l,m,n}$ are cyclic
  with period $N$, except for the identity $Y_{0,0,0}$.
  Further we observe that the $N+1$ operators
  \[
    X,XZ,XZ^2,\ldots,XZ^{N-1},Z
  \] 
  are pairwise complementary. It follows that the $N+1$ 
  bases of eigenkets, one for each of the operators in are
  MU. In addition to the eigenbases of $X$ and $Z$ there are
  $N-1$ more such bases.

  For $N$ prime, explicitly, ket $\ket{i,k}$, the $k$ th
  eigenket of the $i$th basis, $XZ^{i}\ket{i,k} =
  \gamma_N^{k}\ket{i,k}$, is given by (for $N$ odd)
  \[
    \ket{i,k}
    = \frac{1}{\sqrt{N}} \sum_{l=0}^{N-1} \gamma_N^{-kl}
    \gamma_N^{il(l-1) / 2} \ket l,
    \, i = 0,1,2,\ldots,N-1
  \] 
  in terms of the reference basis of eigenkets of $Z$. For
  $i = 0$ we have the eigenstates of $X$, $\ket{\hat{k}} =
  \ket{0,k}$. This last equation correctly states the
  eigenkets of $XZ^{i}$ for all odd $N$, but these bases are
  pairwise MU only if $N$ is prime. One can give a similar
  expression for $\ket{i,k}$ when $N$ is even.

  In summary, we can systematically construct $N+1$ bases
  that are MU if $N$ is prime. As noted, the construction
  based on cyclic operators does not work if $N$ is
  composite. Yet if $N = p^{M}$ is a power of a prime, it is
  possible to modify the construction such that it does work
  in a closely analogous way, by replacing the modulo $N$ 
  shifts by shifts of a Galois field arithmetic that treats
  the $N$-dimensional degree of freedom systematically as
  composed of $M$ $p$-dimensional constituents.

  \subsection{Construction of mutually unbiased bases in
  prime power dimensions}

  \section{A new proof for the existence of mutually
  unbiased bases (Bandyopadhyay)}

  A $d$-level quantum system is described by a density
  operator $\rho$ that requires $d^2-1$ real numbers for its
  complete specification. A maximal orthogonal quantum test
  performed on such a system has, without degeneracy, $d$ 
  possible outcomes, providing $d-1$ independent
  probabilities. It follows that in principle one requires
  at least $d+1$ different orthogonal measurements for
  complete state determination.

  Since the quantum mechanical description of a physical
  system is characterized in terms of probabilities of
  outcomes of conceivable experiments consisten with quantum
  formalism, in order to obtain full information about the
  system under consideration we need to perform measreuments
  on a large number of identically prepared copies of the
  system. A minimal set of measreuments can be reasonably
  close to an optimal set if they mutually differ as much as
  possible, thereby ruling out possible overlaps in the
  results.

  It has been shown that measurments in a special class of
  bases, i.e., MUBs not only form a miniaml set but also
  provie the optimal way of determinig a quantum state.
  Mutually unbiased measurements correspond to measurements
  that are as different as they can be. Wootters and Fields
  showed that measurements in MUB provide the minimal as
  well as optimal way of complete specification of the
  density marix. By explicit construction they showed the
  existence of MUB for prime pwoer dimensions and proved
  that for any dimension $d$ there can be at most $d+1$ MUB.

  The author gives a constructive proof of the results
  earlier obtained by Ivanovic, Wootters and Fields with a
  distinct method. This method offers an interesting
  connection between maximal commuting bases of orthogonal
  unitary matrices and mutually unbaised bases. And another
  advantage is the they provide an explicit construction of
  the MUB observables as tensor products of the Pauli
  amtrices for dimensions $d = 2^{m}$. When $d = 2$ the
  mutually unbiased oeprators are the three Pauli matrices,
  but this cannot be generalized in a straightforward way to
  higher dimension.

  Let $B_1 = \{\ket{\phi_1},\ldots,\ket{\phi_d}\}$ and $B_2
  = \{\ket{\psi_1},\ldots,\ket{\psi_d}\}$ be two orthonormal
  bases in the $d$ dimensional state space. They are said to
  mutually unbiased bases if and only if
  $|\braket{\phi_i|\psi_j}| = \frac{1}{\sqrt{d}}$ for every
  $i,j = 1,\ldots,d$. A set of $\{B_1,\ldots,B_m\}$ of
  orthonormal bases in $\C^{d}$ is called a set of mutually
  unbaised bases if each pair of bases $B_i$ and $B_j$ are
  mutually unbiased.

  First the authors show the existence of $p+1$ MUBs int he
  space $\C^{p}$ for $p$ prime, a result which is first
  shown by Ivanovic in an explicit definition of the bases.
  The authors here show that these bases are in fact bases
  each consisting of eigenvectors of the unitary operators
  \[
    Z,X,XZ,\ldots,XZd^{-1},
  \] 
  where $X$ and $Z$ are generalizations of Pauli oeprators
  to the quantum systems with more than two states.

  They later show that there is a useful connetion between
  mutually unbaised bases and special types of basesa for
  the space of square matrices. These bases consist of
  orthogonal unitary matrices which can be grouped in
  maximal classes of commuting matrices. They then transfer
  these ideas to the construction of MUBs over  $\C^{d}$ 
  where $d$ is a prime power. The idea behind the
  construction is as follows. When $d = p^{m}$, imagine the
  system consists of $m$ subsystems each of dimension $p$.
  Then the total number of measurements on the whole system,
  viewed as performing measurment on every subsytem in their
  respective $MUB$ is $(p+1)^{m}$. They then show that these
  $(p+1)^{m}$ oeprators fall into $p^{m}+1$ maximal
  noncomuuting classes where members of each class commute
  among themeselves. The bases formed by eigenvectors of
  each such mutually noncummuting class are mutually
  unbiased.

  \subsection{Construction of sets of MUB for prime
  dimensions}

  Ivanovic for the first time showed that for any prime
  dimension $d$ there is a set of $d+1$ mutually unbiased
  bases. In that paper the bases are given explicitly. Here
  we show that there is a nice symmetrical structure behind
  these bases, and their existence can be derived as a
  consequence of propoerites of Pauli operators on $d$-state
  systems.

  \begin{theorem}
    Let $B_1 = \{\ket{\phi_1},\ldots,\ket{\phi_d}\}$ be an
    orthonormal basis in $\C^{d}$. Suppose that there is a
    unitary operator $V$ such that $V\ket{\phi_j} = \beta_j
    \ket{\phi_{j+1}}$, where $|\beta_j| = 1$ and
    $\ket{\phi_{d+1}} = \ket{\phi_1}$; i.e., $V$ applies a
    cyclic shift modulo a phase on the elements of the basis
    $B_1$. Assume that the orthonormal basis $B_2 =
    \{\ket{\psi_1},\ldots,\ket{\psi_d}\}$ consists of
    eigenvectors of $V$. Then $B_1$ and $B_2$ are MUB.
  \end{theorem}

  For the moment, we suppose that $d$ is a prime number, and
  all algebraic oeprations are modulo $d$. We consider
  $\{\ket 0, \ket 1, \ldots, \ket{d-1}\}$ as the standard
  basis of $\C^{d}$. We define the unitary operators $X_d$ 
  and $Z_d$ over $\C^{d}$, as a natural generalization of
  Pauli operators $\sigma_x$ and $\sigma_z$:
  \begin{align*}
    X_d \ket j &= \ket{j+1},\\
    Z_d \ket j &= \omega^{j} \ket j,
  \end{align*}
  where $\omega$ is the $d$-th root of unity, $\omega =
  e^{2\pi i / d}$. We are interested in the unitary
  operators of the form $X_d Z_d^{k}$ whose action is given
  by
  \[
    X_d Z_d^{k} \ket j
    = \left( \omega \right)^{j} \ket{j+1}.
  \] 
  For $0 \leq k,l \leq d - 1$, the eigenvectors of $X_d
  Z_d^{k}$ are cyclically shifted under the action of $X_d
  Z_d^{l}$. Noting that the standard basis is the set of
  eigenvetors of $Z_d$ we have the following first result.
  \begin{theorem}
    For any prime $d$, the set of the bases each consisting
    of the eigenvectors of
    \[
      Z_d, X_d, X_d Z_d, X_d Z_d^2,\ldots,X_d Z_d^{d-1}
    \] 
    form a set of $d+1$ mutually unbiased bases.
  \end{theorem}

  \subsection{Bases for unitary operators and MUB}

  Here Bandyopaihyay studies the close relation between MUB
  and a special type of bases for $\mathcal M_d(\C)$.

  \begin{lemma}
    There are at most $d$ pairwise orthogonal commuting
    unitary matrices in $\mathcal M_d(\C)$.
  \end{lemma}

  Let $\mathcal B = \{U_1,U_2,\ldots,U_{d^2}\}$ be a basis
  of unitary matrices for $\mathcal M_d(\C)$. Without loss
  of generality, we can assume that $U_1 = 1$. We say that
  the basis $\mathcal{B}$ is a \textit{maximal commuting
  basis} for $\mathcal{M}_d(\C)$ if $\mathcal{B}$ can be
  partitioned as
  \[
    \mathcal{B}
    = \{1\} \bigcup_{} \mathcal{C_1} \bigcup_{} \ldots
    \bigcup_{} \mathcal{C}_{d+1},
  \] 
  where each class $\mathcal{C}_j$ contains exactly $d-1$ 
  commuting matrices from $\mathcal{B}$. We note that $\{1\}
  \bigcup_{} \mathcal{C}_j$ is a set of $d$ commuting
  orthogonal unitary matrices.

  \begin{theorem}
    If there is a maximal commuting basis of orthogonal
    unitary matrices in $\mathcal{M}_d(\C)$, then there is a
    set of $d+1$ mutually unbiased bases.
  \end{theorem}

  \begin{theorem}
    Let $B_1,\ldots,B_m$ be a set of MUB in $\C^{d}$. Then
    there are $m$ classes
    $\mathcal{C}_1,\ldots,\mathcal{C}_m$ each consisting of
    $d$ commuting unitary matrices such that matrices in
    $\mathcal{C}_1 \bigcup_{} \ldots \bigcup_{}
    \mathcal{C}_m$ are pairwise orthogonal.
  \end{theorem}

  \subsection{Construction of a set of MUB for prime powers}

  To construct a maximal set of MUB in $\C^{p^{m}}$ where
  $p$ is a prime number, we consider the Hilbert space
  $\mathcal{H}$ as a tensor product of $m$ copies of
  $\C^{p}$,
  \[
    \mathcal{H} = \C^{p} \otimes \ldots \otimes \C^{p}.
  \] 
  Just like the case of $\C^{p}$, we build a set of MUB as
  the sets of eigenvectors of special types of unitary
  operators on the background space $\mathcal{H}$. On the
  space $\C^{p}$ we considered the generalized Pauli
  operators $X_p$ and $Z_p$. On the space $\mathcal{H}$, we
  consider the tensor products of operators $X_p$ and $Z_p$.

  Consider the finite field $\mathbb F_p =
  \{0,1,\ldots,p-1\}$. Let $\omega = e^{2\pi i / d}$ be a
  primitive $p$-th root of unity. Then
  \[
    Z_p X_p = \omega X_p Z_p.
  \] 
  Therefore, if $U_1 = X_p^{k_1} Z_p^{l_1}$ and $U_2 =
  X_p^{k_2}Z_p^{l_2}$, then
  \[
    U_2 U_1 = \omega^{k_1l_2 - k_2l_1} U_1 U_2.
  \] 
  It is of our interest, only the unitary operators on
  $\mathcal{H}$ of the form
  \[
    U = M_1 \otimes \ldots \otimes M_m
  \] 
  where $M_j = X_p^{k_j} Z_p^{l_j}$ for $0 \leq k_j,l_j \leq
  p - 1$. To describe an oeprator of this form, it is
  enought to specify the integers $k_j$ and $l_j$. So we can
  represent such an operator by the following vector of
  length $2m$ over the field $\mathbb F_p$:
  \[
    (k_1,\ldots,k_m,l_1,\ldots,l_m),
  \] 
  or
  \[
    X_p(k_1,\ldots,k_m)Z_p(l_1,\ldots,l_m).
  \] 
  If $\alpha = (k_1,\ldots,k_m)$ and $\beta =
  (l_1,\ldots,l_m)$, then $\alpha, \beta \in \mathbb
  F_p^{m}$ and the operators are
  \[
    X_p(\alpha) Z_p(\beta).
  \] 

  The Pauli group $\mathbb P(p,m)$ is the group of all
  unitary operators on $\mathcal{H} = \C^{p} \otimes \cdots
  \C^{p}$ of the form
  \[
    \omega^{j} X_p(\alpha) Z_p(\beta),
  \] 
  for some integer $j \geq 0$ and vectors $\alpha, \beta \in
  \mathbb F_p^{m}$ where $\omega = \exp(2\pi i / p)$. For
  now we are only concerned with the subset $\mathbb
  P_0(p,m)$ of the operators with $j = 0$. If the operators
  $U$ and $U'$ in $\mathbb P_0(p,m)$ are represented by the
  vectors
  \[
    (k_1,\ldots,k_m,l_1,\ldots,l_m)
  \] 
  and
  \[
    (k_1',\ldots,k_m',l_1',\ldots,l_m'),
  \] 
  then $U$ and $U'$ are commuting if and only if
  \[
    \sum_{j=1}^{m} k_j l_j' - \sum_{j=1}^{m} k_j'l_j = 0
    \mod p.
  \] 
  The standard basis of the Hilbert space $\C^{p} \otimes
  \cdots \otimes \C^{p}$ consists of the vectors $\ket{j_1
  \cdots j_m}$ where $(j_1,\ldots,j_m) \in \mathbb
  F_p^{m}$. Then
  \[
    X_p(\alpha)Z_p(\beta) \ket{j_1\cdots j_m}
    = \omega^{j_1\beta_1+\ldots+j_m\beta_m}
    \ket{(j_1+\alpha_1)\cdots(j_m+\alpha_m)}.
  \] 
  Equivalently
  \[
    X_p(\alpha)Z_p(\beta)\ket a
    = \omega^{\alpha \cdot \beta} \ket{a+\alpha},
  \] 
  \[
    X_p(\alpha)Z_p(\beta)
    = \sum_{a \in \mathbb F_p^{m}}^{} \omega^{a \cdot \beta}
    \ket{a+\alpha} \bra{a}
  \] 
  where the operations are modulo $p$.

  \begin{theorem}
    Let $U = X_p(\alpha)Z_p(\beta)$ and $U' =
    X_p(\alpha')Z_p(\beta')$ be operators in $\mathbb
    P_0(p,m)$. If $U \neq U'$, i.e., $(\alpha,\beta) \neq
    (\alpha',\beta')$, then the operators $U$ and $U'$ are
    orthogonal.
  \end{theorem}

  \subsection{The general construction}

  The maximal commuting orthogonal basis for
  $\mathcal{M}_{p^{m}}(\C)$ with partition of the form is
  such that each class $\{1\} \bigcup_{} \mathcal{C}_j$ is a
  linear space of oeprators in the Pauli group $\mathbb
  P(p,m)$. Let
  \[
    X_p(\alpha_1)Z_p(\beta_1), \ldots,
    X_p(\alpha_{p^{m}})Z_p(\beta_{p^{m}})
  \] 
  be the operators in the class $\{1\} \bigcup_{}
  \mathcal{C}_j$. We say that this class is linear if the
  set of the vectors
  \[
    \mathcal{E}_{j}
    =
    \{(\alpha_1,\beta_1),\ldots,(\alpha_{p^{m}},\beta_{p^{m}})\}
  \] 
  form an $m$-dimensional subspace of $\mathbb{F_p^{2m}}$.
  In this case, to specify a linear class, it is enough to
  present a basis for the subspace $\mathcal{E}_j$. SUch a
  basis can be represented by an $m \times (2m)$ matrix, so
  instaed of listing all operators in the class
  $\mathcal{C}_1,\ldots,\mathcal{C}_{p^{m}+1}$, we can list
  the $p^{m}+1$ matrices representing the bases of these
  classes.

  Specifically, the bases of linear classes of oeprators in
  this construction are represented by the matrices
  \[
    (0_m, 1_m), (1_m,A_1), \ldots, (1_m,A_{p^{m}}),
  \] 
  where $0_m$ is the all-zero matrix of order $m$ and each
  $A_j$ is an $m \times m$ matrix over $\mathbb{F}_p$. The
  following lemma gives a simple necessary and sufficient
  condition for operators in each class to be commuting.

  \begin{lemma}
    Let $S$ be a set of $m$ operators in $\mathbb{P}_0(p,m)$ 
    and $S$ be represented by the matrix $(1_m,A)$ where $A$ 
    is an $m \times m$ matrix over $\mathbb{F}_p$. Then the
    operators in $S$ are pairwise commuting if and only if
    $A$ is a symmetric matrix.
  \end{lemma}

  \begin{theorem}
    Let $\{A_1,\ldots,A_l\}$ be a set of symmetric $m \times
    m$ matrices over $\mathbb{F}_p$ such that $\det(A_j-A_k)
    \neq 0$ for every $1 \leq j < k \leq l$. Then there is a
    set of $l+1$ mutually unbiased bases on $\C^{p^{m}}$.
  \end{theorem}

  To construct $p^{m}+1$ mutually unbiased bases in
  $\C^{p^{m}}$, we only need to find $m$ symmetric
  nonsingular matrices $B_1,\ldots,B_m \in
  \mathcal{M}_m(\C)$ such that the matrix $\sum_{j=1}^{m}
  b_j B_j$ is also nonsingular, for every nonzero vector
  $(b_1,\ldots,b_m) \in \mathbb{F}_p^{m}$. 

  Wootters and Fields have found the following general
  construction for the matrices $B_1,\ldots,B_m$. Let
  $\gamma_1,\ldots,\gamma_m$ be a basis of
  $\mathbb{F}_{p^{m}}$ as a vector space over
  $\mathbb{F}_p$. Then any element $\gamma_i \gamma_j \in
  \mathbb{F}_{p^{m}}$ can be written uniquely as
  \[
    \gamma_i \gamma_j
    = \sum_{l=1}^{m} b_{ij}^{l} \gamma_l.
  \] 
  Then $B_l = \left( b_{ij}^{l} \right)$.

  \section{Mutually unbiased bases, generalized spin
  matrices and sperability}

  The goal of this article is to provide an algorithm for
  deriving explicit solutions to the MUB problem for $d =
  p^{n}$. It is important to emphasize that the methodology
  gives a specific solution of the MUB problem for $d =
  p^{n}$. Once such a solution is in hand, there are many
  ways to construct other mutually unbaised bases.

  Let $d$ denote the dimension of the finite dimensional
  complex Hilbert space $H$, and the unitary matrices acting
  on $H$ are indexed by subscripts $(j,k)$. Fix an
  orthonormal basis $\{\ket j, j = 0,\ldots,d-1\}$.

  \begin{definition}
    Let $0 \leq j,k < d$. Then define the matrix
    \[
      S_{j,k}
      = \sum_{m=0}^{d-1} \omega^{mj} \ket m \bra{m+k},
    \] 
    where $\omega = \exp(2\pi i / d)$.
  \end{definition}
  The matrices $S_{j,k}$ have null trace except for
  $S_{0,0}$ which is the identity matrix. This set of
  matrices is closed under mutliplication, up to scalar
  multiples of powers of $\omega$.

  \begin{lemma}
    The matrices satisfy $S_{j,k} S_{a,b} = \omega^{ka}
    S_{j+1,k+b}$. Thus, $S_{j,k}$ and $S_{a,b}$ commut if
    and only if $ka = jb$ up to an additive multiple of $d$.
  \end{lemma}

  These matrices are unitary and are also orthogonal to one
  another with respect to the Frobenius inner product on the
  space of $d \times d$ complex matrices, $\Tr(A^{*}B)$.
  Also $(S_{j,k})^{*} = \omega^{jk} S_{-j,-k}$.

  The set $\{S_{j,k} : 0 \leq j,k \leq d - 1\}$ is a set of
  $d^2$ unitary matrices that forms an orthogonal basis for
  the space of $d \times d$ matrices and is closed under
  multiplication, up to multiples of powers of $\omega$.
  Thus they can be regarded as analogues of the Pauli spin
  matrices. These matrices satisfy the initial requirements
  of Bandyopiahyay's theorem, which is based on the fact
  that commuting unitary matrices can be simultaneosly
  diagonalized.

  \subsection{Spin matrices and the MUB problem for $d$ 
  prime}

  Let $d = p$ where $p$ is a prime number. Recall that
  $S_{j,k}$ and $S_{a,b}$ commut if and only if $ka = jb
  \mod p$. Consider the vector space over the finite field
  $\Z_p$, $V_2(\Z_p) = \{(j,k) : j,k \in \Z_p\}$, and define
  a symplectic product
  \[
    \sigma(u, u')
    = kj' - jk' \mod p.
  \] 
  Thus $S_u$ and $S_{u'}$ commute if and only if the
  symplectic product of their vector indices equals zero.
  Once we have the classes of commuting matrices, we can
  make a direct computation or invoke Bandyopiahyay's
  theorem to argue the existence of a complete set of
  mutually unbiased bases. We can construct these bases
  explicitly in terms of the spin matrices.

  \begin{proposition}
    Let $a \in \Z_p$ and define
    \begin{align*}
      C_a &= \{b(1,0) + ba(0,1) = b(1,a) : b \in \Z_p\},\\
      C_\infty &= \{b(0,1) : b \in \Z_p\}
    .\end{align*}
    There are $p$ vectors in each of these $p+1$ classes and
    $C_r \cap C_s = \{(0,0)\}$ for all $r \neq s$ in $I =
    \{0,1,\ldots,p-1,\infty\}$. If $u,v$ are in $C_r$ then
    $\sigma(u,v) = 0$.
  \end{proposition}

  The $C_t$ can be thought of as lines in a two-dimensional
  space. In addition the vectors in $C_t$ can be written as
  a multiple of a single vector $u_t = (j_t,k_t)$ and $C_t$ 
  is an additive subgroup of $V_2(\Z_p)$. The matrices
  associated with $C_t$ are $\{S_{nu_t} : 0 \leq n < p\}$,
  they commut but do not form a multiplicative subgroup of
  the unitary matrices, but we consider $S_{u_t}$ to be the
  generator of $\{S_{nu_t} : 0 \leq n < p\}$.

  Bandyopihyay's theorem guarantees that the orthonormal
  eigenvectors for each class solve the MUB problem, and we
  can use the indicial notation to express the associated
  orthogonal projections explicitly in terms of the unitary
  matrices. For general computations let us define the
  following.

  \begin{definition}
    Let $0 \leq j,k < d$ and $u = (j,k)$. If $d$ is even and
    both $j$ and $k$ are odd, set $\alpha_u = -\exp(\pi i /
    2) = -\omega^{1 / 2}$. Otherwise set $\alpha_u = 1$.
  \end{definition}

  \begin{definition}
    For each $u = (j,k) \neq (0,0)$ and $0 \leq r < d$,
    define
    \[
      P_u(r)
      = \frac{1}{d} \sum_{m=0}^{d-1} \left(\alpha_u \omega^{r}
      S_u\right)^{m},
    \] 
    where $(\alpha_u \omega^{r} S_u)^{0} = S_{0,0}$.
  \end{definition}

  \begin{proposition}
    For $d$ a prime, $\{P_u(r) : 0  \leq r < d\}$ is a
    complete set of mutually orthogonal projections.
  \end{proposition}

  Also, the $P_u(r)$ have trace one and
  \[
    (\alpha_u \omega^{r} S_u)^{t}
    = \sum_{m=0}^{d-1} \omega^{-mt} P_u(m+r).
  \] 
  And so we have obtained a set of $d$ orthogonal,
  one-dimensional projections. The indices of members of a
  commuting class are multiples of a vector $u_t$. Thus if
  $u = bu_t$, then $P_u(r)$ should be $P_{u_t}(s)$ for some
  $s$.

  \begin{proposition}
    If $p > 2$ is prime and $u = bu_t = b(j_t,k_t)$ with $2
    \leq b < p$, then $P_u(r) = P_{u_t}(s)$ where $s =
    b^{-1}\left( r - j_t k_t (b 2) \right)$ and
    $b^{-1}$ is the multiplicative inverse of $b$ modulo
    $p$.
  \end{proposition}

  It is not hard to prove that
  \[
    \Tr\left( P_u(r) P_{u'}(s) \right) 
    = \frac{1}{d}.
  \] 

  \begin{theorem}
    If $p$ is prime, there is a complete set of $p+1$ 
    mutually unbiased bases $B_a$, $0 \leq a < p$, and
    $B_\infty$ that are the normalized eigenvectors of the
    corresponding sets of commuting spin matrices
    $\{S_{b,ba} : b \in \Z_p\} \leftrightarrow C_a$ and
    $\{S_{0,b} : b \in \Z_p\} \leftrightarrow C_\infty$.
    These bases can be computed from the projections.
  \end{theorem}

  \subsection{The MUB problem for $d = p^2$}

  Before showing the procedure for the case $d = p^{m}$, we
  illustrate the case $d = p^2$. The basic strategy is to
  use the indices of the spin matrices to encode
  commutativity and techniques of vector spaces over finite
  fields to define the appropriate classes. The actual MUB
  bases can then be recovered from the classes of commuting
  spin matrices.

  We now consider the tensor products of the form $S_u
  \otimes S_v$ where commutativity is again encoded by the
  indices so that $S_{u_1} \otimes S_{v_1}$ commutes with
  $S_{u_2} \otimes S_{v_2}$ if and only if
  \[
    \sigma(u_1,u_2) + \sigma(v_1,v_2) = 0 \mod p.
  \]
  No we consider vectors in a four dimensional vector space
  over $\Z_p$, $V_4(\Z_p) = \{w = (j,k,a,b) = (u,v)\}$, and
  we define the symplectic product as
  \[
    \sigma(w_1, w_2)
    = \sigma(u_1,u_2) + \sigma(v_1,v_2).
  \] 
  The solution to the problem of finding the commuting
  classes of spin matrices now reduces to finding the
  classes of vectors $w$ that satisfy $\sigma(w_1,w_2) = 0$.
  For $p$ an odd prime, the procedure to define classes of
  four-vectors with symplectic products equal to zero
  requires a particular non-zero integer $D$ in $\Z_p$. $D$ 
  is defined by the requirement that $D \neq k^2 \mod p$ for
  all $k \in \Z_p$.

  \begin{theorem}
    Let $p$ be an odd prime. Then commuting classes of spin
    matrices are indexed by the following subsets of
    $V_4(\Z_p)$:
    \[
      C_{a_0,a_1}
      = \{(2b_0,a_0b_0+a_1b_1D, 2b_1D, a_0b_1+a_1b_0) :
      b_0,b_1 \in \Z_p\},
    \] 
    \[
      C_\infty = \{(0,b_0,0,b_1) : b_0,b_1 \in \Z_p\}
    \] 
    where $a_0,a_1 \in \Z_p$ and $(j_1,k_1,j_2,k_2)$ 
    corresponds to $S_{j_1,k_1} \otimes S_{j_2,k_2}$.
    $C_{a_0,a_1}$ is a subspace of $V_4(\Z_p)$ with basis
    \[
      G_{a_0,a_1} = \{(2,a_0,0,a_1),(0,a_1D,2D,a_0)\}
    \] 
    and $C_\infty$ has the basis $G_\infty =
    \{(0,1,0,0),(0,0,0,1)\}$. There $p^2+1$ such classes,
    and it is claimed that each class has $p^2$ members,
    that $\sigma(w_1,w_2) = 0$ for vectors in the same
    class, and that the only vector common to any pair of
    classes is $(0,0,0,0)$. 
  \end{theorem}

  \section{Mutually unbiased bases and orthogonal
  decompositions of Lie algebras (Wocjan)}

  The authors establish a connection between the problem of
  constructing maximal collections of mutually unbiased
  bases and an open problem in the theory of Lie algebras.
  They show that a collection of $\mu$ MUBs in $\C^{n}$ 
  gives rise to a collection of $\mu$ Cartan subalgebras of
  the special linea Lie algebra $sl_n(\C)$ that are pairwise
  orthogonal with respect to the Killing form. In
  particular, a complete collection of MUBs in $\C^{n}$ 
  gives rise to a so-called orthogonal decomposition of
  $sl_n(\C)$.

  It is a longstanding conjecture that ODs of $sl_n(\C)$ can
  only exist if $n$ is a prime power. This corroborates the
  general belief that a complete collection of MUBs can only
  exist in prime power dimensions.

  \subsection{Introduction} 

  Two orthonormal bases $\mathcal{B}$ and $\mathcal{B}'$ of
  the Hilbert space $\C^{n}$ are called mutually unbiased if
  and only if
  \[
    |\braket{\phi|\psi}| = 1 / \sqrt{n}
  \] 
  for all $\ket \phi \in \mathcal{B}$ and all $\ket \psi \in
  \mathcal{B}'$. By putting the vectors of the bases
  $\mathcal{B}$ and $\mathcal{B}'$ as columns of matrices
  $M_{\mathcal{B}}$ and $M_{\mathcal{B}'}$, the above
  condition says that $_{\mathcal{B}}^{*}M_{\mathcal{B}'}$ 
  should also be a generalized Hadamard (scaled by $1 /
  \sqrt{n}$). The problem of determining the maximal nmber
  of bases that are mutually unbiased is an open problem. It
  is known that $n+1$ is an upper bound on the number of
  mutually unbiased bases in dimension $n$.

  For prime power dimensions, several constructions attain
  this bound. One construction uses so-called nice error
  bases (Bandyophyay). The belief that the maximal amount of
  MUBs can only be obtained for prime power dimensions is
  reflected in a conjecture on the orthogonal decompositions
  of complex simple Lie algebras into Cartan subalgebras.
  In this way the authors obtain an alternative viewpoint
  and new properties for known constructions of MUBs in
  prime power dimensions, obtained by partitioning nice
  unitary error bases.

  \subsection{MUBs constructions and nice error bases}

  The problem of constructing mutually unbiased bases
  corresponds to partitioning unitary matrices that are
  orthogonal with respect to the trace inner product into
  certain commuting classes. The mutually unbaises bases
  correspond to the common eigenvectors of commuting
  classes.

  A \textit{unitary error basis} $\mathcal{E}$ is a basis of
  the vector space of complex $n \times n$ matrices that is
  orthogonal with respect to the trace inner product. That
  is to say, a set of unitary matrices $\mathcal{E} = \{U_1
  = 1, U_2, \ldots, U_{n^2}\}$ is a unitary error basis if
  and only if
  \[
    \Tr(U_k^{*} U_l) = n \delta_{k,l}, 
    \quad k,l \in \{1,\ldots,n^2\}.
  \] 
  Two constructions of unitary error bases are known. The
  first are nice error bases, a group-theoretic construction
  due to Knill. The second type of unitary error bases are
  shift-and-multiply bases, a combinatorial construction due
  to Werner.

  \begin{definition}
    Let $G$ be a gruop of order $n^2$ with identity element
    $e$. A set $\mathcal{N} = \{U_g \in U_n(\C) : g \in G\}$ 
    is a nice error basis if
    \begin{itemize}
      \item $U_e$ is the identity matrix.
      \item $\Tr(U_g) = 0$ for all $g \in G \setminus
        \{e\}$, and
      \item $U_g U_h = \omega(g,h)U_{gh}$ for all $g,h \in
        G$ where $\omega(g,h)$ are complex numbers of
        modulus one.
    \end{itemize}
  \end{definition}
  The group $G$ is called the index group of the nice error
  basis $\mathcal{N}$ because its elements enumerate the
  elements of $\mathcal{N}$. 

  \begin{lemma}
    Let $\mathcal{C} = \mathcal{C}_1 \cup \ldots \cup
    \mathcal{C}_\mu$ with $\mathcal{C}_k \cap \mathcal{C}_l
    = \{1\}$ for $k \neq l$ be a set of $\mu(n-1) + 1$ 
    unitary matrices that are mutually orthogonal with
    respect to the trace inner product. Furthermore, let
    each class $\mathcal{C}_k$ of the partition of
    $\mathcal{C}$ contain $n$ commuting matrices $U_{k,t}, 0
    \leq t \leq n-1$, where $U_{k,0} = 1$. For fixed $k$,
    let $\mathcal{B}_k$ contain the common eigenvectors
    $\ket{\psi_i^{k}}$ of the matrices $U_{k,j}$. Then the
    bases $\mathcal{B}_k$ form a set of $\mu$ mutuall
    unbiased bases.
  \end{lemma}

  The commuting classes in the above construction are
  maximal. This is beacuase there can be at most $n$
  mutually commuting unitary matrices acting on $\C^{n}$ 
  that are orthogonal with respect to the trace inner
  product. Let $\mathcal{C}$ be a set of mutually commuting
  matrices. Since the matrices in $\mathcal{C}$ are mutually
  commuting, they can be diagonalized simultaneously. The
  trace orthogonality of a unitary error basis implies that
  the diagonals of the elements of $\mathcal{C}$, when
  written in their common eigenbasis, must be pairwise
  orthogonal as vectors in $\C^{n}$ with respect to the
  standard inner product. Since there can be at most $n$ 
  orthogonal vectors, the above commuting classes are
  maximal.

  Unitary error bases consisting of tensor products of
  generalized Pauli matrices (a particular class of nice
  error bases) can be partitioned according to the mentioned
  Lemma so that we obtain a collection of $n+1$ MUBs for
  dimensions $n$ that are prime powers.

  Let $n = p^{e}$ be a prime power. Define the generalized
  Pauli operators acting on $\C^{p}$ as
  \[
    X = \sum_{k=0}^{p-1} \ket k \bra{k+1},
    \quad
    Z = \sum_{k=0}^{p-1} \omega^{k} \ket k \bra k,
  \] 
  where $\omega$ is the $p$th root of unity. For $(x,y) :=
  (x_1,\ldots,x_e,z_1,\ldots,z_e) \in \Z_p^e \times \Z_p^e$ 
  define the tensor product $U^{(x,y)}$ of generalized Pauli
  matrices to be 
  \[
    U^{(x,y)}
    := X^{x_1} Z^{z_1} \otimes \cdots \otimes X^{x_e}
    Z^{z_e},
  \] 
  where $\Z_p = \{0,\ldots,p-1\}$ is the cyclic group of
  order $p$ and $\Z_p^e$ is the direct product of $e$ copies
  of $\Z_p$. Then it is readily verified that the set
  $\mathcal{N} := \{U^{(x,z) : (x,z) \in \Z_p^e \times
  \Z_p^e}\}$ is a unitary error basis for $\C^{n}$. Moreoer,
  it is a nice error basis with index group $G := \Z_p^e
  \times \Z_p^e$.

  \begin{theorem}
    Let $n = p^{e}$ be a prime power dimension. Then the
    nice error basis mentioned earlier consisting of tensor
    products of generalized Pauli matrices can be
    partitioned according to the Lemma into $n+1$ commuting
    classes showing that there are $n+1$ mutually unbiased
    bases.
  \end{theorem}

  Wocjan then goes on to show that the unitary error basis
  as given, consists of monomial matrices and hence the
  complete MUB construction yields monomial MUBs, as will
  any construction based on nice error bases.

  For prime power dimensions $n = p^{e}$, there is only one
  other construction that attains the upper bound of $n+1$ 
  MUBs, this construction is based on exponential sums in
  finite fields and Galois rings that attains the upper
  bound of $n+1$ MUBs. There are 3 cases in this
  construction. At least 2 of these cases are monomial and
  strongly indicates this for the 3rd case as well. This is
  done by showing that the corresponding MUBs can be
  obtained equivalently (after a basis change) by
  partitioning the nice error bases. 

  \begin{conjecture}
    The complete collection of MUBs obtained by Klappenecker
    and Rötteler's construction of mutually unbiased bases
    is also monomial and obtained by partitioning nice error
    bases. This would imply that all known complete
    collections of MUBs have these two properties.
  \end{conjecture}

  \subsection{Lie algebras and orthogonal decompositions}

  Let $\mathcal{L}$ be a Lie algebra. A Cartan subalgebra of
  $\mathcal{L}$ is a maximal subalgebra $\mathcal{H}$ that
  is self-normalizing, i.e., if $[g,h] \in \mathcal{H}$ for
  all $h \in \mathcal{H}$, then $g \in \mathcal{H}$ as well.
  If $\mathcal{L}$ is simple, then all Cartan subalgebras of
  $\mathcal{L}$ are abelian.

  As a vector space, every complex simple Lie algebra
  $\mathcal{L}$ can be decomposed into a direct sum of
  Cartan subalgebras $\mathcal{H}_i$ 
  \[
    \mathcal{L}
    = \mathcal{H}_0 \oplus \mathcal{H}_1 \oplus \cdots
    \oplus
    \mathcal{H}_h,
  \] 
  where $h$ is the Coxeter number. The Killing form
  \[
    K(A,B)
    = \Tr(\ad A \cdot \ad B)
  \] 
  is non degenerate on $\mathcal{L}$, here $\ad A$ denotes a
  linear operator on $\mathcal{L}$ mapping $C \in
  \mathcal{L}$ to $[A,C]$. The same holds for the
  restriction of the Killing form to any Cartan subalgebra
  $\mathcal{H}$. If we require all components of the
  decomposition to be pairwise orthogonal with respect to
  the Killing form $K$,
  \[
    K(\mathcal{H}_i,\mathcal{H}_j) = 0,
    \quad i \neq j,
  \] 
  then we obtain a so called \textit{orthogonal
  decomposition} of $\mathcal{L}$. In the case of the
  special linear Lie algebra $sl_n(\C)$, there is in
  addition a natural link between orthogonal decompositions
  of Lie algebras and orthogonal paris of maximal abelian
  subalgebras in von Neumann algebras.

  \subsection{Establishing the connection between ODs and
  MUBs}

  MUBs give rise to orthogonal Cartan subalgebras and
  viceversa when these Cartan subalgebras are closed under
  the adjoint operation.

  \begin{theorem}
    If there are $\mu$ pairwise unbiased bases
    $\mathcal{B}_1,\mathcal{B}_2,\ldots,\mathcal{B}_\mu$ of
    $\C^{n}$ then there are $\mu$ Cartan subalgebras
    $\mathcal{H}_1,\mathcal{H}_2,\ldots,\mathcal{H}_\mu$ of
    $sl_n(\C)$ that are pairwise orthogonal with respect to
    the Killing form. In particular, a complete collection
    of MUBs in $\C^{n}$ gives rise to an orthogonal
    decomposition of $sl_n(\C)$.
  \end{theorem}

  \begin{theorem}
    Collections of $\mu$ Cartan subalgebras
    $\mathcal{H}_1,\mathcal{H}_2,\ldots,\mathcal{H}_\mu$ of
    $sl_n(\C)$ that are pairwise orthogonal with respect to
    the Killing form and are closed under the adjoint
    operation, i.e., the involutory map $^*$, correspond to
    collections of  $\mu$ mutually unbiased bases
    $\mathcal{B}_1,\mathcal{B}_2,\ldots,\mathcal{B}_\mu$.
  \end{theorem}

  \begin{conjecture}
    The Lie algebra $\mathcal{L} = sl_n(\C)$ has an OD only
    if $n$ is a prime power.
  \end{conjecture}

  Using the last theorems, this conjecture implies that a
  complete collection of $n+1$ MUBs exists only in prime
  power dimensions $n$.

  \section{Multicomplementary operators via finite Fourier
  transform}

  \subsection{Introduction}

  The concept of complementarity is a direct nontrivial
  consequence of the superposition principle and
  distinguishes purely quantm systems from those that may be
  accurately treated classically. The idea of
  complementarity can be loosely formulated by stating that,
  in order to understand a quantum phenomenon completely we
  need a combination of mutually exclusive properties: the
  precise knowledge of one of them implies that all possible
  outcomes in the other are equally probable.

  In finite-dimensional systems, complementarity is
  tantamount to unbiasedness: each eigenstate of any
  measurement is an equal maginutde superposition of the
  eigenstates of any of the complementary measurements. This
  leads naturally to the concepto of MUBs.

  For a $d$-dimensional system it has been found that the
  maximum number of MUBs cannot be greater than $d+1$ and
  this limit is reached if $d$ is prime or power of prime.
  In this paper Klimov et al give an explicit construction
  with a different method that resorts to elementary notions
  of finite field theory.

  \subsection{Complementary operators in prime dimension}

  We consider a system living in a Hilbert space
  $\mathcal{H}_d$, whose dimension $d$ is a prime number.
  Fix a computational basis $\ket n$, $n = 0,\ldots,d-1$ and
  introduce the operators
  \begin{align*}
    X\ket n &= \ket{n+1}\\
    Z\ket n &= \omega^{n}\ket n
  ,\end{align*} 
  where $\omega = \exp(2\pi i / d)$, and the addition of
  índices is understood to be modulo $p$. The operators $X$ 
  and $Z$ are generalizations of the Pauli matrices. They
  generate a group under multiplication known as the
  generalized Pauli group and abey the commutation relations
  \[
    ZX = \omega XZ.
  \] 
  According to the ideas of Bandyophiay, we can find $d+1$ 
  disjoint classes (each one with $d-1$ commuting operators)
  such that the corresponding eigenstates form sets of MUBs.
  The explicit construction starts with the following sets
  of oeprators:
  \begin{align*}
    \{Z^{k}\}, &k=1,\ldots,d-1\\
    \{(XZ^{m})^{k}\} &k=1,\ldots,d-1, \, m = 0,\ldots,d-1.
  \end{align*}
  One can easily check that 
  \[
    \Tr\left( Z^{k}Z^{k'*} \right) = d\delta_{kk'},
    \quad
    \Tr\left( X^{k}X^{k'*} \right) = d\delta_{kk'},
  \] 
  \[
    \Tr\left( (XZ^{m})^{k} (XZ^{m'})^{k'*} \right) 
    = d\delta_{kk'}\delta_{mm'}.
  \] 
  These pairwise orthogonality relations indicate that, for
  every value of $m$, we generate a maximal set of $d-1$ 
  commuting operators and that all these classes are
  disjoint. In addition, the common eigenstates of each
  class $m$ form different sets of unbiased bases. 

  Starting from $Z$, it is possible to obtain any element of
  the form $(XZ^{m})^{k}$ by using a combination of only two
  operators $F$ (finite Fourier transform) and $V$ (diagonal
  transformation):
  \[
    F = \frac{1}{\sqrt{d}} \sum_{n,n'=0}^{d-1} \omega^{nn'}
    \ket n \bra{n'},
  \]
  and for $d$ odd:
  \[
    V = \sum_{n=0}^{d-1} \omega^{-(n^2-n)(d+1) / 2} \ket n
    \bra n.
  \] 
  It is easy to verify that $X = F^{*}ZF$. The position and
  momentum eigenstates are Fourier transform one of the
  other. The case $d=2$ requires modifications.

  \subsection{Complementary operators in composite
  dimensions}

  The construction of the previous section fails if the
  dimension of the system is a power of a prime. The root of
  this failure can be traced to the fact that $\Z_n$ is
  generally not a field. The construction of
  multicomplementary operators cannot procced by simply
  taking powers of some basic elements.

  The field $\mathbb{F}_d$ can be represented as the field
  of equivalence classes of polynomials whose coefficients
  belong to $\Z_p$. The product in the multiplicative group
  $\mathbb{F}_d^*$ is deifned as the product of the
  corresponding polynomials modulo a primitive polynomial of
  degree $n$ irreducible in $\Z_p$. In fact,
  $\mathbb{F}_d^*$ is a cyclic group of order $d-1$: it is
  generated by powers of a primitive element $\alpha$. This
  establishes a natural order for the field elements, and
  using this order we can label the elements of a basis in
  $\mathcal{H}_d$ as
  \[
    \ket 0, \ket \alpha,
    \ket{\alpha^2},\ldots,\ket{\alpha^{d-1}}.
  \] 
  The method consists in using elements of $\mathbb{F}_d$,
  instead of natural numbers, to label the classes of
  complementary opeartors.

  For the additive group in the field $\mathbb{F}_d$ we can
  introduce additive characters as a map that fulfills
  \[
    \chi(\theta_1)\chi(\theta_2) = \chi(\theta_1+\theta_2),
    \quad \theta_1,\theta_2 \in \mathbb{F}_d.
  \] 
  All of these additive characters have the form
  \[
    \chi(\theta)
    = \exp\left( 2\pi i / p \Tr(\theta) \right). 
  \] 
  For future reference,
  \[
    \sum_{\theta \in \mathbb{F}_d}^{} \chi(\theta) = 0,
  \] 
  which leads us to the relation
  \[
    \sum_{k=0}^{d-2} \chi(\alpha^\alpha \theta) =
    d\delta_{\theta,0} - 1.
  \] 
  Klimov continues defining the diagonal operators with
  respect to the bases label by the field elements as
  follows:
  \[
    Z_q
    = \ket 0 \bra 0 + \sum_{k=1}^{d-1} \chi(\alpha^{q+k})
    \ket{\alpha^k}\bra{\alpha^k},
    \quad q = 0,\ldots,d-1.
  \] 
  This definitino implies
  \[
    Z_q \ket{\alpha^k} = 
    \chi(\alpha^{q+k}) \ket{\alpha^k},
  \] 
  and the combination property
  \[
    Z_q Z_{q'}
    = Z_{q+q'}
    = \ket 0 \bra 0
    + \sum_{k=1}^{d-1} \chi\left( \alpha^{q+k} +
      \alpha^{q'+k} \right) \ket{\alpha^k} \bra{\alpha^k}.
  \] 
  These are quite natural generalizations of the properties
  of matrices in the class $\{Z^{k}\}$, since the
  $\ket{\alpha^k}$ are eigenstates of $Z_q$. In similar
  fashion the operators $X_q$ are defined as
  \[
    X_q
    = \sum_{k=1}^{d-1} \ket{\alpha^k + \alpha^q}
    \bra{\alpha^k} + \ket{\alpha^q} \bra 0,
    \quad q = 0,\ldots,d-2,
  \] 
  so that they act as operators shifting $\ket{\alpha^k}$ to
  $\ket{\alpha^k + \alpha^q}$ and satisfy the combination
  property
  \[
    X_qX_{q'}
    = X_{q+q'}
    = \ket{\alpha^q + \alpha^{q'}} \bra 0
    + \sum_{k=1}^{d-1} \ket{\alpha^k + \alpha^q +
      \alpha^{q'}} \bra{\alpha^k}.
  \] 
  The finite Fourier transform $F$, when expressed in terms
  of the baseis $\ket{\alpha^k}$, takes the form
  \[
    F
    = \frac{1}{\sqrt{d}} 
    \left[
      \ket 0 \bra 0 + \sum_{k,k'=1}^{d-1} \chi\left(
      \alpha^{k'+k} \right) \ket{\alpha^{k'}} \bra{\alpha^k}
      + \sum_{k=1}^{d-1} (\ket 0 \bra{\alpha^k} +
      \ket{\alpha^k}\bra 0)
    \right].
  \] 
  This definition satisfies the natural property
  \[
    F^{*}F = FF^{*} = I.
  \] 
  The operator $F$ is set up to transform the operators
  $Z_q$ into $X_q$:
  \[
    X_q = F^{*} Z_q F.
  \] 
  The Weyl form of the canoncial commutation relations is
  now
  \[
    Z_qX_{q'} = \chi\left( \alpha^{q+q'} \right) X_{q'}Z_q.
  \] 
  The operators $Z_q$ and $X_q$ have been designed to be
  $d$-``periodic'', in the sense that
  \[
    Z_d = Z_0,
    \quad
    X_d = X_0.
  \] 
  In fully analogy, we can generate operators from $X_q$ and
  $Z_q$ ; they will be of the form $X_qZ_r$. Linear
  independence and orthoganility are guaranteed. The sets
  \begin{align*}
    \{Z_q\} &q = 0,\ldots,d-2,
    \{X_qZ_{q+r}\} &q,r = 0,\ldots,d-2,
  \end{align*}
  are disjoint and that every element of a set with a fixed
  value $r$ commutes with every other element in the same
  set, i.e., they define multicomplementary operators.

  \subsection{Complementary operators for two qubits}

  Consider two qubits described in a four-dimensional
  Hilbert space $\mathcal{H}_4$. Start with the field
  $\mathbb{F}_4$ obtained from the irreducible polyonomial
  over $\Z_2$, $\theta^2+\theta+1 = 0$, and the primitive
  element $\alpha$ is defined as a root of such polynomial.
  The four elements of $\mathbb{F}_4$ can be written as
  \[
    \{0,1,\alpha,\alpha+1 = \alpha^2\}.
  \] 
  A direct application of the additive characters gives us
  \[
    \chi(0) = 1,
    \,
    \chi(\alpha) = -1,
    \,
    \chi(\alpha^2) = -1,
    \,
    \chi(\alpha^3) = 1.
  \] 
  Lets consider the computational basis for the
  vectors $\ket 0, \ket \alpha, \ket{\alpha^2},
  \ket{\alpha^3}$, the matrices $Z_q$ and $X_q$ are readily
  computed. The rest of sets are routinely obtained.

  \subsection{Multicomplementary operators as tensor
  products}

  One can establish an isomorphism between the form of
  complementary operators mentioned before and its
  representation in terms of direct product of generalized
  Pauli operators. This isomorphism can be put forward by
  showing a one-to-one correspondence between the basis
  $\{\ket 0, \ket{\alpha},
  \ket{\alpha^2},\ldots,\ket{\alpha^{d-1}}\}$ and the
  coefficients of the expansion of the powers of the
  primitive element on, for instance, the polynomial basis,
  formed by $(1,\alpha,\alpha^2,\ldots,\alpha^{n-1})$. The
  representation of the operators as a tensor product is
  possible do to the fact that $\mathbb{F}_d$ is isomorphic
  to $\Z_p \times \cdots \times \Z_p$ with $n$ products.

  \section{Discrete phase-space structure of $n$-qubit
  mutually unbiased bases}

  Klimov et al work out the phase-space structure for a
  system of $n$ qubits. The axes of the phase space are
  labelled by the finite field $GF(2^{n})$ and they
  investigate the gemoetrical structures compatible with the
  notion of unbiasedness.

  Quantum mechanics describes physical systems through the
  density operator $\rho$. To overcome conceptual
  difficulties of the mathematical formulation of the
  quantum system, various phase-space methods have been
  developed. Generally these are presented in terms of
  continuous variables, typically position and momentum.
  However, there are many quantum systems that can be
  appropriately described in a finite-dimensional Hilbert
  space. The ideas presenting the discrete phase space
  picture as a $d \times d$ lattice and the reprsentation of
  the quantum system by means of the Wigner function, has
  progressed recently.  In particular when the dimension is
  a power of a prime, one can label the points in the $d
  \times d$ grid with elements of the finite Galois field
  $GF(d)$. The use of the finite field is a necessity
  because it is the only way to preserve the geometrical
  properties necessary to define the Wigner function.

  In these finite descirptions, the WIgner function, being
  the Weyl representative of the density operator, naturally
  emerges as a function that takes values only at the points
  defining the discrete mesh of the phase space. Although
  the axis observables cannot be complementary in the usual
  sense (their commutator cannot be proportional to the
  identity operator), they will have a closely related
  property: every eigenstate of either one of the them is a
  state of maximum uncertainty with respect to the other.

  It has been shown that the maximum number of MUBs can be
  at most $d+1$. If the dimension $d$ is a prime or power of
  prime, then the maximal number of MUBs can be obtained.
  Different explicit constructions of MUBs in prime power
  dimensions have been suggested in a number of recent
  papers. a

  The construction of MUBs is closely related to the
  possibility of finding $d+1$ disjoint classes, each one
  haveing $d-1$ commuting oeprators, so the corresponding
  eigenstates form sets of MUBs. Nevertheless these MUB
  operators can be organized in serveral different
  nontrivial tables, leading to different factorization
  propeorties of the MUB.

  \subsection{Mutually unbiased bases and discrete phase
  space}

  We start by considering a system living in a Hilbert space
  $\mathcal{H}_d$ whose dimension $d$ is a prime number $p$.
  Two nondegenerate tests are mutually unbiased if the bases
  formed by their eigenstates are MUBs. Therefore the
  measurements of the components of spin 1/2 along $x,y$ and
  $z$ directions are all unbiased.

  Klimov et al fix a computational basis $\ket n$, for $n =
  0,\ldots,d-1$ in the Hilbert spae and introduce the basic
  oeprators
  \[
    X\ket n = \ket{n+1},
    \quad
    Z\ket n = \omega^{n}\ket n,
  \]  
  where $\omega = \exp(2\pi i / d)$ is the $d$-th root of
  unity and addition and multiplication is understood modulo
  $d$. The operators $X$ and $Z$ are generalizations of the
  Pauli matrices and were studied long ago by Weyl. They
  generate a group under multiplication known as the
  generalized Pauli group and obey the commutation rule
  \[
    ZX = \omega XZ.
  \] 
  One can construct MUBs by finding $d+1$ disjoint classes,
  such that the corersponding eigenstates form sets of MUBs.
  Start out with the set of operators
  \[
    \{X^{k}\}, \quad \{Z^{k}X^{m k}\},
  \] 
  with $k = 1,\ldots,d-1$ and $m = 0,\ldots,d-1$. One can
  easily check that
  \[
    \Tr\left( X^{k}X^{k'*} \right) = d\delta_{k,k'},
    \quad 
    \Tr\left( Z^{k}Z^{k'*} \right) = d\delta_{k,k'},
  \] 
  \[
    \Tr\left( (Z^{k}X^{m k})(Z^{k'}X^{m;k})^{*} \right) 
    = d\delta_{k,k'}\delta_{m,m'}.
  \] 
  The pairwise orthogonality relations indicate that, for
  every value of $m$, we generate a maximal set of $d-1$ 
  commuting oeprators and that all these classes are
  disjoint. In addition, the common eigenstates of each
  class $m$ form different sets of MUBs.

  \subsection{Mutually unbiased bases for $n$ qubits}

  When the space dimension $d = p^{n}$ is a power of a prime
  it is natural to view the system as composed of $n$ 
  constituents each of dimension $p$. The main idea consists
  in labeling the states with elements of the finite field
  $GF(2^{n})$, instead of the integers. We denote by $\ket
  \alpha$ with $\alpha \in GF(2^{n})$, an orthonormal basis
  in the Hilbert space of the system. Operationally the
  elements of the basis can be labeled by posers of a
  primitive element,
  \[
    \{\ket 0, \ket \sigma, \ldots,
    \ket{\sigma^{2^{n}-1}=1}\}.
  \] 
  These vectors are eigenvectors of the operators $Z_\beta$ 
  belonging to the generalized Pauli group, whose generators
  are now defined as
  \[
    Z_\beta = \sum_{\alpha \in GF(2^{n})}^{} \chi(\alpha
    \beta) \ket \alpha \bra \alpha,
    \quad
    X_\beta = \sum_{\alpha \in GF(2^{n})}^{} \ket{\alpha +
      \beta} \bra \alpha,
  \] 
  so that
  \[
    Z_\alpha X_\beta = \chi(\alpha\beta) X_\beta Z_\alpha,
  \] 
  where $\chi$ is an additive character. The operators
  $X_\beta$ and $Z_\alpha$ can be factorized into tensor
  products of powers of single particle Pauli operators
  $\sigma_z$ and $\sigma_x$, whose expression in the
  standard basis of the two 
  \[
    \sigma_z = \ket 1 \bra 1 - \ket 0 \bra 0,
    \quad
    \sigma_x = \ket 0 \bra 1 + \ket 1 \bra 0.
  \] 
  This factorization can be carried out by mapping each
  element of $GF(2^{n})$ onto an ordered set of natural
  numbers. A convenient choice for this is the selfdual
  basis, since the finite Fourier transform factorizes then
  into a product of single-particle Fourier operators, which
  leads to
  \[
    Z_\alpha = \sigma_z^{a_1} \otimes \cdots \otimes
    \sigma_z^{a_n},
    \quad
    X_\beta = \sigma_x^{b_1} \otimes \cdots \otimes
    \sigma_x^{b_n},
  \] 
  where $(a_1,\ldots,a_n)$ and $(b_1,\ldots,b_n)$ are the
  corresponding coefficients.

  The simplest geoemetrical structures in the discrete phase
  space are straight lines, i.e., collections of points
  $(\alpha,\beta) \in GF(2^{n}) \times GF(2^{n})$ satisfying
  the relation
  \[
    \xi \alpha \eta \beta = \nu,
  \] 
  where $\xi, \eta$ and $\nu$ are fixed elements of
  $GF(2^{n})$. If the lines are not parallel they cross each
  other. A ray is a line passing through the origin, so its
  equation is
  \[
    \alpha = 0,
    \quad 
    \text{ or }
    \quad 
    \beta = \lambda \alpha.
  \] 
  Given in parametric form,
  \[
    \alpha(\kappa) = \mu \kappa,
    \quad
    \beta(\kappa) = \nu \kappa,
  \] 
  where $\kappa$ is the parameter running through the field.
  The rays are the simplest structures which are additive in
  the sense that
  \[
    \alpha(\kappa + \kappa') = \alpha(\kappa) +
    \alpha(\kappa'),
    \quad
    \beta(\kappa + \kappa') = \beta(\kappa) +
    \beta(\kappa').
  \] 
  This means that by summing the coordinates of the origin
  and of any point in a ray we obtain another point on the
  same ray. In particular, this opens the possibility of
  introducing operators that generate translations along
  these rays.

  These rays have a very remarkable property: the monomials
  $Z_\alpha X_\beta$ (labelled by phase-space points)
  belonging to the same ray commute
  \[
    Z_{\alpha_1} X_{\beta_1=\lambda \alpha_1} Z_{\alpha_2}
    X_{\beta_2 = \lambda\alpha_2}
    = Z_{\alpha_2}X_{\beta_2=\lambda\alpha_2} Z_{\alpha_1}
    X_{\beta_1=\lambda\alpha_1},
  \] 
  and thus, have a common system of eigenvectors
  $\{\ket{\psi_{v,\lambda}}\}$ with $\lambda, v \in
  GF(2^{n})$:
  \[
    Z_\alpha X_{\lambda\alpha} \ket{\psi_{v,\lambda}} =
    \exp(i\xi_{v,\lambda}) \ket{\psi_{v,\lambda}},
  \] 
  where $\lambda$ is fixed and $\exp(i\xi_{v,\lambda})$ is
  the corresponding eigenvalue, so $\ket{\psi_{v,0}} = \ket
  v$ are eigenstates of $Z_\alpha$ (displacement oeprators
  labeled by the ray $\beta = 0$, which we take as the
  horizontal axis). Indeed we have that
  \[
    |\braket{\psi_{v,\lambda}|\psi_{v',\lambda'}}|^2
    = \delta_{\lambda,\lambda'} \delta_{v,v'} + \frac{1}{d}
    (1-\delta_{\lambda,\lambda'}),
  \] 
  and, in consequence, they are MUBs. Since each ray defines
  a set of $2^{n}-1$ commuting operators, if we introduce
  $2^{n}+1$ sets of commuting operators (which from now on
  will be called displacement operators) as
  \[
    \{X_\beta\},
    \quad
    \{Z_\alpha X_{\beta=\lambda\alpha}\},
  \] 
  then we have a whole bundle of $2^{n}+1$ rays (which is
  obtained by varying the slope $\lambda$) that allows us to
  construct a complete set of MUB operators arrange in a
  $(2^{n}-1)\times(2^{n}+1)$ table.

  \section{Derivation of an explicit expression for mutually
  unbiased bases in even adn odd prime power dimensions
  (Durt)}

  It is well-known that, when the dimension of the Hilbert
  space is a prime power, there exists a set of $N+1$ 
  mutually unbiased bases. This set is maximal because it is
  not possible to find more than $N+1$ mutually unbiased
  bases in a $N$ dimensional Hilbert space. A curucial
  element of the construction is the existence of a finite
  commutative division ring (or field) of $N$ elements. As
  it is well known, finite fields with $N$ elements exist if
  and only if the dimension $N$ is a poewr of a prime and a
  derivation of a set of mutually unbaised bases is already
  known in such cases. In this paper, Durt, obtains in a
  synthetic formulation, the expressions for the mutually
  unbiased bases that were derived in the past. In odd prime
  power dimensions, we recover by a slightly different
  approach the epxressions already obtained in the past by
  Ivanonvic and odd prime power dimension $p^{m}$ by
  Wootters and Fields. They also provide a synthetic
  expression that is also valid in even prime powers
  dimensions. They do this by forming the discrete
  Heisenberg-Weyl group (sometimes called the generalized
  Pauli group), a finite group of unitary transformations.

  \subsection{Preliminary concepts}

  Assume that our system is represented by a Hilbert space
  of prime power dimension $N = p^{m}$. The corresponding
  finite field has $N$ elements. These elements shall be
  labeled by the integers $0 \leq i \leq N - 1$, or
  equivalently by a $m$-tuple of integer numbers
  $(i_0,i_1,\ldots,i_{m-1})$ where $i_j$ runs from $0$ to
  $p-1$, obtained from the $p$-ary expansion of $i$, $i =
  \sum_{k=0}^{m-1} i_n p^{n}$. Durt denotes the field
  operations by $\odot$ and $\oplus$. It is always possible
  to label the elements of the field in such a way that the
  addition is equivalent with the addition modulo $p$ 
  componentwise. He then makes the terrible decision to
  denote the  $p$-th root of unity as $\gamma_G = \gamma_G =
  e^{2\pi i / p}$. He starts with the identities,
  \[
    \sum_{j=0}^{N-1} \gamma_G^{(j \odot i}
    = N \delta_{i,0},
  \] 
  \[
    \gamma_G^i \cdot \gamma_G^j
    = \gamma_G^{(i \oplus j)},
  \] 
  virtue of the fact that the addition is the addition
  modulo $p$, componentwise.

  \subsection{Construction of the dual basis}

  . Way too complicated to make much sense.

  \section{Equiangular lines, mutually unbiased bases and
  spin models (Godsil and Roy)}

  \subsection{Introduction}

  Godsil and Roy start by mentioning that recent advances in
  quantum computing has led to renewed interest in certain
  special sets of lines in complex space. A set of $m$ 
  lines in $\C^{d}$ spanned by unit vectors $z_1,\ldots,z_m$ 
  is equiangular if there is a constant $a$ such that
  \[
    |\langle z_i, z_j \rangle| = a.
  \] 
  Very similar no? A pair of bases $x_1,\ldots,x_d$ and
  $y_1,\ldots,y_d$ in $\C^{d}$ is mutually unbiased if they
  are both orthonormal and there is a constant $a$ such that
  \[
    |\langle x_i, y_j \rangle| = a,
  \] 
  for all $i$ and $j$ when $i \neq j$. We can view a set of
  $n$ pairwise mutually unbiased bases in $\C^{d}$ as a set
  of lines in $n$ groups of size $d$ such that distinct
  lines in the same group are orthogonal, and if $x$ and $y$ 
  are unit vectors spanning lines in different groups, then
  \[
    |\langle x,y \rangle| = a.
  \] 
  They mention that it is known that a set of equiangular
  lines in $\C^{d}$ has size at most $d^2$, and that a set
  of mutually unbiased bases contains at most $d+1$ bases.
  They offer some combinatorial constructions of these types
  of lines. Starting witha  certain type of difference set
  in an abelian group, we construct lines from the
  characters of the group restricted to the difference set.
  In the case of mutually unbiased bases, starting with a
  suitable finite commutative semifield we find relative
  difference sets for a group of automorphisms, which give
  rise to a class of mutually unbiased bases. The procedure
  results in maximal sets of bases in $\C^{d}$ for any prime
  power $d$.

  An important part of the work, is that they consider
  equivalence of mutually unbiased bases. Most (but not all)
  of the known maximal sets are equivalent. Their
  construction produces several inequivalent sets, that
  equivalent to those of Calderbank, Cameron, Kantor and
  Seidel; in fact all known maximal sets are encapsulated in
  their work. Also, relating to the work of Wocjan, all
  known constructions are monomial.

  \subsection{Semifields}

  Semiregular relative difference sets are not easy to find,
  and the intention is to construct them from commutative
  semifields. A semifield is a field where multiplication
  need not be associative. Every finite field is a
  semifield. 

  Using a semifield $E$, they construct an incidence
  structure as follows. The point set of the incidence
  structure is just $E \times E$ ; we denote the points by
  ordered pairs $(x,y)$. The line set is a second copy of $E
  \times E$, where we denote a line by $[a,b]$. The element
  $a$ of the line $[a,b]$ is called its \textit{slope}. The
  point $(x,y)$ lies on the line $[a,b]$ if and only if
  \[
    y = a \circ x + b.
  \] 
  For each $c$ in $E$ we adjoin the line consisting of the
  points
  \[
    (c,y), \quad y \in E,
  \] 
  then the resulting incidence structure is the affine plane
  $AG(2,E)$.

  Next they construct some groups of automorphisms. If $a,b
  \in E$, define the map $T_{a,b}$ by
  \[
    T_{a,b}(x,y)
    := (x+a,y+b).
  \] 
  Then, if $(x,y)$ lies on $[u,v]$, then $T_{a,b}(x,y)$ lies
  on $[u,v+b-u \circ a]$. Therefore $T_{a,b}$ is an
  automorphism and the set
  \[
    T := \{T_{a,b} : a,b \in E\}
  \] 
  is an abelian group that acts transitively on points, and
  with each parallel class of lines forming an orbit of
  lines. (A parallel class is the set of lines with a given
  slope.)

  Next they define a map $S_{u,v}$ on lines by
  \[
    S_{u,v}([r,s]) := [r+u,s+v].
  \] 
  It is not hard to show that
  \[
    S_{u,v}(x,y) = (x,y+u\circ x+v).
  \]
  Therefore $S_{u,v}$ is an automorphism and
  \[
    S := \{S_{u,v} : u,v \in E\},
  \] 
  is an abelian group that acts transitively on the lines
  and has the point sets of the lines of infinite slope as
  its point orbits.

  Now define $H_{u,b}$ by
  \[
    H_{u,b} := T_{u,b} S_{u,0}.
  \] 
  Then
  \[
    H_{u,b} H_{v,d} = H_{u+v,b+d+u\circ v}.
  \] 
  Given this it is not hard to show that
  \[
    H := \{H_{u,b} : u,v \in E\}
  \] 
  is a group and that if $E$ is commutative, then $H$ is
  commutative. We also find that $H$ acts transitively on
  points and lines. This result is originally due to Hughes
  in 1956.

  \begin{theorem}
    Let $E$ be a finite commutative semifield of order $q$.
    Then the group $H$ is abelina with order $q^2$, and the
    subset
    \[
      H_0 = \{H_{u,0} : u \in E\}
    \] 
    is a relative difference set in $H$ with parameters
    $(q,q,q,1)$.
  \end{theorem}

  \begin{proposition}
    Let $E$ be a finite commutative semifield of order $q$.
    Then the characters of $H$ restricted to $H_0$, together
    with the standard basis, are a set of $q+1$ mutually
    unbiased bases in $\C^{q}$.
  \end{proposition}

  And now they construct mutually unbiased bases explicitly
  using the chracters of $H$.

  \begin{lemma}
    When $q = p^{n}$ is odd, let $\omega$ be a primitive
    $p$-th root of unity, and let $\langle a,x \rangle$ 
    denote the scalar product from $E \times E$ to $GF(p)$.
    Then
    \[
      \phi_{ab}(H_{x,y}) = \omega^{\langle 2a,x \rangle +
      \langle b,2y-x\circ x \rangle}
    \] 
    is a character of $H$.
  \end{lemma}

  When $q = 2^{n}$ is even, we need more structure. Let
  $\{e_1,\ldots,e_n\}$ be a basis for $E$ over $GF(2)$, let
  $\{\hat{e}_1, \ldots, \hat{e}_n\}$ be a basis for $R$, a
  free module over $\Z_4$. For each $x = \sum x_i e_i$ in
  $E$, $x_i \in \Z_2$, embed $x$ in $R$ as
  \[
    x \mapsto \hat{x} = \sum_{i=1}^{n} x_i \hat{e}_i.
  \] 
  Since $x_i \in \{0,1\}$, any element of $R$ can be written
  uniquely in the form $\hat{x} + 2 \hat{y}$, with $x,y \in
  E$. This map is not an additive homoorphism, but it does
  preserve addition mod 2: for any $x$ and $y$ in $E$,
  \[
    2(\hat{x}+\hat{y}) = 2\overline{x+y}.
  \] 
  Define multiplication on $R$ as follows: let
  \[
    \hat{e}_i \hat{e}_j = \overline{e_i \circ e_j}
  \] 
  for basis elements $\hat{e}_i$ and $\hat{e}_j$, and extend
  linearly to all of $R$. Then multiplication distributes
  over addition, and the embedding preserves multiplication
  mod 2:
  \[
    2\hat{x}\hat{y} = 2\overline{x \circ y}.
  \]
  More and more .. .

  \subsection{Equivalence}

  Equivalence of mutually unbiased bases was introduced by
  Calderbank, Cameron, Kantor and Seidel. The idea is to
  identify a vector in $\C^{n}$ with a point in projective
  space $PG(n-1,\C)$, so that two vectors in $\C^{n}$ are
  considered the same if they span the same $1$-dimensional
  subspace. Two sets of mutually unbiased bases
  $\{B_0,\ldots,B_d\}$ and $\{B_0',\ldots,B_d'\}$ are
  equivalent if there is a unitary operator $U$ mapping the
  first set of bases to the second set (in no particular
  order):
  \[
    \{U(B_0),\ldots,U(B_d)\} = \{B_0',\ldots,B_d'\}.
  \] 
  Note the $U$ preserves angles between lines: for any two
  subspaces $\langle x \rangle$ and $\langle y \rangle$,
  \[
    |\langle Ux, Uy \rangle| = |\langle x,y \rangle|.
  \] 
  Calderbank find several inequivalent mutually-unbiased
  bases (which they refer to as orthogonal frames) using
  symplectic spreads and $\Z_4$-Kerdoc codes. In particular,
  given a symplectic spread $\Sigma$, they show how to
  construct a maximal set of mutually unbiased bases
  $\mathcal F(\Sigma)$. They then show that two sets of
  bases $\mathcal{F}(\Sigma_1)$ and $\mathcal{F}(\Sigma_2)$ 
  are equivalent if and only if there is symplectic
  transformation sending $\Sigma_1$ to $\Sigma_2$. Using
  Kantor's result on inequivalent symplectic spreads,
  Calderbank et al, conclude that a large number of
  inquivalent sets of mutually unbiased bases exist for
  $\C^{n}$ where $n$ is an odd power of 2.

  Since there is a natural correspondence between semifields
  and symplectic semifield spreads, ourt mutually unbiased
  bases are included in those of Calderbank. In fact, all
  known maximal sets fit into that framework.

  And now, they show that the constructions of Alltop,
  Wootters and Fields, Klappenecker and Rötteler, and
  Banyopadhyay et al are all equivalent and are a special
  case of the constructions of Calderbank et al. In the
  process, we show that all known constructions of maximal
  sets of mutually unbiased bases are monomial.

  They express the mutually unbiased bases in terms of
  matrices, where the columns of each $q \times q$ matrix
  form the bases for $\C^{q}$. So, two sets of matrices are
  equivalent if there is unitary map taking one set to the
  other, up to permutations of columns and multipliying any
  column by an element of $\C$ of modulus 1.

  First for the odd case. The earliest construction was due
  to Alltop, although he epxressed his result in different
  terms. Let $\mathbb{F}$ denote the finite field of order
  $q$ and characteristic $p$. As before, $\Tr$ is the
  $GF(p)$ valued trace on $\mathbb{F}$ and $\omega$ is a
  primitive $p$-th root of unity. Definte the matrices
  $A_\alpha$ by
  \[
    A_\alpha := \frac{1}{\sqrt{q}} \left( 
      \omega^{\Tr(x+\alpha)^3+y(x+\alpha)}
    \right)_{x,y},
    \quad x,y \in \mathbb{F}.
  \] 
  If $p > 3$, then $\{A_\alpha : \alpha \in \mathbb{F}\}
  \cup \{I\}$ is a set of $q+1$ mutually unbiased bases for
  $\C^{q}$.

  The next construction was originally due to Ivanovic (in
  the case of prime dimension) and Wootters and Fields (who
  generalized Ivanovic's work to all prime powers). Define
  $W_\alpha$ by
  \[
    W_\alpha
    := \frac{1}{\sqrt{q}} \left( 
      \omega^{\Tr(\alpha x^2 + xy)}
    \right)_{x,y},
    \quad x,y \in \mathbb{F}.
  \] 
  Klappenecker and Rötteler gave a simplified proof that the
  matrices $W_\alpha$ together with the identity matrix form
  a set of mutually unbaised bases.

  \begin{lemma}
    For $p > 3$, $\{A_\alpha : \alpha \in \mathbb{F}\} \cup
    \{I\}$ is equivalent to $\{W_\alpha : \alpha \in
    \mathbb{F}\} \cup \{I\}$.
  \end{lemma}

  Thirdly, Bandyopadhyay et al gave another construction of
  the same bases. Let $\{e_u\}$ denote the standard basis
  $\C^{q}$, indexed by the elements of $\mathbb{F}$. For $a$
  in $\mathbb{F}$, define the following $q \times q$ 
  matrices:
  \begin{align*}
    X(a) &: e_u \mapsto e_{u+a},\\
    Z(a) &: e_u \mapsto \omega^{\Tr(au)} e_u
  .\end{align*}
  Clearly, the standard basis is a complet set of
  eigenvectors for $Z(a)$. It is also easy to verify that
  the vectors
  \[
    f_u = \sum_{v \in \mathbb{F}}^{} \omega^{\Tr(uv)} e_v,
    \quad u \in \mathbb{F}
  \] 
  form a complete set of eigenvectors for $X(a)$.

  Define a map from $\mathbb{F}^2$ to the $d \times d$ 
  matrices as follows
  \[
    D_{(a,b)} 
    := X(a) Z(b).
  \] 
  Up to a phase, these amtrices are sometimes called the
  generalized Pauli matrices. Each $D_{(a,b)}$ is unitary
  and monomial, and $\{D_{(a,b)}\}$, modulo scalar multiples
  of $I$, is isomorphic to $\mathbb{F}^2$ as a group.
  Bandyopadhyay et al partition these matrices into
  commuting sets and show that common eigenvectors must form
  mutally unbiased bases. Those eigenvectros are the bases
  of Wootters and Fields.

  \begin{lemma}
    Let $a,c,d$ and $b = 2ac$ be in $\mathbb{F}$. Then
    \[
      \phi_{c,d}
      = \sum_{x \in \mathbb{F}}^{} \omega^{\Tr(cx^2+2dx)}e_x
    \] 
    is an eigenvector for $D_{a,b}$.
  \end{lemma}

  Now for the case where $q$ is even. The first construction
  was due to Wootters and Fields, but Klappenecker and
  Rötteler gave a simpler description. With $q = 2^{n}$ as
  before let $T$ denote the Teichmüller set of the Galois
  ring $R = GR(4^{n})$ and let $\Tr : R \to \Z_4$ denote the
  trace. Define
  \[
    W_\alpha = \left( i^{\Tr(\alpha+2y)x} \right)_{x,y},
    \quad x,y \in T.
  \] 
  Note that $\Tr(x^2) = \Tr(x)$ in $T$, so after some
  permutation of $\alpha$ these matrices have the form
  \[
    W_\alpha = \left( i^{\Tr(\alpha x^2 +2yx)}
    \right)_{x,y},
    \quad x,y \in T,
  \] 
  which are equivalent to those after the last lemma.

  Again Bandyopadhyay constructed the same bases using Pauli
  matrices. The connection between the bases defined using
  $R = GR(4^{n})$ and the Pauli matrices, defined over
  $\mathbb{F} = GF(2^{n})$, is the natural mod 2 mapping. In
  addition to being a ring homoorphism from $R$ to
  $\mathbb{F}$ it is bijection from $T$ to $\mathbb{F}$.

  It is known that $T$ is multiplicatively closed, and any
  element of $R$ can be written $x+2y$ for $x,y \in T$. Note
  that $(x+2y)^2 = x^2$ is in $T$. Also $(x+y)^2 =
  x^2+y^2+2xy$, so for any $x$ and $y$ in $T$, $x+y+2
  \sqrt{xy}$ is the unique element of $T$ congruent to $x+y
  \mod 2$.

  Using the bijection between $T$ and $\mathbb{F}$, the
  Pauli matrices are, with $a,u \in T$,
  \begin{align*}
    X(a) &: e_u \mapsto e_{u+a+2 \sqrt{ua}} \\
    Z(a) &: e_u \mapsto (-1)^{\Tr(au)} e_u = i^{\Tr(2au)}
    e_u
  .\end{align*} 

  As in the case of $q$ odd, the eigenvectors of $D_{a,b}$ 
  are the bases described by Klappenecker and Rötteler.

  \begin{lemma}
    Let $a,c,d$ and $b = ac$ be in $T$. Then
    \[
      \phi_{c,d} = 
      \sum_{x \in T}^{} i^{\Tr(cx^2+2dx)} e_x
    \] 
    is an eigenvector for $D_{a,b}$.
  \end{lemma}

\end{document}

\documentclass[a4paper]{article}

\usepackage[utf8]{inputenc}
\usepackage[T1]{fontenc}
\usepackage{textcomp}
\usepackage[spanish]{babel}
\usepackage{amsmath, amssymb}
\usepackage{amsthm}
\decimalpoint

\DeclareMathOperator{\R}{\mathbb{R}}
\DeclareMathOperator{\C}{\mathbb{C}}
\DeclareMathOperator{\N}{\mathbb{N}}
\DeclareMathOperator{\Z}{\mathbb{Z}}
\DeclareMathOperator{\GF}{GF}
\DeclareMathOperator{\diag}{diag}

\newtheorem{definition}{Definition}
\newtheorem{theorem}{Theorem}
\newtheorem{proposition}{Proposition}
\newtheorem{lemma}{Lemma}
\newtheorem{corollary}{Corollary}
\newtheorem{example}{Example}

\title{$\Z_4$-Kerdock codes, orthogonal spreads and extremal
euclidean line-sets}
\author{Calderbank, Cameron, Kantor and Seidel - Notes}
\begin{document}
  \maketitle

  The authors were working on Perperata and Zerdock codes,
  which are non-linear binary codes that contain more
  codewords then any linear codes known. Starting from the
  work on the construction of Kerdock codes as binary images
  under certains maps, the present authors find a way to
  explain this map in terms of finite groups and geometries.
  They fix the connections between the geometry of binary
  orthogonal and symplectic vector spaces, and extremal
  complex line-sets. C and S used quadratic forms on
  $\Z_2^{m+1}$ in order to construct a family of
  $\frac{1}{2}N^2$ lines through the origin of $\R^{N}$ 
  where $N = 2^{m+1}$ such that any two are either
  perpendicular or at an angle $\theta$, where $\cos \theta
  = 1 / \sqrt{N}$. Such lines-sets were unions of
  $\frac{1}{2}N$ frames (which they define as $N$ pairwise
  orthogonal $1$-spaces).

  Kantor investigated connections between orthogonal spreads
  and Kerdock sets. When the vector space $\Z_2^{2m+2}$ is
  equipped with the quadratic form
  \[
    x_1x_{m+2} + x_2x_{m+3} + \ldots + x_{m+1}x_{2m+2},
  \] 
  there are $(2^{m+1}-1)(2^{m}+1)$ singular points, i.e.,
  $1$-spaces on which the form vanishes. He defines an
  \textit{orthogonal spread} as a partition of this set of
  points into $2^{m}+1$ totally singular $(m+1)$-spaces.
  They cite a paper where orthogonal spraeds are used to
  construct Kerdock codes and vice versa.

  The bridge between the binary world of orthogonal spreads
  and the real world of frames and line-sets is the
  exraspecial 2-groups. Each such gruop $E$ has order
  $2^{1+2k}$ for some $k$ and arises as a group of
  isometries of the Euclidean space $\R^{2^{k}}$. Also, $E /
  Z(E)$ naturally inherits a quadratic form, producing the
  desired binary geometry.

  Then, they make some changes to the group in order to
  connect \textit{symplectic spreads},  $\Z_4$-Kerdock codes
  and extremal complex line-sets. Symplectic spreads in a
  binary vector space determine $\Z_4$-Kerdock codes, while
  orthogonal spreads determine binary Kerdock codes. And
  finally they consider the analogue of the binary case
  using extrespecial $p$-groups for odd primes $p$, in order
  to relate symplectic spreads and extremal line-sets in
  $\C^{p^{m}}$. It is here where they analyze the
  differences between $p$ odd and $p = 2$.

  \section{Extraspacial 2-groups}

  Let $k$ be a fixed integer and $N = 2^{k}$. A $p$-group is
  a group in which every element has order equal to a power
  of $p$. A finite group is a $p$-group if and only if its
  order is power of $p$. For a prime $p$, a $p$-group $P$ is
  said to be \textit{extraspecial} if the centre $Z(P)$ has
  order $p$ and if $P / Z(P)$ is elementary abelian (and
  hence a vector space over $\Z_p$). Now consider the
  following description of an extraspecial $2$-group $E =
  E_k$ of order $2^{2k+1}$ as an irreducible group of
  orthogonal $N \times N$ matrices with real entries. Since
  $E$ has $2^{2k}$ distinct linear characters and $2^{1+2k}
  = 2^{2k} \cdot 1^2 + (2^{k})^2$, this will be a unique
  faithful irreducible representation of $E$. They also
  construct a group $L$ of real orthogonal transformations
  containing $E$ as a normal subgroup. Elements of $L$ act
  on $E$ by conjugation, fixing the centre $Z(E)$ of order
  $2$. Hence there is a well-defined action on the
  elementary abelian group $\overline{E} = E / Z(E)$ of
  order $2^{2k}$. This action on $\overline{E}$ preserves an
  explicit non-singular quadratic form $Q$.

  Let $V$ denote the vector space $\Z_2^{k}$ and $x \cdot
  y$ the usual dot product. Equip $\R^{N}$ with its usual
  inner product and let $O(\R^{N})$ denote the group of
  orthogonal linear transformations. Label the standard
  basis of $\R^{N}$ as $e_v$, where $v \in V$. Let $b \in
  V$, then define the permutation matrix $X(b)$ and the
  diagonal matrix $Y(b)$ as follows:
  \begin{equation}
    X(b) : e_v \mapsto e_{v+b},
    \quad
    Y(b) : \diag[(-1)^{b \cdot v}]
  \end{equation}
  The groups $X(V) := \{X(b) : b \in V\}$ and $Y(V) :=
  \{Y(b) : b \in V\}$ are contained in $O(\R^{N})$ and are
  isomorphic to the additive group $V$. Let $E := \langle
  X(V) ,Y(V) \rangle$, this is an irreducible subgorup of
  $O(\R^{N})$.
  
\end{document}
